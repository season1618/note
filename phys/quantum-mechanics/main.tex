\documentclass[uplatex]{jsarticle}

\usepackage{amssymb,amsmath}
\usepackage{bm}
\usepackage{amsthm}
\usepackage{physics}

\renewcommand{\epsilon}{\varepsilon}
\newcommand{\R}{\mathbb{R}}
\newcommand{\C}{\mathbb{C}}
\renewcommand{\H}{\mathcal{H}}
\newcommand{\U}{\mathcal{U}}

\newcommand{\inner}[2]{\langle #1, #2 \rangle}
\newcommand{\gen}[1]{\langle #1 \rangle}
\newcommand{\vectwo}[2]{\begin{pmatrix} #1\\ #2 \end{pmatrix}}
\newcommand{\vecfour}[4]{\begin{pmatrix} #1\\ #2\\ #3\\ #4 \end{pmatrix}}

\renewcommand{\(}{\left(}
\renewcommand{\)}{\right)}

\theoremstyle{definition}
\renewcommand{\proofname}{\textbf{証明}}

\newtheorem{ax}{公理}
\newtheorem{dfn}{定義}
\newtheorem{prop}{命題}
\newtheorem{lem}{補題}
\newtheorem{thm}{定理}
\newtheorem{cor}{系}
\newtheorem{ex}{例}

\title{量子力学}
\author{season07001674}
\date{\today}

\begin{document}
\maketitle
\tableofcontents

\section{数学的基礎}

\subsection{量子系}
    \begin{ax}[純粋状態]
        状態空間は可分な無限次元複素ヒルベルト空間$\H$として表される。状態空間の単位ベクトルを状態ベクトルまたは(純粋)状態と呼ぶ。状態$u$と絶対値1の複素数$c$に対して、$u$と$cu$は同じ状態と見なす。
    \end{ax}
    純粋状態の線形結合を重ね合わせ、純粋状態の古典確率混合を混合状態と呼ぶ。純粋状態と混合状態を合わせて量子状態と呼ぶ。

    状態を$\ket{u} \in \H$のように書くときケットベクトルと呼ぶ。$\ket{u}$に対して、$\ket{u}$との内積$f: \H \to C; \ket{v} \mapsto \inner{\ket{u}}{\ket{v}}$を$\bra{u}$と書き、ブラベクトルと呼ぶ。またこの内積$\bra{u} \cdot \ket{v}$を$\braket{u}{v}$と書く。

    \begin{ax}[可観測量]
        可観測量(observable)または物理量は$\H$上の自己共役演算子として表される。
    \end{ax}

    \begin{dfn}[固有状態]
        可観測量の固有ベクトルによって表される状態を固有状態という。
    \end{dfn}

    \begin{dfn}[同時固有状態]
        可観測量$A, B$に対して、$u$が両方の固有状態であるとき、同時固有状態という。
    \end{dfn}
    ある物理量の固有状態が同時に定常状態にもなっている(ハミルトニアンとの同時固有状態である)とき、その物理量の固有値を良い量子数と呼ぶ。

    \begin{ax}[射影仮説]
        状態$u$において可観測量$A$を観測すると、$A$のある固有値$\lambda$が観測され、系の状態は$\lambda$に属する固有空間への射影演算子を$P_\lambda$として$P_\lambda u$に遷移する。遷移する確率は$|P_\lambda u|^2$である(ボルンの規則)。
    \end{ax}
    可観測量が自己共役演算子なので観測される物理量は常に実数となる。

    % \begin{ax}
    %     状態$u$において、どの二つも強可換な可観測量$A_1, \dots, A_n$を観測する。
    % \end{ax}
    強可換でない可観測量について観測は定義されない。

\subsection{系の時間発展}
    \subsubsection{シュレーディンガー描像}
        観測されていない系の時刻$t$における状態ベクトル$u(t)$に対して
            \[u(t) = \U(t, t_0)u(t_0)\]
        なる$\U(t, t_0)$を時間発展演算子という。$\U$は状態ベクトルの長さを保つ演算子なのでユニタリ演算子である。
        \begin{ax}
            量子系のハミルトニアンを$H$とすると、観測がなされない場合の状態$u$の無限小の時間発展は
                \[u(t + dt) = \left(1 - \frac{iH}{\hbar}dt\right)u(t)\]
            で与えられる。
        \end{ax}
        系の時間発展を状態ベクトルの時間発展とする見方をシュレーディンガー描像という。この時間発展演算子は$\H \rightarrow \H$ではないため可観測量ではない。

    \subsubsection{ハイゼンベルク描像}
        可観測量を$A$に対して
        \begin{align*}
            \inner{u(t)}{Au(t)}
            &= \inner{\U(t, t_0)u(t_0)}{A\U(t, t_0)u(t_0)}\\
            &= \inner{u(t_0)}{\U(t, t_0)^{-1}A\U(t, t_0)u(t_0)}
        \end{align*}
        である。ここで
            \[A(t) = \U(t)^{-1}AU(t)\]
        として、状態ベクトルの代わりに可観測量の演算子が時間発展すると見ると
        \begin{align*}
            \dv{A}{t}
            &= \pdv{U^{-1}}{t}AU + U^{-1}A\pdv{U}{t}\\
            &= \(\pdv{U}{t}\)^{-1}AU + U^{-1}A\pdv{U}{t}\\
            &= -\frac{1}{i\hbar}(HU)^{-1}AU + \frac{1}{i\hbar}U^{-1}A(HU)\\
            &= \frac{1}{i\hbar}(U^{-1}A(HU) - (HU)^{-1}AU)\\
            &= \frac{1}{i\hbar}(U^{-1}AHU - U^{-1}H^{-1}AU)\\
            &= \frac{1}{i\hbar}[A, U^{-1}AU]\\
            &= \frac{1}{i\hbar}[A, H]
        \end{align*}
        これをハイゼンベルク方程式という。系の時間発展を可観測量の時間発展とする見方をハイゼンベルク描像という。

    以上を整理すると、量子力学の基本原理は次のようになる。
    \begin{enumerate}
        \item 状態は複素ヒルベルト空間$\H$の単位ベクトルである。
        \item 可観測量は$\H$上の自己共役演算子である。
        \item 状態$u$の可観測量$A$を観測すると、$A$のある固有値$\lambda$が得られ、系の状態は$P_\lambda u$に遷移する。
        \item 状態$u$で$A$を観測した際$\lambda$が得られる確率は$|P_\lambda u|^2$である。
        \item 系の無限小時間発展演算子は$1 - \frac{iH}{\hbar}dt$で与えられる。
    \end{enumerate}

\subsection{二重スリット実験}
    スリットが一つのとき、電子の分布はスリットを中心とした山形となる。スリットが二つのとき、電子の分布は二か所のスリットの位置でピークとなる形ではなく、干渉縞ができる。すなわち、電子は片方のスリットを取ってスクリーンに到達しているのではなく、二重スリットによって干渉した確率分布に従って現れる。一つの電子は観測する度に異なる位置に表れるため、観測していない時は複数の固有状態の重ね合わせであると考えられる。そして観測された瞬間に一つの固有状態に遷移する波束の収縮が起こる。これをコペンハーゲン解釈という。

    一方多世界解釈では波束の収縮は起こらず、デコヒーレンスによって干渉が喪失し、複数の世界に分岐する。

\subsection{不確定性原理}
    状態$u$が与えられたとき、可観測量$A$の不確定性$\Delta A$を標準偏差によって
    \begin{align*}
        \Delta A^2
            &= \ev{(A - \ev{A})^2}\\
            &= \ev{A^2} - 2\ev{A\ev{A}} + \ev*{\ev{A}^2}\\
            &= \ev{A^2} - \ev{A}^2
    \end{align*}
    と定義する。二つの可観測量の不確定性の積を考える。シュワルツの不等式より
    \begin{align*}
        \Delta A^2 \cdot \Delta B^2
            &= |(A - \ev{A})u| \cdot |(B - \ev{B})u|\\
            &\geq |\inner{(A - \ev{A})u}{(B - \ev{B})u}|^2 = |\inner{Au}{Bu}|^2\\
            &\geq |\Im\inner{Au}{Bu}|^2 = |\Im\inner{u}{ABu}|^2\\
            &= \left|\ev{\frac{AB - BA}{2}}\right|^2\\
            &= \frac{|\ev{[A, B]}|^2}{4}
    \end{align*}
    となる。つまり二つの可観測量が非可換のとき不確定性が同時に0にならないことがある。これを不確定性原理という。

    一方、二つの可観測量が交換するとき、それらを同時に決定することが可能である。
    \begin{thm}
        $[A, B] = 0$のとき、$A, B$の同時固有状態からなる完全直交系が存在する。つまり$A, B$の同時対角化が可能である。
    \end{thm}
    \begin{proof}
        $\psi$を$A$の固有値$\lambda$に属する固有状態とする。
            \[AB\psi = BA\psi = B(\lambda\psi) = \lambda B\psi\]
        よって$B\psi$も$\lambda$に属する固有状態となる。つまり$B$は$A$の固有空間$W$をそれ自身に移すので、$B$の固有状態による$W$の完全直交系が取れる。
    \end{proof}

\subsection{波動力学}
    \begin{dfn}[$L^2$空間]
        二乗可積分関数の集合
            \[L^2(X) = \{\psi: X \rightarrow \C \mid \int |\psi(x)|^2 \dd{x} < \infty\}\]
        を$L^2$空間と呼び、内積
            \[\inner{\psi_1}{\psi_2} = \int \psi_1^*\psi_2 \dd{x}\]
        を導入する。このとき$L^2$空間はヒルベルト空間である。
    \end{dfn}
    状態空間を$L^2$空間として定式化したものを波動力学という。このときの状態ベクトルを波動関数と呼ぶ。波動関数が$n$粒子の位置$r_1, r_2, \dots, r_n$の関数$\psi(r_1, r_2, \dots, r_n) \in L^2(\R^{3n})$であるとき、位置表示の波動関数という。多粒子系の波動関数を多体波動関数と呼ぶこともある。

    ただしデルタ関数は$L^2$に属していないため注意が必要である。

    波動関数は二乗可積分であり、二乗可積分な関数は無限遠において0に収束する。

    $L^2$空間上の可観測量は微分演算子として定義できる。
    \begin{dfn}[位置演算子・運動量演算子・軌道角運動量演算子]
        3次元直交座標$(x, y, z)$によって位置表示された一粒子の波動関数に対する位置演算子、運動量演算子、軌道角運動量演算子を次のように定義する。
        \begin{align*}
            \hat{r} &= (\hat{x}, \hat{y}, \hat{z}) = (x, y, z)\\
            \hat{p} &= (\hat{p_x}, \hat{p_y}, \hat{p_z}) = \(-i\hbar\pdv{x}, -i\hbar\pdv{y}, -i\hbar\pdv{z}\)\\
            \hat{L} &= \hat{r} \times \hat{p} = -i\hbar \(y\pdv{z} - z\pdv{y}, z\pdv{x} - x\pdv{z}, x\pdv{y} - y\pdv{x}\)
        \end{align*}
    \end{dfn}
    \begin{proof}
    \end{proof}

    % 軌道角運動量の二乗演算子は
    % \begin{align*}
    %     \hat{L_x}^2
    %     &= -\hbar^2 \(y\pdv{z} - z\pdv{y}\)^2\\
    %     &= -\hbar^2 \(y^2\pdv[2]{z} - y\pdv{y} - yz\pdv{y}\pdv{z} - z\pdv{z} - yz\pdv{y}\pdv{z} + z^2\pdv[2]{y}\)\\
    % \end{align*}
    % \begin{align*}
    %     \hat{L^2}
    %     &= \hat{L_x}^2 + \hat{L_y}^2 + \hat{L_z}^2\\
    %     &= 
    % \end{align*}

    % 軌道角運動量を極座標で書き換える。
    % \begin{align*}
    %     x &= r\sin\theta\cos\phi\\
    %     y &= r\sin\theta\sin\phi\\
    %     z &= r\cos\theta\\
    %     \pdv{x} &= \sin\theta\cos\phi\pdv{r} + r\cos\theta\cos\phi\pdv{\theta} - r\sin\theta\sin\phi\pdv{\phi}\\
    %     \pdv{y} &= \sin\theta\sin\phi\pdv{r} + r\cos\theta\sin\phi\pdv{\theta} + r\sin\theta\cos\phi\pdv{\phi}\\
    %     \pdv{z} &= \cos\theta\pdv{r} - r\sin\theta\pdv{\theta}
    % \end{align*}
    % より
    % \begin{align*}
    %     L_x &= -i\hbar \(r\cos\theta\sin\theta\sin\phi\pdv{r} - r^2\sin^2\theta\sin\phi\pdv{\theta} - r\cos\theta\sin\theta\sin\phi\pdv{r} - r^2\cos^2\theta\sin\phi\pdv{\theta} - r^2\cos\theta\sin\theta\pdv{\phi}\)\\
    %         &= i\hbar \(r^2\sin\phi\pdv{\theta} + r^2\cos\theta\sin\theta\pdv{\phi}\)\\
    %     L_y &= i\hbar \(r^2\cos\phi\pdv{\theta} + r^2\cos\theta\sin\theta\pdv{\phi}\)\\
    %     L_z &= -i\hbar \(
    %         r\sin^2\theta\cos\phi\sin\phi\pdv{r} + r^2\cos\theta\sin\theta\cos\phi\sin\phi\pdv{\theta} + r^2\sin^2\theta\cos^2\phi\pdv{\phi}
    %         - r\sin^2\theta\cos\phi\sin\phi\pdv{r} - r^2\cos\theta\sin\theta\cos\phi\sin\phi\pdv{\theta} + r^2\sin^2\theta\sin^2\phi\pdv{\phi}
    %         \)\\
    %         &= -i\hbar r^2\sin^2\theta\pdv{\phi}
    % \end{align*}
    である。軌道角運動量の大きさは
    \begin{align*}
        % \hat{L^2} &= \hat{L_x}^2 + \hat{L_y}^2 + \hat{L_z}^2\\
        %           &= -\hbar^2 \(r^2\pdv[2]{\theta}
        \hat{L^2} = -\hbar^2 \left[\frac{1}{\sin\theta}\pdv{\theta}\(\sin\theta\pdv{\theta}\) + \frac{1}{\sin^2\theta}\pdv[2]{\phi}\right]
    \end{align*}
    となる。

    各演算子の交換関係は次のようになっている。
    \begin{align*}
        &[x, y] = [y, z] = [z, x] = 0\\
        &[p_x, p_y] = [p_y, p_z] = [p_z, p_x] = 0\\
        &[x, p_x] = [y, p_y] = [z, p_z] = i\hbar\\
        &[L_x, L_y] = i\hbar L_z\\
        &[L^2, L_x] = [L^2, L_y] = [L^2, L_z] = 0
    \end{align*}
    軌道角運動量は各成分を同時に決めることができないので、特定の向きを持ったベクトルを想定するのは正しくない。

    波動関数$\psi(t)$の時間発展は
    \begin{align*}
        \psi(t + dt) &= \left(1 - \frac{iH}{\hbar}\right)\psi(t)\\
        i\hbar\pdv{\psi}{t} = H\psi(t)
    \end{align*}
    となる。これをシュレーディンガー方程式という。

    \begin{dfn}[シュレーディンガー演算子]
    \end{dfn}

    \begin{thm}[エーレンフェストの定理]
            \[\dv{\ev{p}}{t} = -\ev{\nabla V}\]
    \end{thm}
    \begin{proof}
        \begin{align*}
            \dv{\ev{p}}{t}
            &= \dv{t}(\int \psi^* \cdot -i\hbar\nabla \psi \dd[3]{r})\\
            &= \int \(i\hbar\pdv{\psi}{t}\)^* \cdot \nabla\psi - \psi^* \cdot \nabla\(i\hbar\pdv{\psi}{t}\) \dd[3]{r}\\
        \end{align*}
        シュレーディンガー方程式
            \[i\hbar\pdv{\psi}{t} = \(-\frac{\hbar^2}{2m}\Delta + V\)\psi\]
        より
        \begin{align*}
            \dv{\ev{p}}{t}
            &= \int \(-\frac{\hbar^2}{2m}\Delta + V\)\psi^* \cdot \nabla\psi - \psi^* \cdot \nabla\left[\(-\frac{\hbar^2}{2m}\Delta + V\)\psi\right] \dd[3]{r}\\
            &= \int \psi^* \cdot - \nabla V \psi \dd[3]{r}\\
            &= -\ev{\nabla V}
        \end{align*}
        となる。
    \end{proof}
    つまり量子力学において物理量の期待値を考えると、古典力学におけるニュートンの運動方程式が導かれる。

\subsection{行列力学}
    \begin{dfn}[$l^2$空間]
        ノルムが有限な無限次元の列ベクトルの集合
            \[l^2 = \{a \in \C^\infty \mid a_1^2 + a_2^2 + \dots < \infty\}\]
        を$l^2$空間と呼び、内積
            \[\inner{a}{b} = a_1^*b_1 + a_2^*b_2 + \cdots\]
        を導入する。このとき$l^2$空間はヒルベルト空間である。
    \end{dfn}
    $N$粒子の系に対して、状態空間を$l^2$として定式化したものを行列力学という。$l^2$空間におけるケットベクトル$\ket{\psi} = (\psi_1, \psi_2, \dots)^\top \in l^2$に対してブラベクトルは$\bra{\psi} = (\psi_1^*, \psi_2^*, \dots)$となる。可観測量は無限次元の行列である。

% \subsection{経路積分}

% 光の粒子性 黒体放射、光電効果、コンプトン効果
% 電子の波動性 二重スリット実験、電子線回折、デイヴィソン=ガーマーの実験
% 物理量は確率的に決まる 二重スリット実験
% 物質は波のように振る舞う(確率混合ではなく重ね合わせ) 二重スリット実験
% 物理量はしばしば離散的な値となる 線スペクトル
% エネルギーの最小値が正であり得る(零点エネルギー) 線スペクトル
% 観測によって状態が遷移する(波束の収縮) ?
% 同時に観測できる物理量の組合せには制限がある(不確定性原理) 零点振動、シュテルン=ゲルラッハの実験
\section{固有値問題}

\subsection{時間に依存しないシュレーディンガー方程式}
    ハミルトニアン$H$が時間に依存しないとき、$H$と$i\hbar\pdv{t}$は交換するので、同時固有状態$\psi_1, \psi_2, \dots, \psi_n$を持つ。それぞれの固有値を$E_1, E_2, \dots, E_n$とすれば
        \[H(r)\psi_i(r, t) = i\hbar\pdv{t}\psi_i(r, t) = E_i\psi_i(r, t)\]
    となる。右側の方程式を解くと
        \[\psi(r, t) = \phi_i(r)e^{-i\frac{E_i}{\hbar}t}\]
    であり、第一式に代入すると
        \[H\phi_i(r) = E_i\phi_i(r)\]
    となる。これを時間に依存しないシュレーディンガー方程式という。$\phi_i(r)$も波動関数と呼ぶ。
    
    一般解は、定数$c_n$を用いて
        \[\psi(r, t) = \sum_n c_n\psi_n(r, t) = \sum_n c_n\phi_n(r)e^{-i\frac{E_n}{\hbar}t}\]
    となる。$\psi(x, t)$の規格化条件は
        \[\int \psi^*\psi \dd[3]{r} = \sum_m\sum_n c_mc_n\phi_m^*\phi_n e^{i\frac{E_m - E_n}{\hbar}t} = 1\]
    エルミート演算子の異なる固有値に属する固有関数の内積は0になるので
        \[\sum_n c_n^2 = 1\]
    となる。初期条件から係数が決定し、確率分布は一定である。

% \subsection{演算子の交換関係}
% アーベル群の全ての元の同時固有状態からなる完全直交系が存在する。

\subsection{対称性と固有値問題}
    ハミルトニアンと$G_i$が交換する、つまり
        \[[H, G_i] = 0\]
    のとき
        \[HG_i\phi = G_iH\phi = G_iE\phi = EG_i\phi\]
    より、ハミルトニアンの固有値$E$に属する任意の固有関数$\phi$に対して$G_i\phi$もまた$E$に属する固有関数である。

    同時固有状態を利用すると、波動関数の形を制限することで、シュレーディンガー方程式を解く際の計算量を減らしたり、縮退の解消を対称群によって記述したりすることができる。

\subsection{空間反転対称性}
    % ハミルトニアン$H(x)$が時間に依存しない偶関数であるとき、固有値$E$と固有関数$\phi(x)$に対して
    % \begin{align*}
    %     H(x)\phi(x) &= E\phi(x)\\
    %     H(x)\phi(-x) &= E\phi(-x)
    % \end{align*}
    % より$\phi(-x)$も同じ固有値$E$に属する固有関数である。$\phi(x), \phi(-x)$が線形従属なら定数$s$を用いて$\phi(-x) = s\phi(x)$だから
    % \begin{gather*}
    %     \phi(x) = s\phi(-x) = s^2\phi(x)\\
    %     s = \pm 1
    % \end{gather*}
    % より$\phi(x)$は偶関数または奇関数である。$\phi(x), \phi(-x)$が線形独立なら
    %     \[\phi_+(x) = \frac{\phi(x) + \phi(-x)}{2},\ \phi_-(x) = \frac{\phi(x) - \phi(-x)}{2}\]
    % とすれば、線形独立な固有関数として偶関数と奇関数を選ぶことができる。従って固有関数を偶関数または奇関数と仮定して良い。

    一次元ハミルトニアン$H(x)$が空間反転に対して対称、つまり偶関数であるとき
    \begin{align*}
        G &= \{R_+, R_-\}\\
        R_+&: \phi(x) \mapsto \phi(x)\\
        R_-&: \phi(x) \mapsto \phi(-x)
    \end{align*}
    とすると、$G$の元はいずれも$H$と交換する。$R_-$の固有値を$s$とおくと
        \[R_-^2\phi(x) = s^2\phi(x)\]
    より$s = \pm 1$。$s = 1$のとき偶関数であり、$s = -1$のとき奇関数である。つまり、偶関数と奇関数からなる完全系が存在する。

\subsection{波動関数の分解}
    \begin{thm}
        ハミルトニアンが$H(p_1, \dots, p_m, q_1, \dots, q_n) = H_1(p_1, \dots, p_m) + H_2(q_1, \dots, q_n)$のように分解できるとき、$H$の固有関数として、$H_1, H_2$の固有値と固有関数を用いて$\phi(p_1, \dots, p_m, q_1, \dots, q_n) = \phi_1(p_1, \dots, p_m)\phi_2(q_1, \dots, q_n)$という形のものからなる完全系が存在する。
    \end{thm}
    \begin{proof}
        $H$はエルミート演算子なので
        \begin{align*}
            H_1(p_i) + H_2(q_j) &= H_1^\dagger(p_i) + H_2^\dagger(q_j)\\
            H_1(p_i) - H_1^\dagger(p_i) &= -(H_2(q_j) - H_2^\dagger(q_j))
        \end{align*}
        左辺は$p_i$のみに依存し、右辺は$q_j$のみに依存するので、両辺を定数$k$とおくことができる。
        \begin{align*}
            \phi^*(H_1 - H_1^\dagger)\phi &= \phi^* k \phi\\
            \phi^*(H_1^\dagger - H_1)\phi &= \phi^* k^* \phi\\
            \frac{k + k^*}{2} &= 0
        \end{align*}
        つまり
            \[k = \frac{k + k^*}{2} + \frac{k - k^*}{2} = \frac{k - k^*}{2}\]
        と表せる。このとき
            \[H = \(H_1 - \frac{k}{2}\) + \(H_2 + \frac{k}{2}\)\]
        と分解でき、それぞれの項はエルミート演算子である。

        エルミート演算子には固有関数からなる完全系が存在し、それぞれ$\{\phi_{1i}\},\ \{\phi_{2j}\}$とおく。$H$の固有値$\lambda$に属する固有関数$\phi$を
            \[\phi = \sum_{i, j} \phi_{1i}\phi_{2j}\]
        と展開できる。ここで
            \[H(\phi_{1i}\phi_{2j}) = (H_1 + H_2)(\phi_{1i}\phi_{2j}) = (\lambda_1 + \lambda_2)\phi_{1i}\phi_{2j}\]
        となり、$\phi_{1i}\phi_{2j}$は$H$の固有関数である。またその固有値は$\lambda = \lambda_1 + \lambda_2$である。
    \end{proof}
    \begin{thm}
        ハミルトニアンが$H(p_1, \dots, p_m, q_1, \dots, q_n) = H_1(p_1, \dots, p_m) + H_2(q_1, \dots, q_n)$のように分解でき、$H_1$のある固有関数系が完全系を成すとき、$H$の固有値と固有関数は$H_1, H_2$の固有値と固有関数を用いて$\lambda = \lambda_1 + \lambda_2,\ \phi(p_1, \dots, p_m, q_1, \dots, q_n) = \phi_1(p_1, \dots, p_m)\phi_2(q_1, \dots, q_n)$と書ける。
    \end{thm}
    \begin{proof}
        $H$の固有値$\lambda$に属する固有関数$\phi(p_1, \dots, p_m, q_1, \dots, q_n)$を$H_1$の固有関数完全系$\{\phi_{11}, \phi_{12}, \dots, \phi_{1m}\}$を用いて
            \[\phi(p_i, q_j) = \sum_i \phi_{1i}(p_i)\phi_{2i}(q_j)\]

            \[H\phi(p_i, q_j) = (H_1 + H_2)\phi(p_i, q_j) = \sum_i \lambda_{1i}\phi_{1i}\phi_{2i} + \sum_i \phi_{1i}(p_i)H_2\phi_{2i}(q_j)\]
        \begin{align*}
            H_1\phi_1\phi_2 &= \lambda_1\phi_1\phi_2\\
            H_2\phi_1\phi_2 &= \lambda_2\phi_1\phi_2\\
            (H_1 + H_2)\phi_1\phi_2 &= (\lambda_1 + \lambda_2)\phi_1\phi_2
        \end{align*}
        より$\phi_1\phi_2$は$H$の固有関数である。
    \end{proof}
\section{一粒子系}

\subsection{平面波}
    ポテンシャルが存在しないとき、時間に依存しないシュレーディンガー方程式は
        \[-\frac{\hbar^2}{2m}\Delta\phi(r) = E\phi(r)\]
    である。波数ベクトル$k$、角振動数$\omega = E / \hbar$とおけば
        \[\psi(r, t) = Ae^{i\(k \cdot r - \omega t\)}\]
    であり平面波を表す。
        \[\(-i\hbar\pdv{x}, -i\hbar\pdv{y}, -i\hbar\pdv{z}\)\psi(r, t) = \hbar k\psi(r, t)\]
    となるので、平面波は運動量の固有状態である。つまりド・ブロイ波の式$p = \hbar k = h / \lambda,\ E = \hbar\omega = h\nu$を満たす。

\subsection{ガウス波束}
    波動関数
    \begin{align*}
        \psi(r, t) &= \int \rho(k)e^{i(k \cdot r - \omega t)} \dd{k}\\
        \rho(k) &= \(\frac{\sigma^2}{2\pi^3}\)^{3/4}\exp\{-\sigma^2(k - k_0)^2 - i(k - k_0) \cdot r_0\}
    \end{align*}
    をガウス波束という。

\subsection{一次元井戸型ポテンシャル}
    $L > 0, V_0 > 0$に対して、一次元井戸型ポテンシャル
        \[V(x) =
            \begin{cases}
                V_0 & (x < -L)\\
                0 & (-L < x < L)\\
                V_0 & (L < x)
            \end{cases}
        \]
    を考える。時間に依存しないシュレーディンガー方程式は
        \[
            \begin{cases}
                -\frac{\hbar^2}{2m}\dv[2]{\phi}{x} = E\phi & (|x| < L)\\
                -\frac{\hbar^2}{2m}\dv[2]{\phi}{x} = (E - V_0)\phi & (|x| > L)
            \end{cases}
        \]
    となる。ポテンシャルが偶関数だから固有関数は偶関数または奇関数として良く、境界条件も$x = L$でのみ考えれば良い。

    (1)$E < 0$のとき

    $2mE / \hbar^2 = - k_1^2, 2m(E - V_0) / \hbar^2 = - k_2^2 \ (k_1, k_2 > 0)$とおくと解は
    \begin{align*}
        \phi(x) = 
        \begin{cases}
            C_+ e^{k_2x} + C_- e^{- k_2x} & (x < -L)\\
            A_+ e^{k_1x} + A_- e^{- k_1x} & (-L < x < L)\\
            B_+ e^{k_2x} + B_- e^{- k_2x} & (L < x)
        \end{cases}\\
        k_2^2 - k_1^2 = \frac{2mV_0}{\hbar^2}
    \end{align*}
    となる。無限遠で0に収束するので$C_- = B_+ = 0$である。
    \begin{align*}
        \phi_\pm(x) =
        \begin{cases}
            \pm B e^{k_2x} & (x < -L)\\
            A e^{k_1x} \pm A e^{- k_1x} & (-L < x < L)\\
            B e^{- k_2x} & (L < x)
        \end{cases}
    \end{align*}
    境界条件より
    \begin{align*}
        \phi_\pm(L) = A e^{k_1L} \pm A e^{-k_1L} = B e^{-k_2L}\\
        \phi_\pm'(L) = k_1 A e^{k_1L} \mp k_1 A e^{-k_1L} = -k_2 B e^{-k_2L}\\
    \end{align*}
    第一式を$k_2$倍して辺々足すと
    \begin{align*}
        k_1 A (e^{k_1L} + e^{-k_1L})
        \therefore B = A (e^{k_1L} \pm e^{- k_1L}) e^{k_2L}
    \end{align*}
    従って
    \begin{align*}
        \phi_+(x) =
        \begin{cases}
            A (e^{k_1L} + e^{- k_1L}) e^{k_2(L + x)}\\
            A (e^{k_1x} + e^{k_1x})\\
            A (e^{k_1L} + e^{- k_1L}) e^{k_2(L - x)}
        \end{cases}\\
        \phi_-(x) =
        \begin{cases}
            - A (e^{k_1L} - e^{- k_1L}) e^{k_2(L + x)}\\
            A (e^{k_1x} - e^{k_1x})\\
            A (e^{k_1L} - e^{- k_1L}) e^{k_2(L - x)}
        \end{cases}
    \end{align*}

    (2)$0 < E < V_0$のとき

    $2mE / \hbar^2 = k_1^2, 2m(E - V_0) / \hbar^2 = - k_2^2 \ (k_1, k_2 > 0)$とおくと解は
    \begin{gather*}
        \phi(x) = 
        \begin{cases}
            B_+ e^{k_2x} + B_- e^{- k_2x} & (x < -L)\\
            A_+ e^{ik_1x} + A_- e^{- ik_1x} & (-L < x < L)\\
            C_+ e^{k_2x} + C_- e^{- k_2x} & (L < x)
        \end{cases}\\
        k_1^2 + k_2^2 = \frac{2mV_0}{\hbar^2}
    \end{gather*}
    となる。無限遠で0に収束するので$B_- = C_+ = 0$である。

    \begin{align*}
        \phi_\pm(x) =
        \begin{cases}
            \pm B e^{k_2x} & (x < -L)\\
            A (e^{ik_1x} \pm e^{-ik_1x}) / 2 & (-L < x < L)\\
            B e^{- k_2x} & (L < x)
        \end{cases}
    \end{align*}
    境界条件より
    \begin{align*}
        \phi_\pm(L) &= A (e^{ik_1L} \pm e^{-ik_1L}) / 2 = B e^{- k_2L}\\
        \phi_\pm'(L) &= ik_1 A (e^{ik_1L} \mp e^{-ik_1L}) / 2 = - k_2 B e^{- k_2L}
    \end{align*}
    定数は
        \[B = A \frac{(e^{ik_1L} \pm e^{-ik_1L})}{2} e^{k_2L}\]
    また
        \[k_1 \tan(k_1L) = k_2,\ k_1 / \tan(k_1L) = - k_2\]
    が成り立つので$k_1^2 + k_2^2 = 2mV_0 / \hbar^2$と合わせて$k_1, k_2$及びエネルギー固有値$E$が求まる。

    従って
    \begin{align*}
        \phi_+(x) =
        \begin{cases}
            A \cos(k_1L) e^{k_2(L + x)} & (x < -L)\\
            A \cos(k_1x) & (-L < x < L)\\
            A \cos(k_1L) e^{k_2(L - x)} & (L < x)
        \end{cases}\\
        \phi_-(x) =
        \begin{cases}
            - A \sin(k_1L) e^{k_2(L + x)} & (x < -L)\\
            A \sin(k_1x) & (-L < x < L)\\
            A \sin(k_1L) e^{k_2(L - x)} & (L < x)
        \end{cases}
    \end{align*}
    このようにポテンシャルによって波動関数が無限遠で0に収束する状態を束縛状態と呼び、離散スペクトルを持つ。運動エネルギーが負の領域にも正の確率が分布している。

    $V_0 \to \infty$のとき$k_2 \to \infty$なので、$|x| > L$において
    % \begin{align*}
    %     A \cos(k_1L) e^{k_2(L - x)} \to A \cos(k_1L)\\
    %     - k_2 A \cos(k_1L) e^{k_2(L - x)} \to - k_2 A \cos(k_1L)
    % \end{align*}
    波動関数が恒等的に0となるものを除外すると、境界条件より
    \begin{align*}
        \phi_+(L) &= A \cos(k_1L) = 0\\
        \therefore k_1L &= \frac{\pi}{2}n \ (n = 1, 3, 5, \dots)\\
        \phi_-(L) &= A \sin(k_1L) = 0\\
        \therefore k_1L &= \frac{\pi}{2}n \ (n = 2, 4, 6, \dots)
    \end{align*}
    となる。よってエネルギー固有値は
        \[E_n = \frac{\hbar^2}{2m}\(\frac{n\pi}{2L}\)^2 \ (n = 1, 2, 3, \dots)\]
    である。エネルギーが最小の状態を基底状態、それ以外を励起状態という。量子力学では、基底状態においてもエネルギーが0とならず、零点エネルギーと呼ばれる。

    (3)$V_0 < E$のとき

    $2mE / \hbar^2 = k_1^2, 2m(E - V_0) / \hbar^2 = k_2^2 \ (k_1, k_2 > 0)$とおくと解は
    \begin{gather*}
        \phi(x) = 
        \begin{cases}
            C_+ e^{ik_2x} + C_- e^{- ik_2x} & (x < -L)\\
            A_+ e^{ik_1x} + A_- e^{- ik_1x} & (-L < x < L)\\
            B_+ e^{ik_2x} + B_- e^{- ik_2x} & (L < x)
        \end{cases}\\
        k_1^2 - k_2^2 = \frac{2mV_0}{\hbar^2}
    \end{gather*}
    となる。$\phi(x)$は偶関数または奇関数なので
    \begin{gather*}
        \phi_\pm(x) = 
        \begin{cases}
            \pm B_- e^{ik_2x} \pm B_+ e^{- ik_2x} & (x < -L)\\
            A e^{ik_1x} \pm A e^{- ik_1x} & (-L < x < L)\\
            B_+ e^{ik_2x} + B_- e^{- ik_2x} & (L < x)
        \end{cases}
    \end{gather*}
    となる。境界条件より
    \begin{align*}
        \phi_\pm(x) &= A (e^{ik_1L} \pm e^{-ik_1L}) = B_+ e^{ik_2L} + B_- e^{-ik_2L}\\
        \phi_\pm'(x) &= ik_1 A (e^{ik_1L} \mp e^{-ik_1L}) = ik_2 B_+ e^{ik_2L} - ik_2 B_- e^{-il_2+}
    \end{align*}
    第一式$ik_2$倍して足し引きすると
    \begin{align*}
        2 ik_1 A e^{ik_1L} = 2 ik_2 B_+ e^{ik_2L}\\
        B_+ = \frac{k_1}{k_2} A e^{i(k_1 - k_2)L}\\
        \pm 2 ik_1 A e^{-ik_1L} = 2 ik_2 B_- e^{-ik_2L}\\
        B_- = \frac{k_1}{k_2} A e^{-i(k_1 - k_2)L}
    \end{align*}
    このように波動関数が無限遠で平面波に漸近する状態を散乱状態と呼び、連続スペクトルを持つ。
    % \begin{align*}
    %     - \frac{\hbar^2}{2m}\dv[2]{\phi}{x} &= E\phi & (0 < x < L)\\
    %     \de[^2\phi]{x^2} &= - \frac{2mE}{\hbar}\phi\\
    %     \phi(x) = A\cos\frac{\sqrt{2mE}}{\hbar}x + B\sin\frac{\sqrt{2mE}}{\hbar}x
    % \end{align*}
    % 境界条件より
    % \begin{align*}
    %     \phi(0) &= A = 0\\
    %     \phi(L) &= B\sin\frac{\sqrt{2mE}L}{\hbar} = 0
    % \end{align*}
    % $B = 0$のとき波動関数が恒等的に0になるので不適。つまり
    %     \[\frac{\sqrt{2mE}L}{\hbar} = n\pi (n = 1, 2, \ldots)\]
    % 規格化条件から係数$B$が求まる。
    % \begin{align*}
    %     \int_0^L \sin^2\frac{\sqrt{2mE}}{\hbar} dx
    %     &= \int_0^L \sin\frac{n\pi}{L} dx\\
    %     &= \int_0^L \frac{1 - \cos\frac{2n\pi}{L}}{2} dx\\
    %     &= \frac{L}{2}
    % \end{align*}
    % より$B = \sqrt{\frac{2}{L}}$となる。よって固有関数とエネルギー固有値は
    % \begin{gather*}
    %     \phi_n(x) &= \sqrt{\frac{2}{L}}\sin\frac{\sqrt{2mE}}{\hbar}x\\
    %     E_n &= \frac{\pi^2\hbar^2}{2mL^2}n^2
    % \end{gather*}

\subsection{一次元調和振動子}
    ポテンシャルエネルギーを
        \[V(x) = \frac{m\omega^2}{2}x^2\]
    とする。時間に依存しないシュレーディンガー方程式は
        \[\(-\frac{\hbar^2}{2m}\dv[2]{x} + \frac{m\omega^2}{2}x^2\)\phi(x) = E\phi\]
    となる。ここで
    \begin{align*}
        \xi &= \sqrt{\frac{m\omega}{\hbar}}x\\
        \epsilon &= \frac{2}{\hbar\omega}E\\
        H(\xi) &= \phi(\xi)e^{\xi^2 / 2}
    \end{align*}
    とおくと、$\dv*[2]{x} = (m\omega / \hbar)\dv*{\xi}$より
    \begin{align*}
        \dv[2]{\xi}(H(\xi)e^{-\xi^2 / 2}) + (\epsilon - \xi^2)(H(\xi)e^{-\xi^2 / 2}) &= 0\\
        \dv[2]{H}{\xi}e^{-\xi^2 / 2} + 2\dv{H}{\xi} \cdot -\xi e^{-\xi^2 / 2} + H(\xi) \cdot (-e^{-\xi^2 / 2} + \xi^2e^{-\xi^2 / 2}) + (\epsilon - \xi^2)H(\xi)e^{-\xi^2 / 2}&= 0\\
        \dv[2]{H}{\xi} - 2\xi\dv{H}{\xi} + (-1 + \xi^2)H(\xi) + (\epsilon - \xi^2)H(\xi) &= 0\\
        \dv[2]{H}{\xi} - 2\xi\dv{H}{\xi} + (\epsilon - 1)H(\xi) &= 0
    \end{align*}
    この微分方程式は$\epsilon = 2n + 1\ (n = 0, 1, 2, \dots)$のときに限り解が存在し、エルミート多項式$H_n(\xi)$と呼ばれる。従って固有関数とエネルギー固有値は
    \begin{align*}
        \phi_n(x) &= H_n(x)e^{\frac{m\omega}{2\hbar}x^2}\\
        E_n &= \hbar\omega\(n + \frac{1}{2}\)
    \end{align*}
    となる。

\subsection{球対称ポテンシャル}
    球面極座標$(r, \theta, \phi)$において、動径のみに依存するポテンシャル$V(r)$を考える。三次元極座標ラプラシアンは
        \[\Delta = \frac{1}{r^2}\pdv{r}\(r^2\pdv{r}\) + \frac{1}{r^2\sin\theta}\pdv{\theta}\(\sin\theta\pdv{\theta}\) + \frac{1}{r^2\sin^2\theta}\pdv[2]{\phi}\]
    なので、波動関数を$\psi(r, \theta, \phi)$とすればシュレーディンガー方程式は
    \begin{align*}
        -\frac{\hbar^2}{2\mu}\left[\frac{1}{r^2}\pdv{r}\(r^2\pdv{r}\) + \frac{1}{r^2\sin\theta}\pdv{\theta}\(\sin\theta\pdv{\theta}\) + \frac{1}{r^2\sin^2\theta}\pdv[2]{\phi}\right]\psi + V(r)\psi &= E\psi\\
        \left[\pdv{r}\(r^2\pdv{r}\) + \frac{1}{\sin\theta}\pdv{\theta}\(\sin\theta\pdv{\theta}\) + \frac{1}{\sin^2\theta}\pdv[2]{\phi}\right]\psi + \frac{2\mu r^2(E - V(r))}{\hbar^2}\psi &= 0
    \end{align*}
    $\psi$に作用する演算子が$r$のみに依存する部分と$\theta, \phi$のみに依存する部分の和なので、それぞれの固有値$\lambda, -\lambda$と固有関数$R(r), Y(\theta, \phi)$を用いて
    \begin{align*}
        &\left[\pdv{r}\(r^2\pdv{r}\) + \frac{2\mu r^2(E - V(r))}{\hbar^2}\right]R(r) = \lambda R(r)\\
        &\left[\frac{1}{\sin\theta}\pdv{\theta}\(\sin\theta\pdv{\theta}\) + \frac{1}{\sin^2\theta}\pdv[2]{\phi}\right]Y(\theta, \phi) = -\lambda Y(\theta, \phi)\\
        &\psi(r, \theta, \phi) = R(r)Y(\theta, \phi)
    \end{align*}
    と書ける。
    
    第二式に更に$\sin^2\theta$を掛ける。
        \[\left[\sin\theta\pdv{\theta}\(\sin\theta\pdv{\theta}\) + \pdv[2]{\phi}\right]Y(\theta, \phi) + \lambda \sin^2\theta Y(\theta, \phi) = 0\]
    $Y(\theta, \phi)$に作用する演算子が$\theta$のみに依存する部分と$\phi$のみに依存する部分の和なので、それぞれの固有値$m^2, -m^2$と固有関数$\Theta(\theta), \Phi(\phi)$を用いて
    \begin{align*}
        &\left[\sin\theta\pdv{\theta}\(\sin\theta\pdv{\theta}\) + \lambda \sin^2\theta\right]\Theta(\theta) = m^2 \Theta(\theta)\\
        &\pdv[2]{\Phi(\phi)}{\phi} = -m^2 \Phi(\phi)\\
        &Y(\theta, \phi) = \Theta(\theta)\Phi(\phi)
    \end{align*}
    と書ける。

    結局シュレーディンガー方程式は以下の三つの式に還元される。
    \begin{align*}
        &\left[\pdv{r}\(r^2\pdv{r}\) + \frac{2\mu r^2(E - V(r))}{\hbar^2}\right]R(r) = \lambda R(r)\\
        &\left[\sin\theta\pdv{\theta}\(\sin\theta\pdv{\theta}\) + \lambda \sin^2\theta\right]\Theta(\theta) = m^2 \Theta(\theta)\\
        &\pdv[2]{\Phi(\phi)}{\phi} = -m^2 \Phi(\phi)\\
        &\psi(r, \theta, \phi) = R(r)\Theta(\theta)\Phi(\phi)
    \end{align*}
    
    $\Phi(\phi)$の一般解は
        \[
            \Phi(\phi) =
            \begin{cases}
                Ae^{im\phi} + Be^{-im\phi} & (m^2 > 0)\\
                C\phi + D & (m^2 = 0)\\
                Ee^{\sqrt{-m^2}\phi} + Fe^{-\sqrt{-m^2}\phi} & (m^2 < 0)
            \end{cases}
        \]
    で与えられる。波動関数は連続なので$\Phi(0) = \Phi(2\pi)$より、$m^2 > 0$のとき$m = 1, 2, \dots$、$m^2 = 0$のとき$C = 0$である。第一式は$m = 0$とすれば第二式も含んでいる。また二項はそれぞれ単独でも解を成している。規格化すると結局
    \[
        \begin{aligned}
            \Phi(\phi) &= \frac{1}{2\pi}e^{im\phi} & (m = \dots, -2, -1, 0, 1, 2, \dots)
        \end{aligned}
    \]
    となる。

    二番目の方程式は、$l$を$l \geq |m|$を満たす整数として$\lambda = l(l + 1)$のときに限り解が存在し、ルジャンドル陪関数及びルジャンドル多項式を用いて
    \begin{align*}
        \Theta_{lm}(\theta) &= (-1)^{\frac{m + |m|}{2}} \sqrt{l + \frac{1}{2}} \sqrt{\frac{(l - |m|)!}{(l + |m|)!}} P_l^{|m|}(\cos\theta)\\
        P_l^{|m|}(x) &= (1 - x^2)^{\frac{|m|}{2}} \frac{\dd{}^{|m|}P_l(x)}{\dd{x}^{|m|}}\\
        P_l(x) &= \frac{1}{2^l} \frac{\dd{}^l}{\dd{x}^l} (x^2 - 1)^l
    \end{align*}
    と表せる。
    $Y_l^m(\theta, \phi) = \Theta(\theta)\Phi(\phi)$は球面調和関数となる。

    つまり動径成分の方程式は
        \[\left[\pdv{r}\(r^2\pdv{r}\) + \frac{2\mu r^2(E - V(r))}{\hbar^2}\right]R(r) = l(l + 1)R(r)\]
    となる。これを分解する前の形に合わせて表示すると
        \[-\frac{\hbar^2}{2\mu}\frac{1}{r^2}\pdv{r}\(r^2\pdv{R}{r}\) + \left[V(r) + \frac{l(l + 1)\hbar^2}{2\mu r^2}\right]R(r) = ER(r)\]
    となる。元々のポテンシャルに加えらえた$l(l + 1)\hbar^2 / 2\mu r^2$は遠心力ポテンシャルを表している。

    上で求めた$\psi(r, \theta, \phi) = R(r)\Theta(\theta)\Phi(\phi)$は$L^2$と$L_z$の同時固有状態となっている。$Y(\theta, \phi)$の固有方程式に登場した演算子が$-L^2 / \hbar^2$に一致することを用いれば簡単に計算することができる。
    \begin{align*}
        L^2 (R(r)Y(\theta, \phi)) &= l(l + 1)\hbar^2 R(r)Y(\theta, \phi)\\
        L_z (R(r)\Theta(\theta)\Phi(\phi)) &= m\hbar R(r)\Theta(\theta)\Phi(\phi)
    \end{align*}
    つまり、軌道角運動量の二乗は$l(l + 1)\hbar^2$、$z$成分は$m\hbar$である。

\subsection{球面波}
    ポテンシャルが存在しないとき、波動関数の動径成分が満たす方程式は
        \[\left[\dv{r}\(r^2\dv{r}\) + \frac{2\mu r^2E}{\hbar^2}\right]R(r) = l(l + 1) R(r)\]
    である。$k^2 = 2\mu r^2E / \hbar^2, \xi = kr$を用いると$\dv*{r} = k\dv*{\xi}$より
    \begin{align*}
        k\dv{\xi}\(\frac{\xi^2}{k^2}\dv{R}{r}\) + \xi^2R(r) &= l(l + 1)R(r)\\
        \dv[2]{R}{r} + \frac{2}{\xi}\dv{R}{r} + \left[1 - \frac{l(l + 1)}{\xi^2}\right]R(r) &= 0
    \end{align*}
    これは球ベッセル微分方程式と呼ばれており、一般解は球ベッセル関数$j_l(\xi)$と球ノイマン関数$n_l(\xi)$を用いて
        \[R_l(r) = Aj_l(kr) + Bn_l(kr)\]
    と表される。球ベッセル関数は原点で正則だが、球ノイマン関数は原点で発散する。また、第一種球ハンケル関数と第二種球ハンケル関数
    \begin{align*}
        h^1_l(\xi) &= j_l(\xi) + in_l(\xi)\\
        h^2_l(\xi) &= j_l(\xi) - in_l(\xi)
    \end{align*}
    を用いて表すこともできる。これは球面波を意味し、前者は外向きの進行波、後者は内向きの進行波を表す。球面波は($z$方向の)軌道角運動量の固有状態である。

\subsection{水素原子}
    二つの点電荷$e_1, e_2$がクーロン力を及ぼし合っているときの波動関数を求める。それぞれの位置が$\bm{r}_1, \bm{r}_2$のときハミルトニアンは
    \begin{align*}
        H &= -\frac{\hbar^2}{2m_1}\Delta_1 - \frac{\hbar^2}{2m_2}\Delta_2 + \frac{k}{|\bm{r}_2 - \bm{r}_1|}\\
        \Delta_1 &= \pdv[2]{x_1} + \pdv[2]{y_1} + \pdv[2]{z_1}\\
        \Delta_2 &= \pdv[2]{x_2} + \pdv[2]{y_2} + \pdv[2]{z_2}\\
        k &= \frac{e_1e_2}{4\pi\epsilon_0}
    \end{align*}
    となる。
    \begin{align*}
        \bm{r_G} &= \frac{m_1\bm{r}_1 + m_2\bm{r}_2}{m_1 + m_2}\\
        \bm{r} &= \bm{r}_2 - \bm{r}_1\\
        \mu &= \frac{m_1m_2}{m_1 + m_2}
    \end{align*}
    とおくと
    \begin{align*}
        \bm{r}_1 &= \bm{r}_G - \frac{m_2}{m_1 + m_2}\bm{r}\\
        \bm{r}_2 &= \bm{r}_G + \frac{m_1}{m_1 + m_2}\bm{r}
    \end{align*}
    である。
    \begin{align*}
        \pdv{x_1}
        &= \pdv{x_G}{x_1}\pdv{x_G} + \pdv{x}{x_1}\pdv{x}\\
        &= \frac{m_1}{m_1 + m_2}\pdv{x_G} - \pdv{x}\\
        \pdv[2]{x_1}
        &= \frac{m_1^2}{(m_1 + m_2)^2}
        \pdv[2]{x_G} - \frac{2m_1}{m_1 + m_2} \pdv[2]{x_G}{x} + \pdv[2]{x}\\
        -\frac{\hbar^2}{2m_1}\Delta_1
        &= -\frac{m_1\hbar^2}{2(m_1 + m_2)^2}\Delta_G + \frac{\hbar^2}{m_1 + m_2}\nabla_G \cdot \nabla_r - \frac{\hbar^2}{2m_1}\Delta_r
    \end{align*}
    同様に
        \[-\frac{\hbar^2}{2m_2}\Delta_2 = -\frac{m_2\hbar^2}{2(m_1 + m_2)^2}\Delta_G - \frac{\hbar^2}{m_1 + m_2}\nabla_G \cdot \nabla_r - \frac{\hbar^2}{2m_2}\Delta_r\]
    なのでハミルトニアンは
        \[H = -\frac{\hbar^2}{2(m_1 + m_2)}\Delta_G - \frac{\hbar^2}{2\mu}\Delta_r + \frac{k}{|r|}\]
    となる。ハミルトニアンが重心のみに依存する項と相対座標のみに依存する項に分解できるので、波動関数を$\psi(R, r) = \psi_G(R)\psi_r(r)$と変数分離できる。つまり、古典力学と同様、二体問題は一体問題に帰着される。重心成分は系全体を一つの自由粒子と見なしたときの波動関数と一致する。よって以降は相対座標のみに依存する部分を考える。

    水素原子において中心電荷が固定されている場合のハミルトニアンは
        \[H = -\frac{\hbar^2}{2\mu}\Delta + \frac{k}{r}\]
    である。つまり波動関数の動径成分が満たす方程式は
        \[\left[\pdv{r}\(r^2\pdv{r}\) + \frac{2\mu r^2(E - k / r)}{\hbar^2}\right]R(r) = l(l + 1) R(r)\]
    動径方向の解はラゲールの陪多項式を用いて
    \begin{align*}
        R_{nl}(r) &= -\(\frac{2\mu k}{n\hbar^2}\)^{3/2} \sqrt{\frac{(n-l-1)!}{2\mu [(n+l)!]^3}} \exp(-\frac{\mu k}{n\hbar^2}r) \(\frac{2\mu k}{n\hbar^2}r\)^l L^{2l+1}_{n+l}\(\frac{\mu k}{n\hbar^2}r\)\\
        L^s_t(x) &= \\
        n &= l + 1, l + 2, \dots
    \end{align*}
    $n, l, m$をそれぞれ主量子数、方位量子数、磁気量子数という。エネルギー固有値は主量子数によって決まり
        \[E_n = -\frac{\mu k^2}{2\hbar^2}\frac{1}{n^2} = -\frac{\mu e^4}{32\pi^2\epsilon_0^2\hbar^2}\frac{1}{n^2}\]
    である。従って水素原子のエネルギー準位は縮退を起こしている。主量子数はK殻、L殻、M殻、N殻...に対応し、方位量子数はs軌道、p軌道、d軌道、f軌道...に対応する。エネルギー準位の差は
    \begin{align*}
        \frac{E_n - E_m}{hc}
        &= \frac{m}{2\hbar^2}\(\frac{e^2}{4\pi\epsilon_0}\)^2\\
        &= \frac{\mu e^4}{8\pi\epsilon_0^2h^3c}
    \end{align*}
    となりリュードベリの公式に一致した。
\section{角運動量の理論}

\subsection{角運動量代数}
    無次元の演算子$j$が次のような交換関係を満たすとする。
        \[[j_x, j_y] = ij_z, \quad [j_y, j_z] = ij_x, \quad [j_z, j_x] = ij_y\]
    このとき
        \[\bm{j}^2 = j_x^2 + j_y^2 + j_z^2\]
    とおくと
    \begin{align*}
        [\bm{j}^2, j_z]
            &= (j_x^2 + j_y^2 + j_z^2)j_z - j_z(j_x^2 + j_y^2 + j_z^2)\\
            &= j_x^2j_z + j_y(j_zj_y + ij_x) - (j_xj_z + ij_y)j_x - j_zj_y^2\\
            &= -j_x[j_z, j_x] + [j_y, j_z]j_y\\
            &= -j_x \cdot ij_y + ij_x \cdot j_y\\
            &= 0
    \end{align*}
    である。そして上昇演算子(生成演算子)と下降演算子(消滅演算子)を導入する。
        \[j_+ = j_x + ij_y, \quad j_- = j_x - ij_y\]
    いくつかの計算を示す。
    \begin{align*}
        j_-j_+
            &= (j_x - ij_y)(j_x + ij_y)\\
            &= j_x^2 + ij_xj_y - ij_yj_x + j_y^2\\
            &= j_x^2 + j_y^2 + i[j_x, j_y]\\
            &= j_x^2 + j_y^2 - j_z\\
            &= \bm{j}^2 - j_z^2 - j_z\\
        j_+j_-
            &= (j_x + ij_y)(j_x - ij_y)\\
            &= j_x^2 - ij_xj_y + ij_yj_x + j_y^2\\
            &= j_x^2 + j_y^2 - i[j_x, j_y]\\
            &= j_x^2 + j_y^2 + j_z\\
            &= \bm{j}^2 - j_z^2 + j_z\\
        [j_z, j_{\pm}]
            &= [j_z, j_x \pm ij_y]\\
            &= [j_z, j_x] \pm i[j_z, j_y]\\
            &= ij_y \pm j_x\\
            &= \pm (j_x \pm ij_y)\\
            &= \pm j_{\pm}
    \end{align*}

    $\bm{j}^2, j_z$は同時固有状態を持つ。それぞれの固有値が$\lambda, m$であるときの固有状態を$\ket{\lambda, m}$と書く。$j_+, j_-$は$\bm{j}^2$の固有値を変えない。$j_z$の固有値については
    \begin{align*}
        j_zj_{\pm}\ket{\lambda, m}
            &= (j_{\pm}j_z \pm j_{\pm})\ket{\lambda, m}\\
            &= j_{\pm}(j_z \pm 1)\ket{\lambda, m}\\
            &= (m \pm 1)j_{\pm}\ket{\lambda, m}
    \end{align*}
    となる。従って、$j_+$を作用させると固有値$m + 1$の固有状態に移行し、$j_-$を作用させると固有値$m - 1$の固有状態に移行する。
    \begin{thm}
        $j_z$の固有値$m$には最大値と最小値が存在する。
    \end{thm}
    \begin{proof}
        \begin{align*}
            \lambda &= \ev{\bm{j}^2}{\lambda, m}\\
                    &= \ev{j_x^2 + j_y^2}{\lambda, m} + m^2\\
                    &\geq m^2 \geq 0
        \end{align*}
        最大値$j$に対して$j_+\ket{\lambda, j} = 0$だから
        \begin{align*}
            \lambda\ket{\lambda, j}
                &= \bm{j}^2\ket{\lambda, j}\\
                &= (j_-j_+ + j_z + j_z^2)\ket{\lambda, j}\\
                &= j(j + 1)\ket{\lambda, j}\\
            \lambda &= j(j + 1)
        \end{align*}
        最小値$j - n$に対して$j_-\ket{\lambda, j - n} = 0$だから
        \begin{align*}
            \lambda\ket{\lambda, j - n}
                &= \bm{j}^2\ket{\lambda, j - n}\\
                &= (j_+j_- - j_z + j_z^2)\ket{\lambda, j - n}\\
                &= (j - n)(j - n - 1)\ket{\lambda, j - n}\\
            \lambda &= (j - n)(j - n - 1)
        \end{align*}
    \end{proof}
    従って
    \begin{align*}
        j(j + 1) &= (j - n)(j - n - 1)\\
        j &= -(2n + 1)j + n(n + 1)\\
        j &= \frac{n}{2}
    \end{align*}
    より、$j$は非負の半整数であり、$\lambda$に対して一意である。固有値は$-j, -(j - 1), \dots, j - 1, j$を取る。

    \begin{align*}
        |c|^2   &= \ev{j_+^\dagger j_+}{\lambda, m}\\
                &= \ev{j_-j_+}{\lambda, m}\\
                &= \ev{\bm{j}^2 - j_z^2 - j_z}{\lambda, m}\\
                &= j(j + 1) - m^2 - m\\
                &= (j - m)(j + m + 1)\\
        |c'|^2  &= \ev{j_-^\dagger j_-}{\lambda, m}\\
                &= \ev{j_+j_-}{\lambda, m}\\
                &= \ev{\bm{j}^2 - j_z^2 + j_z}{\lambda, m}\\
                &= j(j + 1) - m^2 + m\\
                &= (j + m)(j - m + 1)
    \end{align*}
    より
        \[c = \sqrt{(j - m)(j + m + 1)}, \quad c' = \sqrt{(j + m)(j - m + 1)}\]
    とすれば良い。

\subsection{スピン}
    量子力学においてスピンと呼ばれる内部自由度が存在する。粒子の状態は位置または運動量だけでは決定せず、スピンを持ち出す必要がある。

    粒子はスピン演算子$S$に対して必ず$S^2$の固有状態となっている。固有値$s(s + 1)\hbar$に属する$S^2$の固有状態にある粒子はスピン$s$を持つという。例えば電子はスピン$1/2$であることが実験で確認されている。
    
    波動関数はそれぞれの空間座標とスピン座標の関数として表現できる。また、状態空間は元々の空間とスピノル空間$D_s$のテンソル積となる。
    \[\psi(r_1, \sigma_1, \dots, r_n, \sigma_n) = (\psi(r_1, r_2, \dots, r_n), m_1, m_2, \dots, m_n) \in L^2(\R^{3n}) \otimes D_s\]

\subsection{角運動量の合成}
    $\{c_0x^t + c_1x^{t-1}y + \dots + c_ty^y\}$によって得られる表現を$D_j\ (j = t/2)$と書く。

    有限次元ベクトル空間$V$について、線形演算子$j_x, j_y, j_z: V \rightarrow V$が

    $R_{2j+1} = \gen{v_{-j}, v_{-(j-1)}, \dots, v_{j-1}, v_j}$は既約である。これは$D_j$と同型である。

    全角運動量演算子を
        \[j = j_1 \otimes 1 + 1 \otimes j_2\]
    で定める。
    
    半整数$j_1, j_2$に対し
        \[D_{j_1} \otimes D_{j_2} = D_{|j_1-j_2|} \oplus D_{|j_1-j_2|+1} \oplus \dots \oplus D_{j_1+j_2}\]
    である。$D_{j_1} \otimes D_{j_2}$の基底として$\{\ket{j_1, m_1} \otimes \ket{j_2, m_2} \mid |m_1| \leq j_1, |m_2| \leq j_2\}$を考えることができる。
        \[j_z\ket{j_1, m_1} \otimes \ket{j_2, m_2} = (m_1 + m_2)\ket{j_1, m_1} \otimes \ket{j_2, m_2}\]
    より、$m$に属する$j_z$の固有ベクトルとして、$\ket{j_1, m_1}\ket{j_2, m_2}\ (m_1 + m_2 = m)$ある。$j_1 \leq j_2$なら固有ベクトルの数はそれぞれ
    \[
        \begin{cases}
            j_1 + j_2 - |m| & (m \leq j_1 - j_2)\\
            2j_1 + 1 & (j_1 - j_2 < m < -j_1 + j_2)\\
            j_1 + j_2 - |m| & (-j_1 + j_2 \leq m)
        \end{cases}
    \]
    である。最大の固有値$j_1 + j_2$の固有ベクトルがただ一つ存在し、下降演算子を適用することで$D_{j_1+j_2}$が得られる。直交補空間を考えれば各固有値の固有ベクトルが一つずつ減る。残った最大の固有値$j_1 + j_2 - 1$の固有ベクトルに対して同様に繰り返すことで、$D_{j_1+j_2-1}, \dots, D_{|j_1-j_2|}$が得られる。

    $\ket{j, m} \in D_{j_1} \otimes D_{j_2}$を
        \[\ket{j, m} = \sum C^{jm}_{j_1m_1j_2m_2} \ket{j_1, m_1} \otimes \ket{j_2, m_2}\]
    と表したとき、係数$C^{jm}_{j_1m_1j_2m_2}$をクレブシュ=ゴルダン係数と呼ぶ。

\subsection{2電子系}
    電子のスピンは$1/2$なので、$s_z$の固有値は$+1/2, -1/2$の二つである。それぞれの固有状態を$\ket{\alpha}, \ket{\beta}$とする。つまり
        \[s_z\ket{\alpha} = \frac{1}{2}\ket{\alpha}, \quad s_z\ket{\beta} = -\frac{1}{2}\ket{\beta}\]
    である。2電子のスピンの合成$s \otimes 1 + 1 \otimes s$を考えると、$D_{1/2} \otimes D_{1/2} = D_0 \oplus D_1$と分解される。$D_1$の固有状態は
    \begin{align*}
        \ket{1, 1} &= \ket{\alpha} \otimes \ket{\alpha}\\
        \ket{1, 0} &= \frac{1}{\sqrt{2}}(\ket{\alpha} \otimes \ket{\beta} + \ket{\beta} \otimes \ket{\alpha})\\
        \ket{1, -1} &= \ket{\beta} \otimes \ket{\beta}
    \end{align*}
    となる。これは2電子の入れ替えに対して対称である。$D_0$の固有状態は
        \[\ket{0, 0} = \frac{1}{\sqrt{2}}(\ket{\alpha} \otimes \ket{\beta} - \ket{\beta} \otimes \ket{\alpha})\]
    となる。これは2電子に入れ替えに対して反対称である。

    ハミルトニアン
        \[H_{ss} = \alpha s_1 \cdot s_2\]
    によって定義される作用をスピン-スピン相互作用という。また
        \[H_{so} = \beta l \cdot s\]
    をスピン軌道相互作用という。

\subsection{ゼーマン相互作用}
    電磁ポテンシャルのハミルトニアンは$\sigma = (\sigma_x, \sigma_y, \sigma_z)$をパウリ行列として
        \[H = \frac{[\sigma \cdot (p - eA)]^2}{2m} + e\phi\]
    である。

\subsection{スピン軌道相互作用}
        \[H_{\text{SO}} = \alpha L \cdot S\]
    軌道角運動量とスピン角運動量の相互作用をスピン軌道相互作用という。
    \begin{align*}
        L \cdot S(L_x + S_x)
            &= (L_xS_x + L_yS_y + L_zS_z)(L_x + S_x)\\
            &= L_x^2S_x + L_yL_xS_y + L_zL_xS_z + L_xS_x^2 + L_yS_yS_x + L_zS_zS_x\\
        (L_x + S_x)L \cdot S
            &= (L_x + S_x)(L_xS_x + L_yS_y + L_zS_z)\\
            &= L_x^2S_x + L_xL_yS_y + L_xL_zS_z + L_xS_x^2 + L_yS_xS_y + L_zS_xS_z\\
        [L \cdot S, L_x + S_x]
            &= -[L_x, L_y]S_y + [L_z, L_x]S_z - L_y[S_x, S_y] + L_z[S_z, S_x]\\
            &= -i\hbar L_zS_y + i\hbar L_yS_z - i\hbar L_yS_z + i\hbar L_zS_y\\
            &= 0
    \end{align*}
    $y, z$についても同様なので、$[L \cdot S, L + S] = 0$つまりスピン軌道相互作用は全角運動量$J = L + S$と交換する。
    
    $L_i, S_i$は交換するので
    \begin{align*}
        J^2 &= (L + S)^2\\
            &= (L_x + S_x, L_y + S_y, L_z + S_z)^2\\
            &= (L_x + S_x)^2 + (L_y + S_y)^2 + (L_z + S_z)^2\\
            &= L^2 + 2 L \cdot S + S^2
    \end{align*}
    つまり
        \[L \cdot S = \frac{1}{2}(J^2 - L^2 - S^2)\]
    である。

    粒子の方位量子数が$l$、スピン量子数が$s$のとき、$D_l \otimes D_s = D_{|l-s|} \oplus \dots \oplus D_{l+s}$より、全角運動量の量子数$j$は$l - s \leq j \leq l + s$を取る。固有状態はクレブシュ=ゴルダン係数を用いて
        \[\ket{j, m_j} = \sum C^{jm_j}_{lm_lsm_s} \ket{l, m_l} \otimes \ket{s, m_s}\]
    と表される。このときのエネルギー固有値は
    \begin{align*}
        2 L \cdot S \ket{j, m_j}
            &= J^2 \ket{j, m_j} - (L^2 + S^2) \sum C^{jm_j}_{lm_lsm_s} \ket{l, m_l} \otimes \ket{s, m_s}\\
            &= j(j + 1)\hbar^2 \ket{j, m_j} - (l(l + 1)\hbar^2 + s(s + 1)\hbar^2) \sum C^{jm_j}_{lm_lsm_s} \ket{l, m_l} \otimes \ket{s, m_s}\\
            &= [j(j + 1) - l(l + 1) - s(s + 1)]\hbar^2 \ket{j, m_j}
    \end{align*}
    より$\alpha/2 [j(j + 1) - l(l + 1) - s(s + 1)]\hbar^2$となる。したがって$j$の値によってエネルギーが分裂する。スピン軌道相互作用によって起こされるこのようなエネルギーの微細な分裂を微細構造という。

    パッシェン=バック効果

\subsection{ラーモア歳差運動}
    $z$軸方向に一様な磁場$B$がかかっている場合を考える。軌道運動を無視しスピンのみを考えると、ハミルトニアンは
        \[H_s = -\mu B s_z\]
    である。$s_z$の固有状態$\ket{\alpha}, \ket{\beta}$が$H_s$の固有状態であり、
    \begin{align*}
        H_s\ket{\alpha} &= -\mu B s_z\ket{\alpha} = -\frac{\mu B}{2}\ket{\alpha}\\
        H_s\ket{\beta} &= -\mu B s_z\ket{\beta} = \frac{\mu B}{2}\ket{\beta}
    \end{align*}
    である。エネルギー固有値はそれぞれ$E_\alpha = -\mu B/2, E_\beta = \mu B/2$である。つまり波動関数は
        \[\ket{\sigma(t)} = c_1e^{-i\frac{E_\alpha}{\hbar}t}\ket{\alpha} + c_2e^{-i\frac{E_\beta}{\hbar}t}\ket{\beta}\]
    となる。$t = 0$のとき
        \[\ket{\sigma} = \cos\frac{\theta}{2}\ket{\alpha} + \sin\frac{\theta}{2}e^{i\delta}\ket{\beta}\]
    であるとすると
        \[c_1 = \cos\frac{\theta}{2}, \quad c_2 = \sin\frac{\theta}{2}e^{i\delta}\]
    となる。つまり
        \[\ket{\sigma(t)} = \cos\frac{\theta}{2}e^{-i\frac{E_\alpha}{\hbar}t}\ket{\alpha} + \sin\frac{\theta}{2}e^{i\delta}e^{-i\frac{E_\beta}{\hbar}t}\ket{\beta}\]
    である。

    \begin{alignat*}{2}
        s_x\ket{\alpha} &= \frac{1}{2}\ket{\beta}, &\quad s_x\ket{\beta} &= \frac{1}{2}\ket{\alpha}\\
        s_y\ket{\alpha} &= \frac{i}{2}\ket{\beta}, & s_y\ket{\beta} &= -\frac{i}{2}\ket{\alpha}\\
        s_z\ket{\alpha} &= \frac{1}{2}\ket{\alpha}, & s_z\ket{\beta} &= -\frac{1}{2}\ket{\beta}
    \end{alignat*}
    なので
    \begin{align*}
        s_x\ket{\sigma(t)} &= \frac{1}{2}\cos\frac{\theta}{2}e^{i\frac{\mu B}{2\hbar}t}\ket{\beta} + \frac{1}{2}\sin\frac{\theta}{2}e^{i\delta}e^{-i\frac{\mu B}{2\hbar}t}\ket{\alpha}\\
        s_y\ket{\sigma(t)} &= \frac{i}{2}\cos\frac{\theta}{2}e^{i\frac{\mu B}{2\hbar}t}\ket{\beta} - \frac{i}{2}\sin\frac{\theta}{2}e^{i\delta}e^{-i\frac{\mu B}{2\hbar}t}\ket{\alpha}\\
        s_z\ket{\sigma(t)} &= \frac{1}{2}\cos\frac{\theta}{2}e^{i\frac{\mu B}{2\hbar}t}\ket{\alpha} - \frac{1}{2}\sin\frac{\theta}{2}e^{i\delta}e^{-i\frac{\mu B}{2\hbar}t}\ket{\beta}
    \end{align*}
    となるから、
    \begin{align*}
        \ev{s_x}{\sigma(t)}
            &= \(\cos\frac{\theta}{2}e^{-i\frac{\mu B}{2\hbar}t}\bra{\alpha} + \sin\frac{\theta}{2}e^{-i\delta}e^{i\frac{\mu B}{2\hbar}t}\bra{\beta}\)\(\frac{1}{2}\cos\frac{\theta}{2}e^{i\frac{\mu B}{2\hbar}t}\ket{\beta} + \frac{1}{2}\sin\frac{\theta}{2}e^{i\delta}e^{-i\frac{\mu B}{2\hbar}t}\ket{\alpha}\)\\
            &= \frac{1}{2}\cos\frac{\theta}{2}\sin\frac{\theta}{2} e^{i\delta} e^{-i\frac{\mu B}{\hbar}t} + \frac{1}{2}\cos\frac{\theta}{2}\sin\frac{\theta}{2} e^{-i\delta} e^{i\frac{\mu B}{\hbar}t}\\
            &= \frac{1}{2} \sin\theta \cos\(\frac{\mu B}{\hbar}t - \delta\)\\
        \ev{s_y}{\sigma(t)}
            &= \(\cos\frac{\theta}{2}e^{-i\frac{\mu B}{2\hbar}t}\bra{\alpha} + \sin\frac{\theta}{2}e^{-i\delta}e^{i\frac{\mu B}{2\hbar}t}\bra{\beta}\)\(\frac{i}{2}\cos\frac{\theta}{2}e^{i\frac{\mu B}{2\hbar}t}\ket{\beta} - \frac{i}{2}\sin\frac{\theta}{2}e^{i\delta}e^{-i\frac{\mu B}{2\hbar}t}\ket{\alpha}\)\\
            &= -\frac{i}{2}\cos\frac{\theta}{2}\sin\frac{\theta}{2} e^{i\delta} e^{-i\frac{\mu B}{\hbar}t} + \frac{i}{2}\cos\frac{\theta}{2}\sin\frac{\theta}{2} e^{-i\delta} e^{i\frac{\mu B}{\hbar}t}\\
            &= -\frac{1}{2} \sin\theta \sin\(\frac{\mu B}{\hbar}t - \delta\)\\
        \ev{s_z}{\sigma(t)}
            &= \(\cos\frac{\theta}{2}e^{-i\frac{\mu B}{2\hbar}t}\bra{\alpha} + \sin\frac{\theta}{2}e^{-i\delta}e^{i\frac{\mu B}{2\hbar}t}\bra{\beta}\)\(\frac{1}{2}\cos\frac{\theta}{2}e^{i\frac{\mu B}{2\hbar}t}\ket{\alpha} - \frac{1}{2}\sin\frac{\theta}{2}e^{i\delta}e^{-i\frac{\mu B}{2\hbar}t}\ket{\beta}\)\\
            &= \frac{1}{2}\cos^2\frac{\theta}{2} - \frac{1}{2}\sin^2\frac{\theta}{2}\\
            &= \frac{1}{2} \cos\theta
    \end{align*}
    となる。つまり、粒子のスピンが磁場方向に傾いている場合、スピンの各方向の期待値が指すベクトルは、磁場方向に水平な円周の上を角振動数$\mu B/\hbar$で回転する。これをラーモア歳差運動という。

\subsection{ランダウ準位}
\subsection{アハラノフ=ボーム効果}
    二重スリット実験において、電子の波動関数はスリットAを通るものとスリットBを通るものの重ね合わせで表される。
        \[\psi = \psi_A + \psi_B\]

\subsection{ラビ振動}
% 湯川相互作用、クライン・仁科の式、朝永・ラッティンジャー液体、ゲルマン西島の関係式、近藤効果
\section{近似法}

\subsection{時間に依存しない摂動論(縮退のない場合)}
    考えている系のシュレーディンガー方程式は厳密には解けないが、ハミルトニアンが厳密に解ける主要項と微小な摂動項の和で表される場合、エネルギー固有値や固有状態を漸近展開することで解の性質を調べることができる。このような近似の方法を摂動論と呼ぶ。

    時間に依存しないハミルトニアンを
    \begin{gather*}
        H\ket{\phi_n} = E_n\ket{\phi_n}\\
        H = H_0 + \lambda V
    \end{gather*}
    と置く。エネルギー固有値と固有状態を$\lambda$について漸近展開する。\footnote{$\lambda$によって冪級数展開できることは仮定に過ぎない。$\lambda$で冪級数展開できないような系も知られており、非摂動な系と呼ばれている。}
    \begin{align*}
        E_n &= \sum_{i=0}^\infty \lambda^i E_n^i\\
        \ket{\phi_n} &= \sum_{i=0}^\infty \lambda^i \ket{\phi_n^i}
    \end{align*}\footnote{$E_n^i, \ket{\phi_n^i}$の$i$は添え字であることに注意。}
    シュレーディンガー方程式に代入すると
    \begin{align*}
        (H_0 + \lambda V)\sum_{i=0}^\infty \lambda^i \ket{\phi_n^i} &= \(\sum_{i=0}^\infty \lambda^i E_n^i\)\(\sum_{i=0}^\infty \lambda^i \ket{\phi_n^i}\)\\
        \sum_{i=0}^\infty (\lambda^i H_0\ket{\phi_n^i} + \lambda^{i+1} V\ket{\phi_n^i}) &= \sum_{i=0}^\infty \sum_{j+k=i} \lambda^i E_n^j \ket{\phi_n^k}\\
        H_0\ket{\phi_n^0} + \sum_{i=1}^\infty \lambda^i (H_0\ket{\phi_n^i} + V\ket*{\phi_n^{i-1}}) &= \sum_{i=0}^\infty \sum_{j+k=i} \lambda^i E_n^j \ket{\phi_n^k}\\
    \end{align*}
    $\lambda$の0,1,2次の項について比較する。
    \begin{align*}
        H_0\ket{\phi_n^0} &= E_n^0\ket{\phi_n^0}\\
        H_0\ket{\phi_n^1} + V\ket{\phi_n^0} &= E_n^0\ket{\phi_n^1} + E_n^1\ket{\phi_n^0}\\
        H_0\ket{\phi_n^2} + V\ket{\phi_n^1} &= E_n^0\ket{\phi_n^2} + E_n^1\ket{\phi_n^1} + E_n^2\ket{\phi_n^0}
    \end{align*}
    0次の項の式は主要項のハミルトニアンの時間に依存しないシュレーディンガー方程式であり、固有値$E_n^0$と固有状態$\ket{\phi_n^0}$は既知と仮定する。

    一次摂動を考える。両辺に左から$\bra{\phi_m^0}$を掛けると、$\bra{\phi_m^0}H_0 = (H_0\ket{\phi_m^0})^\dagger = (E_m^0\ket{\phi_m^0})^\dagger = \bra{\phi_m^0}E_m^0$だから
    \begin{align*}
        E_m^0\braket{\phi_m^0}{\phi_n^1} + \mel{\phi_m^0}{V}{\phi_n^0} &= E_n^0\braket{\phi_m^0}{\phi_n^1} + E_n^1\delta_{mn}\\
        (E_n^0 - E_m^0)\braket{\phi_m^0}{\phi_n^1} &= \mel{\phi_m^0}{V}{\phi_n^0} - E_n^1\delta_{mn}
    \end{align*}
    $m = n$として
        \[E_n^1 = \ev{V}{\phi_n^0}\]
    エネルギー固有値は
    \begin{align*}
        E_n &\simeq E_n^0 + E_n^1\\
            &= \ev{H_0}{\phi_n^0} + \ev{V}{\phi_n^0}\\
            &= \ev{H}{\phi_n^0}
    \end{align*}
    となる。また$m \neq n$として
        \[\braket{\phi_m^0}{\phi_n^1} = \frac{\mel{\phi_m^0}{V}{\phi_n^0}}{E_n^0 - E_m^0}\]
    なので、$c_n^1$を定数として
        \[\ket{\phi_n^1} = c_n^1\ket{\phi_n^0} + \sum_{m \neq n} \frac{\mel{\phi_m^0}{V}{\phi_n^0}}{E_n^0 - E_m^0}\ket{\phi_m^0}\]
    となる。

    次に二次摂動を考える。同様に左から$\bra{\phi_m^0}$を掛ける。それぞれ
    \begin{align*}
        \mel{\phi_m^0}{H_0}{\phi_n^2} &+ \mel{\phi_m^0}{V}{\phi_n^1}\\
            &= \mel{\phi_m^0}{H_0}{\phi_n^2} + \bra{\phi_m^0}V \(c_n^1\ket{\phi_n^0} + \sum_{k \neq n} \frac{\mel{\phi_k^0}{V}{\phi_n^0}}{E_n^0 - E_k^0} \ket{\phi_k^0}\)\\
            &= E_m^0\braket{\phi_m^0}{\phi_n^2} + c_n^1\mel{\phi_m^0}{V}{\phi_n^0} + \sum_{k \neq n} \frac{\mel{\phi_k^0}{V}{\phi_n^0}}{E_n^0 - E_k^0} \mel{\phi_m^0}{V}{\phi_k^0}\\
        \mel{\phi_m^0}{E_n^0}{\phi_n^2} &+ \mel{\phi_m^0}{E_n^1}{\phi_n^1} + \mel{\phi_m^0}{E_n^2}{\phi_n^0}\\
            &= \mel{\phi_m^0}{E_n^0}{\phi_n^2} + \bra{\phi_m^0}E_n^1 \(c_n^1\ket{\phi_n^0} + \sum_{k \neq n} \frac{\mel{\phi_k^0}{V}{\phi_n^0}}{E_n^0 - E_k^0} \ket{\phi_k^0}\) + \mel{\phi_m^0}{E_n^2}{\phi_n^0}\\
            &= E_n^0\braket{\phi_m^0}{\phi_n^2} + c_n^1E_n^1\delta_{mn} + \sum_{k \neq n} \frac{\mel{\phi_k^0}{V}{\phi_n^0}}{E_n^0 - E_k^0} E_n^1\delta_{mk} + E_n^2\delta_{mn}
    \end{align*}
    よって
    \begin{align*}
        (E_n^0 - E_m^0)\braket{\phi_m^0}{\phi_n^2} &= c_n^1(\mel{\phi_m^0}{V}{\phi_n^0} - E_n^1\delta_{mn}) + \sum_{k \neq n} \frac{\mel{\phi_k^0}{V}{\phi_n^0}}{E_n^0 - E_k^0}(\mel{\phi_m^0}{V}{\phi_k^0} - E_n^1\delta_{mk}) - E_n^2\delta_{mn}
    \end{align*}
    $m = n$として
    \begin{align*}
        E_n^2 &= c_n^1(\mel{\phi_n^0}{V}{\phi_n^0} - E_n^1) + \sum_{k \neq n} \frac{\mel{\phi_k^0}{V}{\phi_n^0}}{E_n^0 - E_k^0}(\mel{\phi_n^0}{V}{\phi_k^0} - E_n^1\delta_{nk})\\
              &= \sum_{m \neq n} \frac{|\mel{\phi_m^0}{V}{\phi_n^0}|^2}{E_n^0 - E_m^0}
    \end{align*}
    基底状態$n = 1$に対しては$E_n^2 \leq 0$となるので常に引力として働く。また$m \neq n$として
        \[\braket{\phi_m^0}{\phi_n^2} = c_n^1\frac{\mel{\phi_m^0}{V}{\phi_n^0}}{E_n^0 - E_m^0} - E_n^1\frac{\mel{\phi_m^0}{V}{\phi_n^0}}{(E_n^0 - E_m^0)^2} + \sum_{k \neq n} \frac{\mel{\phi_m^0}{V}{\phi_k^0}\mel{\phi_k^0}{V}{\phi_n^0}}{(E_n^0 - E_m^0)(E_n^0 - E_k^0)}\]
    なので、$c_n^2$を定数として
        \[\ket{\phi_n^2} = c_n^2\ket{\phi_n^0} + \sum_{m \neq n}\left[c_n^1\frac{\mel{\phi_m^0}{V}{\phi_n^0}}{E_n^0 - E_m^0} - E_n^1\frac{\mel{\phi_m^0}{V}{\phi_n^0}}{(E_n^0 - E_m^0)^2} + \sum_{k \neq n} \frac{\mel{\phi_m^0}{V}{\phi_k^0}\mel{\phi_k^0}{V}{\phi_n^0}}{(E_n^0 - E_m^0)(E_n^0 - E_k^0)}\right]\ket{\phi_m^0}\]
    である。

\subsection{時間に依存しない摂動論(縮退のある場合)}
    $H_0$におけるエネルギー固有値$E_n^0$の固有状態$\ket{\phi_n}$が、摂動によってエネルギー固有値$E_{n,\alpha}\ (\alpha = 1, 2, \dots, \nu)$及び固有状態$\ket{\phi_{n,\alpha}}$に分裂したとする。つまり
    \begin{gather*}
        H\ket{\phi_{n,\alpha}} = E_{n,\alpha}\ket{\phi_{n,\alpha}}\\
        H = H_0 + \lambda V
    \end{gather*}
    である。エネルギー固有値と固有状態を$\lambda$について漸近展開する。
    \begin{align*}
        E_{n,\alpha} &= \sum_{i=0}^\infty \lambda^i E_{n,\alpha}^i\\
        \ket{\phi_{n,\alpha}} &= \sum_{i=0}^\infty \lambda^i \ket{\phi_{n,\alpha}^i}
    \end{align*}
    ここで$E_{n,\alpha}^0 = E_n^0$だが、縮退が解けた直後$\lambda \to 0$としたときの固有状態$\ket{\phi_{n,\alpha}^0}$は未知である。縮退のない場合と同様にシュレーディンガー方程式に代入する。
    \begin{align*}
        H_0\ket{\phi_{n,\alpha}^0}
            &= E_{n,\alpha}^0\ket{\phi_{n,\alpha}^0}\\
            &= E_n^0\ket{\phi_{n,\alpha}^0}\\
        H_0\ket{\phi_{n,\alpha}^1} + V\ket{\phi_{n,\alpha}^0}
            &= E_{n,\alpha}^0\ket{\phi_{n,\alpha}^1} + E_{n,\alpha}^1\ket{\phi_{n,\alpha}^0}\\
            &= E_n^0\ket{\phi_{n,\alpha}^1} + E_{n,\alpha}^1\ket{\phi_{n,\alpha}^0}
    \end{align*}
    0次の項の式は主要項のハミルトニアンの時間に依存しないシュレーディンガー方程式であり、$\ket{\phi_{n,\alpha}^0}$は$H_0$の固有状態である。

    一次摂動を考える。$H_0$の$E_n^0$に属する固有空間の正規直交基底$\{\ket{u_1}, \ket{u_2}, \dots, \ket{u_\nu}\}$を取ると、$\ket{\phi_{n,\alpha}^0}$は$\ket{u_j}$の線形結合で一意に書くことができて、
        \[\ket{\phi_{n,\alpha}^0} = \sum_{j=1}^\nu c_j\ket{u_j}\]
    とする。1次の項の式に代入して
        \[H_0\ket{\phi_{n,\alpha}^1} + V \sum_{j=1}^\nu c_j\ket{u_j} = E_n^0\ket{\phi_{n,\alpha}^1} + E_{n,\alpha}^1 \sum_{j=1}^\nu c_j\ket{u_j}\]
    左から$\bra{u_i}$を掛けると、$\bra{u_i}H_0 = \bra{u_i}E_n$なので
        \[\sum_{j=1}^\nu \mel{u_i}{V}{u_j}c_j = E_{n,\alpha}^1c_i\]
    となる。つまり、縮退が解けた後のエネルギー準位と、解けた直後の固有状態は、ポテンシャル$V$の表現行列の固有値と固有ベクトルに一致する。これを永年方程式という。

\paragraph{シュタルク効果}
    水素原子に$z$軸方向の一様静電場を掛ける。摂動項は
        \[V = -eEz = -eEr\cos\theta\]
    である。$n = 2$において基底$\{\ket{2,0,0}, \ket{2,1,0}, \ket{2,1,1}, \ket{2,1,-1}\}$を取ると$V$の表現行列は
    \[
        \begin{bmatrix}
            \mel{2,0,0}{V}{2,0,0} & \mel{2,0,0}{V}{2,1,0} & \mel{2,0,0}{V}{2,1,1} & \mel{2,0,0}{V}{2,1,-1}\\
            \mel{2,1,0}{V}{2,0,0} & \mel{2,1,0}{V}{2,1,0} & \mel{2,1,0}{V}{2,1,1} & \mel{2,1,0}{V}{2,1,-1}\\
            \mel{2,1,1}{V}{2,0,0} & \mel{2,1,1}{V}{2,1,0} & \mel{2,1,1}{V}{2,1,1} & \mel{2,1,1}{V}{2,1,-1}\\
            \mel{2,1,-1}{V}{2,1,-1} & \mel{2,1,-1}{V}{2,1,0} & \mel{2,1,-1}{V}{2,1,1} & \mel{2,1,-1}{V}{2,1,-1}
        \end{bmatrix}
        =
        \begin{bmatrix}
            0 & 3eEa_B & 0 & 0\\
            3eEa_B & 0 & 0 & 0\\
            0 & 0 & 0 & 0\\
            0 & 0 & 0 & 0
        \end{bmatrix}
    \]
    となる。したがって、固有値$0$に属する固有状態$c_1\ket{2,1,1} + c_2\ket{2,1,-1}\ (|c_1|^2 + |c_2|^2 = 1)$、固有値$3eEa_B$に属する固有状態$1/\sqrt{2}(\ket{2,0,0} + \ket{2,1,0})$、固有値$-3eEa_B$に属する固有状態$1/\sqrt{2}(\ket{2,0,0} - \ket{2,1,0})$が存在する。したがって一次摂動では、2s, 2p軌道は$E_2^0, E_2^0 - 3eEa_B, E_2^0 + 3eEa_B$の三つのエネルギー準位に分裂する。

    原子や分子に一様な電場をかけたときにエネルギー準位が分裂する現象をシュタルク効果と呼ぶ。

\subsection{時間に依存する摂動論}
    摂動項が時間に依存する場合を考える。
        \[H(t) = H_0 + \lambda V(t)\]
    $H_0$の固有値と固有状態を$E_n^0, \ket{\phi_n^0}$とする。$\{\ket{\phi_n^0}\}$は基底を成すので、時間に依存するシュレーディンガー方程式
        \[H(t)\ket{\psi(t)} = i\hbar\pdv{t}\ket{\psi(t)}\]
    の解は
        \[\ket{\psi(t)} = \sum_n c_n(t) \exp\(-i\frac{E_n^0}{\hbar}t\) \ket{\phi_n^0}\]
    と一意に展開できる。代入すると
    \begin{align*}
        (H_0 + \lambda V(t)) \sum_m c_m(t) \exp\(-i\frac{E_m^0}{\hbar}t\) \ket{\phi_m^0} &= i\hbar \pdv{t} \sum_m c_m(t) \exp\(-i\frac{E_m^0}{\hbar}t\)\ket{\phi_m^0}\\
        \sum_m (E_m^0 + \lambda V(t)) c_m(t) \exp\(-i\frac{E_m^0}{\hbar}t\) \ket{\phi_m^0} &= i\hbar \sum_m \(\dv{c_m}{t} - i\frac{E_m^0}{\hbar}c_m(t)\) \exp\(-i\frac{E_m^0}{\hbar}t\) \ket{\phi_m^0}\\
        i\hbar \sum_m \dv{c_m}{t} \exp\(-i\frac{E_m^0}{\hbar}t\) \ket{\phi_m^0} &= \lambda \sum_m c_m(t) \exp\(-i\frac{E_m^0}{\hbar}t\) V(t)\ket{\phi_m^0}
    \end{align*}
    左から$\bra{\phi_n^0}\exp\(i\frac{E_n^0}{\hbar}t\)$を掛けると
    \begin{align*}
        \dv{c_n}{t} &= \frac{\lambda}{i\hbar} \sum_m \exp\(-i\frac{E_m^0 - E_n^0}{\hbar}t\) \mel{\phi_n^0}{V(t)}{\phi_m^0} c_m(t)\\
    \end{align*}
    である。$c_n(t)$を$\lambda$によって漸近展開する。
        \[c_n(t) = \sum_{i=0}^\infty \lambda^i c_n^i(t)\]
    $c_n(t)$の微分方程式に代入すると
    \begin{align*}
        \dv{t} \sum_{i=0}^\infty \lambda^i c_n^i(t) &= \frac{\lambda}{i\hbar} \sum_m \exp\(-i\frac{E_m^0 - E_n^0}{\hbar}t\) \mel{\phi_n^0}{V(t)}{\phi_m^0} \sum_{i=0}^\infty \lambda^i c_m^i(t)\\
        \dv{c_n^0}{t} + \sum_{i=0}^\infty \lambda^{i+1} \dv{c_n^{i+1}}{t} &= \frac{1}{i\hbar} \sum_{i=0}^\infty \lambda^{i+1} \sum_m \exp\(-i\frac{E_m^0 - E_n^0}{\hbar}t\) \mel{\phi_n^0}{V(t)}{\phi_m^0} c_m^i(t)
    \end{align*}
    $\lambda$の次数で比較すれば
    \begin{align*}
        \dv{c_n^0}{t} &= 0\\
        \dv{c_n^{i+1}}{t} &= \frac{1}{i\hbar} \sum_m \exp\(-i\frac{E_m^0 - E_n^0}{\hbar}t\) \mel{\phi_n^0}{V(t)}{\phi_m^0} c_m^i(t)
    \end{align*}
    となる。$c_n^0$は一定であり、$\sum_n |c_n^0|^2 = 1$を満たす。

    摂動ポテンシャルとして
    \[
        V(t) =
        \begin{cases}
            0 & (t < 0)\\
            V & (t \geq 0)
        \end{cases}
    \]
    を考える。$t < 0$で状態が$\ket{\phi_m^0}$だったとして、時刻$t(\geq 0)$で状態$\ket{\phi_n^0}\ (E_n^0 \neq E_m^0)$が観測される確率、つまり$\ket{\phi_m^0}$から$\ket{\phi_n^0}$への遷移確率$P_n(t)$を求める。$c_k^0 = \delta_{km}$なので、
    \begin{align*}
        \dv{c_n^1}{t}
            &= \frac{1}{i\hbar} \sum_k \exp\(-i\frac{E_k^0 - E_n^0}{\hbar}t\) \mel{\phi_n^0}{V(t)}{\phi_k^0} c_k^0\\
            &= \frac{1}{i\hbar} \exp\(-i\frac{E_m^0 - E_n^0}{\hbar}t\) \mel{\phi_n^0}{V(t)}{\phi_m^0}\\
        c_n^1(t)
            &= \frac{1}{i\hbar} \left[c_n^1(0) + \int_0^t \exp\(-i\frac{E_m^0 - E_n^0}{\hbar}t'\) \mel{\phi_n^0}{V}{\phi_m^0} \dd{t'}\right]\\
            &= \left[\exp\(-i\frac{E_m^0 - E_n^0}{\hbar}t'\) \frac{\mel{\phi_n^0}{V}{\phi_m^0}}{E_m^0 - E_n^0}\right]_0^t\\
            &= \frac{\mel{\phi_n^0}{V}{\phi_m^0}}{E_m^0 - E_n^0} \left[\exp\(-i\frac{E_m^0 - E_n^0}{\hbar}t\) - 1\right]\\
        c_n(t)
            &\simeq c_n^0 + c_n^1(t)\\
            &= \frac{\mel{\phi_n^0}{V}{\phi_m^0}}{E_m^0 - E_n^0} \left[\exp\(-i\frac{E_m^0 - E_n^0}{\hbar}t\) - 1\right]
    \end{align*}
    である。$e^{i\theta} - 1 = e^{i\theta/2}(e^{i\theta/2} - e^{-i\theta/2}) = 2ie^{i\theta/2}\sin(\theta/2)$より遷移確率は、
    \begin{align*}
        P_n(t)
            &= |c_n(t)|^2\\
            &\simeq \(\frac{|\mel{\phi_n^0}{V}{\phi_m^0}|}{E_m^0 - E_n^0}\)^2 \cdot 4\sin^2\(\frac{E_m^0 - E_n^0}{2\hbar}t\)\\
            &= \frac{4|\mel{\phi_n^0}{V}{\phi_m^0}|^2}{(E_m^0 - E_n^0)^2} \sin^2\(\frac{E_m^0 - E_n^0}{2\hbar}t\)
    \end{align*}
    となる。エネルギー準位の差が小さいほど遷移確率は大きい。また単位時間当たりの遷移確率は
    \begin{align*}
        \lim_{t \to \infty} \frac{P_n(t)}{t} = \frac{2\pi}{\hbar} |\mel{\phi_n^0}{V}{\phi_m^0}|^2 \delta(E_m^0 - E_n^0)
    \end{align*}
    これをフェルミの黄金律と呼ぶ。

\subsection{変分法}
    \begin{thm}
        ハミルトニアン$H$とその基底エネルギー$E_0$について、任意の状態$\ket{\phi}$に対して、不等式
            \[\ev{H}{\phi} \geq E_0\]
        が成り立つ。等号成立は$\psi$が基底状態のときのみである。
    \end{thm}
    \begin{thm}
        ハミルトニアン$H$の固有状態は$\ev{H}[\phi] = \ev{H}{\phi}$の停留点である。
    \end{thm}
    上記の定理より、状態を適当なパラメータで表し停留点を求めることで、固有状態を近似的に求めることができる。これを変分法と呼ぶ。

\footnote{ブラケット記法でこのように書くと冗長だが、慣習的にこのように書く。}

\subsection{WKB近似}
    位相を$S$として波動関数を
        \[\phi(x) = e^{\frac{i}{\hbar}S(x)}\]
    と仮定する。
    \begin{align*}
        \dv[2]{\phi}{x}
            &= \dv{x}\(\frac{i}{\hbar}\dv{S}{x}e^{\frac{i}{\hbar}S(x)}\)\\
            &= \frac{i}{\hbar}\dv[2]{S}{x}e^{\frac{i}{\hbar}S(x)} - \frac{1}{\hbar^2}\(\dv{S}{x}\)^2e^{\frac{i}{\hbar}S(x)}
    \end{align*}
    より
    \begin{align*}
        \left[-\frac{\hbar^2}{2m}\dv[2]{x} + V(x)\right]\phi(x) &= E\phi(x)\\
        \frac{1}{2m}\left[-i\hbar\dv[2]{S}{x} + \(\dv{S}{x}\)^2\right] + V(x) &= E
    \end{align*}
    $\hbar \to 0$の極限でこの方程式はハミルトン=ヤコビ方程式に帰着し、$S(x)$は位相に対応する。この方程式は$\dv*{S}{x}$に関するリッカチの微分方程式である。
        \[S(x) = S_0(x) + \(\frac{\hbar}{i}\)S_1(x) + \(\frac{\hbar}{i}\)^2S_2(x) + \cdots\]
    と展開すると運動量$p(x) = \sqrt{2m(E - V(x))}$を用いて
    \begin{align*}
        S_0(x) &= \pm \frac{i}{\hbar}\int_0^x p(x')dx'\\
        S_1(x) &= \log \frac{1}{\sqrt{p(x)}} + C_1
    \end{align*}
    となる。0次は古典力学の解に対応する。$p(x)$を複素関数とみなし閉路に沿って積分すると
        \[\oint p \dd{x} \simeq S(x) = nh\]
    となりゾンマーフェルトの量子条件が導かれる。これをWKB(Wentzel Kramers Brillouin)近似という。
\section{多粒子系}

\subsection{同種粒子系}
    量子力学において同種粒子を区別することはできない(不可弁別性)。したがって同種粒子系の状態はある対称性を持ったものに限られる。
    
    $N$個の同種粒子からなる系の状態$\psi(r_1, \dots, r_N)$を考える。粒子$i$と粒子$j$を入れ替える演算子$P_{ij}$を
        \[P_{ij}\psi(\dots, r_i, \dots, r_j, \dots) = \psi(\dots, r_j, \dots, r_i, \dots)\]
    とする。系の状態は粒子の入れ替えによって観測可能な違いを生じない。したがって$P_{ij}\psi = e^{i\theta}\psi$である。$P_{ij}^2 = 1$より結局$P_{ij}\psi = \pm \psi$である。つまり同種粒子系の状態は任意の$i, j$について$P_{ij}$の固有状態つまり同時固有状態である。同時固有状態に縮退がないなら、$P_{ij}P_{jk}P_{ki} = P_{jk}$より$P_{ij}$の固有値は等しい。つまり固有値が全て$1$(完全対称)か固有値がすべて$-1$(完全反対称)となる。前者をボース粒子(ボソン)、後者をフェルミ粒子(フェルミオン)と呼ぶ。ボース粒子とフェルミ粒子は超選択則により混合しない。
    
    ここで以下の定理が成り立つ。
    \begin{thm}[スピンと統計の定理]
        整数スピンの粒子は完全対称、半整数スピンの粒子は完全反対称である。
    \end{thm}

    多粒子系の状態を一粒子状態のテンソル積を対称化または反対称化した状態で表されることがある。固有値$\lambda$に属する一粒子の固有状態を$\ket{\lambda}$とし、$\ket{\lambda}$にある粒子が$n_\lambda$ ($\sum_\lambda n_\lambda = N$)個あるときの固有状態を$\ket{\{n_\lambda\}}$と書くことにする。ボース粒子系の固有状態は、
        \[\ket{\{n_\lambda\}_\lambda} = \frac{1}{\sqrt{N! \prod_i n_i!}} \sum_{\sigma \in S_N} \ket{\lambda_{\sigma(1)}} \otimes \dots \otimes \ket{\lambda_{\sigma(N)}}\]
    フェルミ粒子系の固有状態は、
        \[\ket{\{n_\lambda\}_\lambda} = \frac{1}{\sqrt{N!}} \sum_{\sigma \in S_N} (-1)^\sigma \ket{\lambda_{\sigma(1)}} \otimes \dots \otimes \ket{\lambda_{\sigma(N)}}\]
    となる。フェルミ粒子の場合、ある$\lambda$で$n_\lambda > 1$のとき0になる。つまり二つの粒子が同じ固有状態となることはない(パウリの排他原理)。

\subsection{フォック空間}
    通常、多粒子系の状態空間は一粒子状態空間のテンソル積の真部分空間となる。したがって多粒子系の問題を解くときは次のような二段階の手順を踏むことになる。
    \begin{enumerate}
        \item 一粒子系の状態を解く。
        \item 一粒子状態のテンソル積を対称化/反対称化して多粒子系の状態を作る。
    \end{enumerate}
    しかし初めから多粒子系の状態空間として対称/反対称なものを想定すればこのような手続きは不要である。このような空間をフォック空間という。また、同種粒子は区別できないので、一粒子固有状態にいくつの粒子があるかを示す占有数表示を用いるのが自然である。

    一粒子状態空間$V$に対し、$N$粒子の対称/反対称な状態空間は
        \[F_\pm^N = S_\pm \H^{\otimes N}\]
    となる。

    一粒子状態を$\ket{i}$ ($i = 1, 2, \dots$)で表現する。一粒子状態$\ket{i}$にある粒子が$n_i$個あるような$N$個の同種粒子系の状態を$\ket{n_1, n_2, \dots}$と書くことにする。$N$個の粒子に番号を付け、粒子$\alpha$が一粒子状態$\ket{\phi_\alpha}$にあるとして、ボース粒子系の固有状態を
        \[\ket{n_1, n_2, \dots} = \frac{1}{\sqrt{N! \prod_i n_i!}} \sum_{\sigma \in S^N} \ket{\phi_{\sigma(1)}} \otimes \dots \otimes \ket{\phi_{\sigma(N)}}\]
    フェルミ粒子系の固有状態を
        \[\ket{n_1, n_2, \dots} = \frac{1}{\sqrt{N!}} \sum_{\sigma \in S^N} (-1)^\sigma \ket{\phi_{\sigma(1)}} \otimes \dots \otimes \ket{\phi_{\sigma(N)}}\]
    と定義する。この定義が番号に付け方に依らないことは明らかである。$\ket{0, 0, \dots}$は真空を表す。

    場の扱い。生成消滅を伴う現象。多粒子系の計算。物質と場の統一。

\subsection{生成消滅演算子}
    $i$番目の一粒子状態に対して、ボース粒子及びフェルミ粒子系の生成消滅演算子$d_i^\dagger: F_\pm^N \to F_\pm^{N+1}, d_i: F_\pm^N \to F_\pm^{N-1}$を
    \begin{align*}
        d_i^\dagger \ket{n_1, n_2, \dots, n_i, \dots} &= \sqrt{n_i + 1}\ket{n_1, n_2, \dots, n_i + 1, \dots}\\
        d_i \ket{n_1, n_2, \dots, n_i, \dots} &= \sqrt{n_i}\ket{n_1, n_2, \dots, n_i - 1, \dots}
    \end{align*}
    と定義する。ボース粒子系には交換関係
        \[[d_i^\dagger, d_i^\dagger] = 0, \quad [d_i, d_j] = 0, \quad [d_i, d_j^\dagger] = \delta_{ij}\]
    が成り立ち、フェルミ粒子系では反交換関係
        \[\{d_i^\dagger, d_i^\dagger\} = 0, \quad \{d_i, d_j\} = 0, \quad \{d_i, d_j^\dagger\} = \delta_{ij}\]
    が成り立つ。また個数演算子$\hat{n_i}$を
        \[\hat{n_i} = d_i^\dagger d_i\]
    とする。

    生成演算子を用いると同種粒子系の状態は
        \[\ket{n_1, n_2, \dots} = \prod_{i = 1}^\infty \frac{1}{\sqrt{n_i!}} (d_i^\dagger)^{n_i} \ket{0, 0, \dots}\]
    となる。

\subsection{第二量子化}
    系を占有数表示で議論するために演算子を生成消滅演算子を用いて書き直す。同種粒子系において、$n$粒子状態にのみ作用する演算子を$n$体演算子という。$A$を一粒子状態に作用する演算子として、一体演算子
    \begin{align*}
        A^1 &= \sum_{\alpha = 1}^N A_\alpha\\
        A_\alpha &= I \otimes \dots \otimes A \otimes \dots \otimes I
    \end{align*}
    は
    \begin{align*}
        A^1\ket{n_1, n_2, \dots}
            &= \sum_\alpha A_\alpha \frac{1}{\sqrt{N! \prod_k n_k!}} \sum_{\sigma \in S^N} \ket{\phi_{\sigma(1)}} \otimes \dots \otimes \ket{\phi_{\sigma(\alpha)}} \otimes \dots \otimes \ket{\phi_{\sigma(N)}}\\
            &= \frac{1}{\sqrt{N! \prod_k n_k!}} \sum_\alpha \sum_{\sigma \in S^N} \ket{\phi_{\sigma(1)}} \otimes \dots \otimes A\ket{\phi_{\sigma(\alpha)}} \otimes \dots \otimes \ket{\phi_{\sigma(N)}}\\
            &= \frac{1}{\sqrt{N! \prod_k n_k!}} \sum_\alpha \sum_{\sigma \in S^N} \ket{\phi_{\sigma(1)}} \otimes \dots \otimes \left(\sum_i \mel{i}{A}{\phi_{\sigma(\alpha)}} \ket{i}\right) \otimes \dots \otimes \ket{\phi_{\sigma(N)}}\\
            &= \frac{1}{\sqrt{N! \prod_k n_k!}} \sum_i \sum_{\sigma \in S^N} \sum_\alpha \mel{i}{A}{\phi_{\sigma(\alpha)}} \ket{\phi_{\sigma(1)}} \otimes \dots \otimes \ket{i} \otimes \dots \otimes \ket{\phi_{\sigma(N)}}\\
            &= \frac{1}{\sqrt{N! \prod_k n_k!}} \sum_{i, j} \mel{i}{A}{j} \sum_{\sigma \in S^N} \sum_{\alpha, \phi_{\sigma(\alpha)} = j} \ket{\phi_{\sigma(1)}} \otimes \dots \otimes \ket{i} \otimes \dots \otimes \ket{\phi_{\sigma(N)}}\\
            &= \sum_i \mel{i}{A}{i} \frac{1}{\sqrt{N! \prod_k n_k!}} \sum_{\sigma \in S^N} \sum_{\alpha, \phi_{\sigma(\alpha)} = i} \ket{\phi_{\sigma(1)}} \otimes \dots \otimes \ket{i} \otimes \dots \otimes \ket{\phi_{\sigma(N)}}\\
            &\quad + \sum_{i \neq j} \mel{i}{A}{j} \frac{1}{\sqrt{N! \prod_k n_k!}} \sum_{\sigma \in S^N} \sum_{\alpha, \phi_{\sigma(\alpha)} = j} \ket{\phi_{\sigma(1)}} \otimes \dots \otimes \ket{i} \otimes \dots \otimes \ket{\phi_{\sigma(N)}}
    \end{align*}
    各$\sigma$で$\phi_{\sigma(\alpha)} = j$となる$\alpha$は$n_j$個あるので、
    \begin{align*}
        A^1\ket{n_1, n_2, \dots}
            &= \sum_i \mel{i}{A}{i} n_i\ket{n_1, n_2, \dots} + \sum_{i \neq j} \mel{i}{A}{j} n_j\sqrt{\frac{n_i + 1}{n_j}}\ket{n_1, \dots, n_i + 1, \dots, n_j - 1, \dots}\\
            &= \sum_i \mel{i}{A}{i} n_i\ket{n_1, n_2, \dots} + \sum_{i \neq j} \mel{i}{A}{j} \sqrt{(n_i + 1)n_j}\ket{n_1, \dots, n_i + 1, \dots, n_j - 1, \dots}
    \end{align*}
    より
    \begin{align*}
        A^1 &= \sum_{i, j} \mel{i}{A}{j} d_i^\dagger d_j
    \end{align*}
    となる。同様に$A_{\alpha\beta}$を粒子$\alpha, \beta$の組にのみ作用する演算子として、二体演算子
    \begin{align*}
        A^2 &= \sum_{\alpha < \beta} A_{\alpha\beta}\\
        A_{\alpha\beta} &= I \otimes \dots \otimes A \otimes \dots \otimes B \otimes \dots \otimes I
    \end{align*}
    は
    \begin{align*}
        A^2 = \frac{1}{2}\sum_{i, j, k, l} \mel{i,j}{A}{k,l} d_i^\dagger d_j^\dagger d_k d_l
    \end{align*}
    となる。一般の$n$体演算子についても同様。
    
    ここで場の演算子$\psi(r) \colon F_\pm^N \to F_\pm^{N-1}, \psi^\dagger(r) \colon F_\pm^N \to F_\pm^{N+1}$
    \begin{align*}
        \psi^\dagger(r) &= \sum_i d_i^\dagger \phi_i^*(r)\\
        \psi(r) &= \sum_i \phi_i(r)d_i
    \end{align*}
    を導入する。ボース粒子系の場合
    \begin{align*}
        [\psi^\dagger(r_1), \psi^\dagger(r_2)] &= \sum_{i, j} \phi_i^*(r_1)\phi_j^*(r_2) [d_i^\dagger, d_j^\dagger] = 0\\
        [\psi(r_1), \psi(r_2)] &= \sum_{i, j} \phi_i(r_1)\phi_j(r_2) [d_i, d_j] = 0\\
        [\psi(r_1), \psi^\dagger(r_2)] &= \sum_{i, j} \phi_i(r_1)\phi_j^*(r_2) [d_i, d_j^\dagger] = \delta(r_1 - r_2)
    \end{align*}
    フェルミ粒子系の場合
    \begin{align*}
        \{\psi^\dagger(r_1), \psi^\dagger(r_2)\} &= \sum_{i, j} \phi_i^*(r_1)\phi_j^*(r_2) \{d_i^\dagger, d_j^\dagger\} = 0\\
        \{\psi(r_1), \psi(r_2)\} &= \sum_{i, j} \phi_i(r_1)\phi_j(r_2) \{d_i, d_j\} = 0\\
        \{\psi(r_1), \psi^\dagger(r_2)\} &= \sum_{i, j} \phi_i(r_1)\phi_j^*(r_2) \{d_i, d_j^\dagger\} = \delta(r_1 - r_2)
    \end{align*}
    が成り立つ。最後の交換/反交換関係には
        \[\int \left[\sum_i \phi_i(r_1)\phi_i^*(r_2)\right]f(r_2) dr_2 = \sum_i \phi_i(r_1) \int \phi_i^*(r_2)f(r_2) dr_2 = f(r_1)\]
    より
        \[\sum_i \phi_i(r_1)\phi_i^*(r_2) = \delta(r_1 - r_2)\]
    を用いた。場の演算子を用いると
    \begin{align*}
        A^1 &= \psi^\dagger A \psi\\
        A^2 &= {\psi^\dagger}^2 A \psi^2
    \end{align*}
    と書ける。このような演算子の表示を変換する手続きを第二量子化という。
\section{量子化学}

量子化学は、量子力学を用いて原子核と電子からなる多体系の問題を扱う学問である。任意の2粒子間のクーロンポテンシャルを考慮してシュレーディンガー方程式を解析的に解くのは困難であるため、様々な近似法が考案されている。
\begin{description}
    \item[原子価結合法(Valence Bond theory, VB)] 電子を特定の原子軌道に属するものとして化学結合を説明する。
    \item[分子軌道法(Molecular Orbital method, MO)] 電子を分子軌道に属するものとして化学結合を説明する。
    \begin{description}
        \item[LCAO法(Linear Combination of Atomic Orbital)] 分子軌道を原子軌道の線形結合として近似。
    \end{description}
    % \item[分子力学法(Molecular Mechanics, MM)] 分子の安定性を原子間のポテンシャルエネルギーの総和によって計算する。
    % \item[密度汎関数理論(Density Function Theory, DFT)] 電子密度から多体電子系の性質を計算する。
\end{description}

\subsection{多体問題}
    $n$個の質点系を考える。それぞれの質量を$m_1, m_2, \dots, m_n$、座標を$r_1, r_2, \dots, r_n$とおく。ポテンシャルが相対座標のみに依存するとき、ハミルトニアンは
        \[H = -\sum_i \frac{\hbar^2}{2m_i}\Delta_i + V(r_i - r_j)\]
    である。重心及び相対座標を
    \begin{align*}
        r_G &= \frac{m_1r_1 + m_2r_2 + \dots + m_nr_n}{m_1 + m_2 + \dots + m_n}\\
        r'_i &= r_i - r_G
    \end{align*}
    また全質量を$M = m_1 + m_2 + \dots + m_n$とすると
    \begin{align*}
        \pdv{x_i}
            &= \pdv{x_G}{x_i}\pdv{x_G} + \sum_j \pdv{x'_j}{x_i}\pdv{x'_j}\\
            &= \frac{m_i}{M}\pdv{x_G} + \pdv{x'_i} - \frac{m_i}{M} \sum_j \pdv{x'_j}\\
        \pdv[2]{x_i}
            &= \frac{m_i^2}{M^2}\pdv[2]{x_G} + \pdv[2]{{x'_i}} + \frac{m_i^2}{M^2}\(\sum_j \pdv{x'_j}\)^2\\
            &\quad + 2\frac{m_i}{M}\pdv{x_G}\pdv{x'_i} - 2\frac{m_i^2}{M^2} \sum_j \pdv{x_G}\pdv{x'_j} - 2\frac{m_i}{M} \sum_j \pdv{x'_i}\pdv{x'_j}
    \end{align*}
    となる。$y, z$についても同様。$x, y, z$の和を取ると
    \begin{align*}
        \Delta_i &= \frac{m_i^2}{M^2}\Delta_G + \Delta'_i + \frac{m_i^2}{M^2}\(\sum_j \nabla'_j\)^2 + 2\frac{m_i}{M}\nabla_G \cdot \nabla'_i - 2\frac{m_i^2}{M^2} \sum_j \nabla_G \cdot \nabla'_j - 2\frac{m_i}{M} \sum_j \nabla'_i \cdot \nabla'_j\\
        -\frac{\hbar^2}{2m_i}\Delta_i &= -\frac{m_i\hbar^2}{2M^2}\Delta_G - \frac{\hbar^2}{2m_i}\Delta'_i - \frac{m_i\hbar^2}{2M^2}\(\sum_j \nabla'_j\)^2 - \frac{\hbar^2}{M}\nabla_G \cdot \nabla'_i + \frac{m_i\hbar^2}{M^2} \sum_j \nabla_G \cdot \nabla'_j + \frac{\hbar^2}{M} \sum_j \nabla'_i \cdot \nabla'_j
    \end{align*}
    $i$について和を取ると
    \begin{align*}
        -\sum_i \frac{\hbar^2}{2m_i}\Delta_i
            &= -\frac{\hbar^2}{2M}\Delta_G - \sum_i \frac{\hbar^2}{2m_i}\Delta'_i - \frac{\hbar^2}{2M}\(\sum_i \nabla'_i\)^2 - \frac{\hbar^2}{M}\nabla_G \cdot \sum_i \nabla'_i + \frac{\hbar^2}{M} \sum_j \nabla_G \cdot \nabla'_j + \frac{\hbar^2}{M} \sum_{i,j} \nabla'_i \cdot \nabla'_j\\
            &= -\frac{\hbar^2}{2M}\Delta_G - \sum_i \frac{\hbar^2}{2m_i}\Delta'_i + \frac{\hbar^2}{2M}\(\sum_j \pdv{x'_j}\)^2
    \end{align*}
    従ってハミルトニアンは
        \[H = -\frac{\hbar^2}{2M}\Delta_G - \sum_i \frac{\hbar^2}{2m_i}\Delta'_i + \frac{\hbar^2}{2M}\(\sum_i \nabla'_i\)^2 + V(r'_i - r'_j)\]
    となる。重心のみに依存する項と相対座標のみに依存する項に分解できるので、波動関数を$\phi(r_G, r'_1, \dots, r'_n) = \phi_G(r_G)\phi_r(r'_1, \dots, r'_n)$と変数分離できる。重心成分は系全体を一つの自由粒子と見なしたときの波動関数と一致する。よって以降は相対座標のみに依存する部分を考える。

\subsection{多電子原子}
    中心の原子核と$n$個の電子からなる系を考える。原子核は質量$\mu_s$、電荷$Ze$の質点であり、電子は質量$\mu$、電荷$-e$の質点であるとする。原子核は電子に比べて十分重いので$M$に反比例する項は無視する。更に、重心からの相対座標を原子核からの相対座標で置き換え、それぞれ$r_1, \dots, r_n$とおく。系のハミルトニアンは
        \[H = -\frac{\hbar^2}{2\mu_s}\Delta_0 - \sum_{i=1}^n \frac{\hbar^2}{2\mu}\Delta_i - \sum_{i=1}^n \frac{Ze^2}{4\pi\epsilon_0}\frac{1}{|r_i|} + \sum_{i,j=1}^n \frac{e^2}{4\pi\epsilon_0}\frac{1}{|r_i - r_j|}\]
    ここで、ある電子が他の電子から受ける力が平均的に相殺され、原子核方向への成分のみが残ると考え、電子間相互作用を無視できるとする。核電荷の大きさを補正して$\bar{Z}e$とおくと
        \[H = -\frac{\hbar^2}{2\mu_0}\Delta_0 - \sum_{i=1}^n \frac{\hbar^2}{2\mu}\Delta_i - \sum_{i=1}^n \frac{1}{4\pi\epsilon_0}\frac{\bar{Z}e^2}{|r_i|}\]
    となる。これを一電子近似という。ハミルトニアンは$r_i\ (0 \leq i \leq n)$のみに依存する項に分解できるので、$\phi(r_0, r_1, \dots, r_n) = \phi_0(r_0)\phi_1(r_1) \dots \phi_n(r_n)$と変数分離できる。よってシュレーディンガー方程式は
        \[-\frac{\hbar^2}{2m}\Delta_i\phi_i - \frac{\bar{Z}e^2}{4\pi\epsilon_0}\frac{1}{|r_i|}\phi_i = E_i\phi_i\]
    このとき
        \[E = \sum_i E_i\]
    である。

    それぞれの電子は原子核からのクーロンポテンシャルに従って運動するから、水素原子と同様に量子数$(n, l, m, s)$によって指定される固有状態を持つ。電子はフェルミ粒子だから反対称性より同じ固有状態には一個までしか入ることができない。これをパウリの排他原理という。特に安定した状態では、エネルギーの低い準位から順に入ることになる。

\subsection{水素分子イオン}
    LCAO法を用いて水素分子の波動関数を計算する。

    陽子二個と電子一個からなる系を考える。陽子間の距離を$R$、陽子と電子の距離をそれぞれ$r_a, r_b$とすると、ハミルトニアンは
        \[H = -\frac{\hbar^2}{2m}\Delta - \frac{e^2}{4\pi\epsilon_0}\frac{1}{r_a} - \frac{e^2}{4\pi\epsilon_0}\frac{1}{r_b} + \frac{e^2}{4\pi\epsilon_0}\frac{1}{R}\]
    である。原子軌道として基底状態すなわち1s軌道$\phi^{1s} = \frac{1}{\sqrt{\pi a_0^3}}e^{-r/a_0}$を考える。それぞれの原子軌道を
    \begin{align*}
        \phi_a(r) = \phi^{1s}\(r + \frac{R}{2}\)\\
        \phi_b(r) = \phi^{1s}\(r - \frac{R}{2}\)
    \end{align*}
    とする。つまり
    \begin{align*}
        \(-\frac{\hbar^2}{2m}\Delta - \frac{e^2}{4\pi\epsilon_0}\frac{1}{r_a}\)\phi_a(r) &= E^{1s}\phi_a(r)\\
        \(-\frac{\hbar^2}{2m}\Delta - \frac{e^2}{4\pi\epsilon_0}\frac{1}{r_b}\)\phi_b(r) &= E^{1s}\phi_b(r)
    \end{align*}
    分子軌道はそれらの線形結合として
        \[\phi = c_a\phi_a + c_b\phi_b\]
    と表せると仮定する。対称性よりそれぞれの軌道に属する確率は等しいので、$c_a = \pm c_b$である。規格化条件から
    \begin{align*}
        c_a^2 \int (\phi_a \pm \phi_b)^*(\phi_a \pm \phi_b) \dd{V} &= 1\\
        c_a^2 \int ({\phi_a}^*\phi_a \pm {\phi_a}^*\phi_b \pm {\phi_b}^*\phi_a + {\phi_b}^*\phi_b) \dd{V} &= 1
    \end{align*}
    積分の第一項と第四項は共に1である。また、$\phi_a, \phi_b$は実関数なので、第二項と第三項の重なり積分は実数$S$と置く。
    \begin{gather*}
        c_a^2 (2 \pm 2S) = 1\\
        \therefore c_a = c_b = \frac{1}{\sqrt{2(1 + S)}},\ c_a = -c_b = \frac{1}{\sqrt{2(1 - S)}}
    \end{gather*}
    つまり分子軌道は
        \[\phi^+ = \frac{1}{\sqrt{2(1 + S)}}(\phi_a + \phi_b),\ \phi^- = \frac{1}{\sqrt{2(1 - S)}}(\phi_a - \phi_b)\]
    となる。エネルギー固有値は求められないので、ハミルトニアンから期待値を計算する。ここで
    \[
        \begin{aligned}
            J &= \frac{e^2}{4\pi\epsilon_0} \left[-\int \frac{\phi_a^*\phi_a}{r_b} \dd{r} + \frac{1}{R}\right] = \frac{e^2}{4\pi\epsilon_0}\left[-\int \frac{\phi_b^*\phi_b}{r_a} \dd{r} + \frac{1}{R}\right] & (クーロン積分)\\
            K &= \frac{e^2}{4\pi\epsilon_0}\left[-\int \frac{\phi_b^*\phi_a}{r_b} \dd{r} + \frac{S}{R}\right] = \frac{e^2}{4\pi\epsilon_0}\left[-\int \frac{\phi_a^*\phi_b}{r_a} \dd{r} + \frac{S}{R}\right] & (共鳴積分)
        \end{aligned}
    \]
    とおくと
    \begin{align*}
        &\int (\phi_a \pm \phi_b)^* H (\phi_a \pm \phi_b) \dd{r}\\
        &= \int (\phi_a \pm \phi_b)^* \(-\frac{\hbar^2}{2m}\Delta - \frac{1}{4\pi\epsilon_0}\frac{e^2}{r_a} - \frac{1}{4\pi\epsilon_0}\frac{e^2}{r_b} + \frac{1}{4\pi\epsilon_0}\frac{e^2}{R}\) (\phi_a \pm \phi_b) \dd{r}\\
        &= \int (\phi_a \pm \phi_b)^* \left\{\(E_{1s} - \frac{e^2}{4\pi\epsilon_0}\frac{1}{r_b} + \frac{e^2}{4\pi\epsilon_0}\frac{1}{R}\)\phi_a \pm \(E_{1s} - \frac{e^2}{4\pi\epsilon_0}\frac{e^2}{r_a} + \frac{e^2}{4\pi\epsilon_0}\frac{1}{R}\)\phi_b\right\} \dd{r}\\
        &= E^{1s} + J \pm SE^{1s} \pm  K \pm SE^{1s} \pm K + E^{1s} + J\\
        &= 2(1 \pm S)E^{1s} + 2(J \pm K)
    \end{align*}
    つまり
    \begin{align*}
        E^+ &= E^{1s} + \frac{J + K}{1 + S}\\
        E^- &= E^{1s} + \frac{J - K}{1 - S}
    \end{align*}
    となる。どのような$R$に対しても$E^+ < E^{1s} < E^-$となるので、$H$と$H^+$が別々に存在するよりも水素分子イオンとして存在した方が安定である。$\phi^+$を結合性軌道、$\phi^-$を反結合性軌道と呼ぶ。$\phi^+$は二つの原子核の間に電子が存在するため、クーロン力により二つの原子核を結び付け安定化すると考えられる。
\section{散乱理論}

\subsection{量子系の散乱}
    時刻$t$における状態$u(t)$に対して、
        \[Su(-\infty) = u(\infty)\]
    となるユニタリ演算子を散乱演算子という。散乱演算子を特定の基底について行列表示したものをS行列という。

\subsection{リップマン=シュウィンガー方程式}
    散乱する前の波動関数$\phi$は$H_0\phi = E\phi$を満たす。散乱後ハミルトニアンは$H = H_0 + V$となり、散乱の前後でエネルギーは保存するので波動関数$\psi$は
        \[(H_0 + V)\psi = E\psi\]
    を満たす。$V \to 0$で$\psi \to \phi$となるので解として
    \begin{align*}
        (E - H_0)\psi &= V\psi = (E - H_0)\phi + V\psi\\
        \psi &= \phi + \frac{1}{E - H_0}V\psi
    \end{align*}
    が考えられる。正確には演算子$1 / (E - H_0)$の分母に$i\epsilon$を加え$\epsilon \to 0$とする。
        \[\psi = \phi + \frac{1}{E - H_0 \pm i\epsilon}V\psi\]
    となる。これはグリーン関数
        \[(E - H_0)G(\bm{r}, \bm{r'}) = \delta(\bm{r} - \bm{r'})\]
    を用いると
        \[\psi(\bm{r}) = \phi(\bm{r}) + \int G(\bm{r}, \bm{r'})V(\bm{r'})\psi(\bm{r'}) d\bm{r'}\]
    と書ける。第一項は入射波、第二項は散乱波を表している。グリーン関数は
        \[G^\pm(\bm{r}, \bm{r'}) = -\frac{2m}{\hbar^2}\frac{1}{4\pi}\frac{e^{\pm ik|\bm{r} - \bm{r'}|}}{|\bm{r} - \bm{r'}|} \quad \left(k^2 = \frac{2mE}{\hbar^2}\right)\]
    となる。ポテンシャルが無限遠で0に漸近するとして、$r \to \infty$の漸近形を考える。$r \gg r'$なので、
    \begin{align*}
        |\bm{r} - \bm{r'}|
            &= \sqrt{r^2 - 2\bm{r} \cdot \bm{r'} + r'^2}\\
            &= \sqrt{\left(r - \frac{\bm{r} \cdot \bm{r'}}{r}\right)^2 + r'^2 - \frac{(\bm{r} \cdot \bm{r'})^2}{r^2}}\\
            &\simeq r - \bm{\hat{r}} \cdot \bm{r'}\\
        G^\pm(\bm{r}, \bm{r'})
            &\simeq -\frac{2m}{\hbar^2}\frac{1}{4\pi}\frac{e^{\pm ik(r - \bm{\hat{r}} \cdot \bm{r'})}}{r}
    \end{align*}
    となる。$G^+(\bm{r}, \bm{r'})$は外向きの散乱、$G^-(\bm{r}, \bm{r'})$は内向きの散乱を表す。実際には内向きの散乱が起こる系を準備するのは困難であるから、ここでは外向きの散乱のみを考える。入射波を$\phi(r) = e^{i\bm{k} \cdot \bm{r}}$とし、$\bm{k'} = k\bm{\hat{r}}$とおくと、リップマン=シュウィンガー方程式は
    \begin{align*}
        \psi(\bm{r}) &\simeq e^{i\bm{k} \cdot \bm{r}} + f(\bm{k}, \bm{k'})\frac{e^{ikr}}{r} \quad (r \to \infty)\\
        f(\bm{k}, \bm{k'}) &= -\frac{2m}{\hbar^2}\frac{1}{4\pi} \int e^{-i\bm{k'} \cdot \bm{r'}}V(\bm{r'})\psi(\bm{r'}) d\bm{r'}
    \end{align*}
    となる。$f(\bm{k}, \bm{k'})$を散乱振幅という。

\subsection{ボルン近似}
    リップマン=シュウィンガー方程式において、$\psi(r)$をポテンシャルについて摂動展開する。つまり
    \begin{gather*}
        \psi = \sum_{n = 0}^\infty \psi_n\\
        \psi_0(\bm{r}) = e^{i\bm{k} \cdot \bm{r}}, \quad \psi_{n + 1}(\bm{r}) = \int G^+(\bm{r}, \bm{r'})V(\bm{r'})\psi_n(\bm{r'}) d\bm{r'}
    \end{gather*}
    である。一次の項まで展開する。これを(第一)ボルン近似という。さらに無限遠で0に漸近するポテンシャルを考えると
    \begin{align*}
        \psi(\bm{r})
            &\simeq \psi_0(\bm{r}) + \psi_1(\bm{r})\\
            &= e^{i\bm{k} \cdot \bm{r}} + \int G^+(\bm{r}, \bm{r'})V(\bm{r'})e^{i\bm{k} \cdot \bm{r'}} d\bm{r'}\\
            &\simeq e^{i\bm{k} \cdot \bm{r}} - \frac{2m}{\hbar^2}\frac{1}{4\pi} \left[\int e^{i(\bm{k} - k\bm{k'}) \cdot \bm{r'}}V(\bm{r'}) d\bm{r'}\right] \frac{e^{ikr}}{r}
    \end{align*}
    つまり散乱振幅はポテンシャルのフーリエ変換で与えられる。
        \[f(\bm{k}, \bm{k'}) \simeq -\frac{m}{2\pi\hbar^2}\hat{V}(\bm{k'} - k\bm{k'})\]

\subsection{球対称ポテンシャル}
    無限遠で0に漸近する球対称ポテンシャル$V(r)$による散乱を考える。ここでは定常波を想定する。入射波を$z$軸方向に進行する平面波$\psi_i(r) = e^{ikz}$、散乱波を$\psi_s(r)$とすると、全体の波動関数は$\psi(r) = \psi_i(r) + \psi_s(r)$である。入射波、散乱波、波動関数はいずれも軸対称なのでルジャンドル多項式で展開できる。これを部分波展開という。波動関数を
        \[\psi(r) = \sum_{l = 0}^\infty R_l(r)P_l(\cos\theta)\]
    とすると動径波動関数は
    \begin{align*}
        \left[\dv[2]{r} + \frac{2}{r}\dv{r} - \frac{l(l + 1)}{r^2} - \frac{2m}{\hbar^2}V(r) + k^2\right]R_l(r) = 0\\
        E = \frac{\hbar^2k^2}{2m}
    \end{align*}
    を満たす。無限遠での漸近形を考えるとポテンシャルは無視できる。また$\rho = kr$とおくと
        \[\left[\dv[2]{\rho} + \frac{2}{\rho}\dv{\rho} - \frac{l(l + 1)}{\rho^2} + 1\right]R_l(\rho/k) = 0\]
    となる。これは球面波と同じ方程式であり、解は第一種及び第二種球ハンケル関数の線形結合となる。前者が外向きの球面波、後者が内向きの球面波を表す。
    \begin{align*}
        h_l^1(\rho) &= j_l(\rho) + in_l(\rho) \to (-i)^{l+1}\frac{e^{i\rho}}{\rho} \quad (\rho \to \infty)\\
        h_l^2(\rho) &= j_l(\rho) - in_l(\rho) \to i^{l+1}\frac{e^{-i\rho}}{\rho} \quad (\rho \to \infty)
    \end{align*}
    したがって波動関数の漸近形は
        \[\psi(r, \theta) \to \sum_{l = 0}^\infty \left[a_l(\theta)\frac{e^{ikr}}{r} + b_l(\theta)\frac{e^{-ikr}}{r}\right]P_l(\cos\theta)\]
    という形で書くことができる。
    
    また入射波$\psi_i(r) = e^{ikz} = e^{ikr\cos\theta}$は
        \[\psi_i(r, \theta) \to \sum_{l = 0}^\infty \frac{2l + 1}{2ik}\left[\frac{e^{ikr}}{r} - (-1)^l\frac{e^{-ikr}}{r}\right]P_l(\cos\theta) \quad (r \to \infty)\]
    したがって散乱波の漸近形は
        \[\psi_s(r, \theta) \to f(\theta)\frac{e^{ikr}}{r} \quad (r \to \infty)\]
    となる。

    散乱振幅を
        \[f(\theta) = \sum_{l = 0}^\infty \frac{2l + 1}{k}f_lP_l(\cos\theta)\]
    と展開すると、全体の波動関数は
        \[\psi(r) \to \sum_{l = 0}^\infty \frac{2l + 1}{2ik}\left[(1 + 2if_l)\frac{e^{ikr}}{r} - (-1)^l\frac{e^{-ikr}}{r}\right]P_l(\cos\theta) \quad (r \to \infty)\]
    となる。ここでS行列を$S_l = 1 + 2if_l$と定める。

    粒子流は
        \[J = -\frac{i\hbar}{2m}(\psi^*\nabla\psi - (\nabla\psi^*)\psi)\]
    である。入射波は
    \begin{align*}
        J_i &= -\frac{i\hbar}{2m}(\psi_i^*\nabla\psi_i - (\nabla\psi_i^*)\psi_i)\\
            &= -\frac{i\hbar}{2m}(e^{-ikz}(0, 0, ike^{ikz}) - (0, 0, -ike^{-ikz})e^{ikz})\\
            &= \frac{\hbar}{m}(0, 0, k)
    \end{align*}
    よって入射フラックスは$\hbar k / m$。散乱波は
    \begin{align*}
        J_s &= -\frac{i\hbar}{2m}(\psi_s^*\nabla\psi_s - (\nabla\psi_s^*)\psi_s)\\
            &= -\frac{i\hbar}{2m}\left(f(\theta)^*\frac{e^{-ikr}}{r} \cdot f(\theta)ik\frac{e^{ikr}}{r}\hat{r} + f(\theta)^*ik\frac{e^{-ikr}}{r}\hat{r} \cdot f(\theta)\frac{e^{-ikr}}{r}\right) + O\left(\frac{1}{r^3}\right)\\
            &= |f(\theta)|^2\frac{\hbar k}{m}\frac{\hat{r}}{r^2}
    \end{align*}
    よって散乱フラックスの漸近は$|f(\theta)|^2\hbar k / m$。つまり散乱断面積は$|f(\theta)|^2$であり散乱振幅の絶対値の二乗となる。部分波展開を用いると
        \[\dv{\sigma}{\Omega} = |f(\theta)|^2
            = \sum_{l, l' = 0}^\infty \frac{(2l + 1)(2l' + 1)}{k^2}f_lf_{l'}^*P_l(\cos\theta)P_{l'}(\cos\theta)\]
    なので、全断面積は
    \begin{align*}
        \sigma_T
            &= \int_{-1}^1 \sum_{l, l' = 0}^\infty \frac{(2l + 1)(2l' + 1)}{k^2}f_lf_{l'}P_l(\cos\theta)P_{l'}(\cos\theta) \cdot 2\pi d(\cos\theta)\\
            &= \frac{2\pi}{k^2} \sum_{l = 0}^\infty 2(2l + 1)|f_l|^2\\
            &= \frac{4\pi}{k^2} \sum_{l = 0}^\infty (2l + 1)|f_l|^2\\
            &= \frac{4\pi}{k^2} \sum_{l = 0}^\infty (2l + 1)\sin^2\delta_l
    \end{align*}
    また
        \[f(0) = \frac{1}{k}\sum_{l = 0}^\infty (2l + 1)e^{i\delta_l}\sin\delta_l\]
    なので、
        \[\sigma_T = \frac{4\pi}{k}\Im f(0)\]
    となる。これを光学定理という。
\section{相対論的量子力学}

\subsection{ディラック方程式}
    スピン$1/2$、電荷$q$を持つ粒子は以下のディラック方程式に従う。
    \begin{gather*}
        i\hbar\frac{1}{c}\pdv{t}\psi(x) = \left[\alpha \cdot (-i\hbar\nabla - qA) + \beta mc + \frac{q\phi}{c}\right]\psi(x)\\
        \alpha^i = \begin{pmatrix}
            0 & \sigma^i\\
            \sigma^i & 0
        \end{pmatrix}, \quad
        \beta = \begin{pmatrix}
            \sigma^0 & 0\\
            0 & -\sigma^0
        \end{pmatrix}
    \end{gather*}
    $\alpha, \beta$は$4 \times 4$のエルミート行列であり、波動関数$\psi(x)$は4成分ベクトルである。ディラック方程式とその複素共役を取ったものにそれぞれエルミート共役を掛けると、
    \begin{align*}
        i\hbar \psi^\dagger\frac{1}{c}\pdv{t}\psi
            &= \psi^\dagger \left[\alpha \cdot (-i\hbar\nabla - qA) + \beta mc + \frac{q\phi}{c}\right] \psi(x)\\
        -i\hbar \psi^\top\frac{1}{c}\pdv{t}\psi^*
            &= \psi^\top \left[\alpha^* \cdot (i\hbar\nabla - qA) + \beta^* mc + \frac{q\phi}{c}\right] \psi^*(x)\\
            &= \psi^\dagger \left[\alpha \cdot (i\hbar\nabla - qA) + \beta mc + \frac{q\phi}{c}\right] \psi(x)
    \end{align*}
    辺々引いて2で割ると
        \[i\hbar\frac{1}{c}\pdv{t}[\psi^\dagger(x)\psi(x)] = -i\hbar\nabla \cdot [\psi^\dagger(x) \alpha \psi(x)]\]
    となる。
        \[\rho(x) = |\psi(x)|^2, \quad j(x) = c \psi^\dagger(x)\alpha\psi(x)\]
    とおくと、連続の方程式
        \[\pdv{\rho}{t} + \nabla \cdot j = 0\]
    を満たす。$\rho(x) \geq 0$なので、$\rho(x)$を確率密度、$j(x)$を確率密度流と解釈することができる。

    また、共変微分とガンマ行列
    \begin{gather*}
        D_\mu = \partial_\mu - \frac{q}{i\hbar}A_\mu\\
        \gamma^0 = \beta, \quad \gamma^i = \beta\alpha^i
    \end{gather*}
    を用いると、ディラック方程式は
    \begin{align*}
        i\hbar\beta\frac{1}{c}\pdv{t}\psi(x) &= \left[\beta\alpha \cdot (-i\hbar\nabla - qA) + \beta^2 mc + \beta\frac{q\phi}{c}\right]\psi(x)\\
        i\hbar\gamma^0\left(\frac{1}{c}\pdv{t} - \frac{q}{i\hbar}\frac{\phi}{c}\right)\psi(x) &= -i\hbar\gamma^i \left(\partial_i - \frac{q}{i\hbar}A_i\right)\psi(x) + mc\psi(x)\\
        i\hbar \gamma^\mu D_\mu \psi(x) &= mc\psi(x)
    \end{align*}
    となる。

    それぞれの変数はローレンツ変換
        \[\Lambda = \exp \begin{pmatrix}
            0 & -\eta_x & -\eta_y & -\eta_z\\
            -\eta_x & 0 & \theta_z & -\theta_y\\
            -\eta_y & -\theta_z & 0 & \theta_x\\
            -\eta_z & \theta_y & -\theta_x & 0
        \end{pmatrix}\]
    に対して
    \begin{align*}
        x'^\mu &= \Lambda^\mu_\nu x^\nu\\
        D'_\mu &= \Lambda_\mu^\nu D_\nu\\
        \psi'(x') &= \exp\left(-\frac{\eta}{2} \cdot \alpha + i\frac{\theta}{2} \cdot \Sigma\right)\psi(x)
    \end{align*}
    となる。ただし
        \[\Sigma = -i(\alpha^2\alpha^3, \alpha^3\alpha^1, \alpha^1\alpha^2)\]
    である。$\psi(x) \mapsto \psi'(x')$のような変換に従う量をディラックスピノルと呼ぶ。

    ローレンツブーストに対して、$(\eta \cdot \alpha)^2 = |\eta|^2I$より、
    \begin{align*}
        \exp\left(-\frac{\eta}{2} \cdot \alpha\right)
            &= \cosh\left(\frac{\eta}{2} \cdot \alpha\right) - \sinh\left(\frac{\eta}{2} \cdot \alpha\right)\\
            &= \cosh\frac{|\eta|}{2} - \hat{\eta} \cdot \alpha \sinh\frac{|\eta|}{2}\\
            &= \begin{pmatrix}
                \cosh\frac{|\eta|}{2} & -\hat{\eta} \cdot \sigma \sinh \frac{|\eta|}{2}\\
                -\hat{\eta} \cdot \sigma \sinh\frac{|\eta|}{2} & \cosh\frac{|\eta|}{2}\\
            \end{pmatrix}
    \end{align*}
    である。また空間回転に対して、$(\theta \cdot \Sigma)^2 = |\theta|^2I$より、
    \begin{align*}
        \exp\left(i\frac{\theta}{2} \cdot \Sigma\right)
            &= \cos\left(\frac{\theta}{2} \cdot \Sigma\right) + i\sin\left(\frac{\theta}{2} \cdot \Sigma\right)\\
            &= \cos\frac{|\theta|}{2} + i\hat{\theta} \cdot \Sigma \sin\frac{|\theta|}{2}\\
            &= \begin{pmatrix}
                \cos\frac{|\theta|}{2} + i\hat{\theta} \cdot \sigma \sin\frac{|\theta|}{2} & 0\\
                0 & \cos\frac{|\theta|}{2} + i\hat{\theta} \cdot \sigma \sin\frac{|\theta|}{2}
            \end{pmatrix}
    \end{align*}
    である。

\subsection{物理量の演算子}
    ディラックスピノルに対する位置演算子、運動量演算子、軌道角運動量演算子は、各成分ごとに作用させるとして良い。
    
    ディラック方程式において、スピン角運動量演算子は
        \[S = \frac{\hbar}{2}\Sigma = -\frac{i\hbar}{2}(\alpha^2\alpha^3, \alpha^3\alpha^1, \alpha^1\alpha^2) = \frac{\hbar}{2}\begin{pmatrix}
            \sigma^i & 0\\
            0 & \sigma^i
        \end{pmatrix}\]
    と書ける。
    \begin{align*}
        S^2 &= \frac{\hbar^2}{4}((\Sigma^1)^2 + (\Sigma^2)^2 + (\Sigma^3)^2) = \frac{3}{4}\hbar^2I
    \end{align*}
    であり、ディラック方程式がスピン$1/2$の粒子の方程式であることが分かる。
    
    % 固有値は$\pm \hbar/2$で、それぞれの固有状態は
    % \begin{align*}
    %     \ket{\alpha_x} &= \frac{1}{\sqrt{2}}\vectwo{1}{1}, & \ket{\beta_x} &= \frac{1}{\sqrt{2}}\vectwo{1}{-1}\\
    %     \ket{\alpha_y} &= \frac{1}{\sqrt{2}}\vectwo{1}{i}, & \ket{\beta_y} &= \frac{1}{\sqrt{2}}\vectwo{1}{-i}\\
    %     \ket{\alpha_z} &= \vectwo{1}{0}, & \ket{\beta_z} &= \vectwo{0}{1}
    % \end{align*}
    % となる。

    また、運動量方向のスピン角運動量
        \[h = S \cdot \frac{p}{|p|} = \frac{\hbar}{2}\Sigma \cdot \frac{p}{|p|}\]
    をヘリシティという。$p/|p| = (\sin\theta\cos\phi, \sin\theta\sin\phi, \cos\theta)$に対して、ヘリシティの固有値$\pm \hbar/2$に属する固有状態は
    \begin{align*}
        \psi_\uparrow &= \vectwo{\cos\frac{\theta}{2}}{e^{i\phi}\sin\frac{\theta}{2}}\\
        \psi_\downarrow &= \vectwo{e^{-i\phi}\sin\frac{\theta}{2}}{-\cos\frac{\theta}{2}}
    \end{align*}
    として
    \begin{align*}
        \vectwo{\psi_+}{k\psi_+}, \quad \vectwo{\psi_-}{k\psi_-}
    \end{align*}
    となる。$\psi_\uparrow$のとき、スピン角運動量と運動量が平行であり右巻きと呼ばれる。$\psi_\downarrow$のとき、スピン角運動量と運動量が反平行であり左巻きと呼ばれる。

    また、
    \begin{align*}
        [H, h]
            &= \left[c\alpha \cdot p + \beta mc^2, \frac{\hbar}{2}\Sigma \cdot \frac{p}{|p|}\right]\\
    \end{align*}
    \begin{align*}
        [H, h]
            &= \left[H, S \cdot \frac{p}{|p|}\right]\\
            &= [H, S] \cdot \frac{p}{|p|}\\
            &= i\hbar c \sum_{i,j,k} \epsilon^{ijk}\alpha^jp^k\frac{p^i}{|p|}
        [\alpha \cdot p, \Sigma \cdot p]
            &= -i[\alpha^1p_x + \alpha^2p_y + \alpha^3p_z, \alpha^2\alpha^3p_x + \alpha^3\alpha^1p_y + \alpha^1\alpha^2p_z]\\
            &= 0\\
        [\beta, \Sigma \cdot p]
            &= -i[\beta, \alpha^2\alpha^3p_x + \alpha^3\alpha^1p_y + \alpha^1\alpha^2p_z]\\
            &= 0
    \end{align*}
    より自由粒子の場合ヘリシティは保存する。

        \[\gamma^5 = i\gamma^0\gamma^1\gamma^2\gamma^3\]
    をカイラリティという。

\subsection{非相対論的極限}
    波動関数を
        \[\psi(x) = \vectwo{\psi_+(x)}{\psi_-(x)} e^{-imc^2t/\hbar}\]
    とおくと、ディラック方程式は
    \begin{align*}
        i\hbar \pdv{t}\vectwo{\psi_+}{\psi_-} e^{-imc^2t/\hbar} + mc^2 \vectwo{\psi_+}{\psi_-}e^{-imc^2t/\hbar}
            &= \left[\begin{pmatrix} 0 & \sigma\\ \sigma & 0 \end{pmatrix} \cdot (pc - qcA) + \begin{pmatrix} \sigma^0 & 0\\ 0 & -\sigma^0 \end{pmatrix} mc^2 + q\phi\right] \vectwo{\psi_+}{\psi_-} e^{-imc^2t/\hbar}\\
            &= \vectwo{\sigma \cdot (pc - qcA)\psi_- + (+mc^2 + q\phi)\psi_+}{\sigma \cdot (pc - qcA)\psi_+ + (-mc^2 + q\phi)\psi_-} e^{-imc^2t/\hbar}\\
        i\hbar \pdv{t}\vectwo{\psi_+}{\psi_-} + mc^2 \vectwo{\psi_+}{\psi_-}
            &= \vectwo{\sigma \cdot (pc - qcA)\psi_- + (+mc^2 + q\phi)\psi_+}{\sigma \cdot (pc - qcA)\psi_+ + (-mc^2 + q\phi)\psi_-}
    \end{align*}
    成分ごとに書き下すと、
    \begin{align*}
        i\hbar\pdv{t}\psi_+ &= \sigma \cdot (pc - qcA)\psi_- + q\phi\psi_+\\
        \left(2mc^2 + i\hbar\pdv{t}\right)\psi_- &= \sigma \cdot (pc - qcA)\psi_+ + q\phi\psi_-
    \end{align*}
    非相対論的極限において運動エネルギー$i\hbar\pdv*{t}$と静電エネルギー$q\phi$は静止エネルギー$mc^2$に比べて無視できる。つまり
    \begin{align*}
        i\hbar\pdv{t}\psi_+
            &= \sigma \cdot (-i\hbar c\nabla - qcA)\psi_- + q\phi\psi_+\\
            &= \sigma \cdot (-i\hbar c\nabla - qcA) \frac{\sigma \cdot (-i\hbar c\nabla - qcA)}{2mc^2}\psi_+ + q\phi\psi_+\\
            &= \left[\frac{(\sigma \cdot (-i\hbar\nabla - qA))^2}{2m} + q\phi\right]\psi_+
    \end{align*}
    となり、パウリ方程式が導かれる。$\psi_+(x)$はワイルスピノルと呼ばれる。$(\sigma \cdot a)(\sigma \cdot b) = a \cdot b + i\sigma \cdot a \times b$を用いると、
    \begin{align*}
        (\sigma \cdot (-i\hbar\nabla - qA))^2
            &= (-i\hbar\nabla - qA)^2 + i\sigma \cdot (-i\hbar\nabla - qA) \times (-i\hbar\nabla - qA)\\
            &= (-i\hbar\nabla - qA)^2 + i\sigma \cdot (-\hbar^2\nabla \times \nabla + i\hbar\nabla \times qA + qA \times i\hbar\nabla)\\
            &= (-i\hbar\nabla - qA)^2 + i\sigma \cdot (i\hbar\nabla \times qA)\\
            &= (-i\hbar\nabla - qA)^2 - q\hbar \sigma \cdot B
    \end{align*}
    なので
        \[i\hbar\pdv{t}\psi_+ = \left[\frac{(-i\hbar\nabla - qA)^2}{2m} + q\phi - \frac{q\hbar}{2m}\sigma \cdot B\right]\psi+\]
    となる。第三項はパウリ項と呼ばれ、スピン角運動量と磁場の相互作用を表す。$\mu_B = q\hbar / 2m$は電子の磁気モーメントである。特に磁場が$z$成分しか持たないときは、
    \begin{align*}
        i\hbar\pdv{t}\psi_\uparrow &= \left[\frac{(-i\hbar\nabla - qA)^2}{2m} + q\phi - \frac{q\hbar}{2m}B_z\right]\psi_\uparrow\\
        i\hbar\pdv{t}\psi_\downarrow &= \left[\frac{(-i\hbar\nabla - qA)^2}{2m} + q\phi + \frac{q\hbar}{2m}B_z\right]\psi_\downarrow
    \end{align*}
    となる。

\subsection{球対称ポテンシャル}
    球対称ポテンシャル$V(r)$を考える。電荷を持たない粒子に対するハミルトニアンは$H = c\alpha \cdot p + \beta mc^2 + V(r)$である。
    \begin{align*}
        [H, L^i] &= -i\hbar c \sum_{j,k} \epsilon^{ijk}\alpha^jp^k\\
        [H, S^i] &= i\hbar c \sum_{j,k} \epsilon^{ijk}\alpha^jp^k\\
        [H, J^i] &= 0
    \end{align*}
    軌道角運動量とスピン角運動量は保存量ではないが、全角運動量は保存量となる。

    \begin{align*}
        L^2 &= (J - S)^2 = J^2 - 2J \cdot S + S^2\\
        L_z &= J_z - S_z\\
        L \cdot \sigma
            &= \sum_i L^i \cdot \sigma^i\\
            &= \begin{pmatrix}
                L_z & L_x - iL_y\\
                L_x + iL_y & -L_z
            \end{pmatrix}
            = \begin{pmatrix}
                L_z & L_-\\
                L_+ & -L_z
            \end{pmatrix}
    \end{align*}
        
    \begin{align*}
        L^2\psi
            &= (J^2 - 2J \cdot S + S^2)\psi\\
            &= j(j + 1)\hbar^2 - \begin{pmatrix}
                m\hbar & J_- & 0 & 0\\
                J_+ & -m\hbar & 0 & 0\\
                0 & 0 & m\hbar & J_-\\
                0 & 0 & J_+ & -m\hbar
            \end{pmatrix}\hbar\psi + \frac{3}{4}\hbar^2\psi\\
            &= \begin{pmatrix}
                j(j + 1) - m + 3/4 & J_- & 0 & 0\\
                J_+ & j(j + 1) + m + 3/4 & 0 & 0\\
                0 & 0 & j(j + 1) - m + 3/4 & J_-\\
                0 & 0 & J_+ & j(j + 1) + m + 3/4
            \end{pmatrix}\hbar^2\psi\\
            &= (J^2 - 2L \cdot S - S^2)\psi\\
            &= j(j + 1)\hbar^2 - \begin{pmatrix}
                (m - 1/2)\hbar & L_- & 0 & 0\\
                L_+ & -(m + 1/2)\hbar & 0 & 0\\
                0 & 0 & (m - 1/2)\hbar & L_-\\
                0 & 0 & L_+ & -(m + 1/2)\hbar
            \end{pmatrix}\hbar\psi - \frac{3}{4}\hbar^2\psi\\
        L_z\psi
            &= (J_z - S_z)\psi\\
            &= \left(m\hbar - \frac{\hbar}{2}\begin{pmatrix} \sigma^3 & 0\\ 0 & \sigma^3 \end{pmatrix}\right)\psi\\
            &= \begin{pmatrix}
                m - 1/2 & 0 & 0 & 0\\
                0 & m + 1/2 & 0 & 0\\
                0 & 0 & m - 1/2 & 0\\
                0 & 0 & 0 & m + 1/2
            \end{pmatrix}\hbar\psi
    \end{align*}
    より
    \begin{align*}
        \psi = \vecfour{Y_{l-1/2}^{m-1/2}(\theta, \phi)}{Y_{l-1/2}^{m+1/2}(\theta, \phi)}{Y_{l+1/2}^{m-1/2}(\theta, \phi)}{Y_{l+1/2}^{m+1/2}(\theta, \phi)}, \vecfour{Y_{l+1/2}^{m-1/2}(\theta, \phi)}{Y_{l+1/2}^{m+1/2}(\theta, \phi)}{Y_{l-1/2}^{m-1/2}(\theta, \phi)}{Y_{l-1/2}^{m+1/2}(\theta, \phi)}
    \end{align*}
    スピノル球関数
        \[\mathcal{Y}_{jm}^\pm(\theta, \phi) = \vectwo{a Y_{j \mp 1/2}^{m - 1/2}(\theta, \phi)}{b Y_{j \mp 1/2}^{m + 1/2}(\theta, \phi)}\]

\end{document}