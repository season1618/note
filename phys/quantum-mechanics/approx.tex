\section{近似法}

\subsection{時間に依存しない摂動論(縮退のない場合)}
    考えている系のシュレーディンガー方程式は厳密には解けないが、ハミルトニアンが厳密に解ける主要項と微小な摂動項の和で表される場合、エネルギー固有値や固有状態を漸近展開することで解の性質を調べることができる。このような近似の方法を摂動論と呼ぶ。

    時間に依存しないハミルトニアンを
    \begin{gather*}
        H\ket{\phi_n} = E_n\ket{\phi_n}\\
        H = H_0 + \lambda V
    \end{gather*}
    と置く。エネルギー固有値と固有状態を$\lambda$について漸近展開する。\footnote{$\lambda$によって冪級数展開できることは仮定に過ぎない。$\lambda$で冪級数展開できないような系も知られており、非摂動な系と呼ばれている。}
    \begin{align*}
        E_n &= \sum_{i=0}^\infty \lambda^i E_n^i\\
        \ket{\phi_n} &= \sum_{i=0}^\infty \lambda^i \ket{\phi_n^i}
    \end{align*}\footnote{$E_n^i, \ket{\phi_n^i}$の$i$は添え字であることに注意。}
    シュレーディンガー方程式に代入すると
    \begin{align*}
        (H_0 + \lambda V)\sum_{i=0}^\infty \lambda^i \ket{\phi_n^i} &= \(\sum_{i=0}^\infty \lambda^i E_n^i\)\(\sum_{i=0}^\infty \lambda^i \ket{\phi_n^i}\)\\
        \sum_{i=0}^\infty (\lambda^i H_0\ket{\phi_n^i} + \lambda^{i+1} V\ket{\phi_n^i}) &= \sum_{i=0}^\infty \sum_{j+k=i} \lambda^i E_n^j \ket{\phi_n^k}\\
        H_0\ket{\phi_n^0} + \sum_{i=1}^\infty \lambda^i (H_0\ket{\phi_n^i} + V\ket*{\phi_n^{i-1}}) &= \sum_{i=0}^\infty \sum_{j+k=i} \lambda^i E_n^j \ket{\phi_n^k}\\
    \end{align*}
    $\lambda$の0,1,2次の項について比較する。
    \begin{align*}
        H_0\ket{\phi_n^0} &= E_n^0\ket{\phi_n^0}\\
        H_0\ket{\phi_n^1} + V\ket{\phi_n^0} &= E_n^0\ket{\phi_n^1} + E_n^1\ket{\phi_n^0}\\
        H_0\ket{\phi_n^2} + V\ket{\phi_n^1} &= E_n^0\ket{\phi_n^2} + E_n^1\ket{\phi_n^1} + E_n^2\ket{\phi_n^0}
    \end{align*}
    0次の項の式は主要項のハミルトニアンの時間に依存しないシュレーディンガー方程式であり、固有値$E_n^0$と固有状態$\ket{\phi_n^0}$は既知と仮定する。

    一次摂動を考える。両辺に左から$\bra{\phi_m^0}$を掛けると、$\bra{\phi_m^0}H_0 = (H_0\ket{\phi_m^0})^\dagger = (E_m^0\ket{\phi_m^0})^\dagger = \bra{\phi_m^0}E_m^0$だから
    \begin{align*}
        E_m^0\braket{\phi_m^0}{\phi_n^1} + \mel{\phi_m^0}{V}{\phi_n^0} &= E_n^0\braket{\phi_m^0}{\phi_n^1} + E_n^1\delta_{mn}\\
        (E_n^0 - E_m^0)\braket{\phi_m^0}{\phi_n^1} &= \mel{\phi_m^0}{V}{\phi_n^0} - E_n^1\delta_{mn}
    \end{align*}
    $m = n$として
        \[E_n^1 = \ev{V}{\phi_n^0}\]
    エネルギー固有値は
    \begin{align*}
        E_n &\simeq E_n^0 + E_n^1\\
            &= \ev{H_0}{\phi_n^0} + \ev{V}{\phi_n^0}\\
            &= \ev{H}{\phi_n^0}
    \end{align*}
    となる。また$m \neq n$として
        \[\braket{\phi_m^0}{\phi_n^1} = \frac{\mel{\phi_m^0}{V}{\phi_n^0}}{E_n^0 - E_m^0}\]
    なので、$c_n^1$を定数として
        \[\ket{\phi_n^1} = c_n^1\ket{\phi_n^0} + \sum_{m \neq n} \frac{\mel{\phi_m^0}{V}{\phi_n^0}}{E_n^0 - E_m^0}\ket{\phi_m^0}\]
    となる。

    次に二次摂動を考える。同様に左から$\bra{\phi_m^0}$を掛ける。それぞれ
    \begin{align*}
        \mel{\phi_m^0}{H_0}{\phi_n^2} &+ \mel{\phi_m^0}{V}{\phi_n^1}\\
            &= \mel{\phi_m^0}{H_0}{\phi_n^2} + \bra{\phi_m^0}V \(c_n^1\ket{\phi_n^0} + \sum_{k \neq n} \frac{\mel{\phi_k^0}{V}{\phi_n^0}}{E_n^0 - E_k^0} \ket{\phi_k^0}\)\\
            &= E_m^0\braket{\phi_m^0}{\phi_n^2} + c_n^1\mel{\phi_m^0}{V}{\phi_n^0} + \sum_{k \neq n} \frac{\mel{\phi_k^0}{V}{\phi_n^0}}{E_n^0 - E_k^0} \mel{\phi_m^0}{V}{\phi_k^0}\\
        \mel{\phi_m^0}{E_n^0}{\phi_n^2} &+ \mel{\phi_m^0}{E_n^1}{\phi_n^1} + \mel{\phi_m^0}{E_n^2}{\phi_n^0}\\
            &= \mel{\phi_m^0}{E_n^0}{\phi_n^2} + \bra{\phi_m^0}E_n^1 \(c_n^1\ket{\phi_n^0} + \sum_{k \neq n} \frac{\mel{\phi_k^0}{V}{\phi_n^0}}{E_n^0 - E_k^0} \ket{\phi_k^0}\) + \mel{\phi_m^0}{E_n^2}{\phi_n^0}\\
            &= E_n^0\braket{\phi_m^0}{\phi_n^2} + c_n^1E_n^1\delta_{mn} + \sum_{k \neq n} \frac{\mel{\phi_k^0}{V}{\phi_n^0}}{E_n^0 - E_k^0} E_n^1\delta_{mk} + E_n^2\delta_{mn}
    \end{align*}
    よって
    \begin{align*}
        (E_n^0 - E_m^0)\braket{\phi_m^0}{\phi_n^2} &= c_n^1(\mel{\phi_m^0}{V}{\phi_n^0} - E_n^1\delta_{mn}) + \sum_{k \neq n} \frac{\mel{\phi_k^0}{V}{\phi_n^0}}{E_n^0 - E_k^0}(\mel{\phi_m^0}{V}{\phi_k^0} - E_n^1\delta_{mk}) - E_n^2\delta_{mn}
    \end{align*}
    $m = n$として
    \begin{align*}
        E_n^2 &= c_n^1(\mel{\phi_n^0}{V}{\phi_n^0} - E_n^1) + \sum_{k \neq n} \frac{\mel{\phi_k^0}{V}{\phi_n^0}}{E_n^0 - E_k^0}(\mel{\phi_n^0}{V}{\phi_k^0} - E_n^1\delta_{nk})\\
              &= \sum_{m \neq n} \frac{|\mel{\phi_m^0}{V}{\phi_n^0}|^2}{E_n^0 - E_m^0}
    \end{align*}
    基底状態$n = 1$に対しては$E_n^2 \leq 0$となるので常に引力として働く。また$m \neq n$として
        \[\braket{\phi_m^0}{\phi_n^2} = c_n^1\frac{\mel{\phi_m^0}{V}{\phi_n^0}}{E_n^0 - E_m^0} - E_n^1\frac{\mel{\phi_m^0}{V}{\phi_n^0}}{(E_n^0 - E_m^0)^2} + \sum_{k \neq n} \frac{\mel{\phi_m^0}{V}{\phi_k^0}\mel{\phi_k^0}{V}{\phi_n^0}}{(E_n^0 - E_m^0)(E_n^0 - E_k^0)}\]
    なので、$c_n^2$を定数として
        \[\ket{\phi_n^2} = c_n^2\ket{\phi_n^0} + \sum_{m \neq n}\left[c_n^1\frac{\mel{\phi_m^0}{V}{\phi_n^0}}{E_n^0 - E_m^0} - E_n^1\frac{\mel{\phi_m^0}{V}{\phi_n^0}}{(E_n^0 - E_m^0)^2} + \sum_{k \neq n} \frac{\mel{\phi_m^0}{V}{\phi_k^0}\mel{\phi_k^0}{V}{\phi_n^0}}{(E_n^0 - E_m^0)(E_n^0 - E_k^0)}\right]\ket{\phi_m^0}\]
    である。

\subsection{時間に依存しない摂動論(縮退のある場合)}
    $H_0$におけるエネルギー固有値$E_n^0$の固有状態$\ket{\phi_n}$が、摂動によってエネルギー固有値$E_{n,\alpha}\ (\alpha = 1, 2, \dots, \nu)$及び固有状態$\ket{\phi_{n,\alpha}}$に分裂したとする。つまり
    \begin{gather*}
        H\ket{\phi_{n,\alpha}} = E_{n,\alpha}\ket{\phi_{n,\alpha}}\\
        H = H_0 + \lambda V
    \end{gather*}
    である。エネルギー固有値と固有状態を$\lambda$について漸近展開する。
    \begin{align*}
        E_{n,\alpha} &= \sum_{i=0}^\infty \lambda^i E_{n,\alpha}^i\\
        \ket{\phi_{n,\alpha}} &= \sum_{i=0}^\infty \lambda^i \ket{\phi_{n,\alpha}^i}
    \end{align*}
    ここで$E_{n,\alpha}^0 = E_n^0$だが、縮退が解けた直後$\lambda \to 0$としたときの固有状態$\ket{\phi_{n,\alpha}^0}$は未知である。縮退のない場合と同様にシュレーディンガー方程式に代入する。
    \begin{align*}
        H_0\ket{\phi_{n,\alpha}^0}
            &= E_{n,\alpha}^0\ket{\phi_{n,\alpha}^0}\\
            &= E_n^0\ket{\phi_{n,\alpha}^0}\\
        H_0\ket{\phi_{n,\alpha}^1} + V\ket{\phi_{n,\alpha}^0}
            &= E_{n,\alpha}^0\ket{\phi_{n,\alpha}^1} + E_{n,\alpha}^1\ket{\phi_{n,\alpha}^0}\\
            &= E_n^0\ket{\phi_{n,\alpha}^1} + E_{n,\alpha}^1\ket{\phi_{n,\alpha}^0}
    \end{align*}
    0次の項の式は主要項のハミルトニアンの時間に依存しないシュレーディンガー方程式であり、$\ket{\phi_{n,\alpha}^0}$は$H_0$の固有状態である。

    一次摂動を考える。$H_0$の$E_n^0$に属する固有空間の正規直交基底$\{\ket{u_1}, \ket{u_2}, \dots, \ket{u_\nu}\}$を取ると、$\ket{\phi_{n,\alpha}^0}$は$\ket{u_j}$の線形結合で一意に書くことができて、
        \[\ket{\phi_{n,\alpha}^0} = \sum_{j=1}^\nu c_j\ket{u_j}\]
    とする。1次の項の式に代入して
        \[H_0\ket{\phi_{n,\alpha}^1} + V \sum_{j=1}^\nu c_j\ket{u_j} = E_n^0\ket{\phi_{n,\alpha}^1} + E_{n,\alpha}^1 \sum_{j=1}^\nu c_j\ket{u_j}\]
    左から$\bra{u_i}$を掛けると、$\bra{u_i}H_0 = \bra{u_i}E_n$なので
        \[\sum_{j=1}^\nu \mel{u_i}{V}{u_j}c_j = E_{n,\alpha}^1c_i\]
    となる。つまり、縮退が解けた後のエネルギー準位と、解けた直後の固有状態は、ポテンシャル$V$の表現行列の固有値と固有ベクトルに一致する。これを永年方程式という。

\paragraph{シュタルク効果}
    水素原子に$z$軸方向の一様静電場を掛ける。摂動項は
        \[V = -eEz = -eEr\cos\theta\]
    である。$n = 2$において基底$\{\ket{2,0,0}, \ket{2,1,0}, \ket{2,1,1}, \ket{2,1,-1}\}$を取ると$V$の表現行列は
    \[
        \begin{bmatrix}
            \mel{2,0,0}{V}{2,0,0} & \mel{2,0,0}{V}{2,1,0} & \mel{2,0,0}{V}{2,1,1} & \mel{2,0,0}{V}{2,1,-1}\\
            \mel{2,1,0}{V}{2,0,0} & \mel{2,1,0}{V}{2,1,0} & \mel{2,1,0}{V}{2,1,1} & \mel{2,1,0}{V}{2,1,-1}\\
            \mel{2,1,1}{V}{2,0,0} & \mel{2,1,1}{V}{2,1,0} & \mel{2,1,1}{V}{2,1,1} & \mel{2,1,1}{V}{2,1,-1}\\
            \mel{2,1,-1}{V}{2,1,-1} & \mel{2,1,-1}{V}{2,1,0} & \mel{2,1,-1}{V}{2,1,1} & \mel{2,1,-1}{V}{2,1,-1}
        \end{bmatrix}
        =
        \begin{bmatrix}
            0 & 3eEa_B & 0 & 0\\
            3eEa_B & 0 & 0 & 0\\
            0 & 0 & 0 & 0\\
            0 & 0 & 0 & 0
        \end{bmatrix}
    \]
    となる。したがって、固有値$0$に属する固有状態$c_1\ket{2,1,1} + c_2\ket{2,1,-1}\ (|c_1|^2 + |c_2|^2 = 1)$、固有値$3eEa_B$に属する固有状態$1/\sqrt{2}(\ket{2,0,0} + \ket{2,1,0})$、固有値$-3eEa_B$に属する固有状態$1/\sqrt{2}(\ket{2,0,0} - \ket{2,1,0})$が存在する。したがって一次摂動では、2s, 2p軌道は$E_2^0, E_2^0 - 3eEa_B, E_2^0 + 3eEa_B$の三つのエネルギー準位に分裂する。

    原子や分子に一様な電場をかけたときにエネルギー準位が分裂する現象をシュタルク効果と呼ぶ。

\subsection{時間に依存する摂動論}
    摂動項が時間に依存する場合を考える。
        \[H(t) = H_0 + \lambda V(t)\]
    $H_0$の固有値と固有状態を$E_n^0, \ket{\phi_n^0}$とする。$\{\ket{\phi_n^0}\}$は基底を成すので、時間に依存するシュレーディンガー方程式
        \[H(t)\ket{\psi(t)} = i\hbar\pdv{t}\ket{\psi(t)}\]
    の解は
        \[\ket{\psi(t)} = \sum_n c_n(t) \exp\(-i\frac{E_n^0}{\hbar}t\) \ket{\phi_n^0}\]
    と一意に展開できる。代入すると
    \begin{align*}
        (H_0 + \lambda V(t)) \sum_m c_m(t) \exp\(-i\frac{E_m^0}{\hbar}t\) \ket{\phi_m^0} &= i\hbar \pdv{t} \sum_m c_m(t) \exp\(-i\frac{E_m^0}{\hbar}t\)\ket{\phi_m^0}\\
        \sum_m (E_m^0 + \lambda V(t)) c_m(t) \exp\(-i\frac{E_m^0}{\hbar}t\) \ket{\phi_m^0} &= i\hbar \sum_m \(\dv{c_m}{t} - i\frac{E_m^0}{\hbar}c_m(t)\) \exp\(-i\frac{E_m^0}{\hbar}t\) \ket{\phi_m^0}\\
        i\hbar \sum_m \dv{c_m}{t} \exp\(-i\frac{E_m^0}{\hbar}t\) \ket{\phi_m^0} &= \lambda \sum_m c_m(t) \exp\(-i\frac{E_m^0}{\hbar}t\) V(t)\ket{\phi_m^0}
    \end{align*}
    左から$\bra{\phi_n^0}\exp\(i\frac{E_n^0}{\hbar}t\)$を掛けると
    \begin{align*}
        \dv{c_n}{t} &= \frac{\lambda}{i\hbar} \sum_m \exp\(-i\frac{E_m^0 - E_n^0}{\hbar}t\) \mel{\phi_n^0}{V(t)}{\phi_m^0} c_m(t)\\
    \end{align*}
    である。$c_n(t)$を$\lambda$によって漸近展開する。
        \[c_n(t) = \sum_{i=0}^\infty \lambda^i c_n^i(t)\]
    $c_n(t)$の微分方程式に代入すると
    \begin{align*}
        \dv{t} \sum_{i=0}^\infty \lambda^i c_n^i(t) &= \frac{\lambda}{i\hbar} \sum_m \exp\(-i\frac{E_m^0 - E_n^0}{\hbar}t\) \mel{\phi_n^0}{V(t)}{\phi_m^0} \sum_{i=0}^\infty \lambda^i c_m^i(t)\\
        \dv{c_n^0}{t} + \sum_{i=0}^\infty \lambda^{i+1} \dv{c_n^{i+1}}{t} &= \frac{1}{i\hbar} \sum_{i=0}^\infty \lambda^{i+1} \sum_m \exp\(-i\frac{E_m^0 - E_n^0}{\hbar}t\) \mel{\phi_n^0}{V(t)}{\phi_m^0} c_m^i(t)
    \end{align*}
    $\lambda$の次数で比較すれば
    \begin{align*}
        \dv{c_n^0}{t} &= 0\\
        \dv{c_n^{i+1}}{t} &= \frac{1}{i\hbar} \sum_m \exp\(-i\frac{E_m^0 - E_n^0}{\hbar}t\) \mel{\phi_n^0}{V(t)}{\phi_m^0} c_m^i(t)
    \end{align*}
    となる。$c_n^0$は一定であり、$\sum_n |c_n^0|^2 = 1$を満たす。

    摂動ポテンシャルとして
    \[
        V(t) =
        \begin{cases}
            0 & (t < 0)\\
            V & (t \geq 0)
        \end{cases}
    \]
    を考える。$t < 0$で状態が$\ket{\phi_m^0}$だったとして、時刻$t(\geq 0)$で状態$\ket{\phi_n^0}\ (E_n^0 \neq E_m^0)$が観測される確率、つまり$\ket{\phi_m^0}$から$\ket{\phi_n^0}$への遷移確率$P_n(t)$を求める。$c_k^0 = \delta_{km}$なので、
    \begin{align*}
        \dv{c_n^1}{t}
            &= \frac{1}{i\hbar} \sum_k \exp\(-i\frac{E_k^0 - E_n^0}{\hbar}t\) \mel{\phi_n^0}{V(t)}{\phi_k^0} c_k^0\\
            &= \frac{1}{i\hbar} \exp\(-i\frac{E_m^0 - E_n^0}{\hbar}t\) \mel{\phi_n^0}{V(t)}{\phi_m^0}\\
        c_n^1(t)
            &= \frac{1}{i\hbar} \left[c_n^1(0) + \int_0^t \exp\(-i\frac{E_m^0 - E_n^0}{\hbar}t'\) \mel{\phi_n^0}{V}{\phi_m^0} \dd{t'}\right]\\
            &= \left[\exp\(-i\frac{E_m^0 - E_n^0}{\hbar}t'\) \frac{\mel{\phi_n^0}{V}{\phi_m^0}}{E_m^0 - E_n^0}\right]_0^t\\
            &= \frac{\mel{\phi_n^0}{V}{\phi_m^0}}{E_m^0 - E_n^0} \left[\exp\(-i\frac{E_m^0 - E_n^0}{\hbar}t\) - 1\right]\\
        c_n(t)
            &\simeq c_n^0 + c_n^1(t)\\
            &= \frac{\mel{\phi_n^0}{V}{\phi_m^0}}{E_m^0 - E_n^0} \left[\exp\(-i\frac{E_m^0 - E_n^0}{\hbar}t\) - 1\right]
    \end{align*}
    である。$e^{i\theta} - 1 = e^{i\theta/2}(e^{i\theta/2} - e^{-i\theta/2}) = 2ie^{i\theta/2}\sin(\theta/2)$より遷移確率は、
    \begin{align*}
        P_n(t)
            &= |c_n(t)|^2\\
            &\simeq \(\frac{|\mel{\phi_n^0}{V}{\phi_m^0}|}{E_m^0 - E_n^0}\)^2 \cdot 4\sin^2\(\frac{E_m^0 - E_n^0}{2\hbar}t\)\\
            &= \frac{4|\mel{\phi_n^0}{V}{\phi_m^0}|^2}{(E_m^0 - E_n^0)^2} \sin^2\(\frac{E_m^0 - E_n^0}{2\hbar}t\)
    \end{align*}
    となる。エネルギー準位の差が小さいほど遷移確率は大きい。また単位時間当たりの遷移確率は
    \begin{align*}
        \lim_{t \to \infty} \frac{P_n(t)}{t} = \frac{2\pi}{\hbar} |\mel{\phi_n^0}{V}{\phi_m^0}|^2 \delta(E_m^0 - E_n^0)
    \end{align*}
    これをフェルミの黄金律と呼ぶ。

\subsection{変分法}
    \begin{thm}
        ハミルトニアン$H$とその基底エネルギー$E_0$について、任意の状態$\ket{\phi}$に対して、不等式
            \[\ev{H}{\phi} \geq E_0\]
        が成り立つ。等号成立は$\psi$が基底状態のときのみである。
    \end{thm}
    \begin{thm}
        ハミルトニアン$H$の固有状態は$\ev{H}[\phi] = \ev{H}{\phi}$の停留点である。
    \end{thm}
    上記の定理より、状態を適当なパラメータで表し停留点を求めることで、固有状態を近似的に求めることができる。これを変分法と呼ぶ。

\footnote{ブラケット記法でこのように書くと冗長だが、慣習的にこのように書く。}

\subsection{WKB近似}
    位相を$S$として波動関数を
        \[\phi(x) = e^{\frac{i}{\hbar}S(x)}\]
    と仮定する。
    \begin{align*}
        \dv[2]{\phi}{x}
            &= \dv{x}\(\frac{i}{\hbar}\dv{S}{x}e^{\frac{i}{\hbar}S(x)}\)\\
            &= \frac{i}{\hbar}\dv[2]{S}{x}e^{\frac{i}{\hbar}S(x)} - \frac{1}{\hbar^2}\(\dv{S}{x}\)^2e^{\frac{i}{\hbar}S(x)}
    \end{align*}
    より
    \begin{align*}
        \left[-\frac{\hbar^2}{2m}\dv[2]{x} + V(x)\right]\phi(x) &= E\phi(x)\\
        \frac{1}{2m}\left[-i\hbar\dv[2]{S}{x} + \(\dv{S}{x}\)^2\right] + V(x) &= E
    \end{align*}
    $\hbar \to 0$の極限でこの方程式はハミルトン=ヤコビ方程式に帰着し、$S(x)$は位相に対応する。この方程式は$\dv*{S}{x}$に関するリッカチの微分方程式である。
        \[S(x) = S_0(x) + \(\frac{\hbar}{i}\)S_1(x) + \(\frac{\hbar}{i}\)^2S_2(x) + \cdots\]
    と展開すると運動量$p(x) = \sqrt{2m(E - V(x))}$を用いて
    \begin{align*}
        S_0(x) &= \pm \frac{i}{\hbar}\int_0^x p(x')dx'\\
        S_1(x) &= \log \frac{1}{\sqrt{p(x)}} + C_1
    \end{align*}
    となる。0次は古典力学の解に対応する。$p(x)$を複素関数とみなし閉路に沿って積分すると
        \[\oint p \dd{x} \simeq S(x) = nh\]
    となりゾンマーフェルトの量子条件が導かれる。これをWKB(Wentzel Kramers Brillouin)近似という。