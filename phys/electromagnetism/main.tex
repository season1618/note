\documentclass{jsarticle}
\usepackage{amssymb,amsmath}
\usepackage{physics}

\newcommand{\de}[2][]{\frac{d #1}{d #2}}
\newcommand{\pd}[2][]{\frac{\partial #1}{\partial #2}}

\renewcommand{\epsilon}{\varepsilon}

\let\grad\relax
\let\div\relax
\DeclareMathOperator{\grad}{grad}
\DeclareMathOperator{\rot}{rot}
\DeclareMathOperator{\div}{div}

\let\Re\relax
\DeclareMathOperator{\Re}{Re}

\title{電磁気学}
\author{season07001674}
\date{\today}

\begin{document}
\maketitle
\tableofcontents

\input maxwell1.tex % 真空中のマクスウェル方程式
\input maxwell2.tex % 物質中のマクスウェル方程式
\input potential.tex % 電磁ポテンシャル
\input conserve.tex % 電磁場の保存量
\input wave1.tex % 真空中の電磁波
\input wave2.tex % 物質中の電磁波
\input optics.tex % 波動光学

\end{document}