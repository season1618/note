\documentclass{jsarticle}
\usepackage{amssymb,amsmath}
\usepackage{physics}

\renewcommand{\epsilon}{\varepsilon}

\let\grad\relax
\let\div\relax
\DeclareMathOperator{\grad}{grad}
\DeclareMathOperator{\rot}{rot}
\DeclareMathOperator{\div}{div}

\let\Re\relax
\DeclareMathOperator{\Re}{Re}

\title{電磁気学}
\author{season07001674}
\date{\today}

\begin{document}
\maketitle
\tableofcontents

\section{真空中のマクスウェル方程式}

電気に関するのクーロンの法則$F = Qq / 4\pi\epsilon_0r^2$から静電場$E = q / 4\pi\epsilon_0r^2$を定義する。電荷の間に働く力には線形性が成り立つと仮定する。即ち複数の電場の影響は重ね合わせで計算できる。

\subsection{ガウスの法則}
    電荷を覆う閉曲面$S$と電荷を中心とする単位球面を考える。$S$に垂直な電場の成分を積分すると
    \begin{align*}
        \int E \cdot n dS
            &= \frac{q}{4\pi\epsilon_0}\frac{1}{|r|^3}\int |r||n|\cos\theta dS'\\
            &= \frac{q}{4\pi\epsilon_0}\frac{1}{|r|^2}\int dS'\\
            &= \frac{q}{\epsilon_0}
    \end{align*}
    より閉曲面の形に依らない。電荷密度$\rho$を用いると$q = \int \rho dV$なのでガウスの発散定理より、
    \begin{align*}
        \int \div E dV &= \int E \cdot n dS = \frac{1}{\epsilon_0} \int \rho dV\\
        \div E &= \frac{\rho}{\epsilon_0}
    \end{align*}
    となる。これを電場に関するガウスの法則という。

\subsection{ファラデーの法則}
    静電場中を閉曲線$s$に沿って電荷を移動させることを考える。エネルギー保存則より$F = qE$の線積分は0となる。ストークスの定理より
    \begin{align*}
        \int \rot E \cdot ndS &= \oint E \cdot ds = 0\\
        \rot E &= 0
    \end{align*}
    ファラデーの電磁誘導の法則より、コイルに発生する起電力$V$はコイルを貫く磁束$\Phi$の変化に比例する。
        \[V = -\de[\Phi]{t}\]
    起電力は回路によって測られるが、近接作用の立場からコイルの有無に関わらず電場が存在していると考える。起電力は回路に沿って電場を線積分したものなので、
        \[\int E \cdot ds = -\pd{t} \int B \cdot dS\]
    ストークスの定理より
    \begin{align*}
        \int \rot E \cdot dS = \int E \cdot ds &= -\pd{t} \int B \cdot dS\\
        \int \left(\rot E + \pd[B]{t}\right) \cdot dS &= 0\\
        \rot E + \pd[B]{t} &= 0
    \end{align*}

\subsection{磁束保存の法則}
    ビオ=サバ―ルの法則
        \[dB = \frac{\mu_0}{4\pi}\frac{Ids \times r}{|r|^3}\]
    を拡張し
        \[B(r) = \frac{\mu_0}{4\pi} \int \frac{i(r') \times (r - r')}{|r - r'|^3} dr'\]
    とする。ここでベクトルポテンシャル
        \[A = \frac{\mu_0}{4\pi} \int \frac{i(r')}{|r - r'|}dr'\]
    を導入すると、$B = \rot A$と表せるので、
        \[\div B = \div\rot A = 0\]

\subsection{アンペールの法則}
    アンペールの法則
        \[\rot B = \mu_0i\]
    の両辺の$\div$をとると、
        \[\mu_0\div i = 0\]
    となる。これは電荷の保存則$\div i = -\pd[\rho]{t}$に反しているので、これを満たすように書き換える。$\div E = -\frac{\rho}{\epsilon_0}$より、アンペールの法則に付け加えると
        \[\rot B - \epsilon_0\mu_0\pd[E]{t} = \mu_0 i\]
    となる。左辺第二項を移行して
        \[\rot B = \mu_0\left(i + \epsilon_0\pd[E]{t}\right)\]
    と書いたとき、右辺第二項$\epsilon_0\pd[E]{t}$を電荷の移動を伴わないある種の電流と見なせる。これを変位電流と呼ぶ。

\subsection{ローレンツ力}
    平行に流れる二つの電流間には引力が働き、アンペールの力という。その大きさは
        \[|F| = \frac{\mu_0}{2\pi}\frac{I_1 I_2}{r}\]
    である。ここで、力は電流の間に直接働くのではなく、一方の電流が作った磁場によってもう一方の電荷が力を受けると解釈する。力の方向も考慮すれば
        \[dF = Ids \times B\]
    となる。$1\rm{m}$あたり$n$個の電荷があるとすれば、$I = qnv$なので
        \[dF = qnvds \times B\]
    $nds$は微小長さあたりの電荷の個数を表すので、電荷一つあたりの力は、
        \[F = qv \times B\]
    電場による力と合わせて
        \[F = q(E + v \times B)\]
    となる。これをローレンツ力という。

\subsection{物理量・単位・定数}
    電磁気学が発展した当時は$\mathrm{cgs}$単位系が用いられていた。後に$\mathrm{SI}$単位系が定義された。

    電気素量$e$及び電荷の単位$\mathrm{C}$(クーロン)は$e = 1.602176634 \time 10^{−19} \mathrm{C}$とすることにより厳密に定義される。電流を表す$\mathrm{A}$(アンペア)は、電磁気に関する唯一の$\mathrm{SI}$基本単位であり、ある断面積を1秒間に1クーロンの電荷が流れるときの電流と定義される。つまり$\mathrm{C} = \mathrm{A} \cdot \mathrm{s}$である。

    電場は$F = qE$より正の単位電荷を空間内に置いたときに受ける力とする。電束密度$D$はガウスの法則
        \[\oint D \cdot dS = q\]
    によって定義される。電位は静磁場において定義される量であり、ある点を規準として測られる。二点間の電位差はその間を単位電荷を移動させるときの仕事であり、静磁場において経路に依存しない。時間変化する磁場においては電位はスカラーポテンシャルに拡張される。電圧・起電力の単位である$\mathrm{V}$(ボルト)は、導体の二点間を1クーロンの電荷を運ぶときに必要な仕事が1ジュールであるときの、二点間の電圧である。$\mathrm{V} = \mathrm{J} / \mathrm{C} = \mathrm{m}^2 \cdot \mathrm{kg} \cdot \mathrm{s}^{-3} \cdot \mathrm{A}^{-1}$となる。

    磁束密度$B$はローレンツ力の式
        \[dF = Idl \times B\]
    によって試験電流の受ける力によって定義する。ある閉曲面$S$を貫く磁束$\Phi$は
        \[\Phi = \int_S B \cdot dS\]
    となる。磁束の単位$\mathrm{Wb}$(ウェーバ)はファラデーの電磁誘導の法則
        \[V = -\pd[\Phi]{t}\]
    に基づいて、1$\mathrm{V}$の誘導起電力を生じるのに必要な1$\mathrm{s}$あたりの磁束の変化量として定義される。磁場$H$はアンペールの法則
        \[\oint H \cdot dl = I\]
    を満たす量である。 % 真空中のマクスウェル方程式
\section{物質中の電磁場}

\subsection{電気双極子モーメント}
    大きさの等しい正と負の電荷が少しの距離を置いて存在している系を電気双極子と呼ぶ。電荷をそれぞれ$+q, -q$とする。負電荷の位置から正電荷の位置に引いたベクトルを$s$として、電気双極子モーメント$p$を
        \[p = qs\]
    と定義する。
    電気双極子が一様な電場$E$に置かれているとする。$s$と$E$のなす角が$\theta$であるときの電気双極子の位置エネルギーは$\theta = \pi / 2$のときを基準として
    \begin{align*}
        U   &= -\left[q|E|\frac{s}{2}\cos\theta + (-q|E|) \cdot -\frac{s}{2}\cos\theta\right]\\
            &= -p \cdot E
    \end{align*}
    電気双極子の作る電位を求める。電気双極子の中心を原点として、正電荷と負電荷をそれぞれ$(-s/2, 0, 0), (+s/2, 0, 0)$の位置に置く。$r = (x, y, z)$における電位は
    \begin{align*}
        U(r) &= \frac{q}{4\pi\epsilon_0}\(\frac{1}{r_1} - \frac{1}{r_2}\)\\
        r_1 &= \sqrt{r^2 + (s/2)^2 - rs\cos\theta}\\
        r_2 &= \sqrt{r^2 + (s/2)^2 + rs\cos\theta}
    \end{align*}
    ここで$s \ll r$として近似すると
    \begin{align*}
        \frac{1}{r_1}
            &= \frac{1}{\sqrt{r^2 + (s/2)^2 - rs\cos\theta}}\\
            &= \frac{1}{r}\(1 + \(\frac{s}{2r}\)^2 - \frac{s}{r}\cos\theta\)^{-1/2}\\
            &\simeq \frac{1}{r}\(1 - \frac{1}{2}\llr{\(\frac{s}{2r}\)^2 - \frac{s}{r}\cos\theta}\)\\
            &= \frac{1}{r}\(1 - \frac{s^2}{8r^2} + \frac{s}{2r}\cos\theta\)
    \end{align*}
    同様に
        \[\frac{1}{r_2} \simeq \frac{1}{r}\(1 - \frac{s^2}{8r^2} - \frac{s}{2r}\cos\theta\)\]
    だから
    \begin{align*}
        U(r)
            &\simeq \frac{q}{4\pi\epsilon_0}\frac{s}{r^2}\cos\theta\\
            &= \frac{p \cdot r}{4\pi\epsilon_0r^3}
    \end{align*}
    電場は
    \begin{align*}
        E_x
            &= -\pd{x}\(\frac{1}{4\pi\epsilon_0}\frac{p_x x + p_y y + p_z z}{(x^2+y^2+z^2)^{3/2}}\)\\
            &= -\frac{1}{4\pi\epsilon_0}\llr{\frac{p_x}{(x^2 + y^2 + z^2)^{3/2}} - \frac{3}{2}(p_x x + p_y y + p_z z)(x^2 + y^2 + z^2)^{-5/2}(2x)}\\
            &= -\frac{1}{4\pi\epsilon_0}\llr{\frac{p_x}{r^3} - \frac{3x(p \cdot r)}{r^5}}
    \end{align*}
    $y, z$も同様なので、結局
        \[E(r) = \frac{1}{4\pi\epsilon_0}\llr{\frac{p}{r^3} - \frac{3r(p\cdot r)}{r^5}}\]
    となる。

\subsection{磁気モーメント}
    モノポールが存在するとして、等しい大きさのNとSの磁荷を少しの距離を置いて配置したものを磁気双極子という。磁荷をそれぞれ$+q_m(N), -q_m(S)$とする。単位は$Wb$で、$F = q_m H$が成り立つ。S極からN極へ向かうベクトルを$s$として、磁気双極子モーメントと次のように定義する。
        \[m = q_m s\]
    電気双極子と同じように、エネルギーと磁場はそれぞれ、
    \begin{align*}
        U &= -m \cdot H\\
        H(r) &= -\frac{1}{4\pi\mu_0}\llr{\frac{m}{r^3} - \frac{3r(m\cdot r)}{r^5}}
    \end{align*}
    となる。

    ところが現代ではモノポールは存在しないとするのが主流なので、代わりに微小な円形電流を考える。この円形電流の作る磁場は十分遠方では磁気双極子が作るものと同じと見なせる。磁気モーメント$m'$を、
        \[U = -m' \cdot B\]
    が成り立つようにすると、$m' = m / \mu_0$とすれば良いことがわかる。すると磁束密度は、
        \[B(r) = -\frac{\mu_0}{4\pi}\llr{\frac{m'}{r^3} - \frac{3r(m'\cdot r)}{r^5}}\]
    となる。

\subsection{誘電体と電束密度}
    絶縁体(不導体)に静電場をかけると内部の電荷がその方向に偏る。これを分極という。絶縁体をそのように見たときこれを誘電体という。外部の電場によって誘電体が分極すると全体的に電荷が弱まる。そのとき単位面積あたりに通過した正電荷の量と方向を分極ベクトル$P$と呼ぶ。このとき積分形のガウスの法則は
        \[\int \epsilon_0 E \cdot dS = q - \int P \cdot dS\]
    となる。第二項は
        \[\int P \cdot dS = \int \div P dV\]
    と書き換えることができる。誘起された電荷密度$-\div P$を分極電荷という。電束密度を
        \[D = \epsilon_0 E + P\]
    と定義すると
    \begin{align*}
        \int (\epsilon_0 E + P) \cdot dS &= q\\
        \div D &= \rho
    \end{align*}
    となる。多くの物質では$P = \epsilon_0\chi E$という関係が近似的に成り立つ。$\chi$は電気感受率と呼ばれる。比較的弱い電場では電場に比例した分極を起こすが、大きい電場をかけると絶縁破壊が起きて電気が流れてしまう。$\epsilon = \epsilon_0(1 + \chi)$と置くと、$D = \epsilon E$と表せる。$\epsilon$を物質の誘電率という。常に$\chi > 0$なので$\epsilon > \epsilon_0$である。また$\epsilon / \epsilon_0$を比誘電率という。

\subsection{磁性体と磁場}
    永久磁石などの磁場の原因は電子が作る分子電流である。物質内部の分子電流は普段は別々の方向を向いているが、静磁場がかかると一つの向きに揃う。これを磁化という。物質を磁場に対する性質から見たとき、これを磁性体と呼ぶ。外部の磁場と同じ向きに磁場が誘起されるとき常磁性、逆向きのとき反磁性という。誘電体中で静磁場がかかると分子電流が向きを揃え、結果的に電流も変化する。このとき増加した磁気モーメントを磁化ベクトル$J$と呼ぶ。電流密度は
        \[\rot J = \mu_0i_m\]
    また、分極ベクトルの時間変化によっても電流が発生する。密度$\rho$の電荷が$u$だけ変位したとき、発生する電流は
    \begin{gather*}
        i_p = \rho\pd[u]{t} = \pd[P]{t}\\
        \rot B - \epsilon\mu_0\pd[E]{t} = \mu_0 (i + i_m)
    \end{gather*}
    $i_p$を分極電流と呼ぶ。磁場を
        \[H = \frac{1}{\mu_0}(B - J)\]
    と定義するとアンペールの法則は、
    \begin{align*}
        \frac{1}{\mu_0}\rot B - \epsilon_0 \pd[E]{t}
            &= i + i_m + i_p\\
            &= i + \frac{1}{\mu_0}\rot J + \pd[P]{t}\\
        \frac{1}{\mu_0}\rot(B - J) &- \pd[(\epsilon E + P)]{t} = i\\
        \rot H &- \pd[D]{t} = i
    \end{align*}
    となる。多くの物質では$J = \mu_0\chi_m H$という関係が近似的に成り立つ。$\chi_m$は磁化率と呼ばれる。磁場が一定以上強くなると分子電流の向きがすべて揃い、それ以上磁化ベクトルが大きくならないため、この比例関係は崩れる。$\mu = \mu_0(1 + \chi_m)$と置くと、$B = \mu H$と表せる。$\mu$を物質の透磁率という。磁性体の場合は$\mu$が正にも負にもなりうる。また$\mu / \mu_0$を比透磁率という。

\subsection{物質中のマクスウェル方程式}
    以上をまとめると物質中のマクスウェル方程式は次のようになる。
    \begin{align*}
        \div D &= \rho\\
        \div B &= 0\\
        \rot E &+ \pd[B]{t} = 0\\
        \rot H &- \pd[D]{t} =  i
    \end{align*} % 物質中のマクスウェル方程式
\section{電磁ポテンシャル}

\subsection{電磁ポテンシャル}
    $\div B = 0$だからポアンカレの補題より
        \[B = \rot A\]
    となるベクトル場$A$が存在する。これをベクトルポテンシャルという。ファラデーの法則を書き換えると
        \[\rot E + \pd[B]{t} = \rot\(E + \pd[A]{t}\) = 0\]
    である。再びポアンカレの補題より
        \[E + \pd[A]{t} = -\grad\phi\]
    となるスカラー場$\phi$が存在する。これをスカラーポテンシャルという。スカラーポテンシャル$\phi$とベクトルポテンシャル$A$のことを電磁ポテンシャルと呼ぶ。この二つを使って電場と磁場を
    \begin{align*}
        E &= -\grad \phi - \pd[A]{t}\\
        B &= \rot A
    \end{align*}
    と表せば、ファラデーの法則と磁束保存の法則は自動的に満たされることになる。

    電場と磁場が与えられたとき、それを満たす電磁ポテンシャルは一意には決まらない。微分可能な任意の関数$\chi$を用いて
    \begin{align*}
        \phi &\rightarrow \phi - \pd[\chi]{t}\\
        A &\rightarrow A + \grad\chi\\
    \end{align*}
    と変換しても電磁場の形は変わらない。この変換をゲージ変換という。電磁ポテンシャルを決定するために用いられる次の条件をそれぞれローレンツゲージ、クーロンゲージという。
    \begin{align*}
        \div A + \epsilon_0\mu_0\pd[\phi]{t} &= 0\\
        \div A &= 0\\
    \end{align*}

\subsection{電磁ポテンシャルによるマクスウェル方程式}
    残ったマクスウェル方程式
    \begin{align*}
        \div E = \frac{\rho}{\epsilon_0}\\
        \rot B - \epsilon_0\mu_0\pd[E]{t} = \mu_0 i\\
    \end{align*}
    を電磁ポテンシャルを用いて書き換える。
    \begin{align*}
        \div \(-\grad\phi - \pd[A]{t}\) &= \frac{\rho}{\epsilon_0}\\
        \Delta \phi + \pd{t}\div A &= -\frac{\rho}{\epsilon_0}\\
        \(\Delta - \epsilon_0\mu_0\pd[^2]{t^2}\)A &- \grad\(\div A + \epsilon_0\mu_0\pd[\phi]{t}\) = -\mu_0 i\\
    \end{align*}
    ここでローレンツゲージを仮定すると
    \begin{align*}
        \(\Delta-\epsilon_0\mu_0\pd[^2]{t^2}\)\phi &= -\frac{\rho}{\epsilon_0}\\
        \(\Delta-\epsilon_0\mu_0\pd[^2]{t^2}\)A &= -\mu_0 i\\
    \end{align*}
    となる。まとめると
    \begin{align*}
        \(\Delta-\epsilon_0\mu_0\pd[^2]{t^2}\)\phi &= -\frac{\rho}{\epsilon_0}\\
        \(\Delta-\epsilon_0\mu_0\pd[^2]{t^2}\)A &= -\mu_0 i\\
        \div A + \epsilon_0\mu_0\pd[\phi]{t} &= 0\\
    \end{align*}
    となる。これをローレンツゲージにおけるマクスウェル方程式という。

    また、先程の式でクーロンゲージを仮定すると
    \begin{align*}
        \Delta \phi &= -\frac{\rho}{\epsilon_0}\\
        -\epsilon_0\mu_0\pd{t}\grad\phi &+ \(\Delta - \epsilon_0\mu_0\pd[^2]{t^2}\)A = -\mu_0 i\\
    \end{align*}
    となる。これをクーロンゲージにおけるマクスウェル方程式という。

\subsection{荷電粒子のラグランジアン}
    ローレンツ力を電磁ポテンシャルを使って表すと
        \[F = q(E + v \times B) = q\(-\grad\phi - \pd[A]{t} + v \times \rot A\)\]
    である。これをオイラー・ラグランジュ方程式に合うように変形していく。
    \begin{align*}
        \nabla(v \cdot A) &= (v \cdot \nabla)A + v \times (\nabla \times A)\\
        \de[A]{t} &= \pd[A]{t} + (v \cdot \nabla)A\\
    \end{align*}
    より
    \begin{align*}
        F &= -q \mlr{\nabla(\phi - v\cdot A) + \de[A]{t}}\\
        F_i &= q\mlr{\de{t}\pd[(\phi - v\cdot A)]{v_i} - \pd[(\phi - v\cdot A)]{x_i}}\\
    \end{align*}
    したがって、荷電粒子のラグランジアンは、
        \[L = \frac{1}{2}mv^2 - q(\phi - v\cdot A)\]
    である。

\subsection{電磁場のラグランジアン} % 電磁ポテンシャル
\section{電磁場の保存量}

力学において、系の保存量は全て粒子に帰属するものだった。電磁気学においては、電磁場もエネルギーや運動量を持つと考える。基本的には、ローレンツ力による多数の荷電粒子の運動方程式をマクスウェル方程式を用いて書き換えていく。

\subsection{電磁場のエネルギー}
    \subsubsection{静電場のエネルギー}
        電荷は同種の電気が反発力に逆らって集まっているため、運動していなくてもエネルギーを持っている。場の理論において、ポテンシャルエネルギーは物質ではなく場に蓄えられていると考える。つまり、全エネルギーは物質の運動エネルギーと場のエネルギーの和である。

        半径$r$の球面上に電荷$q$が分布している状態を考える。電荷が0の状態から始めて無限遠から微小電荷を近づけると
        \begin{align*}
            U(r)
                &= \int_0^q \frac{q'}{4\pi\epsilon_0 r} dq'\\
                &= \frac{q^2}{8\pi\epsilon_0 r}
        \end{align*}
        となる。ところで球殻の外側の電場は半径に依らず一定である。よって、半径$r, r + dr$の帯電球のエネルギーの差を取れば球殻内のエネルギー密度$u$が分かる。
        \begin{align*}
            U(r + dr) - U(r) &= -u \cdot 4\pi r^2dr\\
            u &= -\frac{1}{4\pi r^2}\de[U]{r}\\
            &= \frac{1}{4\pi r^2}\frac{q^2}{8\pi\epsilon_0 r^2} = \frac{q^2}{32\pi^2\epsilon_0 r^4}\\
            &= \frac{1}{2}\epsilon_0 E^2
        \end{align*}

    \subsubsection{静磁場のエネルギー}
    \subsubsection{電磁場のエネルギー}
        運動方程式
            \[m_i\de[v_i]{t} = q_i(E(r_i) + v_i \times B(r_i))\]
        の両辺と$v_i$との内積を取る。
            \[\de{t}\(\frac{1}{2}m_iv_i^2\) = q_iv_i \cdot E(r_i)\]
        領域$V$内の全粒子について総和を取れば
        \begin{align*}
            \de{t}\(\sum_i \frac{1}{2}m_iv_i^2\)
                &= \sum_i q_iv_i \cdot E(r_i)\\
                &= \sum_i \int q_iv_i\delta(r' - r_i) \cdot E(r') dV\\
                &= \int i(r') \cdot E(r') dV
        \end{align*}
        電場$E$に距離をかけたものが電圧であり、電流密度$i$に面積をかけたものが電流だから、右辺はジュール熱を意味する。つまりジュール熱とは荷電粒子の運動エネルギーである。右辺をさらに変形すると
        \begin{align*}
            E \cdot i
                &= E \cdot \(\rot H - \pd[D]{t}\)\\
                &= H \cdot \rot E - \div(E \times H) - \pd[(E \cdot D)]{t}\\
                &= -\pd[(E \cdot D)]{t} - \pd[(H \cdot B)]{t} - \div(E \times H)
        \end{align*}
        エネルギー密度$u = (E \cdot D + H \cdot B) / 2$、$S = E \times H$とすれば、
            \[-\pd[u]{t} = E \cdot i + \div S\]
        これをポインティング(Poynting)の定理と呼ぶ。$S$はポインティングベクトルと呼ばれており、電磁場のエネルギー流の密度を表している。つまり、領域内のエネルギーの減少量は、荷電粒子の運動エネルギーと放出された電磁波のエネルギーの和に等しい。

\subsection{マクスウェルの応力}
    領域$V$内にある電荷がその外側から受ける力を考える。ローレンツ力の式を少し変えると$dF = \rho(r)E(r) + i(r) \times B(r) dV$となる。
        \[F = \int_V \rho(r)E(r) + i(r)\times B(r)dV\]
    ここで
    \begin{align*}
        \div D &= \rho\\
        \rot H &- \pd[D]{t} = i
    \end{align*}
    を代入すれば
    \begin{align*}
        F   &= \int \left[E \div D + \(\rot H - \pd[D]{t}\) \times B\right] dV\\
            &= \int \left[E \div D - B \times \rot H - \pd[D]{t} \times B\right] dV\\
            &= \int \(\epsilon_0E \div E - \frac{1}{\mu_0}B \times \rot B - \epsilon_0E \times \rot E - \epsilon_0\mu_0\de{t}(E \times H)\) dV
    \end{align*}
    $\div B = 0$なので、対称性を保つために$B \div B$という項を付け加える。
    \begin{align*}
        F   &= \int \left[\epsilon_0(E \div E - E \times \rot E) + \frac{1}{\mu_0}(B \div B - B \times \rot B)\right] dV\\
            &\quad - \frac{1}{c^2}\de{t} \int (E \times H) dV
    \end{align*}
    右辺第二項の積分の中身はポインティングベクトルである。マイナスの符号は電磁波が放出したことによる反作用を表している。ここで近接作用的な見方をするために、第一項を面積分に変換することを考える。つまり、積分の中身をあるベクトル場の発散として表すことができないか検討する。$x$成分だけを取り出せば
    \begin{align*}
        (E\div E - E \times \rot E)_x
            &= E_x\(\pd[E_x]{x} + \pd[E_y]{y} + \pd[E_z]{z}\) - E_y\(\pd[E_y]{x} - \pd[E_x]{y}\) + E_z\(\pd[E_x]{z} - \pd[E_z]{x}\)\\
            &= E_x\pd[E_x]{x} - E_y\pd[E_y]{x} - E_z\pd[E_z]{x}\\
            &\quad + E_x\pd[E_y]{y} + E_y\pd[E_x]{y}\\
            &\quad + E_x\pd[E_z]{z} + E_z\pd[E_x]{z}\\
            &= \frac{1}{2}\pd[E_x^2]{x} - \frac{1}{2}\pd[E_y^2]{x} - \frac{1}{2}\pd[E_z^2]{x}\\
            &\quad + \pd[(E_xE_y)]{y} + \pd[(E_xE_z)]{z}\\
            &= \pd[(E_x^2 - \frac{1}{2}E^2)]{x} + \pd[(E_xE_y)]{y} + \pd[(E_xE_z)]{z}
    \end{align*}
    従って領域内の力は
    \begin{align*}
        T &= T_e + T_m\\
        T_e &= \epsilon_0
        \begin{bmatrix}
            E_x^2 - \frac{1}{2}E^2 & E_xE_y & E_xE_z\\
            E_yE_x & E_y^2 - \frac{1}{2}E^2 & E_yE_z\\
            E_zE_x & E_zE_y & E_z^2 - \frac{1}{2}E^2\\
        \end{bmatrix}\\
        T_m &= \frac{1}{\mu_0}
        \begin{bmatrix}
            B_x^2 - \frac{1}{2}B^2 & B_xB_y & B_xB_z\\
            B_yB_x & B_y^2 - \frac{1}{2}B^2 & B_yB_z\\
            B_zB_x & B_zB_y & B_z^2 - \frac{1}{2}B^2\\
        \end{bmatrix}
    \end{align*}
    として
    \begin{align*}
        F   &= \int \div T dV - \frac{1}{c^2}\de{t}\int S dV\\
            &= \int T \cdot dS - \frac{1}{c^2}\de{t}\int S dV
    \end{align*}
    と表される。この$T$をマクスウェルの応力テンソルという。

    電場のみの状態を考えたとき、微小面$dS$に試験電荷を置いたときにかかる力のベクトルが$T_e \cdot dS$である。$T_e$は対称テンソルであり、直交する固有ベクトルを持つ。電場$E$は当然その固有ベクトルの一つである。電場方向には引っ張り合う力が働き、電場の垂直方向には圧縮する力が働く。

\subsection{電磁場の運動量}
    多粒子系の運動方程式
        \[\sum_i m_i\dd[r_i]{t} = \sum_i \int \left[q_i\delta(r' - r_i)E(r') + q_i\delta(r' - r_i)\de[r_i]{t} \times B(r')\right] dV\]
    に
    \begin{align*}
        \div D &= \sum_i q_i\delta(r' - r_i)\\
        \rot H - \pd[D]{t} &= \sum_i q_i\delta(r' - r_i)\de[r_i]{t}\\
    \end{align*}
    を代入すると、右辺は領域$V$内に働く電磁場による力と等しいので、マクスウェルの応力を用いると
    \begin{align*}
        \sum_i m_i\dd[r_i]{t} = \int T \cdot dS - \frac{1}{c^2}\de{t}\int (E \times H) dV\\
        \de{t}\left[\sum_i m_i\de[r_i]{t} + \frac{1}{c^2}\int S dV\right] = \int T \cdot dS
    \end{align*}
    となる。左辺第二項が電磁場の運動量であると解釈できる。

\subsection{電磁場の角運動量}
    \begin{align*}
        \de{t}\left[\sum_i r_i \times m_i\de[r_i]{t}\right]
            &= \sum_i r_i \times m_i\dd[r_i]{t}\\
            &= \sum_i r_i \times q_i\(E(r_i) + \de[r_i]{t} \times B(r_i)\)\\
            &= \sum_i \int q_i\delta(r' - r_i)r' \times \(E(r') + \de[r_i]{t} \times B(r')\) dV\\
            &= \int r' \times \(\div T - \frac{1}{c^2}\de{t}(E \times H)\) dV
    \end{align*}
    ここで
    \begin{align*}
        (r' \times \div T)_x
            &= \left\{(x, y, z) \times (\div T_x, \div T_y, \div T_z)\right\}_x\\
            &= y\div T_z - z\div T_y\\
            &= \div (yT_z) - \div T_{zy} - \div (zT_y) + \div T_{yz}
    \end{align*}
    $T$は対称テンソルなので相殺される。つまり
        \[(r' \times \div T)_x = \div (yT_z - zT_y)\]
    である。
        \[r' \times T = (yT_z - zT_y, zT_x - xT_z, xT_y - yT_x)\]
    と定義すれば
    \begin{align*}
        \de{t}\left[\sum_i r_i \times m_i\de[r_i]{t}\right]
            &= \int \div(r' \times T) dV - \frac{1}{c^2}\de{t}\int r' \times S dV\\
            &= \int r' \times T \cdot dS - \frac{1}{c^2}\de{t}\int r' \times S dV
    \end{align*}
    従って
        \[\de{t}\left[\sum_i r_i \times m_i\de[r_i]{t} + \frac{1}{c^2}\int r' \times S dV\right] = \int r' \times T \cdot dS\]
    となる。左辺第二項が電磁場の角運動量である。
 % 電磁場の保存量
\section{真空中の電磁波}

\subsection{波動方程式}
    真空中において物質が存在しないとする。電荷密度と電流密度は0となるので
    \begin{align*}
        \div E &= 0\\
        \div B &= 0\\
        \rot E &+ \pd[B]{t} = 0\\
        \rot B &- \epsilon_0\mu_0\pd[E]{t} = 0
    \end{align*}
    第3式の両辺の$\rot$を計算すると
    \begin{align*}
        \rot\rot E + \pd{t}(\rot B) &= 0\\
        \grad\div E - \Delta E + \pd{t}\left(\epsilon_0\mu_0\pd[E]{t}\right) &= 0\\
        \left(-\epsilon_0\mu_0\pdv[2]{t} + \Delta\right)E &= 0
    \end{align*}
    第4式も同様に
        \[\left(-\epsilon_0\mu_0\pdv[2]{t} + \Delta\right)B = 0\]
    となる。この形の微分方程式は波動方程式と呼ばれる。電磁場が空間を伝わっていく現象を電磁波という。特に真空中の光速は
        \[c = \frac{1}{\sqrt{\epsilon_0\mu_0}}\]
    であり、$c$と表す。この値が当時の光速の測定値と一致したことは、光が電磁波であることの有力な証拠となった。物質中の場合は、電磁場によって電荷密度の偏りや電流密度が発生することを無視すれば、真空中と同じ形の方程式になる。物質中の光速は
        \[v = \frac{1}{\sqrt{\epsilon\mu}}\]
    である。物質中の光速は真空中の光速より遅くなり、それらの比
        \[n = \frac{c}{v} = \frac{\sqrt{\epsilon\mu}}{\sqrt{\epsilon_0\mu_0}}\]
    を(絶対)屈折率という。

\subsection{平面波}
    波数$k$、振動数$\omega$の平面波は
        \[E(r, t) = f(k \cdot r - \omega t)\]
    と表される。速さは$\omega / |k|$である。ガウスの法則に代入すると
    \begin{align*}
        \div E
        &= \div f(k \cdot r - \omega t)\\
        &= k_xf'_x + k_yf'_y + k_zf'_z\\
        &= k \cdot f' = 0
    \end{align*}
    従って電場は電磁波の進行方向に対して変化しない。つまり電磁波の平面波解は横波である。
        \[(E_x, E_y, E_z) = (f_x(k \cdot r - \omega t), f_y(k \cdot r - \omega t), f_z(k \cdot r - \omega t))\]
    をファラデーの法則に代入する。
        \[-\pd[B_x]{t} = k_yf_z'(k \cdot r - \omega t) - k_zf_y'(k \cdot r - \omega t)\]
    だから
        \[\left(\pd[B_x]{t}, \pd[B_y]{t}, \pd[B_z]{t}\right) = -k \times f'(k \cdot r - \omega t)\]
    となる。この両辺を積分する。積分定数は静磁場を意味するので省略すると
        \[(B_x, B_y, B_z) = \frac{k}{\omega} \times f(k \cdot r - \omega t)\]
    つまり電場と磁場は互いに直交する方向に振動し、全く同じ形で伝わっていくことになる。

    % \begin{align*}
    %     E(r, t) &= E_0(r)\exp(i(k \cdot r - \omega t))\\
    %     B(r, t) &= B_0(r)\exp(i(k \cdot r - \omega t))\\
    % \end{align*}
    % という形の解を考える。$E_0, B_0$は複素数を成分に持つベクトル場である。物質中のマクスウェル方程式
    % \begin{align*}
    %     \div E &= 0\\
    %     \div B &= 0\\
    %     \rot E &+ \pd[B]{t} = 0\\
    %     \rot B &- \epsilon(\omega)\mu(\omega)\pd[E]{t} = 0\\
    % \end{align*}
    % に代入する。
    % \begin{align*}
    %     \pd[E]{x} &= \pd[E_0]{x}\exp(i(k \cdot r - \omega t)) + ik_xE_0\exp(i(k \cdot r - \omega t))\\
    %     \pd[E]{t} &= -i\omega E_0\exp(i(k \cdot r - \omega t))\\
    % \end{align*}
    % となる。$y, z$或いは$B$についても同様であり
    %     \[\div E_0 = \div B_0 = 0\]
    % となる。
    % \begin{align*}
    %     \div E &= \div E_0 \exp(i(k \cdot r - \omega t)) + ik \cdot E_0\exp(i(k \cdot r - \omega t)) = 0\\
    %     \div E_0 &= 0\, k \cdot E_0 = 0\\
    % \end{align*}
    % \begin{align*}
    %     \rot E &= \rot E_0 \exp(i(k \cdot r - \omega t)) + ik \times E_0\exp(i(k \cdot r - \omega t)) - i\omega B_0\exp(i(k \cdot r - \omega t)) = 0\\
    %     \rot E_0 = 0\, k \times E_0 = \omega B_0\\
    % \end{align*}

\subsection{球面波}
    球対称な波動を考える。球座標におけるラプラシアンの角度に依存しない部分は
        \[\Delta f = \frac{1}{r}\pd[^2]{r^2}(rf)\]
    で表される。
    \begin{align*}
        \frac{1}{r}\pd[^2]{r^2}(rE) &= \frac{1}{c^2}\pd[^2E]{t^2}\\
        \pd[^2]{r^2}(rE) &= \frac{1}{c^2}\pd[^2]{t^2}(rE)
    \end{align*}
    これは$rE$に関する一次元波動方程式である。
        \[E(r, t) = \frac{1}{|r|}{f(k|r| - \omega t)}\]
    とおくと、この式は原点から拡散または収束するような波を表している。振幅は原点からの距離に反比例する。ガウスの法則より
    \begin{gather*}
        \div E = -\frac{x + y + z}{(x^2 + y^2 + z^2)^{3/2}}f(k|r| - \omega t) + \frac{1}{|r|}\frac{k(x + y + z)}{(x^2 + y^2 + z^2)^{1/2}}f'(k|r| - \omega t) = 0\\
        \frac{1}{|r|}f(k|r| - \omega t) = kf'(k|r| - \omega t)
    \end{gather*}
    電場と磁場は垂直だが、そのどちらも動径方向と垂直になるとは限らない。$r$と電場が垂直なものをTE球面波(transverse electric wave)、磁場と垂直なものをTM球面波(transverse magnetic wave)という。

\subsection{偏光}
    電磁波の平面波解において、電場と磁場の振動方向は進行方向に垂直な面内で2次元の自由度を持っている。これを偏光という。$k$に垂直な直交する単位ベクトル$e_1, e_2$を用いて$E_0 = E_1e_1 + E_2e_2$とおく。$E_1, E_2$の偏角をそれぞれ$\alpha_1, \alpha_2$とすると、それぞれの成分は
    \begin{align*}
        E(r, t)
        &= (E_1e_1 + E_2e_2)\exp(i(k \cdot r - \omega t))\\
        &= |E_1|\exp(i(k \cdot r - \omega t + \alpha_1))e_1 + |E_2|\exp(i(k \cdot r - \omega t + \alpha_2))\\
        &= |E_1|\cos(\theta + \alpha_1)e_1 + |E_2|\cos(\theta + \alpha_2)e_2
    \end{align*}
    より
    \begin{align*}
        E_x = |E_1|(\cos\alpha_1\cos\theta - \sin\alpha_1\sin\theta)\\
        E_y = |E_2|(\cos\alpha_2\cos\theta - \sin\alpha_2\sin\theta)
    \end{align*}
    $\cos\theta, \sin\theta$について解くと
    \begin{align*}
        \cos\theta = \frac{E_x}{|E_1|}\sin\alpha_2 - \frac{E_y}{|E_2|}\sin\alpha_1\frac{1}{\sin(\alpha_2 - \alpha_1)}\\
        \sin\theta = \frac{E_x}{|E_1|}\cos\alpha_2 - \frac{E_y}{|E_2|}\cos\alpha_1\frac{1}{\sin(\alpha_2 - \alpha_1)}
    \end{align*}
    となる。
        \[\cos^2\theta + \sin^2\theta = \frac{\frac{E_x^2}{|E_1|^2} - \frac{2E_xE_y}{|E_1E_2|}\cos(\alpha_2 - \alpha_1) + \frac{E_y^2}{|E_2|^2}}{\sin^2(\alpha_2 - \alpha_1)}\]
    位相差を$\delta = \alpha_2 - \alpha_1$とおくと
        \[\frac{E_x^2}{|E_1|^2} - \frac{2E_xE_y}{|E_1E_2|}\cos\delta + \frac{E_y^2}{|E_2|^2} = \sin^2\delta\]
    となる。判別式は
        \[\frac{1}{|E_1|^2}\frac{1}{|E_2|^2} - \frac{\cos^2\delta}{|E_1E_2|^2} = \frac{\sin^2\delta}{|E_1E_2|^2} > 0\]
    だから、電場の振動は楕円を描く。これを楕円偏光という。$E_1, E_2$の絶対値が等しく、直交するとき円偏光、$E_1, E_2$の位相差が$\pi$の整数倍であるとき直線偏光という。
    \begin{align*}
        E^+ = \frac{E_1 - iE_2}{2}, \quad E^- = \frac{E_1 + iE_2}{2}
    \end{align*}
    とおくと
    \begin{align*}
        E(r, t)
        &= ((E^+ + E^-)e_1 + i(E^+ - E^-)e_2)\exp(i(k \cdot r - \omega t))\\
        &= (E^+e_1 + iE^+e_2)\exp(i(k \cdot r - \omega t)) + (E^-e_1 - iE^-e_2)\exp(i(k \cdot r - \omega t))
    \end{align*}
    となるので、二つの円偏光に分解することができる。第一項が左偏光、第二項が右偏光に対応する。

\subsection{遅延ポテンシャルと先進ポテンシャル}
    電磁場の原因が電荷密度$\rho$と電流密度$i$のみで、かつそれらが自己場の影響を受けないとした場合に、ローレンツゲージにおけるマクスウェル方程式の解を求める。
    \begin{align*}
        \left(-\frac{1}{c^2}\pd[^2]{t^2} + \Delta\right)\phi &= -\frac{\rho}{\epsilon_0}\\
        \left(-\frac{1}{c^2}\pd[^2]{t^2} + \Delta\right)A &= -\mu_0 i
    \end{align*}
    二つの解があったとき、それらの差は波動方程式の解なので、特殊解に任意の電磁波を重ねたものとなる。

    まず両辺をフーリエ変換する。
    \begin{align*}
        \phi(r, t) &= \frac{1}{2\pi}\int_{-\infty}^{\infty} \phi(r, \omega)e^{i\omega t}d\omega\\
        \rho(r, t) &= \frac{1}{2\pi}\int_{-\infty}^{\infty} \rho(r, \omega)e^{i\omega t}d\omega
    \end{align*}
    として
    \begin{align*}
        \frac{\omega^2}{c^2}\phi(r, \omega)e^{i\omega t} + \Delta \phi(r, \omega)e^{i\omega t} &= -\frac{\rho(r, \omega)}{\epsilon_0}e^{i\omega t}\\
        \left(\Delta + \frac{\omega^2}{c^2}\right)\phi(r, \omega) &= -\frac{\rho(r, \omega)}{\epsilon_0}
    \end{align*}
    この解は
        \[\phi(r, \omega) = \frac{1}{4\pi\epsilon_0}\int \frac{Ae^{i\omega|r - r'|/c} + Be^{i\omega|r - r'|/c}}{|r - r'|}\rho(r, \omega)dr'^3\]
    である。これを逆フーリエ変換して
    \begin{align*}
        \phi(r, t)
        &= \frac{1}{2\pi}\int_{-\infty}^{\infty} \frac{1}{4\pi\epsilon_0}\int \frac{Ae^{i\omega|r - r'|/c} + Be^{-i\omega|r - r'|/c}}{|r - r'|}\rho(r, \omega)dr'^3 e^{i\omega t}d\omega\\
        &= \frac{1}{4\pi\epsilon_0}\int \frac{1}{2\pi}\int_{-\infty}^{\infty} \frac{Ae^{i\omega(t + |r - r'|/c)} + Be^{i\omega(t - |r - r'|/c)}}{|r - r'|}\rho(r, \omega) d\omega dr'^3\\
        &= \frac{1}{4\pi\epsilon_0}\int \frac{A\rho(r', t + |r - r'|/c) + B\rho(r', t - |r - r'|/c)}{|r - r'|} dr'^3
    \end{align*}
    第一項を先進ポテンシャル、第二項を遅延ポテンシャルという。ベクトルポテンシャルについても同様なので、まとめると
    \begin{align*}
        \phi(r, t) &= \frac{1}{4\pi\epsilon_0}\int \frac{\rho(r', t - |r - r'|/c)}{|r - r'|} dr'^3\\
        A(r, t) &= \frac{\mu_0}{4\pi}\int \frac{i(r', t - |r - r'|/c)}{|r - r'|} dr'^3
    \end{align*}
    である。遅延ポテンシャルから電磁場を求めたものはジェフィメンコ方程式と呼ばれている。
    % 第一項は近接項で、静電場と同じ形をしている。第二項は放射項で、電磁波による効果を表している。ジェフィメンコ方程式は電荷の保存則を満たすとは限らない。

\subsection{リエナール=ヴィーヘルト・ポテンシャル}
    遅延ポテンシャルから運動する点電荷の作る電磁場を求める。点電荷の軌道を$s(t)$とすると、電荷密度と電流密度はそれぞれ次のように表せる。
    \begin{align*}
        \rho(r, t) &= q\delta(r - s(t))\\
        i(r, t) &= q\dot{s}(t)\delta(r - s(t))
    \end{align*}
    これを遅延ポテンシャルに代入する。
    \begin{align*}
        \phi(r, t)
        &= \frac{1}{4\pi\epsilon_0}\int \frac{q\delta(r' - s(t - |r - r'|/c))}{|r - r'|} dr'^3\\
        &= \frac{1}{4\pi\epsilon_0}\int\int \frac{q\delta(r' - s(t'))\delta(t' - (t - |r - r'|/c))}{|r - r'|} dt'dr'^3\\
        &= \frac{1}{4\pi\epsilon_0}\int\int \frac{q\delta(r' - s(t'))\delta(t' - (t - |r - r'|/c))}{|r - r'|} dr'^3dt'\\
        &= \frac{1}{4\pi\epsilon_0}\int \frac{q\delta(t' - (t - |r - s(t')|/c))}{|r - s(t')|} dt'
    \end{align*}
    ここで
        \[f(t') = t' - \left(t - \frac{|r - s(t')|}{c}\right)\]
    とおくと
        \[f'(t') = 1 + \frac{(r - s(t')) \cdot -\dot{s}(t')}{c|r - s(t')|} = 1 - \frac{(r - s(t')) \cdot \dot{s}(t')}{c|r - s(t')|}\]
    である。$f(t_r) = 0$として積分を計算すれば
    \begin{align*}
        \phi(r, t)
        &= \frac{q}{4\pi\epsilon_0}\frac{1}{|r - s(t_r)|(1 + (r - s(t_r)) \cdot \dot{s}(t_r) / (c|r - s(t_r)|))}\\
        &= \frac{q}{4\pi\epsilon_0}\frac{1}{|r - s(t_r)| + (r - s(t_r)) \cdot \dot{s}(t_r) / c}
    \end{align*}
    $A(r, t)$についても同様となる。まとめると
    \begin{align*}
        \phi(r, t) &= \frac{q}{4\pi\epsilon_0}\frac{1}{|r - s(t_r)| + (r - s(t_r)) \cdot \dot{s}(t_r) / c}\\
        A(r, t) &= \frac{\mu_0q}{4\pi}\frac{\dot{s}(t_r)}{|r - s(t_r)| + (r - s(t_r)) \cdot \dot{s}(t_r) / c}
    \end{align*}
    これをリエナール=ヴィーヘルト・ポテンシャルという。電磁場を求めると
    \begin{align*}
        E(r, t) &= \frac{q}{4\pi\epsilon_0}\left[\frac{(1 - v^2/c^2)(R/|R| - v/c)}{R^2b^3} + \frac{R \times ((R/|R| - v/c) \times a)}{R^2b^3c^2}\right]\\
        b &= 1 - \frac{R \cdot v}{|R|c}
    \end{align*}
    である。

\subsection{等速運動する点電荷}
    リエナール=ヴィーヘルト・ポテンシャルから等速運動する点電荷の作る電磁ポテンシャルを求める。点電荷の軌道を$s(t) = (vt, 0, 0)$とすると
    \begin{align*}
        t - t_r &= \frac{\sqrt{(x - vt_r)^2 + y^2 + z^2}}{c}\\
        c^2(t - t_r)^2 &= (x - vt_r)^2 + y^2 + z^2\\
        (c^2 - v^2)t_r^2 - 2(c^2t - xv)t_r &= x^2 + y^2 + z^2 - c^2t^2
    \end{align*}
    だから
    \begin{align*}
        \phi(r, t)
        &= \frac{q}{4\pi\epsilon_0}\frac{1}{c(t - t_r) - v/c \cdot (x - vt_r)}\\
        &= \frac{q}{4\pi\epsilon_0}\frac{1}{(-c + v^2/c)t_r + ct - vx/c}\\
        A(r, t)
        &= \frac{\mu_0q}{4\pi}\frac{(v, 0, 0)}{(-c + v^2/c)t_r + ct - vx/c}
    \end{align*}
    ここで
    \begin{align*}
        \left\{\left(-c + \frac{v^2}{c}\right)t_r + ct - \frac{vx}{c}\right\}^2
        &= c^2\left\{-\left(1 - \frac{v^2}{c^2}\right)t_r + t - \frac{vx}{c^2}\right\}^2\\
        &= c^2\left\{\left(1 - \frac{v^2}{c^2}\right)^2t_r^2 - 2\left(1 - \frac{v^2}{c^2}\right)\left(t - \frac{vx}{c^2}\right)t_r + \left(t - \frac{vx}{c^2}\right)^2\right\}\\
        &= \left(1 - \frac{v^2}{c^2}\right)(x^2 + y^2 + z^2 - c^2t^2) + \left(ct - \frac{vx}{c}\right)^2\\
        &= x^2 - 2xvt + v^2t^2 + \left(1 - \frac{v^2}{c^2}\right)(y^2 + z^2)\\
        &= (x - vt)^2 + \left(1 - \frac{v^2}{c^2}\right)(y^2 + z^2)
    \end{align*}
    なので
    \begin{align*}
        \phi(r, t) &= \frac{q}{4\pi\epsilon_0}\frac{1}{\sqrt{(x - vt)^2 + \left(1 - \frac{v^2}{c^2}\right)(y^2 + z^2)}}\\
        A(r, t) &= \frac{\mu_0q}{4\pi}\frac{(v, 0, 0)}{\sqrt{(x - vt)^2 + \left(1 - \frac{v^2}{c^2}\right)(y^2 + z^2)}}
    \end{align*}
    となる。点電荷から見たとき、等ポテンシャル面は$yz$方向に引き伸ばされた回転楕円体であり、形を保ったまま点電荷に纏った状態で移動していく。つまり等速運動はエネルギーを必要としない。電場を計算すると
    \begin{align*}
        E(r, t)
        &= -\grad\phi - \pd[A]{t}\\
        &= \frac{q}{4\pi\epsilon_0}\frac{(x - vt, \left(1 - \frac{v^2}{c^2}\right)y, \left(1 - \frac{v^2}{c^2}\right)z)}{\{(x - vt)^2 + \left(1 - \frac{v^2}{c^2}\right)(y^2 + z^2)\}^{3/2}} - \frac{\mu_0q}{4\pi}\frac{(v^2(x - vt), 0, 0)}{\{(x - vt)^2 + \left(1 - \frac{v^2}{c^2}\right)(y^2 + z^2)\}^{3/2}}\\
        &= \frac{q}{4\pi\epsilon_0}\left(1 - \frac{v^2}{c^2}\right)\frac{(x - vt, y, z)}{\{(x - vt)^2 + \left(1 - \frac{v^2}{c^2}\right)(y^2 + z^2)\}^{3/2}}
    \end{align*}
    である。特に$x - vt = 0, y = z = 0$の場合はそれぞれ
    \begin{align*}
        E(r, t) &= \frac{q}{4\pi\epsilon_0}\frac{1}{\sqrt{1 - \frac{v^2}{c^2}}}\frac{(0, y, z)}{\{(y^2 + z^2)\}^{3/2}}\\
        E(r, t) &= \frac{q}{4\pi\epsilon_0}\left(1 - \frac{v^2}{c^2}\right)\frac{(x - vt, 0, 0)}{\{(x - vt)^2\}^{3/2}}
    \end{align*}
    となる。$x$軸から離れた点では速度が大きくなるほど電場も大きくなり、$x$軸上では速度が大きくなるほど電場が小さくなることが分かる。

\subsection{加速運動する点電荷}
    % $r - s(t_r) = R, \dot{s}(t_r) = v, \ddot{s}(t_r) = a$とおくと、リエナール・ヴィーヘルト・ポテンシャルは
    % \begin{align*}
    %     \phi(r, t)
    %     &= \frac{q}{4\pi\epsilon_0}\frac{1}{|r - s(t_r)| - \dot{s}(t_r) \cdot (r - s(t_r)) / c}\\
    %     &= \frac{q}{4\pi\epsilon_0}\frac{1}{|R| - v \cdot R / c}\\
    %     A(r, t)
    %     &= \frac{\mu_0q}{4\pi}\frac{\dot{s}(t_r)}{|r - s(t_r)| - \dot{s}(t_r) \cdot (r - s(t_r)) / c}\\
    %     &= \frac{\mu_0q}{4\pi}\frac{v}{|R| - v \cdot R / c}\\
    % \end{align*}
    次に点電荷が加速する場合に、放射される電磁波を求める。点電荷の位置を$s(t)$、観測者の位置を$r$とする。$R = |r-s(r)|,n = (r-s(r))/|r-s(r)|$とすれば、ポインティングベクトルは、
        \[S(r, t) = \frac{q^2}{16\pi^2\epsilon_0c}\frac{n}{(1 - n \cdot v)^6R^2}[n \times \{(n - v) \times \dot{v}\}]^2\]
    である。放射される電磁波の方向と強度は速度と加速度に依存する。

    加速する電荷から電磁波が放射されることになると、原子核の周りを回る電子はいずれエネルギーを失い、安定した軌道を維持できないことになってしまう。この問題の解決には量子力学が必要となった。 % 真空中の電磁波
\section{物質中の電磁波}

\subsection{誘電体中の電磁波}
    半導体や誘電体などの、電子が原子核に拘束されているような物質に電磁波が入射したときのことを考える。原子核は重いのでほとんど動かないとして良く、分子に拘束された電子を、抵抗を受けながら振動する調和振動子とみなす。これをローレンツ振動子モデルという。電子の変位を$u$、固有振動数を$\omega_0$とすると電子の運動方程式は
        \[\dv[2]{u}{t} = - \omega_0^2 u - \gamma\dv{u}{t} -\frac{eE}{m} \quad (\gamma > 0)\]
    となる。電場が角振動数$\omega$で振動すれば、電子も同じ振動数でそれに追随して動く。
    \begin{align*}
        E(r, t) &= E(r)\exp(-i\omega t)\\
        u(r, t) &= u_(r)\exp(-i\omega t)
    \end{align*}
    と置く。運動方程式に代入すると
    \begin{gather*}
        -\omega^2 u_(r) - i\gamma\omega u(r) + \omega_0^2 u(r) = -\frac{eE(r)}{m}\\
        u(r) = \frac{e}{m(\omega^2 - \omega_0^2 + i\gamma\omega)}E(r)
    \end{gather*}
    単位体積中の分子数が$n$個で、各分子に束縛される電子数が$z$個あるとすれば分極ベクトルは
        \[P(r) = -nzeu = -\frac{nze^2}{m(\omega^2 - \omega_0^2 + i\gamma\omega)}E(r)\]
    電気感受率と誘電率は
    \begin{align*}
        \chi(\omega) &= -\frac{nze^2}{\epsilon_0m(\omega^2 - \omega_0^2 + i\gamma\omega)}\\
        \epsilon(\omega) &= \epsilon_0\left(1 - \frac{nze^2}{m(\omega^2 - \omega_0^2 + i\gamma\omega)}\right)
    \end{align*}
    となる。実際には、複数の固有振動数と減衰係数について足し合わせたものとなる。
    
    誘電率は元々静電場に対して定義されていた。しかし、一定の振動数の振動電場を誘電体にかけたときも、分極ベクトルと電場の間には比例関係が成り立つことが分かった。そこで、誘電率を電場の振動数に依存したものとして定義を拡張する。誘電率が振動数に依存するので、光速も振動数によって変わる。従って、電磁波は時間とともに形が崩れる。これを波の分散という。速さが異なれば屈折率も当然異なるので、異なる物質の境界面で分光する。これも分散と呼ばれる。
    
    電場と電束密度は実部を見れば良いが、それらの比である複素誘電率は実部と虚部の両方が意味を持つ。複素電気感受率$\chi = \chi_1 + i\chi_2$と置くと
    \begin{align*}
        \chi_1 &= -\frac{nze^2}{\epsilon_0m}\frac{\omega^2 - \omega_0^2}{(\omega^2 - \omega_0^2)^2 + \gamma^2\omega^2}\\
        \chi_2 &= \frac{nze^2}{\epsilon_0m}\frac{\gamma\omega}{(\omega^2 - \omega_0^2)^2 + \gamma^2\omega^2}
    \end{align*}
    となる。電場が電子に対してする仕事を考える。まず分極ベクトルは
    \begin{align*}
        P(t)
            &= \Re\chi(\omega)E(t)\\
            &= \Re(\chi_1(\omega) + i\chi_2(\omega))|E_0|\exp(-i(\omega t + \alpha))\\
            &= |E_0|\chi_1(\omega)\cos(\omega t + \alpha) + |E_0|\chi_2(\omega)\sin(\omega t + \alpha)
    \end{align*}
    より変位電流は
    \begin{align*}
        I_p(t) = \pdv{P}{t} = -\omega\chi_1(\omega)|E_0|\sin(\omega t + \alpha) + \omega\chi_2(\omega)|E_0|\cos(\omega t + \alpha)
    \end{align*}
    なので、電場が単位体積単位時間当たりに電子に行う仕事を周期$T = 2\pi / \omega$について平均すると、
    \begin{align*}
        W   &= \frac{\omega}{2\pi}\oint E(t)I_p(t)dt\\
            &= \frac{\omega^2|E_0|^2}{2\pi}\oint -\chi_1(\omega)\cos(\omega t + \alpha)\sin(\omega t + \alpha) + \chi_2(\omega)\cos^2(\omega t + \alpha) dt\\
            &= \frac{\omega^2\chi_2|E_0|^2}{2\pi}\int_0^{2\pi / \omega} \frac{1 + \cos 2\omega t}{2} dt\\
            &= \frac{1}{2}\omega\chi_2(\omega)|E_0|^2
    \end{align*}
    となる。$\chi_2(\omega) > 0$なので、電場は常に正の仕事をする。つまり電場のエネルギーが誘電体に吸収されることを意味する。分子内で加速された電子の運動エネルギーが熱として失われることになる。
        \[\omega\chi_2(\omega) \propto \frac{\omega^2}{(\omega^2 - \omega_0^2)^2 + \gamma^2\omega^2}\]
    より吸収は共鳴点$\omega = \omega_0$で最大となり、電子の振動は電場より$\pi / 2$遅れる。それ以外の低周波や高周波の振動電場は誘電体との相互作用が弱く、ほとんど通り抜ける。また、振動数が小さい場合には$\omega = 0$に近づくので静電場に対する誘電率で近似できる。磁性体の磁化率や透磁率も振動する磁場に対しては振動数に依存するが、強磁性体以外の普通の物質では、$\chi_m$は$\mu_0$に比べて非常に小さく、物質の電磁気的な性質にはほとんど影響を与えない。

\subsection{金属中の電磁波}
    金属やプラズマ中の自由電子には束縛力が働かないため、ローレンツ振動子モデルにおいて$\omega_0 = 0$とした方程式
        \[m\dv[2]{u}{t} =  -\gamma\dv{u}{t} - eE \quad (\gamma > 0)\]
    を考えれば良い。これをドルーデモデルという。電気感受率と誘電率は
    \begin{align*}
        \chi(\omega) &= -\frac{nze^2}{\epsilon_0m(\omega^2 + i\gamma\omega)}\\
        \epsilon(\omega) &= \epsilon_0\left(1 - \frac{nze^2}{m(\omega^2 + i\gamma\omega)}\right)
    \end{align*}
    となる。$\gamma = 0$のとき電子は入射した電場と完全に同位相の電場を作り出すため、誘電率が真空中よりも小さくなる。$\omega = 0$つまり静電場のとき誘電率が無限大だから、電場が完全に打ち消されることが分かる。同様に$\chi = \chi_1 + i\chi_2$として
    \begin{align*}
        \chi_1 &= -\frac{nze^2}{\epsilon_0m}\frac{1}{\omega^2 + \gamma^2}\\
        \chi_2 &= \frac{nze^2}{\epsilon_0m}\frac{1}{\omega^3 / \gamma + \gamma\omega}
    \end{align*}
    となる。

    金属中に電磁波が存在するとき、電子の移動による電流が無視できない。そこで電流密度を考慮してマクスウェル方程式を解く。
    \begin{align*}
        \div E = 0\\
        \div B = 0\\
        \rot E + \pdv{B}{t} &= 0\\
        \rot B - \epsilon\mu\pdv{E}{t} &= \mu i
    \end{align*}
    第一式の$\rot$と第二式の時間偏微分を取り、磁場の項を消去すると
    \begin{align*}
        \grad\div E - \Delta E + \epsilon\mu\pdv[2]{E}{t} &= - \mu \pdv{i(r, t)}{t}\\
        \left(-\epsilon\mu\pdv[2]{t} + \Delta\right)E(r, t) &= \mu\pdv{i(r, t)}{t}
    \end{align*}
    右辺に電流密度の項があるため波動方程式ではなく、もはや同じ形を保って進む解は存在しない。電場と電流密度のラプラス変換を考え、それぞれ
    \begin{align*}
        E(r, t) = E(k, \omega)\exp(i(k \cdot r - \omega t))\\
        i(r, t) = i(k, \omega)\exp(i(k \cdot r - \omega t))
    \end{align*}
    とおく。平面波解が存在しないので、形式的においた波数ベクトルや振動数が実数とは限らないことに注意する。
    \begin{align*}
        \epsilon\mu\pdv[2]{t}E(r, t) &= -\epsilon\mu\omega^2 E(k, \omega)\exp(i(k \cdot r - \omega t))\\
        \Delta E(r, t) &= - (k_x^2 + k_y^2 + k_z^2)E(k, \omega)\exp(i(k \cdot r - \omega t))
    \end{align*}
    だから
        \[(-\epsilon\mu\omega^2 + k_x^2 + k_y^2 + k_z^2)E(k, \omega) = i\mu\omega i(k, \omega)\]
    ドルーデモデルによれば$i = \sigma E$と表される。結局は線形偏微分方程式となる。電磁波が$z$軸方向に進んでいるとすると
    \begin{align*}
        k_z^2 - \epsilon\mu\omega^2 = i\mu\sigma\omega\\
        k_z^2 = \epsilon\mu\omega^2 + i\mu\sigma\omega
    \end{align*}
    $k_z = k_1 + ik_2$とおくと
    \begin{align*}
        k_1 &= \omega\sqrt{\epsilon\mu}\left[\frac{\sqrt{1 + \left(\frac{\sigma}{\epsilon\omega}\right)^2} + 1}{2}\right]\\
        k_2 &= \omega\sqrt{\epsilon\mu}\left[\frac{\sqrt{1 + \left(\frac{\sigma}{\epsilon\omega}\right)^2} - 1}{2}\right]
    \end{align*}
    振動数が小さいときは第一項を無視できる。実際、普通の金属の導電率は$\sigma \simeq 10^7 {\rm \Omega^{-1} \cdot m^{-1}}$、誘電率は$\epsilon \simeq \epsilon_0 \simeq 10^{-11}{\rm C^2 \cdot N^{-1} \cdot m^{-2}}$だから
        \[\omega \ll \frac{\sigma}{\epsilon} \simeq 10^{18}{\rm s^{-1}}\]
    であり、普通の電波($10^4$-$10^{10}$)や可視光($10^{14}$)では第二項だけで十分近似できる。このとき波数は複号を選択して
        \[k_z = \frac{1 + i}{\sqrt{2}}\sqrt{\mu\sigma\omega} = \frac{1 + i}{l}\]
    つまり電場は
        \[E(r, t) = E(k, \omega)\exp\left(i\left(\frac{1 + i}{l}z - \omega t\right)\right) = E(k, \omega)e^{-z/l}\exp(i(z/l - \omega t))\]
    となる。実指数関数が含まれていてフーリエ変換できない関数だったことが分かる。電場は導体に入射すると進行方向に急激に減衰し、$l$の数倍の距離を進むとほとんど消えてしまう。侵入できる深さは先程の近似が成り立つ範囲で振動数が高いほど短い。$\omega = 10^{10}{\rm s^{-1}}$の場合、
        \[l \simeq \left(\frac{2}{4\times 3.14\times 10^{-7}\times 10^7\times 10^{10}}\right)^{1/2} \simeq 10^{-6}\]
    だから、電磁波は導体中にほとんど侵入できない。

    形を保ったまま移動する波が存在しないので、単純な意味での速度を考えることはできない。振動数は
    \begin{align*}
        \epsilon\mu\omega^2
        &= k_1^2 - k_2^2\\
        &= k_1^2 - \frac{\mu^2\sigma^2\omega^2}{4k_1^2}\\
        \omega^2
        &= \frac{k_1^2}{\epsilon\mu + \frac{\mu^2\sigma^2}{4k_1^2}}
    \end{align*}
    位相速度は
        \[\frac{\omega}{k_1} = \frac{1}{\sqrt{\epsilon\mu + \frac{\mu^2\sigma^2}{4k_1^2}}}\]
    群速度は
    \begin{align*}
        \dv{\omega}{k_1}
        &= \frac{1}{\sqrt{\epsilon\mu + \frac{\mu^2\sigma^2}{4k_1^2}}} - \frac{1}{2}\frac{k_1}{\left(\epsilon\mu + \frac{\mu^2\sigma^2}{4k_1^2}\right)^{3/2}} \cdot -2 \frac{\mu^2\sigma^2}{4k_1^3}\\
        &= \frac{1}{\sqrt{\epsilon\mu + \frac{\mu^2\sigma^2}{4k_1^2}}} + \frac{\mu^2\sigma^2}{4k_1^2\left(\epsilon\mu + \frac{\mu^2\sigma^2}{4k_1^2}\right)^{3/2}}
    \end{align*}
    となる。

\subsection{デバイモデル}
    電気双極子モーメント$p$の時間発展を緩和時間を$\tau$として
        \[\tau\dv{P}{t} = -P + \epsilon_0\chi_0 E\]
    と定式化する。
    \begin{align*}
        E(t) &= E(\omega)\exp(-i\omega t)\\
        P(t) &= P(\omega)\exp(-i\omega t)
    \end{align*}
    とおくと
    \begin{align*}
        -i\omega\tau P(\omega)\exp(-i\omega t) &= -P(\omega)\exp(-i\omega t) + \epsilon_0\chi_0E(\omega)\exp(-i\omega t)\\
        P(\omega) &= \frac{\epsilon_0\chi_0}{1 - i\omega\tau}E(\omega)
    \end{align*}
    となる。よって電気感受率と誘電率は
    \begin{align*}
        \chi(\omega) &= \frac{\chi_0}{1 - i\omega\tau}\\
        \epsilon(\omega) &= \epsilon_0\left(1 + \frac{\chi_0}{1 - i\omega\tau}\right)
    \end{align*} % 物質中の電磁波
\section{波動光学}

\subsection{電磁場の境界条件}
    $xy$平面を境に2つの物質が存在しているとする。境界に垂直な長方形の周にファラデーの法則を適用して
        \[\int (E_1 - E_2) \cdot t ds + \int \pd[B]{t} \cdot n dS = 0\]
    長方形を小さくしていけば第二項は0に近づくので
        \[E_1 \cdot t = E_2 \cdot t\]
    同じ領域にアンペールの法則を適用する。電流密度は0として
    \begin{gather*}
        \int (H_1 - H_2) \cdot t ds - \int \pd[D]{t} \cdot n dS = 0\\
        H_1 \cdot t = H_2 \cdot t
    \end{gather*}
    次に境界面の一部を覆うような直方体に電場に関するガウスの法則を適用する。直方体内の電荷を0として
    \begin{gather*}
        \int (D_1 - D_2) \cdot n dS = 0\\
        D_1 \cdot n = D_2 \cdot n
    \end{gather*}
    同じ領域に磁場に関するガウスの法則を適用する。
    \begin{gather*}
        \int (B_1 - B_2) \cdot n dS = 0\\
        B_1 \cdot n = B_2 \cdot n
    \end{gather*}
    となる。よって電磁場の境界条件は
    \begin{itemize}
        \item 境界面に平行な電場は連続
        \item 境界面に平行な磁場は連続
        \item 境界面に垂直な電束密度は連続
        \item 境界面に垂直な磁束密度は連続
    \end{itemize}
    となる。

\subsection{光の反射・屈折}
    $xy$平面を境に2つの媒質が存在しているとする。空間内にはいくつかの平面波が存在しており、それぞれの半空間において$i$番目の平面波の波数と振動数は$k_i, \omega_i$である。
    \begin{align*}
        E(r, t) &=
        \begin{cases}
            \sum_i E_{+i}\exp(i(k_i \cdot r - \omega_i t)) & (z \geq 0)\\
            \sum_i E_{-i}\exp(i(k_i \cdot r - \omega_i t)) & (z \leq 0)\\
        \end{cases}\\
        H(r, t) &=
        \begin{cases}
            \sum_i H_{+i}\exp(i(k_i \cdot r - \omega_i t)) & (z \geq 0)\\
            \sum_i H_{-i}\exp(i(k_i \cdot r - \omega_i t)) & (z \leq 0)\\
        \end{cases}
    \end{align*}
    電磁場の境界条件より
    \begin{align*}
        E_x(x, y, 0, t) &= \sum_i E_{+ix}\exp(i(k_{ix}x + k_{iy}y - \omega_i t)) = \sum_i E_{-ix}\exp(i(k_{ix}x + k_{iy}y - \omega_i t))\\
        E_y(x, y, 0, t) &= \sum_i E_{+iy}\exp(i(k_{ix}x + k_{iy}y - \omega_i t)) = \sum_i E_{-iy}\exp(i(k_{ix}x + k_{iy}y - \omega_i t))\\
        E_z(x, y, 0, t) &= \epsilon_+ \sum_i E_{+iz}\exp(i(k_{ix}x + k_{iy}y - \omega_i t)) = \epsilon_- \sum_i E_{-iz}\exp(i(k_{ix}x + k_{iy}y - \omega_i t))
    \end{align*}
    領域$z \leq 0$で平面波は一つと仮定すると
    \begin{align*}
        \sum_i E_{+ix}\exp(i(k_{ix}x + k_{iy}y - \omega_i t)) &= E_{-x}\exp(i(k_xx + k_yy - \omega t))\\
        \sum_i E_{+iy}\exp(i(k_{ix}x + k_{iy}y - \omega_i t)) &= E_{-y}\exp(i(k_xx + k_yy - \omega t))\\
        \epsilon_+\sum_i E_{+iz}\exp(i(k_{ix}x + k_{iy}y - \omega_i t)) &= \epsilon_-E_{-z}\exp(i(k_xx + k_yy - \omega t))
    \end{align*}
    $x, y, t$で$n$階偏微分し$(x, y, t) = (0, 0, 0)$とすることで$(k_{ix}, k_{iy}, \omega_i) = (k_x, k_y, \omega)$が分かる。ここで$k_y = 0$とする。同じ媒質では$|k| = \omega / v = n\omega / c$が等しいので、領域$z \geq 0$では$k_z = \pm d$の二つの平面波の合成となる。つまり光が透過するとき、波長は変化し振動数は変化しない。波数ベクトル$(k_x, 0, -d), (k_x, 0, d), (k_x, 0, k_z)$の平面波をそれぞれ入射光、反射光、透過光と呼ぶことにする。入射角$\theta_i$、反射角$\theta_r$、屈折角$\theta_t$は次のように表せる。
        \[\sin\theta_i = \sin\theta_r = \frac{k_x}{\sqrt{k_x^2 + d^2}} = \frac{ck_x}{n_+\omega}, \quad \sin\theta_t = \frac{k_x}{\sqrt{k_x^2 + k_z^2}} = \frac{ck_x}{n_-\omega}\]
    従って次が成り立つ。
    \begin{align*}
        \theta_i &= \theta_r \quad (反射の法則)\\
        n_+\sin\theta_i &= n_-\sin\theta_t \quad (屈折の法則)\\
    \end{align*}

    入射光、反射光、透過光が存在している平面を入射面と呼ぶ。それぞれの電磁波は電場が入射面に平行な成分と磁場が入射面に平行な成分に分けることができる。前者をP偏光(parallel)、後者をS偏光(senkrecht)と呼ぶ。入射光の電磁場が
    \begin{align*}
        E_i &= (I_p\cos\theta_i, I_s, I_p\sin\theta_i)\exp(i(k_i \cdot r - \omega t))\\
        H_i &= \frac{1}{\mu_+\omega}(dI_s, -dI_p\cos\theta_i - k_xI_p\sin\theta_i, k_xI_s)\exp(i(k_i \cdot r - \omega t))\\
        E_r &= (R_p\cos\theta_i, R_s, -R_p\sin\theta_i)\exp(i(k_r \cdot r - \omega t))\\
        H_r &= \frac{1}{\mu_+\omega}(-dR_s, dR_p\cos\theta_i + k_xR_p\sin\theta_i, k_xR_s)\exp(i(k_r \cdot r - \omega t))\\
        E_t &= (T_p\cos\theta_t, T_s, T_p\sin\theta_t)\exp(i(k_t \cdot r - \omega t))\\
        H_t &= \frac{1}{\mu_+\omega}(-k_zT_s, k_zT_p\cos\theta_t - k_xT_p\sin\theta_t, k_xT_s)\exp(i(k_t \cdot r - \omega t))
    \end{align*}
    であるとする。境界条件より
    \begin{align*}
        I_p\cos\theta_i + R_p\cos\theta_i &= T_p\cos\theta_t\\
        I_s + R_s &= T_s\\
        \epsilon_+(I_p\sin\theta_i - R_p\sin\theta_i) &= \epsilon_-T_p\sin\theta_t\\
        \frac{dI_s}{\mu_+\omega} - \frac{dR_s}{\mu_+\omega} &= -\frac{k_zT_s}{\mu_-\omega}\\
        \frac{-dI_p\cos\theta_i - k_xI_p\sin\theta_i}{\mu_+\omega} + \frac{dR_p\cos\theta_i + k_xR_p\sin\theta_i}{\mu_+\omega} &= \frac{k_zT_p\cos\theta_t - k_xT_p\sin\theta_t}{\mu_-\omega}\\
        \frac{k_xI_s}{\omega} + \frac{k_xR_s}{\omega} &= \frac{k_xT_s}{\omega}
    \end{align*}
    透磁率が等しいとすると$\epsilon_- / \epsilon_+ = n_-^2 / n_+^2 = \sin^2\theta_i / \sin^2\theta_t$なので
    \begin{align*}
        I_p + R_p &= \frac{\cos\theta_t}{\cos\theta_i}T_p\\
        I_p - R_p &= \frac{\epsilon_-}{\epsilon_+}\frac{\sin\theta_t}{\sin\theta_i}T_p \simeq \frac{\sin\theta_i}{\sin\theta_t}T_p\\
        I_s + R_s &= T_s\\
        I_s - R_s &= \frac{\mu_+}{\mu_-}\frac{\tan\theta_i}{\tan\theta_t}T_s \simeq \frac{\tan\theta_i}{\tan\theta_t}T_s\\
    \end{align*}
    つまり
    \begin{align*}
        T_p &= \frac{2}{\frac{\cos\theta_t}{\cos\theta_i} + \frac{\sin\theta_i}{\sin\theta_t}}I_p\\
            &= \frac{2\cos\theta_i\sin\theta_t}{\cos\theta_i\sin\theta_i + \cos\theta_t\sin\theta_t}I_p\\
            &= \frac{2\cos\theta_i\sin\theta_t}{(\sin2\theta_i + \sin2\theta_t) / 2}I_p\\
            &= \frac{2\cos\theta_i\sin\theta_t}{\sin(\theta_t + \theta_i)\cos(\theta_t - \theta_i)}I_p\\
        T_s &= \frac{2}{1 + \frac{\tan\theta_i}{\tan\theta_t}}I_p\\
            &= \frac{2\tan\theta_t}{\tan\theta_i + \tan\theta_t}I_p\\
            &= \frac{2\cos\theta_i\sin\theta_t}{\sin\theta_i\cos\theta_t + \cos\theta_i\sin\theta_t}\\
            &= \frac{2\cos\theta_i\sin\theta_t}{\sin(\theta_i + \theta_t)}\\
        R_p &= \frac{\cos\theta_t\sin\theta_t - \cos\theta_i\sin\theta_i}{\cos\theta_i\sin\theta_i + \cos\theta_t\sin\theta_t}I_p\\
            &= \frac{\sin(\theta_t - \theta_i)\cos(\theta_t + \theta_i)}{\sin(\theta_t + \theta_i)\cos(\theta_t - \theta_i)}I_p\\
            &= \frac{\tan(\theta_t - \theta_i)}{\tan(\theta_t + \theta_i)}I_p\\
            R_s
            &= \frac{\sin(\theta_t - \theta_i)}{\sin(\theta_t + \theta_i)}I_p
    \end{align*}
    これらをフレネルの式という。つまり入射したP偏光は出射するときもP偏光であり、入射したS偏光は出射するときもS偏光である。反射率及び透過率は入射光に含まれるP偏光とS偏光の割合に依存する。$\theta_i = \pi / 2$のとき$R_p = R_s = 1$で反射率100\%になる。また、$\theta_i + \theta_t = \pi / 2$のとき$R_p = 0$でP偏光の反射率は0となる。このときの入射角$\theta_i$をブリュースター角という。
    \begin{align*}
        \frac{R_p}{R_s}
        &= \frac{\cos(\theta_t + \theta_i)}{\cos(\theta_t - \theta_i)}
    \end{align*}
    S偏光の反射率はP偏光の反射率より高い。したがって偏光グラスなどでS偏光を遮れば、外光のうちS偏光が多く含まれている反射光を集中してカットできる。

\subsection{散乱理論}
    散乱とは、光が物質に入射したとき光を四方八方に放射する現象である。古典的には、入射光によって誘起された電気双極子の振動により二次波が放出される、と説明される。物質を原点に置き、$z$軸の正の方向から光が入射するとする。単位時間に単位面積当たりに入射する粒子数のうち、半径$r$の球面のある立体角$d\Omega$内に散乱される単位時間当たりの粒子数の割合を微分断面積という。光散乱の場合は粒子数の代わりにエネルギーで測る。電磁波は振動するので平均を取る。すなわち、散乱波のポインティングベクトルを$S_s$、入射波のポインティングベクトルを$S_i$とすれば、
        \[\de[\sigma(\theta)]{\Omega} = \frac{|S_s|}{|S_i|}r^2\]
    である。ここで$\theta$は散乱によって$z$軸から逸れた角度であり、散乱角と呼ばれる。これを全立体角で積分したものを全断面積という。
    
\subsection{トムソン散乱}
    自由電子に振動数の低い電磁波が入射したときを考える。電子の運動方程式は、
        \[m\de[u]{t} = eE_i\]
    である。$E_i = E_0\sin\omega t$とすれば、
        \[u = \frac{eE_i}{m\omega^2}\]
    つまり
        \[p(\omega) = \frac{e^2E_0}{m\omega^2}\]
    である。この振動により双極子が誘起され、双極子放射が起こる。$n,E_i$のなす角を$\phi$とすると
    \begin{align*}
        E_s = \frac{\mu}{4\pi}\frac{\omega^2}{r}{n\times(n\times p(\omega))}e^{i\omega(t - r/c)} = \frac{\mu}{4\pi}\frac{e^2E_0}{mr}\sin\phi e^{i\omega(t - r/c)}
    \end{align*}
    それぞれのポインティングベクトルの時間平均は、
    \begin{align*}
        |\overline{S_i}| &= \frac{|\overline{E_i}|^2}{\mu c} = \frac{|E_0|^2}{2\mu c}\\
        |\overline{S_s}| &= \frac{|\overline{E_s}|^2}{\mu c} = \frac{1}{\mu c}\left(\frac{\mu}{4\pi}\frac{\omega^2}{r}|p(\omega)|\right)^2\sin^2\phi = \frac{|E_0|^2}{2\mu c}\left(\frac{\mu}{4\pi}\frac{\omega^2}{r}\alpha\right)^2\sin^2\phi
    \end{align*}
    ただし$\alpha$は分極率である。したがって微分断面積は
    \begin{align*}
        \de[\sigma]{\Omega} = \left(\frac{\mu}{4\pi}\omega^2\alpha\right)^2\sin^2\phi           
    \end{align*}
    無偏光の場合は、散乱角を$\theta$としたとき$n = (\sin\theta, 0, \cos\theta), \quad E_0 / |E_0| = (\cos\psi, \sin\psi, 0)$と置くと
    \begin{align*}
        \cos\phi = n \cdot E_0 / |E_0| = \cos\psi\sin\theta
    \end{align*}
    なので$\psi$について平均を取ると
    \begin{align*}
        \sin^2\phi = 1 - |\cos^2\psi\sin^2\theta| = \frac{1 + \cos^2\theta}{2}
    \end{align*}
    よって
    \begin{align*}
        \de[\sigma]{\Omega} = \left(\frac{\mu}{4\pi}\omega^2\alpha\right)^2 \frac{1 + \cos^2\theta}{2}
    \end{align*}
    となる。$\alpha = e^2 / m\omega^2$を代入すれば
    \begin{align*}
        \de[\sigma]{\Omega} = \left(\frac{e^2}{4\pi mc^2\epsilon_0}\right)^2 \frac{1 + \cos^2\theta}{2} = a_0^2\frac{1 + \cos^2\theta}{2}
    \end{align*}
    である。光の進行方向に対して最も強く散乱することが分かる。$a_0$は静電エネルギーと静止エネルギーが一致する半径を示し、古典的電子半径と呼ばれる。全立体角で積分すれば
    \begin{align*}
        \int_0^\pi\int_0^{2\pi} \frac{1 + \cos^2\theta}{2}\sin\theta d\theta d\phi = \pi \int_0^\pi 2\sin\theta - \sin^3\theta d\theta = \pi \left[\cos\theta + \frac{1}{3}\cos^3\theta\right]_0^\pi = \frac{8\pi}{3}
    \end{align*}
    より
        \[\sigma = \frac{8\pi}{3}a_0^2\]
    となる。このような自由電子による散乱をトムソン散乱と呼ぶ。

\subsection{レイリー散乱}
    微粒子のサイズが光の波長よりも十分に小さいとき、半径$a$の誘電体の双極子モーメントはクラウジウス・モソッティの関係式より
        \[p = 4\pi\epsilon_0\frac{\epsilon - \epsilon_0}{\epsilon + 2\epsilon_0}a^3E_0\]
    である。微分散乱断面積は
    \begin{align*}
        \de[\sigma]{\Omega}
            &= \left(\frac{\mu}{4\pi}\omega^2\alpha\right)^2 \frac{1 + \cos^2\theta}{2}\\
            &= \left(\epsilon_0\mu\omega^2\frac{\epsilon - \epsilon_0}{\epsilon + 2\epsilon_0}a^3\right)^2 \frac{1 + \cos^2\theta}{2}\\
            &= \left(\frac{\epsilon - \epsilon_0}{\epsilon + 2\epsilon_0}\right)^2 \left(\frac{\omega}{c}\right)^4 a^6 \frac{1 + \cos^2\theta}{2}
    \end{align*}
    全断面積は
    \begin{align*}
        \sigma = \frac{8\pi}{3}\left(\frac{\epsilon - \epsilon_0}{\epsilon + 2\epsilon_0}\right)^2 \left(\frac{\omega}{c}\right)^4 a^6
    \end{align*}
    つまり青い光は赤い光より多く散乱される。青空の原因はこのレイリー散乱である。光は空気中の微粒子にぶつかる度に散乱を繰り返し、青い光は垂直方向に離散していく。結果朝焼けや夕焼けが起こる。それでも地平線から離れた上空では青く見え、その中間当たりは白く見える。
    太陽のない場所を見上げたとき、視線の先からやってくる光は散乱光である。

\subsection{ミー散乱}

\subsection{輝度}
    ある面を単位時間当たりに通過するエネルギーを放射束という。電磁波の放射の場合波長ごとの放射束を分光放射束という。光源が広がりを持った場合を考える。放射束を光源表面の面積とその立体角で微分したものを放射輝度という。分光放射輝度も同様である。 % 波動光学

\end{document}