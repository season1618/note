\documentclass{jsarticle}
\usepackage{amssymb,amsmath}

\newcommand{\mlr}[1]{\left\{#1 \right\}}
\newcommand{\llr}[1]{\left[#1 \right]}
\newcommand{\de}[2][]{\frac{d #1}{d #2}}
\newcommand{\dd}[2][]{\frac{d^2 #1}{d #2^2}}
\newcommand{\pd}[2][]{\frac{\partial #1}{\partial #2}}
\newcommand{\pdd}[2][]{\frac{\partial^2 #1}{\partial #2^2}}

\renewcommand{\epsilon}{\varepsilon}
\newcommand{\grad}{\operatorname{grad}}
\newcommand{\rot}{\operatorname{rot}}
\renewcommand{\div}{\operatorname{div}}
\renewcommand{\Re}{\operatorname{Re}}

\title{電磁気学}
\author{season07001674}
\date{\today}

\begin{document}
\maketitle
\tableofcontents

\input maxwell1.tex % 真空中のマクスウェル方程式
\input maxwell2.tex % 物質中のマクスウェル方程式
\input potential.tex % 電磁ポテンシャル
\input conserve.tex % 電磁場の保存量
\input wave1.tex % 真空中の電磁波
\input wave2.tex % 物質中の電磁波
\input optics.tex % 波動光学

\end{document}