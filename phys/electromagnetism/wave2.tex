\section{物質中の電磁波}

\subsection{誘電体中の電磁波}
    半導体や誘電体などの、電子が原子核に拘束されているような物質に電磁波が入射したときのことを考える。原子核は重いのでほとんど動かないとして良く、分子に拘束された電子を、抵抗を受けながら振動する調和振動子とみなす。これをローレンツ振動子モデルという。電子の変位を$u$、固有振動数を$\omega_0$とすると電子の運動方程式は
        \[\dv[2]{u}{t} = - \omega_0^2 u - \gamma\dv{u}{t} -\frac{eE}{m} \quad (\gamma > 0)\]
    となる。電場が角振動数$\omega$で振動すれば、電子も同じ振動数でそれに追随して動く。
    \begin{align*}
        E(r, t) &= E(r)\exp(-i\omega t)\\
        u(r, t) &= u_(r)\exp(-i\omega t)
    \end{align*}
    と置く。運動方程式に代入すると
    \begin{gather*}
        -\omega^2 u_(r) - i\gamma\omega u(r) + \omega_0^2 u(r) = -\frac{eE(r)}{m}\\
        u(r) = \frac{e}{m(\omega^2 - \omega_0^2 + i\gamma\omega)}E(r)
    \end{gather*}
    単位体積中の分子数が$n$個で、各分子に束縛される電子数が$z$個あるとすれば分極ベクトルは
        \[P(r) = -nzeu = -\frac{nze^2}{m(\omega^2 - \omega_0^2 + i\gamma\omega)}E(r)\]
    電気感受率と誘電率は
    \begin{align*}
        \chi(\omega) &= -\frac{nze^2}{\epsilon_0m(\omega^2 - \omega_0^2 + i\gamma\omega)}\\
        \epsilon(\omega) &= \epsilon_0\left(1 - \frac{nze^2}{m(\omega^2 - \omega_0^2 + i\gamma\omega)}\right)
    \end{align*}
    となる。実際には、複数の固有振動数と減衰係数について足し合わせたものとなる。
    
    誘電率は元々静電場に対して定義されていた。しかし、一定の振動数の振動電場を誘電体にかけたときも、分極ベクトルと電場の間には比例関係が成り立つことが分かった。そこで、誘電率を電場の振動数に依存したものとして定義を拡張する。誘電率が振動数に依存するので、光速も振動数によって変わる。従って、電磁波は時間とともに形が崩れる。これを波の分散という。速さが異なれば屈折率も当然異なるので、異なる物質の境界面で分光する。これも分散と呼ばれる。
    
    電場と電束密度は実部を見れば良いが、それらの比である複素誘電率は実部と虚部の両方が意味を持つ。複素電気感受率$\chi = \chi_1 + i\chi_2$と置くと
    \begin{align*}
        \chi_1 &= -\frac{nze^2}{\epsilon_0m}\frac{\omega^2 - \omega_0^2}{(\omega^2 - \omega_0^2)^2 + \gamma^2\omega^2}\\
        \chi_2 &= \frac{nze^2}{\epsilon_0m}\frac{\gamma\omega}{(\omega^2 - \omega_0^2)^2 + \gamma^2\omega^2}
    \end{align*}
    となる。電場が電子に対してする仕事を考える。まず分極ベクトルは
    \begin{align*}
        P(t)
            &= \Re\chi(\omega)E(t)\\
            &= \Re(\chi_1(\omega) + i\chi_2(\omega))|E_0|\exp(-i(\omega t + \alpha))\\
            &= |E_0|\chi_1(\omega)\cos(\omega t + \alpha) + |E_0|\chi_2(\omega)\sin(\omega t + \alpha)
    \end{align*}
    より変位電流は
    \begin{align*}
        I_p(t) = \pdv{P}{t} = -\omega\chi_1(\omega)|E_0|\sin(\omega t + \alpha) + \omega\chi_2(\omega)|E_0|\cos(\omega t + \alpha)
    \end{align*}
    なので、電場が単位体積単位時間当たりに電子に行う仕事を周期$T = 2\pi / \omega$について平均すると、
    \begin{align*}
        W   &= \frac{\omega}{2\pi}\oint E(t)I_p(t)dt\\
            &= \frac{\omega^2|E_0|^2}{2\pi}\oint -\chi_1(\omega)\cos(\omega t + \alpha)\sin(\omega t + \alpha) + \chi_2(\omega)\cos^2(\omega t + \alpha) dt\\
            &= \frac{\omega^2\chi_2|E_0|^2}{2\pi}\int_0^{2\pi / \omega} \frac{1 + \cos 2\omega t}{2} dt\\
            &= \frac{1}{2}\omega\chi_2(\omega)|E_0|^2
    \end{align*}
    となる。$\chi_2(\omega) > 0$なので、電場は常に正の仕事をする。つまり電場のエネルギーが誘電体に吸収されることを意味する。分子内で加速された電子の運動エネルギーが熱として失われることになる。
        \[\omega\chi_2(\omega) \propto \frac{\omega^2}{(\omega^2 - \omega_0^2)^2 + \gamma^2\omega^2}\]
    より吸収は共鳴点$\omega = \omega_0$で最大となり、電子の振動は電場より$\pi / 2$遅れる。それ以外の低周波や高周波の振動電場は誘電体との相互作用が弱く、ほとんど通り抜ける。また、振動数が小さい場合には$\omega = 0$に近づくので静電場に対する誘電率で近似できる。磁性体の磁化率や透磁率も振動する磁場に対しては振動数に依存するが、強磁性体以外の普通の物質では、$\chi_m$は$\mu_0$に比べて非常に小さく、物質の電磁気的な性質にはほとんど影響を与えない。

\subsection{金属中の電磁波}
    金属やプラズマ中の自由電子には束縛力が働かないため、ローレンツ振動子モデルにおいて$\omega_0 = 0$とした方程式
        \[m\dv[2]{u}{t} =  -\gamma\dv{u}{t} - eE \quad (\gamma > 0)\]
    を考えれば良い。これをドルーデモデルという。電気感受率と誘電率は
    \begin{align*}
        \chi(\omega) &= -\frac{nze^2}{\epsilon_0m(\omega^2 + i\gamma\omega)}\\
        \epsilon(\omega) &= \epsilon_0\left(1 - \frac{nze^2}{m(\omega^2 + i\gamma\omega)}\right)
    \end{align*}
    となる。$\gamma = 0$のとき電子は入射した電場と完全に同位相の電場を作り出すため、誘電率が真空中よりも小さくなる。$\omega = 0$つまり静電場のとき誘電率が無限大だから、電場が完全に打ち消されることが分かる。同様に$\chi = \chi_1 + i\chi_2$として
    \begin{align*}
        \chi_1 &= -\frac{nze^2}{\epsilon_0m}\frac{1}{\omega^2 + \gamma^2}\\
        \chi_2 &= \frac{nze^2}{\epsilon_0m}\frac{1}{\omega^3 / \gamma + \gamma\omega}
    \end{align*}
    となる。

    金属中に電磁波が存在するとき、電子の移動による電流が無視できない。そこで電流密度を考慮してマクスウェル方程式を解く。
    \begin{align*}
        \div E = 0\\
        \div B = 0\\
        \rot E + \pdv{B}{t} &= 0\\
        \rot B - \epsilon\mu\pdv{E}{t} &= \mu i
    \end{align*}
    第一式の$\rot$と第二式の時間偏微分を取り、磁場の項を消去すると
    \begin{align*}
        \grad\div E - \Delta E + \epsilon\mu\pdv[2]{E}{t} &= - \mu \pdv{i(r, t)}{t}\\
        \left(-\epsilon\mu\pdv[2]{t} + \Delta\right)E(r, t) &= \mu\pdv{i(r, t)}{t}
    \end{align*}
    右辺に電流密度の項があるため波動方程式ではなく、もはや同じ形を保って進む解は存在しない。電場と電流密度のラプラス変換を考え、それぞれ
    \begin{align*}
        E(r, t) = E(k, \omega)\exp(i(k \cdot r - \omega t))\\
        i(r, t) = i(k, \omega)\exp(i(k \cdot r - \omega t))
    \end{align*}
    とおく。平面波解が存在しないので、形式的においた波数ベクトルや振動数が実数とは限らないことに注意する。
    \begin{align*}
        \epsilon\mu\pdv[2]{t}E(r, t) &= -\epsilon\mu\omega^2 E(k, \omega)\exp(i(k \cdot r - \omega t))\\
        \Delta E(r, t) &= - (k_x^2 + k_y^2 + k_z^2)E(k, \omega)\exp(i(k \cdot r - \omega t))
    \end{align*}
    だから
        \[(-\epsilon\mu\omega^2 + k_x^2 + k_y^2 + k_z^2)E(k, \omega) = i\mu\omega i(k, \omega)\]
    ドルーデモデルによれば$i = \sigma E$と表される。結局は線形偏微分方程式となる。電磁波が$z$軸方向に進んでいるとすると
    \begin{align*}
        k_z^2 - \epsilon\mu\omega^2 = i\mu\sigma\omega\\
        k_z^2 = \epsilon\mu\omega^2 + i\mu\sigma\omega
    \end{align*}
    $k_z = k_1 + ik_2$とおくと
    \begin{align*}
        k_1 &= \omega\sqrt{\epsilon\mu}\left[\frac{\sqrt{1 + \left(\frac{\sigma}{\epsilon\omega}\right)^2} + 1}{2}\right]\\
        k_2 &= \omega\sqrt{\epsilon\mu}\left[\frac{\sqrt{1 + \left(\frac{\sigma}{\epsilon\omega}\right)^2} - 1}{2}\right]
    \end{align*}
    振動数が小さいときは第一項を無視できる。実際、普通の金属の導電率は$\sigma \simeq 10^7 {\rm \Omega^{-1} \cdot m^{-1}}$、誘電率は$\epsilon \simeq \epsilon_0 \simeq 10^{-11}{\rm C^2 \cdot N^{-1} \cdot m^{-2}}$だから
        \[\omega \ll \frac{\sigma}{\epsilon} \simeq 10^{18}{\rm s^{-1}}\]
    であり、普通の電波($10^4$-$10^{10}$)や可視光($10^{14}$)では第二項だけで十分近似できる。このとき波数は複号を選択して
        \[k_z = \frac{1 + i}{\sqrt{2}}\sqrt{\mu\sigma\omega} = \frac{1 + i}{l}\]
    つまり電場は
        \[E(r, t) = E(k, \omega)\exp\left(i\left(\frac{1 + i}{l}z - \omega t\right)\right) = E(k, \omega)e^{-z/l}\exp(i(z/l - \omega t))\]
    となる。実指数関数が含まれていてフーリエ変換できない関数だったことが分かる。電場は導体に入射すると進行方向に急激に減衰し、$l$の数倍の距離を進むとほとんど消えてしまう。侵入できる深さは先程の近似が成り立つ範囲で振動数が高いほど短い。$\omega = 10^{10}{\rm s^{-1}}$の場合、
        \[l \simeq \left(\frac{2}{4\times 3.14\times 10^{-7}\times 10^7\times 10^{10}}\right)^{1/2} \simeq 10^{-6}\]
    だから、電磁波は導体中にほとんど侵入できない。

    形を保ったまま移動する波が存在しないので、単純な意味での速度を考えることはできない。振動数は
    \begin{align*}
        \epsilon\mu\omega^2
        &= k_1^2 - k_2^2\\
        &= k_1^2 - \frac{\mu^2\sigma^2\omega^2}{4k_1^2}\\
        \omega^2
        &= \frac{k_1^2}{\epsilon\mu + \frac{\mu^2\sigma^2}{4k_1^2}}
    \end{align*}
    位相速度は
        \[\frac{\omega}{k_1} = \frac{1}{\sqrt{\epsilon\mu + \frac{\mu^2\sigma^2}{4k_1^2}}}\]
    群速度は
    \begin{align*}
        \dv{\omega}{k_1}
        &= \frac{1}{\sqrt{\epsilon\mu + \frac{\mu^2\sigma^2}{4k_1^2}}} - \frac{1}{2}\frac{k_1}{\left(\epsilon\mu + \frac{\mu^2\sigma^2}{4k_1^2}\right)^{3/2}} \cdot -2 \frac{\mu^2\sigma^2}{4k_1^3}\\
        &= \frac{1}{\sqrt{\epsilon\mu + \frac{\mu^2\sigma^2}{4k_1^2}}} + \frac{\mu^2\sigma^2}{4k_1^2\left(\epsilon\mu + \frac{\mu^2\sigma^2}{4k_1^2}\right)^{3/2}}
    \end{align*}
    となる。

\subsection{デバイモデル}
    電気双極子モーメント$p$の時間発展を緩和時間を$\tau$として
        \[\tau\dv{P}{t} = -P + \epsilon_0\chi_0 E\]
    と定式化する。
    \begin{align*}
        E(t) &= E(\omega)\exp(-i\omega t)\\
        P(t) &= P(\omega)\exp(-i\omega t)
    \end{align*}
    とおくと
    \begin{align*}
        -i\omega\tau P(\omega)\exp(-i\omega t) &= -P(\omega)\exp(-i\omega t) + \epsilon_0\chi_0E(\omega)\exp(-i\omega t)\\
        P(\omega) &= \frac{\epsilon_0\chi_0}{1 - i\omega\tau}E(\omega)
    \end{align*}
    となる。よって電気感受率と誘電率は
    \begin{align*}
        \chi(\omega) &= \frac{\chi_0}{1 - i\omega\tau}\\
        \epsilon(\omega) &= \epsilon_0\left(1 + \frac{\chi_0}{1 - i\omega\tau}\right)
    \end{align*}