\section{電磁場の保存量}

力学において、系の保存量は全て粒子に帰属するものだった。電磁気学においては、電磁場もエネルギーや運動量を持つと考える。基本的には、ローレンツ力による多数の荷電粒子の運動方程式をマクスウェル方程式を用いて書き換えていく。

\subsection{電磁場のエネルギー}
    \subsubsection{静電場のエネルギー}
        電荷は同種の電気が反発力に逆らって集まっているため、運動していなくてもエネルギーを持っている。場の理論において、ポテンシャルエネルギーは物質ではなく場に蓄えられていると考える。つまり、全エネルギーは物質の運動エネルギーと場のエネルギーの和である。

        半径$r$の球面上に電荷$q$が分布している状態を考える。電荷が0の状態から始めて無限遠から微小電荷を近づけると
        \begin{align*}
            U(r)
            &= \int_0^q \frac{q'}{4\pi\epsilon_0 r} dq'\\
            &= \frac{q^2}{8\pi\epsilon_0 r}
        \end{align*}
        となる。ところで球殻の外側の電場は半径に依らず一定である。よって、半径$r, r + dr$の帯電球のエネルギーの差を取れば球殻内のエネルギー密度$u$が分かる。
        \begin{align*}
            U(r + dr) - U(r) &= -u \cdot 4\pi r^2dr\\
            u &= -\frac{1}{4\pi r^2}\de[U]{r}\\
            &= \frac{1}{4\pi r^2}\frac{q^2}{8\pi\epsilon_0 r^2} = \frac{q^2}{32\pi^2\epsilon_0 r^4}\\
            &= \frac{1}{2}\epsilon_0 E^2\\
        \end{align*}

    \subsubsection{静磁場のエネルギー}
    \subsubsection{電磁場のエネルギー}
        運動方程式
            \[m_i\de[v_i]{t} = q_i(E(r_i) + v_i \times B(r_i))\]
        の両辺と$v_i$との内積を取る。
            \[\de{t}\(\frac{1}{2}m_iv_i^2\) = q_iv_i \cdot E(r_i)\]
        領域$V$内の全粒子について総和を取れば
        \begin{align*}
            \de{t}\(\sum_i \frac{1}{2}m_iv_i^2\)
            &= \sum_i q_iv_i \cdot E(r_i)\\
            &= \sum_i \int q_iv_i\delta(r' - r_i) \cdot E(r') dV\\
            &= \int i(r') \cdot E(r') dV\\
        \end{align*}
        電場$E$に距離をかけたものが電圧であり、電流密度$i$に面積をかけたものが電流だから、右辺はジュール熱を意味する。つまりジュール熱とは荷電粒子の運動エネルギーである。右辺をさらに変形すると
        \begin{align*}
            E \cdot i
            &= E \cdot \(\rot H - \pd[D]{t}\)\\
            &= H \cdot \rot E - \div(E \times H) - \pd[(E \cdot D)]{t}\\
            &= -\pd[(E \cdot D)]{t} - \pd[(H \cdot B)]{t} - \div(E \times H)\\
        \end{align*}
        エネルギー密度$u = (E \cdot D + H \cdot B) / 2$、$S = E \times H$とすれば、
            \[-\pd[u]{t} = E \cdot i + \div S\]
        これをポインティング(Poynting)の定理と呼ぶ。$S$はポインティングベクトルと呼ばれており、電磁場のエネルギー流の密度を表している。つまり、領域内のエネルギーの減少量は、荷電粒子の運動エネルギーと放出された電磁波のエネルギーの和に等しい。

\subsection{マクスウェルの応力}
    領域$V$内にある電荷がその外側から受ける力を考える。ローレンツ力の式を少し変えると$dF = \rho(r)E(r) + i(r) \times B(r) dV$となる。
        \[F = \int_V \rho(r)E(r) + i(r)\times B(r)dV\]
    ここで
    \begin{align*}
        \div D &= \rho\\
        \rot H &- \pd[D]{t} = i\\
    \end{align*}
    を代入すれば
    \begin{align*}
        F
        &= \int \left[E \div D + \(\rot H - \pd[D]{t}\) \times B\right]dV\\
        &= \int \left[E \div D - B \times \rot H - \pd[D]{t} \times B\right]dV\\
        &= \int \(\epsilon_0E \div E - \frac{1}{\mu_0}B \times \rot B - \epsilon_0E \times \rot E - \epsilon_0\mu_0\de{t}(E \times H)\)dV
    \end{align*}
    $\div B = 0$なので、対称性を保つために$B \div B$という項を付け加える。
    \begin{align*}
        &= \int \left[\epsilon_0(E \div E - E \times \rot E) + \frac{1}{\mu_0}(B \div B - B \times \rot B)\right]dV - \frac{1}{c^2}\de{t}\(\int (E \times H)dV\)
    \end{align*}
    右辺第二項の積分の中身はポインティングベクトルである。マイナスの符号は電磁波が放出したことによる反作用を表している。ここで近接作用的な見方をするために、第一項を面積分に変換することを考える。つまり、積分の中身をあるベクトル場の発散として表すことができないか検討する。$x$成分だけを取り出せば
    \begin{align*}
        (E\div E - E \times \rot E)_x
        &= E_x\(\pd[E_x]{x} + \pd[E_y]{y} + \pd[E_z]{z}\) - E_y\(\pd[E_y]{x} - \pd[E_x]{y}\) + E_z\(\pd[E_x]{z} - \pd[E_z]{x}\)\\
        &= E_x\pd[E_x]{x} - E_y\pd[E_y]{x} - E_z\pd[E_z]{x}\\
        &\qquad + E_x\pd[E_y]{y} + E_y\pd[E_x]{y}\\
        &\qquad + E_x\pd[E_z]{z} + E_z\pd[E_x]{z}\\
        &= \frac{1}{2}\pd[E_x^2]{x} - \frac{1}{2}\pd[E_y^2]{x} - \frac{1}{2}\pd[E_z^2]{x}\\
        &\qquad + \pd[(E_xE_y)]{y} + \pd[(E_xE_z)]{z}\\
        &= \pd[(E_x^2 - \frac{1}{2}E^2)]{x} + \pd[(E_xE_y)]{y} + \pd[(E_xE_z)]{z}\\
    \end{align*}
    従って領域内の力は
    \begin{align*}
        T &= T_e + T_m\\
        T_e &= \epsilon_0
        \begin{bmatrix}
            E_x^2 - \frac{1}{2}E^2 & E_xE_y & E_xE_z\\
            E_yE_x & E_y^2 - \frac{1}{2}E^2 & E_yE_z\\
            E_zE_x & E_zE_y & E_z^2 - \frac{1}{2}E^2\\
        \end{bmatrix}\\
        T_m &= \frac{1}{\mu_0}
        \begin{bmatrix}
            B_x^2 - \frac{1}{2}B^2 & B_xB_y & B_xB_z\\
            B_yB_x & B_y^2 - \frac{1}{2}B^2 & B_yB_z\\
            B_zB_x & B_zB_y & B_z^2 - \frac{1}{2}B^2\\
        \end{bmatrix}\\
    \end{align*}
    として
    \begin{align*}
        F
        &= \int \div T dV - \frac{1}{c^2}\de{t}\int S dV\\
        &= \int T \cdot dS - \frac{1}{c^2}\de{t}\int S dV\\
    \end{align*}
    と表される。この$T$をマクスウェルの応力テンソルという。

    電場のみの状態を考えたとき、微小面$dS$に試験電荷を置いたときにかかる力のベクトルが$T_e \cdot dS$である。$T_e$は対称テンソルであり、直交する固有ベクトルを持つ。電場$E$は当然その固有ベクトルの一つである。電場方向には引っ張り合う力が働き、電場の垂直方向には圧縮する力が働く。

\subsection{電磁場の運動量}
    多粒子系の運動方程式
        \[\sum_i m_i\dd[r_i]{t} = \sum_i \int \left[q_i\delta(r' - r_i)E(r') + q_i\delta(r' - r_i)\de[r_i]{t} \times B(r')\right] dV\]
    に
    \begin{align*}
        \div D &= \sum_i q_i\delta(r' - r_i)\\
        \rot H - \pd[D]{t} &= \sum_i q_i\delta(r' - r_i)\de[r_i]{t}\\
    \end{align*}
    を代入すると、右辺は領域$V$内に働く電磁場による力と等しいので、マクスウェルの応力を用いると
    \begin{align*}
        \sum_i m_i\dd[r_i]{t} = \int T \cdot dS - \frac{1}{c^2}\de{t}\int (E \times H) dV\\
        \de{t}\left[\sum_i m_i\de[r_i]{t} + \frac{1}{c^2}\int S dV\right] = \int T \cdot dS\\
    \end{align*}
    となる。左辺第二項が電磁場の運動量であると解釈できる。

\subsection{電磁場の角運動量}
    \begin{align*}
        \de{t}\left[\sum_i r_i \times m_i\de[r_i]{t}\right]
        &= \sum_i r_i \times m_i\dd[r_i]{t}\\
        &= \sum_i r_i \times q_i\(E(r_i) + \de[r_i]{t} \times B(r_i)\)\\
        &= \sum_i \int q_i\delta(r' - r_i)r' \times \(E(r') + \de[r_i]{t} \times B(r')\) dV\\
        &= \int r' \times \(\div T - \frac{1}{c^2}\de{t}(E \times H)\) dV\\
    \end{align*}
    ここで
    \begin{align*}
        (r' \times \div T)_x
        &= \left\{(x, y, z) \times (\div T_x, \div T_y, \div T_z)\right\}_x\\
        &= y\div T_z - z\div T_y
        &= \div (yT_z) - \div T_{zy} - \div (zT_y) + \div T_{yz}\\
    \end{align*}
    $T$は対称テンソルなので相殺される。つまり
        \[(r' \times \div T)_x = \div (yT_z - zT_y)\]
    である。
        \[r' \times T = (yT_z - zT_y, zT_x - xT_z, xT_y - yT_x)\]
    と定義すれば
    \begin{align*}
        \de{t}\left[\sum_i r_i \times m_i\de[r_i]{t}\right]
        &= \int \div(r' \times T) dV - \frac{1}{c^2}\de{t}\int r' \times S dV\\
        &= \int r' \times T \cdot dS - \frac{1}{c^2}\de{t}\int r' \times S dV\\
    \end{align*}
    従って
        \[\de{t}\left[\sum_i r_i \times m_i\de[r_i]{t} + \frac{1}{c^2}\int r' \times S dV\right] = \int r' \times T \cdot dS\]
    となる。左辺第二項が電磁場の角運動量である。
