\documentclass{jsarticle}

\usepackage{amsmath,amssymb}
\usepackage{physics}
\usepackage{ascmac}
\usepackage{amsthm}

\let\grad\relax
\let\div\relax
\DeclareMathOperator{\grad}{grad}
\DeclareMathOperator{\rot}{rot}
\DeclareMathOperator{\div}{div}

\newcommand{\ricci}[2]{R^{#1}_{#2}}
\newcommand{\chr}[2]{\left\{#1 \atop #2\right\}}

\renewcommand{\epsilon}{\varepsilon}
\newcommand{\R}{\mathbb{R}}
\newcommand{\C}{\mathbb{C}}
\renewcommand{\H}{\mathcal{H}}
\newcommand{\U}{\mathcal{U}}

\renewcommand{\(}{\left(}
\renewcommand{\)}{\right)}

\theoremstyle{definition}
\renewcommand{\proofname}{\textbf{証明}}

\newtheorem{ax}{公理}
\newtheorem{dfn}{定義}
\newtheorem{prop}{命題}
\newtheorem{lem}{補題}
\newtheorem{thm}{定理}
\newtheorem{cor}{系}
\newtheorem{ex}{例}

\title{相対性理論}
\author{season07001674}
\date{\today}

\begin{document}
\maketitle
\tableofcontents

\part{特殊相対性理論}
マクスウェル方程式はガリレイ変換によって不変ではなかったため、力学と電磁気学の間に矛盾が生じてしまった。マイケルソン=モーレーの実験によって光速は慣性系に依らず一定であることが判明したため、アインシュタインは次の二つの原理を仮定することで力学を修正した。
\begin{description}
    \item[特殊相対性原理] 全ての慣性系で物理法則は同じ
    \item[光速度不変の原理] 光源に依らず光速度は一定
\end{description}

\section{時間と空間の相対性}

\subsection{時間と空間の定義}

\subsection{ローレンツ変換}
    相対性原理と光速度不変の原理から、ガリレイ変換に代わる新たな慣性系間の変換を導出する。

    二つの慣性系$K, K'$を考える。時刻$t = t' = 0$のとき両者の原点は一致していたとする。つまり慣性系の変換として$(0, 0, 0, 0) \mapsto (0, 0, 0, 0)$であるものを考える。時刻0で原点から光が出発して球面上に広がる状態を考えると、両方の系で光速は等しいので、
    \begin{align*}
        x^2 + y^2 + z^2 &= (ct)^2\\
        x'^2 + y'^2 + z'^2 &= (ct')^2
    \end{align*}
    となる。つまり$-(ct)^2 + x^2 + y^2 + z^2$は不変となる。また、二つの慣性系は互いに等速直線運動をしているので、速度を反転させたものは逆変換となるはずであり、線形変換である。よって、このような原点を原点に写し、$-(ct)^2 + x^2 + y^2 + z^2$を保つような線形変換をローレンツ変換と呼ぶ。
    
    $K'$系が$K$系に対して$x$軸方向に速さ$v$で進んでいるとき、ローレンツ変換の具体的な表示を求める。
    \begin{align*}
        ct' &= a_{00}ct + a_{01}x + a_{02}y + a_{03}z\\
        x' &= a_{10}ct + a_{11}x + a_{12}y + a_{13}z\\
        y' &= a_{20}ct + a_{21}x + a_{22}y + a_{23}z\\
        z' &= a_{30}ct + a_{31}x + a_{32}y + a_{33}z
    \end{align*}
    とおく。$(x', y', z') = (0, 0, 0)$のとき$(x, y, z) = (vt, 0, 0)$なので、
    \begin{align*}
        a_{10}c + a_{11}v &= 0\\
        a_{20}c + a_{21}v &= 0\\
        a_{30}c + a_{31}v &= 0
    \end{align*}
    同様に、$(x, y, z) = (0, 0, 0)$のとき$(x', y', z') = (-vt', 0, 0)$なので、
    \begin{align*}
        \frac{a_{10}}{a_{00}} &= -\frac{v}{c}\\
        \frac{a_{20}}{a_{00}} &= 0\\
        \frac{a_{30}}{a_{00}} &= 0
    \end{align*}
    つまり$a_{20} = a_{21} = a_{30} = a_{31} = 0$である。
    
    $y, z$軸は$x$軸の周りに回転させても形は変わらないので、自明に$a_{02} = a_{03} = a_{12} = a_{13} = 0$である。また第3,4式についても適用する。$y, z$軸は対称なので、$a_{22} = a_{33} = p, a_{23} = a_{32} = q$とおくと
    \begin{align*}
        \begin{pmatrix}
            \cos\theta & -\sin\theta\\
            \sin\theta & \cos\theta
        \end{pmatrix}
        \begin{pmatrix}
            p & q\\
            q & p
        \end{pmatrix}
        &=
        \begin{pmatrix}
            p & q\\
            q & p
        \end{pmatrix}
        \begin{pmatrix}
            \cos\theta & -\sin\theta\\
            \sin\theta & \cos\theta
        \end{pmatrix}\\
        \begin{pmatrix}
            p\cos\theta - q\sin\theta & -p\sin\theta + q\cos\theta\\
            p\sin\theta + q\cos\theta & p\cos\theta + q\sin\theta
        \end{pmatrix}
        &=
        \begin{pmatrix}
            p\cos\theta + q\sin\theta & -p\sin\theta + q\cos\theta\\
            p\sin\theta + q\cos\theta & p\cos\theta - q\sin\theta
        \end{pmatrix}
    \end{align*}
    この式は任意の$y, z$及び$\theta$で成立するので$q = 0$。逆変換を考えれば$p^2 = 1$。物理的に妥当なのは$p = 1$である。
    
    開始と同時に原点から$x$軸方向に向かって光が放たれたとすると、$(x, y, z) = (ct, 0, 0)$のとき$(x', y', z') = (ct', 0, 0)$なので、
        \[ct' = (a_{00} + a_{01})ct = (a_{10} + a_{11})ct\]
    同様に$(x, y, z) = (-ct, 0, 0)$のとき$(x', y', z') = (-ct', 0, 0)$なので、
        \[ct' = (a_{00} - a_{01})ct = (-a_{10} + a_{11})ct\]
    つまり$a_{00} = a_{11} = a, a_{01} = a_{10} = b$とおくことができる。速度$v$の符号を反転させれば逆変換になる。しかし係数にどのような形で$v$が含まれているか分からないので、代わりに$x$軸を反転させる。
    \begin{align*}
        ct' &= act - b \cdot -x\\
        -x' &= -bct + a \cdot -x
    \end{align*}
    よって、
    \[
        \begin{pmatrix}
            a & b\\
            b & a
        \end{pmatrix}
        \begin{pmatrix}
            a & -b\\
            -b & a
        \end{pmatrix}
        =
        \begin{pmatrix}
            a^2 - b^2 & 0\\
            0 & a^2 - b^2
        \end{pmatrix}
        =
        \begin{pmatrix}
            1 & 0\\
            0 & 1
        \end{pmatrix}
    \]
    なので$a^2 - b^2 = 1$である。$b / a = -v / c$なので、$a = 1 / \sqrt{1-(v/c)^2}, b = -(v / c) / \sqrt{1-(v/c)^2}$となる。

    従ってローレンツ変換は次のようになる。
    \begin{align*}
        ct' &= \frac{1}{\sqrt{1 - (v / c)^2}}ct - \frac{v / c}{\sqrt{1 - (v / c)^2}}x\\
        x'  &= -\frac{v / c}{\sqrt{1 - (v / c)^2}}ct + \frac{1}{\sqrt{1 - (v / c)^2}}x\\
        y'  &= y\\
        z'  &= z
    \end{align*}
    $v \ll c$の極限でガリレイ変換に一致することが分かる。また$\gamma = 1 / \sqrt{1 - (v / c)^2}$をローレンツ因子と呼ぶ。

    相対性原理とは全ての慣性系で物理法則が不変であるというものだった。特に物理法則はローレンツ変換に対して不変でなければならず、これを特殊相対性原理と呼ぶ。

\subsection{時間の遅れとローレンツ収縮}
    $x' = 0$と置けば$x = vt$を得る。これを$t$について解けば、
        \[t = \frac{t'}{\sqrt{1 - \frac{v^2}{c^2}}}\]
    つまり動いている物体は時間が遅れる。
    
    また$K$系と$K'$系の原点の間の距離を$l, l'$と置くと、$l = x = vt$。また$x = 0$と置けば$x' = -\gamma vt$より$l' = -x' = \gamma vt$なので、
        \[l = \sqrt{1 - \frac{v^2}{c^2}}l'\]
    つまり動いている物体はその長さが進行方向に対して縮む。これをローレンツ収縮という。

\subsection{速度の合成}
    $K'$系が$K$系に対して$x$軸方向に速度$v_1$で進んでいて、$K'$系で質点が$(x', y') = (v_2t'\cos\theta, v_2t'\sin\theta)$で移動しているとする。これをローレンツ変換して、
    \begin{align*}
        ct  &= \gamma ct' + \gamma v_1 / c \cdot v_2t'\cos\theta\\
        x   &= \gamma v_1 / c \cdot ct' + \gamma v_2t'\cos\theta\\
        y   &= v_2t'\sin\theta
    \end{align*}
    よって合成速度$V$は、
    \begin{align*}
        V_x &= \frac{x}{t} = \frac{v_1 + v_2\cos\theta}{1 + \frac{v_1v_2\cos\theta}{c^2}}\\
        V_y &= \frac{y}{t} = \frac{1}{\gamma}\frac{v_2\sin\theta}{1 + \frac{v_1v_2\cos\theta}{c^2}}
    \end{align*}
    $u_1 = v_1 / c, u_2 = v_2 / c, U = V / c$と置くと$U$の絶対値は
    \begin{align*}
        V^2 &= V_x^2 + V_y^2 = \frac{(v_1 + v_2\cos\theta)^2 + (1 - v_1^2 / c^2)v_2^2\sin^2\theta}{\(1 + \frac{v_1v_2\cos\theta}{c^2}\)^2}\\
        U^2 &= \frac{(u_1 + u_2\cos\theta)^2 + (1 - u_1^2)u_2^2\sin^2\theta}{(1 + u_1u_2\cos\theta)^2}\\
            &= \frac{u_1^2 + u_2^2 + 2u_1u_2\cos\theta - u_1^2u_2^2\sin^2\theta}{(1 + u_1u_2\cos\theta)^2}
    \end{align*}
    3点が必ず同一平面上にあることに注意すると、合成速度はベクトルで書くことができて、
        \[U^2 = \frac{u_1^2 + u_2^2 + 2u_1\cdot u_2 - |u_1 \times u_2|^2}{(1 + u_1 \cdot u_2)^2}\]
    となる。これを変形すると、
    \begin{align*}
        U^2 &= \frac{-1 + u_1^2 + u_2^2 - u_1^2u_2^2 + 1 + 2u_1 \cdot u_2 + (u_1 \cdot u_2)^2}
        {(1 + u_1 \cdot u_2)^2}\\
            &= \frac{(1 + u_1 \cdot u_2)^2 - (1 - u_1^2)(1 - u_2^2)}{(1 + u_1 \cdot u_2)^2}\\
        1 - U^2 &= \frac{(1 - u_1^2)(1 - u_2^2)}{(1 + u_1 \cdot u_2)^2}
    \end{align*}
    となるので$u_1, u_2 \leq 1$なら$U \leq 1$より合成速度が光速を超えないことが分かる。

\subsection{光行差}
    光行差とは、移動している観測者が天体を見るとき、天体が移動方向にずれて見える現象またはそのずれを指す。垂直に降っている雨を電車の中から見ると斜めに降っているように見えるのと同じだが、相対論的効果を考慮しなければならない。観測者が速度$v$で移動しており、その進行方向に対して角$\theta$にある天体の光行差を$a$とする。$K$系では時刻$t$における光の位置が$(-ct\cos\theta, -ct\sin\theta)$であるとする。これをローレンツ変換して、
    \begin{align*}
        ct' &= \gamma ct - \gamma v / c \cdot (-ct\cos\theta)\\
        x'  &= -\gamma v / c ct + \gamma (-ct\cos\theta)\\
        y'  &= y = -ct\sin\theta
    \end{align*}
    したがって
    \begin{align*}
        \frac{1}{\tan(\theta - a)}
            &= \frac{x'}{y'} = \frac{-\gamma vt - \gamma ct\cos\theta}{-ct\sin\theta}\\
            &= \frac{1}{\sqrt{1 - (v / c)^2}}\(\frac{v}{c\sin\theta} + \frac{1}{\tan\theta}\)
    \end{align*}
    となる。

\subsection{光のドップラー効果}
    $K$系の原点に光源があり、光が$(ct\cos\theta, ct\sin\theta)$に到達したとする。これをローレンツ変換して、
    \begin{align*}
        ct' &= \gamma ct - \gamma v / c \cdot ct\cos\theta\\
        x'  &= -\gamma v / c \cdot ct + \gamma ct\cos\theta\\
        y'  &= y = ct\sin\theta
    \end{align*}
    光速度不変の原理より、光の進む経路長の比は経過時間の比に等しいので、
    \begin{align*}
        \frac{\sqrt{x'^2 + y'^2}}{ct}
            &= \frac{\gamma ct - \gamma v / c \cdot ct\cos\theta}{ct}\\
            &= \gamma\(1 - \frac{v}{c}\cos\theta\)
    \end{align*}
    となる。二つの慣性系で光の振動する回数は等しいので、
    \begin{align*}
        \lambda' &= \frac{1 - (v / c)\cos\theta}{\sqrt{1 - (v / c)^2}}\lambda\\
        \nu' &= \frac{\sqrt{1 - (v / c)^2}}{1 - (v / c)\cos\theta}\nu
    \end{align*}
    である。特に$\theta = 0^\circ$の時は、
        \[\lambda' = \frac{1 - (v / c)}{\sqrt{1 - (v / c)^2}}\lambda\]
    $v > 0$の時光源に近付いていて、$v < 0$の時に光源から遠ざかっていることに注意すると、近付くときは波長が縮んで青みがかって見え(青方偏移)、遠ざかるときは波長が伸びて赤みがかって見える(赤方偏移)。また$\theta = 90^\circ$の時は、
        \[\lambda' = \frac{1}{\sqrt{1 - (v / c)^2}}\lambda\]
    となって波長が伸びる。これを横ドップラー効果という。
\section{四次元の時空}

\subsection{ミンコフスキー時空}
    ローレンツ変換ではガリレイ変換と異なり、時間と空間が一体となって変換される。そこで相対論では$x, y, z$に時間成分$ct$を加えた4次元時空というものを考える。これをミンコフスキー時空と言い、$(x^0, x^1, x^2, x^3) = (ct, x, y, z)$と表す。相対論では出来事を事象(event)と呼ぶ。事象はミンコフスキー時空の点として表され、世界点という。また質点の運動はこの時空の曲線で表され、世界線という。二つの世界点$(ct_1, x_1, y_1, z_1), (ct_2, x_2, y_2, z_2)$に対して、世界間隔$s_{12}$を
        \[s_{12}^2 = -(ct_2 - ct_1)^2 + (x_2 - x_1)^2 + (y_2 - y_1)^2 + (z_2 - z_1)^2\]
    と定義する。特に原点との世界間隔を世界長さ$s$と言い、
        \[s^2 = -(ct)^2 + x^2 + y^2 + z^2\]
    である。微小な世界間隔は
        \[dx^2 = -(cdt)^2 + dx^2 + dy^2 + dz^2\]
    であり、ミンコフスキー計量
    \[
        \eta_{ij} =
        \begin{pmatrix}
            -1 & 0 & 0 & 0\\
            0 & 1 & 0 & 0\\
            0 & 0 & 1 & 0\\
            0 & 0 & 0 & 1
        \end{pmatrix}
    \]
    を用いると
        \[ds^2 = \eta_{ij}dx^idx^j\]
    となる。ミンコフスキー時空の原点を通過する光は、時間軸から45度傾いた円錐面(実際は面ではない)
        \[s^2 = -(ct)^2 + x^2 + y^2 + z^2 = 0\]
    の上を伝播する。これを光円錐と呼ぶ。光円錐の内側($s^2 < 0$)では、原点における事象と因果関係を持つことができ、時間的(time like)領域と呼ばれる。それに対して外側($s^2 > 0$)は空間的(space like)領域と呼ばれる。

    世界長さ$s$はローレンツ変換によって変化しない不変量である。ローレンツ変換はミンコフスキー時空内でのある種の回転を表していると考えられる。実際$v / c = \tanh\theta$とおくと
    \begin{align*}
        ct' &= +\cosh\theta \cdot ct - \sinh\theta \cdot x\\
        x'  &= -\sinh\theta \cdot ct + \cosh\theta \cdot x
    \end{align*}
    である。

\subsection{固有時}
    注目している物体の静止系で測った時間をその物体の固有時(固有時間)$\tau$と言い、
    \begin{gather*}
        -(cdt)^2 + dx^2 + dy^2 + dz^2 = -(cd\tau)^2\\
        \begin{aligned}
            d\tau^2 &= dt^2 - \frac{dx^2 + dy^2 + dz^2}{c^2}\\
            d\tau &= -\frac{ds}{c}
        \end{aligned}
    \end{gather*}
    である。固有時は世界長さ同様ローレンツ変換によって変わらない不変量である。両辺を$dt^2$で割ると、
    \begin{align*}
        \(\dv{\tau}{t}\)^2
            &= 1 - \frac{1}{c^2}\left[\left(\dv{x}{t}\right)^2 + \left(\dv{y}{t}\right)^2 + \left(\dv{z}{t}\right)^2\right]\\
            &= 1 - \frac{v^2}{c^2}
    \end{align*}
    となる。

\subsection{微分演算子}
    三次元の微分演算子$\nabla$を四次元に拡張する。微分演算子はそれ自身では共変ベクトルだが、テンソルに作用するときは一般にテンソルとはならない。しかし座標変換としてローレンツ変換のような線形変換のみ考えるならば、テンソルと同じように振る舞う。
        \[(\partial_0, \partial_1, \partial_2, \partial_3) = \left(\pdv{x^0}, \pdv{x^1}, \pdv{x^2}, \pdv{x^3}\right) = \left(\frac{1}{c}\pdv{t}, \pdv{x}, \pdv{y}, \pdv{z}\right)\]
    とする。これとミンコフスキー計量との縮約をとると、
        \[(\partial^0, \partial^1, \partial^2, \partial^3) = \left(-\pdv{x^0}, \pdv{x^1}, \pdv{x^2}, \pdv{x^3}\right) = \(-\frac{1}{c}\pdv{t}, \pdv{x}, \pdv{y}, \pdv{z}\)\]
    となる。更にこれらの縮約をとり、
        \[\partial^i\partial_i = -\frac{1}{c^2}\pdv[2]{t} + \pdv[2]{x} + \pdv[2]{y} + \pdv[2]{z}\]
    これはダランベール演算子と呼ばれ$\square$と書く。

\subsection{共変形式}
    特殊相対性原理によれば物理法則はローレンツ変換によって形を変えない。したがって方程式のローレンツ共変性が一見して分かるような形式に書き直せるはずである。このようなものを共変形式(covariant form)と呼ぶ。ニュートン力学における速度や運動量、エネルギーといった物理量はローレンツ変換によって大きく形を変えてしまうため、相対論的な概念に書き換えていく。

    四元速度を
        \[u^i = \dv{x^i}{\tau} = (\gamma c, \gamma v^1, \gamma v^2, \gamma v^3)\]
    と定義する。座標を不変量である固有時で微分しているので反変テンソルである。四元速度の空間成分はニュートン力学の速度に相当する。

    四元運動量を
        \[p^i = mu^i = \gamma m(c, v^1, v^2, v^3)\]
    と定義する。四元速度と同様反変テンソルである。亜光速では$mv^i$の運動量は保存せず、四元運動量が保存することが実験によって確認されている。ここで
        \[(mc)^2 = (p^0)^2 - p_x^2 - p_y^2 - p_z^2 = (p^0)^2 - p^2\]
    なので、テイラー展開して、
    \begin{align*}
        p^0c &= \sqrt{(mc^2)^2 + (pc)^2}\\
             &= mc^2\sqrt{1 + \frac{p^2}{(mc)^2}}\\
             &= mc^2 + \frac{p^2}{2m} + \cdots
    \end{align*}
    第二項は古典力学における運動エネルギーと一致する。そこで$p^0$は物体の全エネルギー$E$を$c$で割ったものだと解釈する。つまり相対論では、物体のエネルギーは四元運動量の時間成分に組み込まれる。
        \[p^i = \left(\frac{E}{c}, p_x, p_y, p_z\right)\]
    である。またエネルギーは
    \begin{align*}
        E^2 &= (mc^2)^2 + (pc)^2\\
            &= (mc^2)^2 + (\gamma mv \cdot c)^2\\
            &= (mc^2)^2 \(1 + \gamma^2\frac{v^2}{c^2}\)\\
            &= (\gamma mc^2)^2\\
        E   &= \gamma mc^2
    \end{align*}
    となる。特に物体が静止しているときは
        \[E = mc^2\]
    となる。
\section{古典物理学の修正}
力学と電磁気学を共変形式に書き直す。

\subsection{運動方程式}
    運動方程式は通常の運動量と力の代わりに四元運動量と四元力を用いて
        \[\dv{p^\mu}{\tau} = f^\mu\]
    となる。四元力の時間成分はエネルギーの変化つまり仕事を表す。自由粒子の場合、
        \[\dv{p^\mu}{\tau} = 0\]
    である。時間成分はエネルギー保存則を表し、空間成分は運動量保存則を表す。
    % 四元力$f^i$を
    %     \[f^i = \dv{p^i}{\tau}\]
    % と定義する。これがそのまま相対論におけるニュートンの運動方程式である。観測者の座標で書き換えると
    % \begin{align*}
    %     f   &= \dv{p}{\tau} = \gamma m\dv{u}{t}\\
    %         &= \gamma m\(\gamma \dv{(c,v)}{t} + \dv{\gamma}{t}(c, v)\)
    % \end{align*}
    % なので、
    % \begin{align*}
    %     \dv{t}\(1 - \frac{v^2}{c^2}\)^{-1/2}
    %         &= -\frac{1}{2}\(1 - \frac{v^2}{c^2}\)^{-3 / 2}\cdot -\frac{1}{c^2}\dv{(v^2)}{t}\\
    %         &= \frac{\gamma^3}{c^2}\dv{(v^2 / 2)}{t}
    % \end{align*}
    % より、
    % \begin{align*}
    %     f^0 &= \gamma m\frac{\gamma^3}{c^2}\dv{(v^2 / 2)}{t}c\\
    %         &= \frac{\gamma^4}{c}\dv{t}\(\frac{1}{2}mv^2\)\\
    %         &= \gamma^4\dv{t}\frac{E}{c}\\
    %     f^i &= \gamma m\(\gamma\dv{v^i}{t} + \frac{\gamma^3}{c^2}\dv{(v^2 / 2)}{t}v^i\)\\
    %         &= \gamma^2\dv{p^i}{t} + \frac{\gamma^4}{c^2}\dv{t}\(\frac{1}{2}mv^2\)v^i\\
    %         &= \gamma^2\dv{p^i}{t} + \gamma^4\dv{t}\frac{E}{c} \cdot \frac{v^i}{c} \quad (i = 1, 2, 3)
    % \end{align*}

\subsection{電磁場}
    マクスウェル方程式は元々ローレンツ共変な方程式なので、修正を加える部分はなく単純に書き換えるだけである。

    電磁場形式のマクスウェル方程式は
    \begin{gather*}
        \begin{aligned}
            \div E &= \frac{\rho}{\epsilon_0}\\
            \div B &= 0
        \end{aligned}\\
        \begin{aligned}
            \rot E + \pdv{B}{t} &= 0\\
            \rot B - \frac{1}{c^2}\pdv{E}{t} &= \mu_0 i
        \end{aligned}
    \end{gather*}
    である。まず第一式と第四式を展開する。
    \begin{align*}
        \partial \cdot (0, E_x, E_y, E_z) &= \frac{\rho}{\epsilon_0}\\
        \partial \cdot (-E_x/c, 0, B_z, -B_y) &= \mu_0 i_x\\
        \partial \cdot (-E_y/c, -B_z, 0, B_x) &= \mu_0 i_y\\
        \partial \cdot (-E_z/c, B_y, -B_x, 0) &= \mu_0 i_z
    \end{align*}
    次に第二式と第三式を展開して、
    \begin{align*}
        \partial \cdot (0, B_x, B_y, B_z) &= 0\\
        \partial \cdot (-B_x, 0, E_z/c, -E_y/c) &= 0\\
        \partial \cdot (-B_y, -E_z/c, 0, E_x/c) &= 0\\
        \partial \cdot (-B_z, E_y/c, -E_x/c, 0) &= 0
    \end{align*}
    ここで、
    \begin{align*}
        F^{\mu\nu} &=
        \begin{bmatrix}
            0 & E_x / c & E_y / c & E_z / c\\
            -E_x / c & 0 & B_z & -B_y\\
            -E_y / c & -B_z & 0 & B_x\\
            -E_z / c & B_y & -B_x & 0
        \end{bmatrix}\\
        G^{\mu\nu} &=
        \begin{bmatrix}
            0 & B_x & B_y & B_z\\
            -B_x & 0 & E_z / c & -E_y / c\\
            -B_y & -E_z / c & 0 & E_x / c\\
            -B_z & E_y / c & -E_x / c & 0
        \end{bmatrix}\\
        j^\mu &= (\rho c, i_x, i_y, i_z)
    \end{align*}
    とすれば、
    \begin{align*}
        \partial_\nu F^{\mu\nu} &= \mu_0 j^\mu\\
        \partial_\nu G^{\mu\nu} &= 0
    \end{align*}
    となる。$F^{\mu\nu}$を電磁テンソル、$j^\mu$を四元電流密度という。
    % 第二式の右辺の0ベクトルはスカラーだが、同時に反変ベクトル、共変ベクトルでもある。したがって形式を整えるために$G^{\mu\nu}$も二階反変テンソルとした。
    $F^{\mu\nu}$とミンコフスキー計量で縮約をとり共変テンソルにすると、
    \begin{align*}
        F_{\mu\nu} 
            &= \eta_{\mu\alpha}\eta_{\nu\beta}F^{\alpha\beta} = \eta_{\mu\mu}\eta_{\nu\nu}F^{\mu\nu}\\
            &= \begin{bmatrix}
                0 & -E_x / c & -E_y / c & -E_z / c\\
                E_x / c & 0 & B_z & -B_y\\
                E_y / c & -B_z & 0 & B_x\\
                E_z / c & B_y & -B_x & 0
            \end{bmatrix}
    \end{align*}
    これを用いて$G^{\mu\nu}$を書き直すと、
    \[
        G^{\mu\nu} =
        \begin{bmatrix}
            0 & -F_{23} & -F_{31} & -F_{12}\\
            F_{23} & 0 & F_{30} & F_{02}\\
            -F_{13} & -F_{30} & 0 & -F_{01}\\
            F_{12} & F_{20} & F_{01} & 0
        \end{bmatrix}
    \]
    となる。したがって
    \begin{gather*}
        \partial_\nu F^{\mu\nu} = \mu_0 j^\mu\\
        \partial_\rho F_{\mu\nu} + \partial_\mu F_{\nu\rho} + \partial_\nu F_{\rho\mu} = 0 \tag{ビアンキの恒等式}
    \end{gather*}
    である(第二式は添え字が重なる場合も成り立つ)。

    電磁場のローレンツ変換は
        \[F'^{ab} = \Lambda^a_\mu \Lambda^b_\nu F^{\mu\nu}\]
    より
    \begin{align*}
        E_x' &= c F'^{01} = c \Lambda^0_\mu \Lambda^1_\nu F^{\mu\nu}\\
             &= c (\gamma \cdot -\gamma v/c F^{00} + \gamma \cdot \gamma F^{01} - \gamma v/c \cdot -\gamma v/c F^{10} + -\gamma v/c \cdot \gamma F^{11})\\
             &= \gamma^2 E_x - \gamma^2 v^2/c^2 E_x\\
             &= E_x\\
        E_y' &= c F'^{02} = c \Lambda^0_\mu \Lambda^2_\nu F^{\mu\nu}\\
             &= c (\gamma F^{02} - \gamma v/c F^{12})\\
             &= \gamma(E_y - v B_z)\\
        E_z' &= c F'^{03} = c \Lambda^0_\mu \Lambda^3_\nu F^{\mu\nu}\\
             &= c (\gamma F^{03} - \gamma v/c F^{13})\\
             &= \gamma(E_z + v B_y)\\
        B_x' &= F'^{23} = \Lambda^2_\mu \Lambda^3_\nu F^{\mu\nu}\\
             &= F^{23}\\
             &= B_x\\
        B_y' &= F'^{31} = \Lambda^3_\mu \Lambda^1_\nu F^{\mu\nu}\\
             &= -\gamma v/c F^{30} + \gamma F^{31}\\
             &= \gamma\(B_y + \frac{v}{c^2} E_z\)\\
        B_z' &= F'^{12} = \Lambda^1_\mu \Lambda^2_\nu F^{\mu\nu}\\
             &= -\gamma v/c F^{02} + \gamma F^{12}\\
             &= \gamma\(B_z - \frac{v}{c^2} E_y\)
    \end{align*}
    となる。コイルに磁石を近付けるとき、コイルの立場では磁場が変化するので電磁誘導によってコイルに電場が発生する。しかし、磁石の立場では近付いてきたコイルの中の電子がローレンツ力を受けて電流を発生させる。このとき電場は生じていない。このように相対論では電磁場も相対的な量となる。$x$軸方向に運動したとき、ローレンツ収縮によって$x$軸方向の長さが$\gamma$倍縮むので、$y, z$方向の電気力線と磁力線の密度も$\gamma$倍になると考えられる。

    % 次に電磁ポテンシャル形式のマクスウェル方程式は
    % \begin{gather*}
    %     \square A - \grad\(\div A + \frac{1}{c^2}\pdv{\phi}{t}\) = -\mu_0i\\
    %     \Delta\phi + \div\pdv{A}{t} = -\frac{\rho}{\epsilon_0}
    % \end{gather*}
    % である。第一式の$\grad$の中身を見れば、新たに$A^\mu = (\phi/c, A_x, A_y, A_z),\ A_\mu = (-\phi/c, A_x, A_y, A_z)$とすれば良いとわかる。$A^\mu, A_\mu$を四元ポテンシャルという。第一式は
    % \[
    %     \begin{aligned}
    %         \square A^\mu - \partial^\mu\partial_\nu A^\nu &= -\mu_0j^\mu & (\mu = 1,2,3)
    %     \end{aligned}
    % \]
    % となる。第二式は
    % \begin{align*}
    %     \(\Delta - \frac{1}{c^2}\pdv[2]{t}\)\phi + \pdv{t}\(\div A + \frac{1}{c^2}\pdv{\phi}{t}\) &= -\frac{\rho}{\epsilon_0}\\
    %     \square \phi + \pdv{t}\partial_\nu A^\nu &= -\frac{\phi}{\epsilon_0}\\
    %     \square A^0 - \partial^0\partial_\nu A^\nu &= -\mu_0j^0
    % \end{align*}
    % となる。二つの式をまとめると
    %     \[\square A^\mu - \partial^\mu\partial_\nu A^\nu = -\mu_0j^\mu\]
    % である。この式を
    %     \[\partial_\nu(\partial^\mu A^\nu - \partial^\nu A^\mu) = \mu_0j^\mu\]
    % と書いて、先ほどの電磁テンソルの式と比較すれば、$F^{\mu\nu} = \partial^\mu A^\nu - \partial^\nu A^\mu + X^{\mu\nu}$だが、$X^{\mu\nu} = 0$となることが分かるので、
    % \begin{align*}
    %     F^{\mu\nu} &= \partial^\mu A^\nu - \partial^\nu A^\mu\\
    %     F_{\mu\nu} &= \partial_\mu A_\nu - \partial_\nu A_\mu
    % \end{align*}
    % となる。$F_{\mu\nu}$は無条件にビアンキの恒等式を満たす。

    電磁テンソル$F_{\mu\nu}$を電磁ポテンシャルで書き直す。
    \begin{align*}
        E &= -\grad\phi - \pdv{A}{t}\\
        B &= \rot A
    \end{align*}
    なので
    \begin{align*}
        F_{i0} &= E_i / c = -\partial_i (\phi / c) - \partial_0 A_i\\
        F_{jk} &= B_i = \partial_j A_k - \partial_k A_j
    \end{align*}
    だから、$A_\mu = (-\phi / c, A_x, A_y, A_z)$とすれば
        \[F_{\mu\nu} = \partial_\mu A_\nu - \partial_\nu A_\mu\]
    となる。先ほどの式に代入するとマクスウェル方程式は
    \begin{align*}
        \partial_\nu(\partial^\mu A^\nu - \partial^\nu A^\mu) = \mu_0 j^\mu\\
        \partial^\mu(\partial_\nu A^\nu) - \square A^\mu = \mu_0 j^\mu
    \end{align*}
    となる。ビアンキの恒等式は自動的に成立している。

    スカラー$\chi$に対し、$A_\mu \mapsto A_\mu + \partial_\mu \chi$がゲージ変換
    \begin{align*}
        \phi &\mapsto \phi - \pdv{\chi}{t}\\
        A &\mapsto A + \div\chi
    \end{align*}
    となっていることが分かる。さらにローレンツ条件
        \[\frac{1}{c^2}\pdv{\phi}{t} + \div A = 0\]
    は
        \[\partial_\mu A^\mu = 0\]
    と書ける。つまりローレンツゲージにおけるマクスウェル方程式は
    \begin{align*}
        \square A^\mu = -\mu_0 j^\mu
    \end{align*}

\subsection{荷電粒子}
    ローレンツ力は電磁気学の中でも力学と接点を持つ概念なので、運動方程式と同様修正する必要がある。
        \[F = q(E + v \times B)\]
    を展開すると、
    \[
        \begin{pmatrix}
            F_x\\
            F_y\\
            F_z
        \end{pmatrix}
        = q
        \begin{pmatrix}
            -E_x/c & 0 & B_z & -B_y\\
            -E_y/c & -B_z & 0 & B_x\\
            -E_z/c & B_y & -B_x & 0
        \end{pmatrix}
        \begin{pmatrix}
            -c\\
            v_x\\
            v_y\\
            v_z
        \end{pmatrix}
    \]
    となる。中央の行列は電磁テンソルと1,2,3行成分が一致している。これに0行目を追加する。さらに通常の力と速度の代わりに四元力と四元速度を用いると
        \[f^\mu = q F^{\mu\nu} u_\nu\]
    である。ローレンツ力の時間成分は電磁場が荷電粒子に対してなす仕事を表す。

\subsection{質点のエネルギー・運動量}
    多粒子系を考える。粒子$i$の質量を$m_i$、位置を$z_i^\mu$、粒子$j$が粒子$i$に及ぼす四元力$f_{ij}^\mu$とすると、運動方程式は
        \[m_i\dv[2]{z_i^\mu}{\tau} = \sum_j f_{ij}^\mu\]
    である。全ての粒子について足し合わせると
        \[\sum_i m_i \dv[2]{z_i^\mu}{\tau} = \sum_{i, j} f_{ij}^\mu\]
    となる。ここで密度分布と速度場、力場を
    \begin{align*}
        \rho(x) &= \sum_i m_i \delta(x - z_i)\\
        u^\mu(x) &= \sum_i \dv{z_i^\mu}{\tau} \delta(x - z_i)\\
        f^\mu(x) &= \sum_i f_i \delta(x - z_i)
    \end{align*}
    と置くと
    \begin{align*}
        \int \rho(x)\pdv{u^\mu}{\tau}(x) dV &= \int f^\mu(x) dV\\
        \rho \pdv{u^\mu}{\tau} &= f^\mu\\
        \rho \pdv{u^\mu}{x^\nu}u^\nu &= f^\mu
    \end{align*}
    である。つまり
        \[\partial_\nu(\rho u^\mu u^\nu) = f^\mu + \rho u^\mu \partial_\nu u^\nu\]
    となる。
    \begin{align*}
        T^{00} &= \gamma^2 mc^2\\
        T^{0i} &= T^{i0} = \gamma^2 mcv^i = \gamma m p^i c\\
        T^{ij} &= \gamma^2 m v^i v^j
    \end{align*}
    である。つまり時間-時間成分はエネルギー、時間-空間成分は運動量を表す。これを多粒子系に拡張する。粒子$i$の質量$m(i)$、位置$z(i)$として
    \begin{align*}
        T^{\mu\nu}(x) = \sum_i \int_{-\infty}^\infty m(i)\dv{z(i)^\mu}{\tau}\dv{z(i)^\nu}{\tau}\delta(x - z(i)) d\tau
    \end{align*}
    となる。これをエネルギー運動量テンソルと呼ぶ。
    \begin{align*}
        \partial_\mu T^{\mu\nu}
            &= \partial_\mu \sum_i \int_{-\infty}^\infty m(i)\dv{z(i)^\mu}{\tau}\dv{z(i)^\nu}{\tau}\delta(x - z(i)) d\tau\\
            &= \sum_i m(i) \int_{-\infty}^\infty \dv{z(i)^\mu}{\tau}\dv{z(i)^\nu}{\tau} \partial_\mu \delta(x - z(i)) d\tau\\
            &= -\sum_i m(i) \int_{-\infty}^\infty \dv{\delta(x - z(i))}{\tau}\dv{z(i)^\nu}{\tau} d\tau\\
            &= -\sum_i m(i) \int_{-\infty}^\infty \dv{\tau}\left(\delta(x - z(i))\dv{z(i)^\nu}{\tau}\right) - \delta(x - z(i))\dv[2]{z(i)^\nu}{\tau} d\tau\\
            &= 0
    \end{align*}
    となる。

\subsection{完全流体のエネルギー・運動量}

\subsection{電磁場のエネルギー・運動量}
    古典電磁気学におけるエネルギーと運動量の保存は
    \begin{align*}
        \pdv{t}\left(\sum_i \frac{1}{2}m_i v_i^2 + \int u dV\right) &= - \int \div S dV\\
        \pdv{t}\left(\sum_i m_i v_i + \frac{1}{c^2}\int S dV\right) &= \int \div T dV
    \end{align*}
    と表される。ただし$u$は電磁場のエネルギー密度、$S$はポインティングベクトル、$T$はマクスウェルの応力テンソルである。
    \begin{align*}
        u &= \frac{E \cdot D + H \cdot B}{2}\\
        S &= E \times H\\
        T &= \epsilon_0 \begin{bmatrix}
            E_x^2 - E^2/2 & E_xE_y & E_xE_z\\
            E_yE_x & E_y^2 - E^2/2 & E_yE_z\\
            E_zE_x & E_zE_y & E_z^2 - E^2/2
        \end{bmatrix}
        +
        \frac{1}{\mu_0} \begin{bmatrix}
            B_x^2 - B^2/2 & B_xB_y & B_xB_z\\
            B_yB_x & B_y^2 - B^2/2 & B_yB_z\\
            B_zB_x & B_zB_y & B_z^2 - B^2/2
        \end{bmatrix}
    \end{align*}
    第一式は
    \begin{align*}
        \pdv{(ct)}\left(\sum_i \frac{1}{2}m_iv_i^2 + \int u dV\right) &= -\frac{1}{c} \int \div S dV\\
        \pdv{(ct)}\left(T_m^{00} + u\right) &= - \frac{1}{c}\div S\\
        \pdv{(ct)}\left(T_m^{00} + u\right) + \frac{1}{c}\left(\pdv{S_x}{x} + \pdv{S_y}{y} + \pdv{S_z}{z}\right) &= 0
    \end{align*}
    第二式は
    \begin{align*}
        \pdv{(ct)}\left(\sum_i m_i v_i c + \frac{1}{c}\int S dV\right) &= \int \div T dV\\
        \pdv{(ct)}\left(T_m^{0i} + \frac{S_i}{c}\right) &= \div T_i\\
        \pdv{(ct)}\left(T_m^{i0} + \frac{S_i}{c}\right) - \left(\pdv{T_{i1}}{x} + \pdv{T_{i2}}{y} + \pdv{T_{i3}}{z}\right) &= 0
    \end{align*}
    つまり
    \begin{align*}
        T_{em} = \begin{bmatrix}
            u & S_x/c & S_y/c & S_z/c\\
            S_x/c & -T_{11} & -T_{12} & -T_{13}\\
            S_y/c & -T_{21} & -T_{22} & -T_{23}\\
            S_z/c & -T_{31} & -T_{32} & -T_{33}
        \end{bmatrix}
    \end{align*}
    とおくと
        \[\partial_\nu(T_m^{\mu\nu} + T_{em}^{\mu\nu}) = 0\]
    となる。$T_{em}^{\mu\nu}$は電磁場のエネルギー・運動量テンソルである。
    \begin{align*}
        u &= \frac{1}{\mu_0}\frac{(E/c)^2 + B^2}{2}\\
        S/c &= \frac{1}{\mu_0 c}(E_yB_z - E_zB_y, E_zB_x - E_xB_z, E_xB_y - E_yB_x)\\
        T &= \frac{1}{\mu_0 c^2} \begin{bmatrix}
            E_x^2 - E^2/2 & E_xE_y & E_xE_z\\
            E_yE_x & E_y^2 - E^2/2 & E_yE_z\\
            E_zE_x & E_zE_y & E_z^2 - E^2/2
        \end{bmatrix}
        +
        \frac{1}{\mu_0} \begin{bmatrix}
            B_x^2 - B^2/2 & B_xB_y & B_xB_z\\
            B_yB_x & B_y^2 - B^2/2 & B_yB_z\\
            B_zB_x & B_zB_y & B_z^2 - B^2/2
        \end{bmatrix}
    \end{align*}
    より${T_{em}}^{\mu\nu}$は
        \[{T_{em}}^{\mu\nu} = \frac{1}{\mu_0}\left(-\eta_{\alpha\beta}F^{\mu\alpha}F^{\nu\beta} + \frac{1}{4}F^{\alpha\beta}F_{\alpha\beta}\eta^{\mu\nu}\right)\]
    と書ける。

\section{保存則}
\subsection{電磁場のエネルギーと運動量}
    荷電粒子の運動方程式は
        \[m \dv[2]{z^\mu}{\tau} = q F^{\mu\nu} u_\nu\]
    である。多粒子系では四元電流密度を用いて
        \[\rho\pdv{u^\mu}{\tau} = F^{\mu\nu}j_\nu\]
    となる。
    \begin{align*}
        \partial_\nu(\rho u^\mu u^\nu) = F^{\mu\lambda}j_\lambda + \rho u^\mu \partial_\nu u^\nu
    \end{align*}
    マクスウェル方程式$\partial_\nu F^{\mu\nu} = \mu_0 j^\mu$より
    \begin{align*}
        \partial_\nu(\rho u^\mu u^\nu)
            &= F^{\mu\lambda}j_\lambda + \rho u^\mu \partial_\nu u^\nu\\
            &= \frac{1}{\mu_0} \eta_{\lambda\sigma} F^{\mu\lambda}\partial_\nu F^{\sigma\nu} + \rho u^\mu \partial_\nu u^\nu\\
    \end{align*}
    
    質点のエネルギー運動量テンソルは
    \begin{align*}
        \partial_\nu {T_m}^{\mu\nu}
            &= \sum_i m(i) \int_{-\infty}^\infty \dv[2]{z(i)^\mu}{\tau}\delta(x - z(i)) d\tau\\
            &= \sum_i \int_{-\infty}^\infty q(i) F^{\mu\nu} u(i)_\nu \delta(x - z(i)) d\tau\\
            &= \int_{-\infty}^\infty F^{\mu\nu} \sum_i q(i)u(i)_\nu \delta(x - z(i)) d\tau\\
            &= F^{\mu\nu}j_\nu
    \end{align*}
    四元電流密度はマクスウェル方程式より$j^\mu = \partial_\nu F^{\mu\nu} / \mu_0$なので
    \begin{align*}
        \partial_\nu {T_m}^{\mu\nu}
            &= F^{\mu\nu} \left(\eta_{\nu\alpha} \frac{1}{\mu_0} \partial_\beta F^{\alpha\beta}\right)\\
            &= \frac{1}{\mu_0} \eta_{\nu\alpha} F^{\mu\nu} \partial_\beta F^{\alpha\beta}\\
            &= \frac{1}{\mu_0} \eta_{\nu\alpha} F^{\mu\nu} \delta_\beta^\mu \partial_\mu F^{\alpha\beta}\\
            &= \frac{1}{\mu_0} \eta_{\sigma\alpha} F^{\mu\sigma} \delta_\beta^\nu \partial_\nu F^{\alpha\beta}\\
            &= \frac{1}{\mu_0} \eta_{\sigma\alpha} F^{\mu\sigma} \partial_\nu F^{\alpha\nu}\\
            &= \frac{1}{\mu_0}(\partial_\nu(\eta_{\sigma\alpha} F^{\mu\sigma} F^{\alpha\nu}) - \eta_{\sigma\alpha}\partial_\nu F^{\mu\sigma} F^{\alpha\nu})\\
            &= \frac{1}{\mu_0}(\partial_\nu(\eta_{\sigma\alpha} F^{\mu\sigma} F^{\alpha\nu}) - \eta^{\mu\beta}\partial_\nu F_{\alpha\beta} F^{\alpha\nu})
    \end{align*}
    またマクスウェル方程式より$-\partial_\nu F_{\alpha\beta} = \partial_\alpha F_{\beta\nu} + \partial_\beta F_{\nu\alpha}$なので
    \begin{align*}
        \partial_\nu {T_m}^{\mu\nu}
            &= \frac{1}{\mu_0}(\partial_\nu(\eta_{\sigma\alpha} F^{\mu\sigma} F^{\alpha\nu}) + \eta^{\mu\beta}(\partial_\alpha F_{\beta\nu} + \partial_\beta F_{\nu\alpha})F^{\alpha\nu})\\
            &= \frac{1}{\mu_0}(\partial_\nu(\eta_{\sigma\alpha} F^{\mu\sigma} F^{\alpha\nu}) + \eta^{\mu\beta}\delta_\alpha^\nu \partial_\nu F_{\beta\lambda}F^{\alpha\lambda} + \frac{1}{2}\eta^{\mu\beta} \partial_\beta(F_{\lambda\alpha}F^{\alpha\lambda}))\\
    \end{align*}

\subsection{変分原理}
    変分原理によれば、物体はラグランジアンを位置、速度、時間の関数$L(x^i, v^i, t)$として、作用
        \[S[x(t)] = \int_{t_1}^{t_2} L(x^i, \dot{x}^i, t) dt\]
    を極小にするような軌道を取る。そのような物体の軌道はオイラー=ラグランジュ方程式
        \[\dv{t}\pdv{L}{\dot{x}^i} - \pdv{L}{x^i} = 0\]
    に従う。相対論では、ラグランジアンを四次元座標$x^i$、四元速度、固有時の関数$L(x^i, u^i, \tau)$として、作用
        \[S[x(\tau)] = \int_{\tau_1}^{\tau_2} L(x^i, \dot{x}^i, \tau) d\tau\]
    を極小にするよう世界線を考える。そしてそのような軌道は
        \[\dv{\tau}\pdv{L}{\dot{x}^i} - \pdv{L}{x^i} = 0\]
    に従う。

    また場の変分原理は、作用を
        \[S[A(ct, x, y, z)] = \int \mathcal{L}(ct, x, y, z) d(ct)dxdydz\]
    
    古典力学では自由質点、荷電粒子、電磁場の作用は
    \begin{align*}
        S &= S_m + S_{qm} + S_{em}\\
        S_m    &= \int \frac{1}{2}mv^2 dt\\
        S_{qm} &= \int -q(\phi - v \cdot A) dt\\
        S_{em} &= 
    \end{align*}
    であった。特殊相対論における作用は
    \begin{align*}
        S &= S_m + S_{qm} + S_{em}\\
        S_m    &= \int -mc\sqrt{-\eta_{\mu\nu} u^\mu u^\nu} d\tau\\
        S_{qm} &= \int q u^\mu A_\mu d\tau\\
        S_{em} &= \int -\frac{1}{4\mu_0}F^{\mu\nu}F_{\mu\nu} + A^\mu j_\mu d(ct)dxdydz
    \end{align*}
    となる。

    上述の方程式が導出されることを確認する。まず荷電粒子の運動は、
    \begin{align*}
        \dv{\tau}\pdv{L_m}{u^\mu} - \pdv{L_m}{x^\mu}
            &= \dv{\tau}\left(-mc \cdot \frac{1}{2}(-\eta_{\mu\nu} u^\mu u^\nu)^{-1/2} \cdot -2\eta_{\mu\nu}u^\nu\right) - 0\\
            &= \dv{\tau}mu_\mu\\
        \dv{\tau}\pdv{L_{qm}}{u^\mu} - \pdv{L_{qm}}{x^\mu}
            &= \dv{\tau} qA_\mu - \partial_\mu(q u^\nu A_\nu)\\
            &= q u^\nu \partial_\nu A_\mu - q u^\nu \partial_\mu A_\nu\\
            &= -q u^\nu F_{\mu\nu}
    \end{align*}
    よって
        \[m \dv[2]{x^\mu}{\tau} = qF^{\mu\nu}u_\nu\]
    となる。

    次に電磁場を考える。ラグランジアン密度を
    \begin{align*}
        \mathcal{L}_\text{em} &= -\frac{1}{4\mu_0}F^{\mu\nu}F_{\mu\nu}\\
        \mathcal{L}_\text{int} &= A^{\mu}j_{\mu}
    \end{align*}
    とする。$F^{\mu\nu}$は反対称テンソルであることに注意すると、
    \begin{align*}
        \pdv{x^\mu}\pdv{\mathcal{L}_\text{em}}{(\partial_\mu A^\nu)} - \pdv{\mathcal{L}_\text{em}}{A^\nu}
            &= -\frac{1}{4\mu_0} \partial^\mu \pdv{\eta_{ik}\eta_{jl}F^{ij}F^{kl}}{(\partial^\mu A^\nu)}\\
            &= -\frac{1}{2\mu_0} \partial^\mu\left(\eta_{ik}\eta_{jl}F^{ij} \pdv{(\partial^k A^l - \partial^l A^k)}{(\partial^\mu A^\nu)}\right)\\
            &= -\frac{1}{2\mu_0}\partial^\mu(\eta_{i\mu}\eta_{j\nu}F^{ij} - \eta_{i\nu}\eta_{j\mu}F^{ij})\\
            &= -\frac{1}{2\mu_0}\partial^\mu(F_{\mu\nu} - F_{\nu\mu})\\
            &= -\frac{1}{\mu_0} \partial^\mu F_{\mu\nu}\\
        \pdv{x^\mu}\pdv{\mathcal{L}_\text{int}}{(\partial_\mu A^\nu)} - \pdv{\mathcal{L}_{int}}{A^\nu}
            &= -\pdv{(A^\mu j_\mu)}{A^\nu}\\
            &= -j_\nu
    \end{align*}
    よって
    \begin{align*}
        \partial_\mu F^{\mu\nu} &= -\mu_0 j^\nu\\
        \partial_\nu F^{\mu\nu} &= \mu_0 j^\mu
    \end{align*}
    となる。

    $L_m$は複雑に見えるが、$(cd\tau)^2 = -\eta_{ij}x^ix^j$なので、
        \[\delta S = -mc^2 \int d\tau = 0\]
    である\footnote{つまり$L_m(x, u, \tau) = -mc^2$としても良いが、この場合ラグランジュ方程式は$x^i, u^i$の条件を導出しない。}。つまり物体は固有時が極大となるような軌道をとる。時間$t$をパラメータとして、ハミルトニアンを求めると、
    \begin{align*}
        H_m &= \gamma mv \cdot v - L_m\\
            &= \gamma mv^2 + mc^2\frac{1}{\gamma}\\
            &= \gamma mc^2\left\{\frac{v^2}{c^2} + \(1 - \frac{v^2}{c^2}\)\right\}\\
            &= \gamma mc^2
    \end{align*}
    となり粒子の全エネルギーになる。

\subsection{ネーターの定理}
    \begin{align*}
        \delta x^\mu &= X^\mu_\nu(x)\epsilon^\nu\\
        \delta \phi_\alpha &= M_{\nu\alpha}^\beta \phi_\beta \epsilon^\nu
    \end{align*}
    \begin{align*}
        \partial_\nu\left(\pdv{L}{\phi_{\alpha\nu}}M_{r\alpha}^\beta \phi_\beta - T^\mu_\lambda X^\lambda_r\right) = 0
    \end{align*}

\part{一般相対性理論}

\section{リーマン幾何学}

\subsection{一般座標系}
    特殊相対論では加速する座標系と重力場を表現できなかった。一般相対論では次の二つの原理から出発し、特殊相対論を加速系と重力場の存在する系に拡張する。
    \begin{itemize}
        \item 一般相対性原理: 物理法則は全ての座標系で同等
        \item 等価原理: 任意の点について、その点で慣性系となるような座標系が存在する
    \end{itemize}
    特殊相対性原理が任意の慣性系で物理法則が同等であることを仮定していたのに対し、一般相対性原理では任意の座標系で物理法則が同等であることを主張している。一般相対性原理を言い換えると、物理法則は共変形式で表されるということができる。

    等価原理は、重力の存在する系でも適当な座標変換をすれば一点では慣性系を取れるということである。このような慣性系を局所慣性系という。さらに局所慣性系$(ct, x, y, z)$において、$t$が時間、$(x, y, z)$が直交座標を表すときミンコフスキー時空となり、局所ローレンツ系という。

    これらに加え、局所慣性系では特殊相対論が成立すると仮定する。したがって特殊相対論の法則を一般座標系に拡張するには、等価原理により局所ローレンツ系へと変換して特殊相対論を適用し、元の座標系に逆変換すれば良い。

\subsection{リーマン計量}
    一般座標系$(x^\mu)$の点$p$において、局所ローレンツ系$(X^\mu)$を取り
        \[ds^2 = \eta_{\mu\nu} dX^\mu dX^\nu\]
    を考える。$ds^2$は慣性系において不変量なので、一般座標系においても不変量である。これを一般座標で表すと
    \begin{align*}
        ds^2
            &= \eta_{\alpha\beta} dX^\alpha dX^\beta\\
            &= \eta_{\alpha\beta} \left(\pdv{X^\alpha}{x^\mu}dx^\mu\right) \left(\pdv{X^\beta}{x^\nu}dx^\nu\right)\\
            &= \left(\eta_{\alpha\beta}\pdv{X^\alpha}{x^\mu}\pdv{X^\beta}{x^\nu}\right) dx^\mu dx^\nu
    \end{align*}
    である。
        \[g_{\mu\nu} = \pdv{X^\alpha}{x^\mu}\pdv{X^\beta}{x^\nu}\eta_{\alpha\beta}\]
    と置けば、時空の線素は
        \[dx^2 = g_{\mu\nu} dx^\mu dx^\nu\]
    となる。線素は微小長さに対して双線形であり、$g_{\mu\nu}$は正定値でない2階対称テンソルである。このように表される距離空間を擬リーマン多様体と呼び、$g_{\mu\nu}$をリーマン計量または計量テンソルという。

    リーマン計量は時空の歪みを表しており、重力ポテンシャルに対応する。

    計量テンソル$g_{\mu\nu}$を行列と見たとき、逆行列を$g^{\mu\nu}$、行列式を$g$と書く。つまり
    \begin{align*}
        (g^{\mu\nu}) &= (g_{\mu\nu})^{-1}\\
        g &= \det(g_{\mu\nu})
    \end{align*}
    である。以後、テンソルの添え字の上げ下げは$g_{\mu\nu}, g^{\mu\nu}$を使って行うこととする。

    $g_{\mu\nu}$の余因子を$m(\mu, \nu)$と置くと、逆行列と行列式は
    \begin{align*}
        (g^{\mu\nu}) &= \frac{1}{g}(m(\mu, \nu))\\
        g &= \sum_{\mu} g_{\mu\nu}m(\mu, \nu)
    \end{align*}
    と書ける。変分を考えると
    \begin{gather*}
        \delta(g^{\mu\lambda}g_{\lambda\nu}) = \delta g^{\mu\lambda} g_{\lambda\nu} + g^{\mu\sigma} \delta g_{\sigma\nu} = 0\\
        \delta g = m(\mu, \nu) \delta g_{\mu\nu} = g g^{\mu\nu} \delta g_{\mu\nu}
    \end{gather*}
    となる。

\subsection{レヴィ=チヴィタ接続}
    一般に多様体上ではベクトルの微分が自然に定まらない。多様体論では、計量とは別に接続係数$\Gamma^\lambda_{\mu\nu}$と呼ばれる量を定義することによって、座標変換に依らない微分を
        \[\nabla_\mu A_\nu = \partial_\mu A_\nu - \Gamma^\lambda_{\mu\nu} A_\lambda\]
    と定義する。$\nabla_\mu$を共変微分という。
    
    % 一般相対論においては、局所ローレンツ系における通常の偏微分を共変微分と定義する。ベクトル$A^\mu(x)$を考える。局所ローレンツ系$(X^\mu)$に変換すると
    % \begin{align*}
    %     A^{||}_\mu(x + dx)\\
    %     A'_\mu(X) &= \pdv{x^\nu}{X^\mu}A_\nu(x)\\
    %     A'_\mu(X + dX)
    %         &= \pdv{x^\nu}{X^\mu}\left(X + dX\right) A_\nu(x + dx)\\
    %         &= \left(\pdv{x^\nu}{X^\mu}\left(X\right) + \pdv{x^\nu}{X^\mu}{X^\lambda}dX^\lambda\right) \left(A_\nu(x) + \partial_\sigma A_\nu dx^\sigma\right)\\
    %         &= \left(\pdv{x^\nu}{X^\mu}\left(X\right) + \pdv{x^\nu}{X^\mu}{X^\lambda}dX^\lambda\right) \left(A_\nu(x) + \partial_\sigma A_\nu \pdv{x^\sigma}{X^\lambda}dX^\lambda\right)\\
    %         &\simeq \pdv{x^\nu}{X^\mu}A_\nu(x) + \pdv{x^\nu}{X^\mu}{X^\lambda} A_\nu(x) dX^\lambda + \partial_\sigma A_\nu \pdv{x^\nu}{X^\mu}(X) \pdv{x^\sigma}{X^\lambda} dX^\lambda
    % \end{align*}
    % したがって
    % \begin{align*}
    %     \nabla_\lambda A_\mu
    %         &= \partial_\sigma A_\nu \pdv{x^\nu}{X^\mu}(X) \pdv{x^\sigma}{X^\lambda} + \pdv{x^\nu}{X^\mu}{X^\lambda} A_\nu(x)\\
    % \end{align*}

    \begin{thm}
        ある点$p$について以下は同値。
        \begin{enumerate}
            \item $\Gamma^\lambda_{\mu\nu}(p) = 0$となるような座標系が存在
            \item $\Gamma^\lambda_{\mu\nu}(p) = \Gamma^\lambda_{\nu\mu}(p)$
        \end{enumerate}
    \end{thm}
    \begin{proof}
        $p$で接続係数が0となるような座標系$(X^\mu)$を取ると
            \[\Gamma^\lambda_{\mu\nu}(p) = -\pdv{x^\lambda}{X^i}(p)\pdv{X^i}{x^\mu}{x^\nu}\left(p\right)\]
        なので対称。逆に接続係数が対称であるとする。$p$を原点に取る。$a^\lambda_{\mu\nu} = -\Gamma^\lambda_{\mu\nu}(p)$として、座標系$(X^\mu)$を$x^\lambda = X^\lambda + 1/2 a^\lambda_{\mu\nu} X^\mu X^\nu$となるように取ると、
            \[X^\lambda = x^\lambda - \frac{1}{2} a^\lambda_{\mu\nu} x^\mu x^\nu + O(x^3)\]
        なので、
            \[\pdv{x^\mu}{X^\nu}\left(p\right) = \delta^\mu_\nu, \quad \pdv{X^\mu}{x^\nu}\left(p\right) = \delta^\mu_\nu, \quad \pdv{x^\sigma}{X^\mu}{X^\nu}\left(p\right) = \frac{a^\sigma_{\mu\nu} + a^\sigma_{\nu\mu}}{2} = a^\sigma_{\mu\nu}\]
        であり、
        \begin{align*}
            \Gamma'^k_{ij}(p)
                &= \pdv{X^k}{x^\lambda}\pdv{x^\mu}{X^i}\pdv{x^\nu}{X^j}\Gamma^\lambda_{\mu\nu}(p) + \pdv{X^k}{x^\sigma}\pdv{x^\sigma}{X^i}{X^j}\\
                &= \delta^k_\lambda \delta^\mu_i \delta^\nu_j \Gamma^\lambda_{\mu\nu}(p) + \delta^k_\lambda a^\sigma_{ij}\\
                &= \Gamma^k_{ij}(p) + a^k_{ij} = 0
        \end{align*}
        となる。
    \end{proof}

    一般相対論においては平行移動によってベクトルの長さは変わらない。また後に述べるように、局所慣性系は数学的には接続係数が0となるような座標系を意味する。よって計量条件と捩れがないことから、許される接続はレヴィ=チヴィタ接続のみとなる(リーマン幾何学の基本定理)。

    \begin{thm}
        レヴィ=チヴィタ接続において以下は同値。
        \begin{enumerate}
            \item $\Gamma^\lambda_{\mu\nu}(p) = 0$
            \item $\partial_\lambda g_{\mu\nu}(p) = 0$
        \end{enumerate}
    \end{thm}
    \begin{proof}
        $\Gamma^k_{ij}(p) = 0$のとき$\Gamma_{kij}(p) = g_{kl}(p)\Gamma^l_{ij}(p) = 0$である。
        \begin{align*}
            \Gamma_{ijk} &= \frac{1}{2}\(\pdv{g_{ki}}{x^j} + \pdv{g_{ij}}{x^k} - \pdv{g_{jk}}{x^i}\)\\
            \Gamma_{jki} &= \frac{1}{2}\(\pdv{g_{ij}}{x^k} + \pdv{g_{jk}}{x^i} - \pdv{g_{ki}}{x^j} \)
        \end{align*}
        辺々足すと
            \[\Gamma_{ijk} + \Gamma_{jki} = \pdv{g_{ij}}{x^k}\]
        より$\partial_k g_{ij}(p) = 0$となる。
    \end{proof}
    
    つまり局所ローレンツ系は数学的には
        \[g_{\mu\nu}(p) = \eta_{\mu\nu}, \quad \partial_\lambda g_{\mu\nu}(p) = 0\]
    と定義される。

    % $g_{ij}$は線形変換しても接続係数は0のままである。$g_{ij}(p)$は実対称行列なので直交行列で対角化可能であり、
    % \begin{gather*}
    %     a^i_k a^j_l g_{ij}(p) = \eta_{kl}\\
    %     a^i_j = \pdv{x^i}{X^j}
    % \end{gather*}
    % となる座標系$(X^i)$が存在する。

    % 等価原理より局所慣性系$(y^i)$を取ることができ
    %     \[\nabla_i f_j(p) = 0\]
    % である。一般座標系$(x^i)$における基底ベクトル$(e_i)$は
    %     \[e_j = \pdv{y^k}{x^j}f_k\]
    % なので
    % \begin{align*}
    %     \nabla_i e_j(p)
    %         &= \nabla_i\(\pdv{y^k}{x^j}f_k\)(p)\\
    %         &= \nabla_i\(\pdv{y^k}{x^j}\)(p) f_k(p) + \pdv{y^k}{x^j}(p) \nabla_i f_k(p)\\
    %         &= \nabla_i\(\pdv{y^k}{x^j}\)(p) f_k(p)\\
    %         &= \nabla_i\(\pdv{y^k}{x^j}\)(p) \pdv{x^l}{y^k}(p)e_l(p)\\
    % \end{align*}
    %     \[\Gamma^k_{ij} = \pdv{x^k}{y^l}\pdv{y^l}{x^i}{x^j}\]

% \subsection{共変微分}
%     反変ベクトル$A^i(X^j)$を座標変換したものが$a^i(x^j)$であるとする。
%     \begin{align*}
%         \pd[a^i]{x^j} &= \pd{x^j}\lr{\pd[x^i]{X^l}A^l}\\
%         &= \ppd{x^i}{x^j}{X^l}A^l
%         +\pd[x^i]{X^l}\pd[X^k]{x^j}\pd[A^l]{X^k}\\
%         \intertext{共変ベクトルも同様に、}
%         \pd[a_i]{x^j} &= \pd{x^j}\lr{\pd[X^l]{x^i}A^l}\\
%         &= \ppd{X^l}{x^i}{x^j}A_l
%         + \pd[X^l]{x^i}\pd[X^k]{x^j}\pd[A_l]{X^k}
%     \end{align*}
%     第二項だけを見ればそれぞれ混合テンソル、共変テンソルのようになっている。ローレンツ変換だけを考えていれば、座標の二階微分である第一項は0になり、微分演算子はテンソルとして振る舞う。そこで新たに、一般の座標変換に対して共変であり、慣性系において通常の微分に一致するような演算子を考える必要がある。
%     \begin{align*}
%         \na_ja^i &= \pd[a^i]{x^j}-\ppd{x^i}{x^j}{X^l}A^l\\
%         &= \pd[a^i]{x^j}-\ppd{x^i}{x^j}{X^l}\pd[X^l]{x^k}a^k\\
%         \na_ja_i &= \pd[a_i]{x^j}-\ppd{X^l}{x^i}{x^j}A_l\\
%         &= \pd[a_i]{x^j}-\ppd{X^l}{x^i}{x^j}\pd[x^k]{X^l}a_k
%     \end{align*}
%     とすれば$X$と$x$の二つの座標系間の変換で不変となる。これを共変微分という。第二項の$a^k,a_k$の係数は接続係数またはアフィン係数と呼ばれる。このままでは他の座標系に依存してしまうので$X$をデカルト座標で固定する。しかし、曲がった空間ではデカルト座標との関係が不定であり、そもそも存在を前提とするわけにはいかないので、計量を使って書き直す。ユークリッド計量を$\delta_{ij}$(クロネッカーのデルタ)とすると、
%         \[g_{ij} = \pd[X^m]{x^i}\pd[X^n]{x^j}\delta_{mn}\]
%     これを微分して、
%         \[\pd[g_{ij}]{x^k} = \delta_{mn}
%         \lr{\ppd{X^m}{x^i}{x^k}\pd[X^n]{x^j}
%         +\pd[X^m]{x^i}\ppd{X^n}{x^j}{x^k}}\]
%     計量は対称テンソルなので、添え字を巡回的に入れ替えて足し引きする。
%         \[\rec{2}\lr{\pd[g_{jk}]{x^i}+\pd[g_{ki}]{x^j}-\pd[g_{ij}]{x^k}}
%         = \delta_{mn}\ppd{X^m}{x^i}{x^j}\pd[X^n]{x^k}\]
%     とすると左辺は第一種クリストッフェル記号そのものであり$\Ga_{kij}$と書く。第二種クリストッフェル記号を$\Ga^l_{ij}$と書き、
%     \begin{align*}
%         \Ga^l_{ij} &= g^{lk}\Ga_{kij}\\
%         &= g^{lk}\delta_{mn}\ppd{X^m}{x^i}{x^j}\pd[X^n]{x^k}\\
%         &= \pd[x^l]{X^u}\pd[x^k]{X^v}\delta^{uv}
%         \delta_{mn}\ppd{X^m}{x^i}{x^j}\pd[X^n]{x^k}\\
%         \intertext{$\delta^{ij},\delta_{ij}=0(i\neq j)$なので}
%         &= \pd[x^l]{X^u}\pd[x^k]{X^u}\ppd{X^m}{x^i}{x^j}\pd[X^m]{x^k}\\
%         &= \ppd{X^m}{x^i}{x^j}\pd[x^l]{X^u}\pd[X^m]{X^u}\\
%         \intertext{$\pd[X^m]{X^u}=\delta^m_u$より}
%         &= \ppd{X^m}{x^i}{x^j}\pd[x^l]{X^m}
%     \end{align*}
%     となって共変ベクトルの共変微分の接続係数となる。一方、反変ベクトルの共変微分の接続係数は、
%     \begin{align*}
%         \ppd{x^i}{x^j}{X^l}\pd[X^l]{x^k} &= \pd{x^j}\lr{\pd[x^i]{X^l}}\pd[X^l]{x^k}\\
%         &= \pd{x^j}\lr{\pd[x^i]{X^l}\pd[X^l]{x^k}}-\pd[x^i]{X^l}\pd{x^j}\pd[X^l]{x^k}\\
%         &= \ppd{x^i}{x^j}{x^k}-\ppd{X^l}{x^j}{x^k}\pd[x^i]{X^l}\\
%         &= -\Ga^i_{jk}
%     \end{align*}
%     となる。つまり接続係数はどちらもクリストッフェル記号で表すことができる。共変微分の定義を書き直すと、
%     \begin{gather*}
%         \na_ja^i = \pd[a^i]{x^j}+\Ga^i_{jk}a^k\\
%         \na_ja_i = \pd[a_i]{x^j}-\Ga^k_{ij}a_k
%     \end{gather*}
%     となる。ミンコフスキー計量の場合接続係数は0になるので、慣性系では通常の微分に一致する。

\subsection{曲率テンソル}
    リーマン幾何学の概念をいくつか導入する。

    共変微分は非可換であり
        \[[\nabla_\mu, \nabla_\nu]A^\lambda = R^\lambda_{\sigma\mu\nu}A^\sigma\]
    と表される。$R^\lambda_{\sigma\mu\nu}$はリーマン曲率テンソルと呼ばれ、クリストッフェル記号を用いると
        \[R^\lambda_{\sigma\mu\nu} = \partial_\mu \Gamma^\lambda_{\sigma\nu} - \partial_\nu \Gamma^\lambda_{\sigma\mu} + \Gamma^\lambda_{\mu\alpha}\Gamma^\alpha_{\sigma\nu} - \Gamma^\lambda_{\nu\alpha}\Gamma^\alpha_{\sigma\mu}\]
    となる。また
        \[R_{\mu\nu} = R^\lambda_{\mu\lambda\nu} = \partial_\lambda \Gamma^\lambda_{\mu\nu} - \partial_\nu \Gamma^\lambda_{\mu\lambda} + \Gamma^\lambda_{\lambda\alpha}\Gamma^\alpha_{\mu\nu} - \Gamma^\lambda_{\nu\alpha}\Gamma^\alpha_{\mu\lambda}\]
    をリッチテンソルといい、対称テンソルである。
        \[R = g^{\mu\nu}R_{\mu\nu}\]
    をリッチスカラーまたはスカラー曲率という。これらはテンソルであり座標系に依らない量である。

    リーマン曲率テンソルはビアンキの恒等式
        \[\nabla_\lambda R^\alpha_{\beta \mu\nu} + \nabla_\mu R^\alpha_{\beta \nu\lambda} + \nabla_\nu R^\alpha_{\beta \lambda\mu} = 0\]
    を満足する。ビアンキの恒等式を変形すると
        \[\nabla_\mu \left(R_{\mu\nu} - \frac{1}{2}Rg_{\mu\nu}\right) = 0\]
    となる。
        \[G_{\mu\nu} = R_{\mu\nu} - \frac{1}{2}Rg_{\mu\nu}\]
    をアインシュタインテンソルという。

    \begin{thm}
        $g_{\mu\nu}(x) = \eta_{\mu\nu} \iff R^\lambda_{\sigma\mu\nu}(x) = 0$である。
    \end{thm}
    \begin{proof}
        必要性は自明。
    \end{proof}

\subsection{計量と重力}
    計量テンソルは重力ポテンシャルに相当し、計量テンソルの一階微分は重力場に相当する。計量テンソルとその一階微分のみからなるクリストッフェル記号は、適当な座標変換によって局所的に0にすることができるため、重力の存在は分からず「見かけの重力」を表す。しかし計量テンソルの二階微分つまり潮汐力を含むリーマン曲率テンソルは、座標変換によって不変であり「真の重力」の情報を内包している。
\section{時空の方程式}

\subsection{測地線の方程式}
    等価原理より局所ローレンツ系$(X^i)$が存在し
        \[\dv[2]{X^i}{\tau} = 0\]
    である。一般座標系$(x^i)$に変換すると
    \begin{align*}
        \dv[2]{X^l}{\tau}
            &= \dv{\tau}\(\sum_i \pdv{X^l}{x^i}\dv{x^i}{\tau}\)\\
            &= \sum_k \pdv{X^l}{x^k}\dv[2]{x^k}{\tau} + \sum_{i, j} \pdv{X^l}{x^i}{x^j}\dv{x^i}{\tau}\dv{x^j}{\tau} = 0
    \end{align*}
    両辺に$\pdv*{x^k}{X^l}$を掛けて$l$について和を取る。
        \[\dv[2]{x^k}{\tau} + \pdv{x^k}{X^l}\pdv{X^l}{x^i}{x^j}\dv{x^i}{\tau}\dv{x^j}{\tau} = 0\]
    一般座標系における接続係数が
        \[\Gamma^k_{ij} = \pdv{x^k}{X^l}\pdv{X^l}{x^i}{x^j}\]
    なので
        \[\dv[2]{x^k}{\tau} + \Gamma^k_{ij}\dv{x^i}{\tau}\dv{x^j}{\tau} = 0\]
    となる。

    つまり局所慣性系は、数学的には時空の一点で接続係数が0になるような座標系である。定理より接続係数は対称

\subsection{電磁場の方程式}
    等価原理より局所ローレンツ系$(X^i)$が存在し、電磁テンソル$F^{\mu\nu}$、四元電流密度$j^\mu$とすると
    \begin{gather*}
        \partial_\nu F^{\mu\nu} = \mu_0 j^\mu\\
        \partial_\rho F_{\mu\nu} + \partial_\mu F_{\nu\rho} + \partial_\nu F_{\rho\mu} = 0
    \end{gather*}
    が成り立つ。ここで$F^{\mu\nu} = \eta^{\mu\alpha}\eta^{\nu\beta}F_{\alpha\beta}$である。これを一般座標系$(x^i)$に変換すると
    \begin{gather*}
        \nabla_\nu F^{\mu\nu} = \mu_0 j^\mu\\
        \nabla_\rho F_{\mu\nu} + \nabla_\mu F_{\nu\rho} + \nabla_\mu F_{\nu\rho} = 0\\
        F^{\mu\nu} = g^{\mu\alpha}g^{\nu\beta}F_{\alpha\beta}
    \end{gather*}
    となる。

    $F_{\mu\nu} = \nabla_\alpha A_\beta - \nabla_\beta A_\alpha$を代入すると
    \begin{align*}
        g^{\mu\alpha}g^{\nu\beta}\nabla_\nu(\nabla_\alpha A_\beta - \nabla_\beta A_\alpha)
            &= g^{\mu\alpha}\nabla_\nu\nabla_\alpha A^\nu - g^{\nu\beta}\nabla_\nu\nabla_\beta A^\mu\\
            &= g^{\mu\alpha}(\nabla_\nu\nabla_\alpha - \nabla_\alpha\nabla_\nu)A^\nu + g^{\mu\alpha}\nabla_\alpha\nabla_\nu A^\nu - g^{\mu\alpha}g^{\nu\beta}\nabla_\beta A^\mu\\
            &= g^{\mu\alpha}R^\nu_{\lambda\nu\alpha}A^\lambda + g^{\mu\alpha}\nabla_\alpha(\nabla_\nu A^\nu) - \square A^\mu\\
            &= R^\mu_\nu A^\nu + g^{\mu\alpha}\nabla_\alpha(\nabla_\nu A^\nu) - \square A^\mu
    \end{align*}
    より電磁ポテンシャルによるマクスウェル方程式は
        \[R^\mu_\nu A^\nu + g^{\mu\alpha}\nabla_\alpha(\nabla_\nu A^\nu) - \square A^\mu = \mu_0 j^\mu\]
    となる。またローレンツゲージ$\nabla_\nu A^\nu = 0$を仮定すると
        \[R^\mu_\nu A^\nu - \square A^\mu = \mu_0 j^\mu\]
    である。

\subsection{重力場の方程式}
    時空の計量と物質分布の関係を導出する。一般相対性理論における重力場の方程式は次の条件の下でニュートン力学に一致するはずである。
    \begin{itemize}
        \item 質点の速さは光速に比べて十分遅い
        \item 重力場は十分弱い
        \item 重力場の時間変化は十分小さい
    \end{itemize}

    時空の計量は物質の分布つまりエネルギー運動量テンソルに依存するはずである。$\nabla_jT_{ij} = 0$だから、計量に関係する量$X_{ij}$で、$X_{ij} = T_{ij}$となるようなものがあれば、$\nabla_jX_{ij} = 0$を満たす。アインシュタインテンソル$G_{ij}$や計量テンソル$g_{ij}$はそのような性質を持っている。ニュートン力学における重力場の方程式はポアソン方程式
        \[\Delta\phi = 4\pi G\rho\]
    である。$\phi$の二階微分は$g_{00}$の二階微分に比例するので、$X_{ij}$は計量の二階微分を含まなければならない。実は、$g_{ij}$の$x$に関する0, 1, 2階微分を含み、$\pdv*[2]{g_{ij}}{x}$の一次式であり、その共変微分が0となるような二階共変テンソルは$G_{ij}$と$g_{ij}$の線形結合に限ることが分かっている。つまり、
        \[G_{ij} + \Lambda g_{ij} = \kappa T_{ij}\]
    である。両辺に$g^{ij}$を掛けて縮約する。
        \[g^{ij}R_{ij} - \frac{1}{2}Rg^{ij}g_{ij} + \Lambda g^{ij}g_{ij} = \kappa g^{ij}T_{ij}\]
    $g^{ij}g_{ij}$はクロネッカーのデルタのトレースなので4である。
    \begin{gather*}
        R - 2R + 4\Lambda = \kappa T\\
        R = 4\Lambda - \kappa T
    \end{gather*}
    これを元の式に代入して、
    \begin{gather*}
        R_{ij} - \frac{1}{2}(4\Lambda - \kappa T)g_{ij} + \Lambda g_{ij} = \kappa T_{ij}\\
        R_{ij} - \Lambda g_{ij} = \kappa\(T_{ij} + \frac{1}{2}Tg_{ij}\)
    \end{gather*}
    である。

    まず加速度系における計量を求める。慣性系$K$に対して加速度系$K'$が$x$軸方向に加速度$g$で進んでいるとする。時刻$t = 0$で両者が一致しているとすれば、十分小さい$t$について、
    \begin{align*}
        t' &= t\\
        x' &= x-\frac{1}{2}gt^2\\
        y' &= y\\
        z' &= z
    \end{align*}
    である。従って慣性系ではミンコフスキー計量なので、
    \begin{align*}
        ds^2 &= -c^2dt^2 + dx^2 + dy^2 + dz^2\\
             &= -c^2dt'^2 + (dx' + gtdt')^2 + dy'^2 + dz'^2\\
             &= \(-1 + \frac{(gt)^2}{c^2}\)(cdt')^2 + 2gtdt'dx' + dx'^2 + dy'^2 + dz'^2
    \end{align*}
    $K'$系の原点では$x = \frac{1}{2}gt'^2$、重力ポテンシャルは$\phi = -gx$なので
    \begin{align*}
        ds^2 &= \(-1 + \frac{2gx}{c^2}\)(cdt')^2 + 2\sqrt{2gx}dt'dx' + dx'^2 + dy'^2 + dz'^2\\
             &= \(-1 - \frac{2\phi}{c^2}\)(cdt')^2 - 2\sqrt{-2\phi}dt'dx' + dx'^2 + dy'^2 + dz'^2
    \end{align*}
    重力ポテンシャルはスカラーなので、加速の方向に依らず$g_{00}= -1 - 2\phi / c^2$である。

    次にリッチテンソルを計算する。$g_{ij} = \eta_{ij} + h_{ij}$とおく。$h_{ij}$は十分小さいとして$h_{ij}$の二次の項を無視する。クリストッフェル記号は、
    \begin{align*}
        \Gamma^k_{ij}
            &= \frac{1}{2}g^{kl}\(\pdv{g_{lj}}{x^i} + \pdv{g_{li}}{x^j} - \pdv{g_{ij}}{x^l}\)\\
            &= \frac{1}{2}\eta_{kl}\(\pdv{h_{lj}}{x^i} + \pdv{h_{li}}{x^j} - \pdv{h_{ij}}{x^l}\)
    \end{align*}
    $h_{00} = -2\phi / c^2$なので
    \begin{align*}
        \Gamma^k_{00}
            &= \frac{1}{2}\eta_{kl}\(\pdv{h_{l0}}{x^0} + \pdv{h_{l0}}{x^0} - \pdv{h_{00}}{x^l}\)\\
            &= -\frac{1}{2}\eta_{kl}\pdv{h_{00}}{x^l}
    \end{align*}
    である。リッチテンソルは後ろの二項が$h_{ij}$の二次になるので無視して
    \begin{align*}
        R_{00}  &= R^k_{0k0}\\
                &= \pdv{\Gamma^k_{00}}{x^k} - \pdv{\Gamma^k_{0k}}{x^0}\\
                &= \pdv{\Gamma^k_{00}}{x^k}\\
                &= -\frac{1}{2}\eta_{kl}\pdv{h_{00}}{x^k}{x^l}\\
                &= -\frac{1}{2}\partial^i\partial_ih_{00}\\
                &= -\frac{1}{2}\Delta h_{00}
    \end{align*}

    次にエネルギー運動量テンソルを計算する。$T_{ij} = \rho u_iu_j$だが、速さが十分小さいので$T_{00} = \rho c^2$で残りの成分は全て無視する。よって$T = g_{ij}T^{ij} = g_{00}\rho c^2$である。したがって方程式は、
    \begin{align*}
        R_{ij} - \Lambda g_{ij} = \kappa\(T_{ij} + \frac{1}{2}Tg_{ij}\)\\
        -\frac{1}{2}\Delta h_{00} - \Lambda g_{00} = \kappa\rho c^2\(1 - \frac{1}{2}g_{00}^2\)
    \end{align*}
    $h_{00} = -2\phi / c^2, g_{00} = -1$を代入すれば、
    \begin{align*}
        \frac{1}{c^2}\Delta\phi + \Lambda = \frac{c^2}{2}\kappa\rho
    \end{align*}
    ポアソン方程式$\Delta\phi = 4\pi G\rho$と比較すると、$\Lambda = 0, \kappa = 8\pi G / c^4$である。つまり、
        \[R_{ij} - \frac{1}{2}Rg_{ij} = \frac{8\pi G}{c^4}T_{ij}\]
    これを重力場の方程式またはアインシュタイン方程式と呼ぶ。

\subsection{弱い重力場}
    弱い重力場$g_{\mu\nu} = \eta_{\mu\nu} + h_{\mu\nu}$ ($|h_{ij}| \ll 1$)において、アインシュタイン方程式を$h_{\mu\nu}$について線形化する。

    クリストッフェル記号は前節と同様
    \begin{align*}
        \Gamma^\lambda_{\mu\nu}
            &= \frac{1}{2}g^{\lambda\sigma}(g_{\nu\sigma, \mu} + g_{\sigma\mu, \nu} - g_{\mu\nu, \sigma})\\
            &= \frac{1}{2}\eta^{\lambda\sigma}(h_{\nu\sigma, \mu} + h_{\sigma\mu, \nu} - h_{\mu\nu, \sigma})
    \end{align*}
    リーマン曲率テンソルは
    \begin{align*}
        R^\lambda_{\sigma\mu\nu}
            &= \partial_\mu \Gamma^\lambda_{\sigma\nu} - \partial_\nu \Gamma^\lambda_{\sigma\mu}\\
            &= \partial_\mu\left(\frac{1}{2}\eta^{\lambda\alpha}(h_{\nu\alpha, \sigma} + h_{\alpha\sigma, \nu} - h_{\sigma\nu, \alpha})\right) - \partial_\nu\left(\frac{1}{2}\eta^{\lambda\alpha}(h_{\mu\alpha, \sigma} + h_{\alpha\sigma, \mu} - h_{\sigma\mu, \alpha})\right)\\
            &= \partial_\mu\left(\frac{1}{2}\eta^{\lambda\alpha}(h_{\nu\alpha, \sigma} - h_{\sigma\nu, \alpha})\right) - \partial_\nu\left(\frac{1}{2}\eta^{\lambda\alpha}(h_{\mu\alpha, \sigma} - h_{\sigma\mu, \alpha})\right)
    \end{align*}
    リッチテンソルは
    \begin{align*}
        R_{\mu\nu}
            &= R^\lambda_{\mu\lambda\nu}\\
            &= \partial_\lambda\left(\frac{1}{2}\eta^{\lambda\alpha}(h_{\nu\alpha, \mu} - h_{\mu\nu, \alpha})\right) - \partial_\nu\left(\frac{1}{2}\eta^{\lambda\alpha}(h_{\lambda\alpha, \mu} - h_{\mu\lambda, \alpha})\right)\\
            &= \frac{1}{2}\eta^{\lambda\alpha}(h_{\nu\alpha, \mu\lambda} - h_{\mu\nu, \alpha\lambda} - h_{\lambda\alpha, \mu\nu} + h_{\mu\lambda, \alpha\nu})
    \end{align*}
    スカラー曲率は
    \begin{align*}
        R   &= g^{\mu\nu}R_{\mu\nu}\\
            &= \frac{1}{2}\eta^{\mu\nu}\eta^{\lambda\alpha}(h_{\nu\alpha, \mu\lambda} - h_{\mu\nu, \alpha\lambda} - h_{\lambda\alpha, \mu\nu} + h_{\mu\lambda, \alpha\nu})\\
            &= \partial^\alpha \partial^\beta h_{\alpha\beta} - \square h
    \end{align*}
    アインシュタインテンソルは
    \begin{align*}
        G_{\mu\nu}
            &= R_{\mu\nu} - \frac{1}{2}Rg_{\mu\nu}\\
            &= \frac{1}{2}\eta^{\lambda\alpha}(h_{\nu\alpha, \mu\lambda} - h_{\mu\nu, \alpha\lambda} - h_{\lambda\alpha, \mu\nu} + h_{\mu\lambda, \alpha\nu})\\
                &\quad - \frac{1}{2}(\partial^\alpha \partial^\beta h_{\alpha\beta} - \square h)\eta_{\mu\nu}\\
    \end{align*}

\subsection{重力場の作用}
    アインシュタインテンソルはゲージ変換に対して不変である。

    重力場の作用は
        \[S_g[g_{\mu\nu}(x)] = \int \frac{c^4}{16\pi G}(R - 2\Lambda) \sqrt{-g}dx^4\]
    であり、アインシュタイン=ヒルベルト作用と呼ばれている。これの変分を計算する。まず
        
    \begin{align*}
        \delta g
            &= g g^{\mu\nu} \delta g_{\mu\nu}\\
        \delta \sqrt{-g}
            &= \frac{1}{2}\sqrt{-g}g^{\mu\nu}\delta g_{\mu\nu}\\
            &= -\frac{1}{2}\sqrt{-g}g_{\mu\nu}\delta g^{\mu\nu}\\
        \delta R_{\mu\nu}
            &= \delta(\partial_\lambda \Gamma^\lambda_{\mu\nu} - \partial_\nu \Gamma^\lambda_{\mu\lambda} + \Gamma^\lambda_{\lambda\alpha}\Gamma^\alpha_{\mu\nu} - \Gamma^\lambda_{\nu\alpha}\Gamma^\alpha_{\mu\lambda})\\
            &= \partial_\lambda \delta\Gamma^\lambda_{\mu\nu} - \partial_\nu \delta\Gamma^\lambda_{\mu\lambda} + \delta\Gamma^\lambda_{\lambda\alpha} \Gamma^\alpha_{\mu\nu} + \Gamma^\lambda_{\lambda\alpha} \delta\Gamma^\alpha_{\mu\nu} - \delta\Gamma^\lambda_{\nu\alpha} \Gamma^\alpha_{\mu\lambda} - \Gamma^\lambda_{\nu\alpha} \delta\Gamma^\alpha_{\mu\lambda}\\
            &= (\partial_\lambda \delta\Gamma^\lambda_{\mu\nu} + \Gamma^\lambda_{\lambda\alpha} \delta\Gamma^\alpha_{\mu\nu} - \Gamma^\alpha_{\mu\lambda} \delta\Gamma^\lambda_{\nu\alpha} - \Gamma^\lambda_{\nu\alpha} \delta\Gamma^\alpha_{\mu\lambda}) - (\partial_\nu \delta\Gamma^\lambda_{\mu\lambda} - \Gamma^\alpha_{\mu\nu} \delta\Gamma^\lambda_{\lambda\alpha})\\
            &= \nabla_\lambda \delta\Gamma^\lambda_{\mu\nu} - \nabla_\nu \delta\Gamma^\lambda_{\mu\lambda}\\
        \delta R
            &= \delta g^{\mu\nu} R_{\mu\nu} + g^{\mu\nu} \delta R_{\mu\nu}\\
            &= R_{\mu\nu} \delta g^{\mu\nu} + g^{\mu\nu} (\nabla_\lambda \delta\Gamma^\lambda_{\mu\nu} - \nabla_\nu \delta\Gamma^\lambda_{\mu\lambda})\\
            &= R_{\mu\nu} \delta g^{\mu\nu} +  \nabla_\lambda (g^{\mu\nu} \delta\Gamma^\lambda_{\mu\nu}) - \nabla_\nu(g^{\mu\nu} \delta\Gamma^\lambda_{\mu\lambda})\\
            &= R_{\mu\nu} \delta g^{\mu\nu} +  \nabla_\lambda(g^{\mu\nu} \delta\Gamma^\lambda_{\mu\nu} - g^{\mu\lambda} \delta\Gamma^\nu_{\mu\nu})\\
        \delta S_g
            &= \frac{c^3}{16\pi G} \int \delta\left[(R - 2\Lambda)\sqrt{-g}\right] dx^4\\
            &= \frac{c^3}{16\pi G} \int (\delta R \sqrt{-g} - (R - 2\Lambda)\delta \sqrt{-g}) dx^4\\
            &= \frac{c^3}{16\pi G} \int \left(R_{\mu\nu} \delta g^{\mu\nu} \sqrt{-g} - (R - 2\Lambda)\frac{1}{2}\sqrt{-g}g_{\mu\nu}\delta g^{\mu\nu}\right) dx^4\\
            &= \frac{c^3}{16\pi G} \int \left(R_{\mu\nu} - \frac{1}{2}Rg^{\mu\nu} + \Lambda g^{\mu\nu}\right) \delta g^{\mu\nu} \sqrt{-g} dx^4
    \end{align*}
    となる。全体の作用
        \[S = S_m + S_g\]
    の変分が0となるので、
    \begin{gather*}
        \begin{aligned}
            \delta S
                &= -\frac{1}{2c} \int T_{\mu\nu} \delta g^{\mu\nu} \sqrt{-g} dx^4\\
                &\quad + \frac{c^3}{16\pi G} \int \left(R_{\mu\nu} - \frac{1}{2}Rg^{\mu\nu} + \Lambda g^{\mu\nu}\right) \delta g^{\mu\nu} \sqrt{-g} dx^4 = 0\\
        \end{aligned}\\
        R_{\mu\nu} - \frac{1}{2}Rg_{\mu\nu} + \Lambda g_{\mu\nu} = \frac{8\pi G}{c^4}T_{\mu\nu}
    \end{gather*}
    となりアインシュタイン方程式が導かれる。
\section{変分原理}
\subsection{変分原理}
    同じことを変分原理からも導出する。特殊相対論における作用は
    \begin{align*}
        S &= S_m + S_{qm} + S_{em}\\
        S_m    &= \int -mc\sqrt{-\eta_{\mu\nu} u^\mu u^\nu} d\tau\\
        S_{qm} &= \int q u^\mu A_\mu d\tau\\
        S_{em} &= \int -\frac{1}{4\mu_0}F^{\mu\nu}F_{\mu\nu} + A^\mu j_\mu d(ct)dxdydz
    \end{align*}
    であった。これらを一般座標系に変換すると
    \begin{align*}
        S &= S_m + S_{qm} + S_{em}\\
        S_m    &= \int -mc\sqrt{-g_{\mu\nu} u^\mu u^\nu} d\tau\\
        S_{qm} &= \int q u^\mu A_\mu d\tau\\
        S_{em} &= \int \left[-\frac{1}{4\mu_0}F^{\mu\nu}F_{\mu\nu} + A^\mu j_\mu\right] \sqrt{-g}dx^4
    \end{align*}
    となる。

    自由粒子の軌道は計量$-g_{\mu\nu}$の導入された四次元擬リーマン多様体上の測地線である。つまり物体の運動方程式はこの擬リーマン多様体における測地線の方程式となる。$-g_{\mu\nu}$のクリストッフェル記号が$g_{\mu\nu}$と同じなので、
        \[\dv[2]{x^\lambda}{\tau} + \Gamma^\lambda_{\mu\nu}\dv{x^\mu}{\tau}\dv{x^\nu}{\tau} = 0\]
    となる。

\subsection{エネルギーと運動量}

\subsection{重力場のゲージ変換}
    無限小の座標変換
        \[x^\mu \mapsto x'^\mu = x^\mu + \xi(x)\]
    を考える。リーマン計量は
    \begin{align*}
        g'_{\mu\nu}(x') &= \pdv{x^\alpha}{x'^\mu}\pdv{x^\beta}{x'^\nu}g_{\alpha\beta}(x)\\
        g'_{\mu\nu}(x + \xi) &= (\delta_\mu^\alpha - \partial_\mu \xi^\alpha)(\delta_\nu^\beta - \partial_\nu \xi^\beta)g_{\alpha\beta}(x)\\
        g'_{\mu\nu}(x) + \partial_\lambda g'_{\mu\nu}\xi^\lambda &= g_{\mu\nu}(x) - \partial_\mu \xi^\alpha g_{\alpha\nu} - \partial_\nu \xi^\beta g_{\mu\beta}\\
        g'_{\mu\nu}(x) + \partial_\lambda g'_{\mu\nu}\xi^\lambda &= g_{\mu\nu}(x) - (\partial_\mu(g_{\alpha\nu}\xi^\alpha) - \xi^\alpha \partial_\nu g_{\alpha\nu}) - (\partial_\nu(g_{\mu\beta}\xi^\beta) - \xi^\beta \partial_\mu g_{\mu\beta})\\
        g'_{\mu\nu}(x) &= g_{\mu\nu}(x) - \partial_\mu \xi_\nu - \partial_\nu \xi_\mu\\
            &\quad + (\partial_\mu g_{\mu\sigma} + \partial_\nu g_{\nu\sigma} - \partial_\sigma g_{\mu\nu})g^{\lambda\sigma}\xi_\lambda\\
        g'_{\mu\nu}(x) &= g_{\mu\nu}(x) - (\partial_\mu\xi_\nu - \Gamma^\lambda_{\mu\nu}\xi_\lambda) - (\partial_\nu\xi_\mu - \Gamma^\lambda_{\mu\nu}\xi_\lambda)\\
            &= g_{\mu\nu}(x) - \nabla_\mu\xi_\nu - \nabla_\nu\xi_\mu
    \end{align*}
    となる。
        \[g_{\mu\nu}(x) \mapsto g'_{\mu\nu}(x) = g_{\mu\nu}(x) - \nabla_\mu\xi_\nu - \nabla_\nu\xi_\mu\]
    をゲージ変換という。



\subsection{エネルギー・運動量テンソル}
    \begin{align*}
        L_\text{int}
    \end{align*}
\section{天体}

\subsection{シュバルツシルト解}
    定常で球対称な天体を考える。外部に物質は存在せず、時空は無限遠で平坦に近づくとする。計量も球対称なので、時間方向の変位$dw^2$、動径方向の変位$dr^2$、球面上の距離$r^2d\theta^2 + r^2\sin^2\theta d\phi^2$だけに依存し、係数は半径のみに依存する。三次元極座標$(x^0, x^1, x^2, x^3) = (w, r, \theta, \phi)$を用いると
        \[ds^2 = -A(r)dw^2 + B(r)dr^2 + C(r)r^2(d\theta^2 + \sin^2\theta d\phi^2)\]
    となる。更に動径方向のスケールを$r \to r / \sqrt{C(r)}$と変更することで$C(r)$は1にできる。ミンコフスキー計量は、$-dw^2 + dx^2 + dy^2 + dz^2 = -dw^2 + dr^2 + r^2(d\theta^2 + \sin^2\theta d\phi^2)$なので、$r \to \infty$で$A(r), B(r) \to 1$となる。つまり
    \begin{align*}
        g_{00} = -A(r), g_{11} = B(r), g_{22} = r^2, g_{33} = r^2\sin^2\theta
    \end{align*}
    である。また
    \begin{align*}
        A(r) &= e^{\nu(r)}\\
        B(r) &= e^{\lambda(r)}
    \end{align*}
    と置く。まずクリストッフェル記号は
        \[\chr{i}{jk} = \frac{1}{2}g^{il}\(\pdv{g_{kl}}{x^j} + \pdv{g_{jl}}{x^k} - \pdv{g_{jk}}{x^l}\)\]
    であった。$l = i$のときだけを考えれば良いので、
    \begin{align*}
        \chr{0}{01} &= \chr{0}{10} = \frac{1}{2}e^{-\nu(r)}\(\pdv{g_{10}}{x^0} + \pdv{g_{00}}{x^1}\) = \frac{1}{2}\nu'\\
        \chr{1}{00} &= \frac{1}{2}e^{\nu - \lambda}\nu'\\
        \chr{1}{11} &= \frac{1}{2}\lambda'\\
        \chr{1}{22} &= -re^{-\lambda}\\
        \chr{1}{33} &= -re^{-\lambda}\sin^2\theta\\
        \chr{2}{12} &= \chr{2}{21} = \frac{1}{r}\\
        \chr{2}{33} &= -\sin\theta\cos\theta\\
        \chr{3}{13} &= \chr{3}{31} = \frac{1}{r}\\
        \chr{3}{23} &= \chr{3}{32} = \frac{1}{\tan\theta}
    \end{align*}
    その他は全て0である。

    まず天体外部の時空の計量について考える。これをシュバルツシルトの外部解という。天体の外部ではエネルギー運動量テンソルは0なのでリッチテンソルも0である。リーマン曲率テンソルは
        \[\ricci{i}{jkl} = \partial_k\Gamma^i_{jl} - \partial_l\Gamma^i_{jk} + \Gamma^m_{jl}\Gamma^i_{km} - \Gamma^m_{jk}\Gamma^i_{lm}\]
    結果だけを示すとリッチテンソルは、
    \begin{align*}
        e^{\lambda - \nu}R_{00} &= \frac{1}{2}\nu'' - \frac{1}{4}\nu'\lambda' + \frac{1}{4}\nu'^2 + \frac{\nu'}{r} = 0\\
        R_{11} &= - \frac{1}{2}\nu'' + \frac{1}{4}\nu'\lambda' - \frac{1}{4}\nu'^2 + \frac{\lambda'}{r} = 0\\
        R_{22} &= 1 - \frac{1}{2}e^{-\lambda}(r\nu' - r\lambda' + 2) = 0\\
        R_{33} &= R_{22}\sin^2\theta = 0
    \end{align*}
    その他は全て0である。あとはこの微分方程式を解けば良い。第一式と第二式を足すと
        \[\frac{\nu' + \lambda'}{r} = 0\]
    より$\nu + \lambda$は定数である。$r \to \infty$で$A(r), B(r) \to 1$なので結局$\nu + \lambda = 0$となる。第三式を書き直すと
        \[\left[1 - \frac{1}{2}r(\lambda' - \nu')\right]e^{-\lambda} = 1\]
    $\lambda' - \nu' = 2\lambda'$より
    \begin{align*}
        (1 - r\lambda')e^{-\lambda} &= 1\\
        (re^{-\lambda})' &= 1
    \end{align*}
    $a$を定数として
        \[re^{-\lambda} = r - a\]
    したがって
    \begin{align*}
        A(r) &= e^{\nu(r)} = 1 - \frac{a}{r}\\
        B(r) &= e^{\lambda(r)} = \(1 - \frac{a}{r}\)^{-1}
    \end{align*}
    である。つまり
        \[g_{00} = -A(r) = -1 + \frac{a}{r}\]
    ニュートン重力では
        \[g_{00} \simeq -1 - \frac{2\phi}{c^2} = -1 + \frac{2GM}{rc^2}\]
    なので
        \[a = \frac{2GM}{c^2}\]
    この$a$をシュバルツシルト半径と呼ぶ。したがってシュバルツシルト計量は、
        \[ds^2 = -\(1 - \frac{a}{r}\)dw^2 + \(1 - \frac{a}{r}\)^{-1}dr^2 + r^2d\theta^2 + r^2\sin^2\theta d\phi^2\]
    となる。
    
    バーコフの定理とは、真空場において球対称解は静的で漸近的平坦であるという定理である。つまり球対称な物質分布が時間変化しても計量は変化せず、重力波を放出しない。つまり外部解はシュバルツシルト解である。球対称に拍動する星は重力波を放出しない。また球殻の内部はミンコフスキー計量によって与えられる。これは球対称に分布する星の内部では重力は厳密に相殺するということであり、ニュートンの球殻定理(Newton's shell theorem)に対応するものである。

\subsection{物体の軌道}
    シュバルツシルト計量から物体の運動を導出するには、シュバルツシルト解の導出の際に求めたおいた接続係数から測地線の方程式を解けば良い。しかし測地線の方程式は変数が混在しているので解くのが煩雑になる。そこで、質量を持つ物体については固有時をパラメータとして使えるので、オイラー=ラグランジュ方程式を解く。

    ラグランジアン
    \begin{align*}
        L   &= -mc \dv{\tau} \sqrt{\(1 - \frac{a}{r}\)dw^2 - \(1 - \frac{a}{r}\)^{-1}dr^2 - r^2(d\theta^2 + \sin^2\theta d\phi^2)}\\
            &= -mc \sqrt{\(1 - \frac{a}{r}\)c^2\dot{t}^2 - \(1 - \frac{a}{r}\)^{-1}\dot{r}^2 - r^2(\dot{\theta}^2 + \sin^2\theta\dot{\phi}^2)}\\
            &= -m \left\{\(1 - \frac{a}{r}\)c^2\dot{t}^2 - \(1 - \frac{a}{r}\)^{-1}\dot{r}^2 - r^2(\dot{\theta}^2 + \sin^2\theta\dot{\phi}^2)\right\}
    \end{align*}
    からオイラー=ラグランジュ方程式を導く。$t, \theta, \phi$については
    \begin{align*}
        \pdv{L}{\dot{t}} &= -2mc^2\(1 - \frac{a}{r}\)\dot{t}\\
        \pdv{L}{\dot{\theta}} &= 2mr^2\dot{\theta}\\
        \pdv{L}{\dot{\phi}} &= 2mr^2\sin^2\theta\dot{\phi}
    \end{align*}
    物体は原点を中心とする平面上を運動するので、$\theta = \pi / 2$として良い。右辺は定数となるので
    \begin{align*}
        \(1 - \frac{a}{r}\)\dot{t} &= \epsilon\\
        r^2\dot{\phi} &= h
    \end{align*}
    とおく。第二式は角運動量の保存を表す。$r$については計量の式から
    \begin{gather*}
        \begin{aligned}
            \(1 - \frac{a}{r}\)c^2\dot{t}^2 - \(1 - \frac{a}{r}\)^{-1}\dot{r}^2 - r^2\dot{\phi}^2 &= c^2\\
            \(1 - \frac{a}{r}\)^2c^2\dot{t}^2 - \(\dv{r}{\tau}\)^2 - \(1 - \frac{a}{r}\)r^2\(\dv{\phi}{\tau}\)^2 &= \(1 - \frac{a}{r}\)c^2
        \end{aligned}\\
        \begin{aligned}
            \(\dv{r}{\tau}\)^2 + r^2\(\dv{\phi}{\tau}\)^2
                &= c^2\epsilon^2 - \(1 - \frac{a}{r}\)c^2 + ar\dot{\phi}^2\\
                &= c^2(\epsilon^2 - 1) + \frac{2GM}{r} + \frac{2GM}{c^2}\frac{h^2}{r^3}
        \end{aligned}
    \end{gather*}
    ここで
    \begin{align*}
        E &= \frac{mc^2(\epsilon^2 - 1)}{2}\\
        U &= -\frac{GMm}{r}\(1 + \frac{h^2}{r^2c^2}\)
    \end{align*}
    とおけば
        \[\frac{1}{2}m \left\{\(\dv{r}{\tau}\)^2 + r^2\(\dv{\phi}{\tau}\)^2\right\} + U(r) = E\]
    第一項は運動エネルギーの極座標表示なので、エネルギー保存則を表す。ポテンシャルエネルギーの第二項が一般相対論的な効果によるもので、ニュートン重力に加えて距離の4乗に反比例する引力が働くことになる。改めて軌道の方程式を書くと
    \begin{gather*}
        \(\dv{r}{\tau}\)^2 + r^2\(\dv{\phi}{\tau}\)^2 = c^2(\epsilon^2 - 1) + \frac{2GM}{r} + \frac{2GMh^2}{r^3c^2}\\
        r^2\dv{\phi}{\tau} = h
    \end{gather*}
    第一式を第二式の二乗で割れば
        \[\frac{1}{r^4}\(\dv{r}{\phi}\)^2 + \frac{1}{r^2} = \frac{c^2(\epsilon^2 - 1)}{h^2} + \frac{2GM}{h^2r} + \frac{2GM}{r^3c^2}\]
    $u = 1 / r$とおけば
        \[\dv{r}{\phi} = -\frac{1}{u^2}\dv{u}{\phi}\]
    なので
        \[\(\dv{u}{\phi}\)^2 + u^2 = \frac{c^2(\epsilon^2 - 1)}{h^2} + \frac{2GM}{h^2}u + \frac{2GM}{c^2}u^3\]
    両辺を$\phi$で微分して
        \[2\dv{u}{\phi}\dv[2]{u}{\phi} + 2u\dv{u}{\phi} = \frac{2GM}{h^2}\dv{u}{\phi} + \frac{6GM}{c^2}u^2\dv{u}{\phi}\]
    両辺を$2\dv*{u}{\theta}$で割ると
    \begin{align*}
        \dv[2]{u}{\phi} &+ u = \frac{GM}{h^2} + \frac{3GM}{c^2}u^2\\
        \dv[2]{u}{\phi} &= \frac{GM}{h^2} - u + \frac{3GM}{c^2}u^2
    \end{align*}
    となる。ニュートン力学におけるケプラー運動の方程式は
        \[\dv[2]{u}{\phi} = \frac{GM}{h^2} - u\]
    であり、原点との距離が近付くほど一般相対論的効果が大きくなる。太陽系の場合、太陽に最も近い惑星である水星の近日点移動に関して、ニュートン力学から計算される数値との間に無視できない誤差が生まれる。この方程式の解は半直弦$l = h^2/GM$、離心率$e$と近点$\alpha$\footnote{$e, \alpha$が積分定数となる。}を用いて
        \[u = \frac{1 + e\cos(\phi - \alpha)}{l}\]
    である。一般相対論的摂動により近点が一周期当たり$\Delta$($\ll 1$)進むとすると、解は
        \[u = \frac{1 + e\cos(1 - \Delta / 2\pi)\phi}{l}\]
    と書ける。相対論的な軌道方程式に代入すると
    \begin{align*}
        \dv[2]{u}{\phi}
            &= -\frac{e}{l}\(1 - \frac{\Delta}{2\pi}\)^2\cos\(1 - \frac{\Delta}{2\pi}\)\phi\\
        \frac{1}{l} - u + \frac{3h^2}{lc^2}u^2
            &= \frac{1}{l} - \frac{1 + e\cos(1 - \Delta / 2\pi)\phi}{l} + \frac{3h^2}{lc^2}\frac{1 + 2e \cos(1 - \Delta / 2\pi)\phi + e^2\cos^2(1 - \Delta / 2\pi)\phi}{l^2}\\
            &= \frac{3h^2}{l^3c^2} - \frac{e}{l}\(1 - \frac{6h^2}{l^2c^2}\)\cos\(1 - \frac{\Delta}{2\pi}\)\phi + \frac{3h^2}{l^3c^2}e^2\cos^2\(1 - \frac{\Delta}{2\pi}\)\phi
    \end{align*}
    $\cos(1 - \Delta / 2\pi)\phi$の係数を比較すると
        \[\(1 - \frac{\Delta}{2\pi}\)^2 = 1 - \frac{6h^2}{l^2c^2}\]
    $\Delta^2$の項を無視すると
    \begin{align*}
        -\frac{\Delta}{\pi} &= -\frac{6h^2}{l^2c^2}\\
        \Delta &= \frac{6\pi h^2}{l^2c^2} = \frac{6\pi GM}{lc^2}
    \end{align*}
    水星の場合、$l = 5.5 \times 10^{10} \mathrm{m}$で公転周期は0.2409年なので、100年当たりの近日点移動は
        \[\Delta \times \frac{360}{2\pi} \times 3600 \times \frac{100}{0.2409} \simeq 43\]
    で約43秒角となる。この値はニュートン力学から計算される近日点移動と実際の観測の差に一致した。

\subsection{物体の落下}
    上で導出した式
        \[\(\dv{r}{\tau}\)^2 + r^2\(\dv{\phi}{\tau}\)^2 = c^2(\epsilon^2 - 1) + \frac{2GM}{r} + \frac{2GMh^2}{r^3c^2}\]
    で角運動量を0とすると
        \[\(\dv{r}{\tau}\)^2 = c^2(\epsilon^2 - 1) + c^2\frac{a}{r}\]
    である。質量を持つ物体は静止した状態を考えることができる。$r_1 < r_2$について$r = r_2$で静止している状態から$r = r_1$に落下するまでの固有時間を求める。
        \[(\epsilon^2 - 1) + \frac{a}{r_2} = 0\]
    だから落下時間は
    \begin{align*}
        \Delta\tau &= \frac{1}{c} \int_{r_1}^{r_2} \left\{\frac{a}{r} - \frac{a}{r_2}\right\}^{-1/2} dr\\
    \end{align*}
    $r = r_2\cos^2\theta$ ($0 \leq \theta \leq \pi/2$)と置換すると
        \[\dv{r}{\theta} = -r_2\sin 2\theta\]
    なので、$\theta \colon \theta_1 \to \theta_2 = 0$で$r \colon r_1 \to r_2$とすれば
    \begin{align*}
        \Delta\tau(r_1, r_2)
            &= \frac{1}{c} \int_{\theta_1}^{\theta_2} \left\{\frac{a}{r_2}\(\frac{1}{\cos^2\theta} - 1\)\right\}^{-1/2} \cdot -2r_2\sin\theta\cos\theta d\theta\\
            &= -\frac{2}{c}\sqrt{\frac{r_2^3}{a}} \int_{\theta_1}^{\theta_2} \cos^2\theta d\theta\\
            &= -\frac{2}{c}\sqrt{\frac{r_2^3}{a}} \left[\frac{\theta}{2} + \frac{\sin 2\theta}{4}\right]_{\theta_1}^{\theta_2}\\
            &= \frac{1}{c}\sqrt{\frac{r_2^3}{a}} \left[\theta_1 + \sin\theta_1\cos\theta_1\right]\\
            &= \frac{1}{c}\sqrt{\frac{r_2^3}{a}} \left[\arccos\sqrt{\frac{r_1}{r_2}} + \sqrt{\frac{r_1}{r_2}\(1 - \frac{r_1}{r_2}\)}\right]
    \end{align*}
    となる。
        \[\Delta\tau(0, r_2) = \frac{\pi}{2c}\sqrt{\frac{r_2^3}{a}}\]
    なので原点まで有限の時間で到達する。

    次に座標時間を求める。
    \begin{align*}
        \dv{t}{\tau} = \frac{\epsilon}{1 - \frac{a}{r}} = \frac{\sqrt{1 - \frac{a}{r_2}}}{1 - \frac{a}{r}}
    \end{align*}
    より
        \[\Delta t = \frac{1}{c} \sqrt{1 - \frac{a}{r_2}} \int_{r_1}^{r_2} \frac{1}{\(1 - \frac{a}{r}\) \sqrt{\frac{a}{r} - \frac{a}{r_2}}} dr\]
    同様に$r = r_2\cos^2\theta$ ($0 \leq \theta \leq \pi/2$)と置換すると
    \begin{align*}
        \Delta t(r_1, r_2)
            &= \frac{1}{c} \sqrt{1 - \frac{a}{r_2}} \int_{\theta_1}^{\theta_2} \frac{1}{\(1 - \frac{a}{r_2\cos^2\theta}\) \sqrt{\frac{a}{r_2\cos^2\theta} - \frac{a}{r_2}}} \cdot -2r_2\sin\theta\cos\theta d\theta\\
            &= \frac{2r_2}{c} \sqrt{\frac{r_2}{a}}\sqrt{1 - \frac{a}{r_2}} \int_{\theta_1}^{\theta_2} \frac{1}{1 - \frac{a}{r_2\cos^2\theta}} \cdot -\cos^2\theta d\theta\\
            &= -\frac{2r_2}{c} \sqrt{\frac{r_2}{a}}\sqrt{1 - \frac{a}{r_2}} \int_{\theta_1}^{\theta_2} \frac{\cos^4\theta}{\cos^2\theta - a / r_2} d\theta
    \end{align*}
    積分の部分だけ取り出すと
    \begin{align*}
        &\int_{\theta_1}^{\theta_2} \frac{\cos^4\theta}{\cos^2\theta - a / r_2} d\theta\\
            &= \int_{\theta_1}^{\theta_2} \cos^2\theta + \frac{a}{r_2} + \(\frac{a}{r_2}\)^2 \frac{1}{\cos^2\theta - a / r_2} d\theta\\
            &= \left[\(\frac{\theta}{2} + \frac{\sin 2\theta}{4}\) + \frac{a}{r_2}\theta + \(\frac{a}{r_2}\)^2 \cdot \frac{1}{2\sqrt{k(1 - k)}} \log \left\|\frac{\sqrt{k}\tan\theta_1 + \sqrt{1 - k}}{\sqrt{k}\tan\theta_1 - \sqrt{1 - k}}\right\|\right]_{\theta_1}^{\theta_2} \quad (k = a / r_2)\\
            &= -\frac{\theta_1}{2} - \frac{\sin\theta_1\cos\theta_1}{2} - \frac{a}{r_2}\theta_1 - \frac{\sqrt{k}^3}{2\sqrt{1 - k}} \log \left\|\frac{\sqrt{k}\sin\theta_1 + \sqrt{1 - k}\cos\theta_1}{\sqrt{k}\sin\theta_1 - \sqrt{1 - k}\cos\theta_1}\right\|\\
            &= -\(1 + \frac{2a}{r_2}\)\frac{\arccos \sqrt{\frac{r_1}{r_2}}}{2} - \frac{1}{2}\sqrt{\frac{r_1}{r_2}\(1 - \frac{r_1}{r_2}\)} - \frac{\sqrt{k}^3}{2\sqrt{1 - k}} \log \left\|\frac{\sqrt{\frac{a}{r_2}\(1 - \frac{r_1}{r_2}\)} + \sqrt{\(1 - \frac{a}{r_2}\)\frac{r_1}{r_2}}}{\sqrt{\frac{a}{r_2}\(1 - \frac{r_1}{r_2}\)} - \sqrt{\(1 - \frac{a}{r_2}\)\frac{r_1}{r_2}}}\right\|
    \end{align*}
    よって
    \begin{align*}
        \Delta t(r_1, r_2)
            &= \frac{r_2 + 2a}{c}\sqrt{\frac{r_2}{a} - 1}\arccos\sqrt{\frac{r_1}{r_2}} + \frac{r_2}{c}\sqrt{\frac{r_2}{a} - 1}\sqrt{\frac{r_1}{r_2}\(1 - \frac{r_1}{r_2}\)}\\
            & \quad + \frac{a}{c} \log \left\|\frac{\sqrt{a(r_2 - r_1)} + \sqrt{(r_2 - a)r_1}}{\sqrt{a(r_2 - r_1)} - \sqrt{(r_2 - a)r_1}}\right\|
    \end{align*}
    第三項は$r_1 \to a$で発散する。つまり外部の観測者から見ると物体は事象の地平面まで到達できない。

\subsection{軌道の安定性}

\subsection{光の軌道}
    光の場合、固有時をパラメータとして用いることはできないので適当なパラメータ$\lambda$を用いてオイラー=ラグランジュ方程式を導く。$t, \theta, \phi$については質量を持つ物体と同様に
    \begin{align*}
        \(1 - \frac{a}{r}\)\dot{t} &= \epsilon\\
        \theta &= \frac{\pi}{2}\\
        r^2\dot{\phi} &= h
    \end{align*}
    となる。$r$については$ds = 0$という条件から
    \begin{align*}
        -\(1 - \frac{a}{r}\)c^2\dot{t}^2 + \(1 - \frac{a}{r}\)^{-1}\dot{r}^2 + r^2\dot{\phi}^2 &= 0\\
        \dot{r}^2 + \(1 - \frac{a}{r}\)r^2\dot{\phi}^2 &= \(1 - \frac{a}{r}\)^2c^2\dot{t}^2\\
        \dot{r}^2 + r^2\dot{\phi}^2 &= c^2\epsilon^2 + \frac{2GM}{r^3c^2}h^2\\
        \frac{1}{r^4}\(\dv{r}{\phi}\)^2 + \frac{1}{r^2} &= \frac{c^2\epsilon^2}{h^2} + \frac{2GM}{r^3c^2}
    \end{align*}
    同様に$u = 1 / r$とおくと
        \[\(\dv{u}{\phi}\)^2 = \frac{c^2\epsilon^2}{h^2} - u^2 + \frac{2GM}{c^2}u^3\]
    両辺を$\phi$で微分して$\dv*{u}{\phi}$で割る。
    \begin{align*}
        2\dv{u}{\phi}\dv[2]{u}{\phi} &= -2u\dv{u}{\phi} + \frac{2GM}{c^2} \cdot 3u^2\dv{u}{\phi}\\
        \dv[2]{u}{\phi} &= -u + \frac{3GM}{c^2}u^2
    \end{align*}

    光が落下する時間を求める。$ds = 0$より、$d\phi = 0$として
        \[\dv{r}{t} = \pm c\(1 - \frac{a}{r}\)\]
    である。$r_1 < r_2$として、$r = r_2$から$r = r_1$に到達するまでの座標時間は
    \begin{align*}
        \Delta t(r_1, r_2)
            &= \frac{1}{c} \int_{r_1}^{r_2} \(1 - \frac{a}{r}\)^{-1} dr\\
            &= \frac{1}{c} \int_{r_1}^{r_2} 1 + \frac{a}{r - a} dr\\
            &= \frac{1}{c} \left[r + a\log(r - a)\right]_{r_1}^{r_2}\\
            &= \frac{r_2 - r_1}{c} + \frac{a}{c}\log\frac{r_2 - a}{r_1 - a}
    \end{align*}
    となる。つまり物体の落下時間と同様$r_1 \to a$で発散する。ただし逆数の対数発散なので、事象の地平面付近までには速やかに到達する。

\subsection{重力赤方偏移}
\section{宇宙モデル}

\subsection{ロバートソン=ウォーカー計量}
    曲率$K$、スケール因子$a(t)$に対して
        \[ds^2 = -dt^2 + a(t)^2\(\frac{dx^2}{1 - Kx^2} + x^2(d\theta^2 + \sin^2\theta d\phi^2)\)\]
    をロバートソン=ウォーカー計量と呼ぶ。

\end{document}