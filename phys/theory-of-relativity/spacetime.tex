\section{時空の方程式}

\subsection{測地線の方程式}
    等価原理より局所ローレンツ系$(X^i)$が存在し
        \[\dv[2]{X^i}{\tau} = 0\]
    である。一般座標系$(x^i)$に変換すると
    \begin{align*}
        \dv[2]{X^l}{\tau}
            &= \dv{\tau}\(\sum_i \pdv{X^l}{x^i}\dv{x^i}{\tau}\)\\
            &= \sum_k \pdv{X^l}{x^k}\dv[2]{x^k}{\tau} + \sum_{i, j} \pdv{X^l}{x^i}{x^j}\dv{x^i}{\tau}\dv{x^j}{\tau} = 0
    \end{align*}
    両辺に$\pdv*{x^k}{X^l}$を掛けて$l$について和を取る。
        \[\dv[2]{x^k}{\tau} + \pdv{x^k}{X^l}\pdv{X^l}{x^i}{x^j}\dv{x^i}{\tau}\dv{x^j}{\tau} = 0\]
    一般座標系における接続係数が
        \[\Gamma^k_{ij} = \pdv{x^k}{X^l}\pdv{X^l}{x^i}{x^j}\]
    なので
        \[\dv[2]{x^k}{\tau} + \Gamma^k_{ij}\dv{x^i}{\tau}\dv{x^j}{\tau} = 0\]
    となる。

    つまり局所慣性系は、数学的には時空の一点で接続係数が0になるような座標系である。定理より接続係数は対称

\subsection{電磁場の方程式}
    等価原理より局所ローレンツ系$(X^i)$が存在し、電磁テンソル$F^{\mu\nu}$、四元電流密度$j^\mu$とすると
    \begin{gather*}
        \partial_\nu F^{\mu\nu} = \mu_0 j^\mu\\
        \partial_\rho F_{\mu\nu} + \partial_\mu F_{\nu\rho} + \partial_\nu F_{\rho\mu} = 0
    \end{gather*}
    が成り立つ。ここで$F^{\mu\nu} = \eta^{\mu\alpha}\eta^{\nu\beta}F_{\alpha\beta}$である。これを一般座標系$(x^i)$に変換すると
    \begin{gather*}
        \nabla_\nu F^{\mu\nu} = \mu_0 j^\mu\\
        \nabla_\rho F_{\mu\nu} + \nabla_\mu F_{\nu\rho} + \nabla_\mu F_{\nu\rho} = 0\\
        F^{\mu\nu} = g^{\mu\alpha}g^{\nu\beta}F_{\alpha\beta}
    \end{gather*}
    となる。

    $F_{\mu\nu} = \nabla_\alpha A_\beta - \nabla_\beta A_\alpha$を代入すると
    \begin{align*}
        g^{\mu\alpha}g^{\nu\beta}\nabla_\nu(\nabla_\alpha A_\beta - \nabla_\beta A_\alpha)
            &= g^{\mu\alpha}\nabla_\nu\nabla_\alpha A^\nu - g^{\nu\beta}\nabla_\nu\nabla_\beta A^\mu\\
            &= g^{\mu\alpha}(\nabla_\nu\nabla_\alpha - \nabla_\alpha\nabla_\nu)A^\nu + g^{\mu\alpha}\nabla_\alpha\nabla_\nu A^\nu - g^{\mu\alpha}g^{\nu\beta}\nabla_\beta A^\mu\\
            &= g^{\mu\alpha}R^\nu_{\lambda\nu\alpha}A^\lambda + g^{\mu\alpha}\nabla_\alpha(\nabla_\nu A^\nu) - \square A^\mu\\
            &= R^\mu_\nu A^\nu + g^{\mu\alpha}\nabla_\alpha(\nabla_\nu A^\nu) - \square A^\mu
    \end{align*}
    より電磁ポテンシャルによるマクスウェル方程式は
        \[R^\mu_\nu A^\nu + g^{\mu\alpha}\nabla_\alpha(\nabla_\nu A^\nu) - \square A^\mu = \mu_0 j^\mu\]
    となる。またローレンツゲージ$\nabla_\nu A^\nu = 0$を仮定すると
        \[R^\mu_\nu A^\nu - \square A^\mu = \mu_0 j^\mu\]
    である。

\subsection{重力場の方程式}
    時空の計量と物質分布の関係を導出する。一般相対性理論における重力場の方程式は次の条件の下でニュートン力学に一致するはずである。
    \begin{itemize}
        \item 質点の速さは光速に比べて十分遅い
        \item 重力場は十分弱い
        \item 重力場の時間変化は十分小さい
    \end{itemize}

    時空の計量は物質の分布つまりエネルギー運動量テンソルに依存するはずである。$\nabla_jT_{ij} = 0$だから、計量に関係する量$X_{ij}$で、$X_{ij} = T_{ij}$となるようなものがあれば、$\nabla_jX_{ij} = 0$を満たす。アインシュタインテンソル$G_{ij}$や計量テンソル$g_{ij}$はそのような性質を持っている。ニュートン力学における重力場の方程式はポアソン方程式
        \[\Delta\phi = 4\pi G\rho\]
    である。$\phi$の二階微分は$g_{00}$の二階微分に比例するので、$X_{ij}$は計量の二階微分を含まなければならない。実は、$g_{ij}$の$x$に関する0, 1, 2階微分を含み、$\pdv*[2]{g_{ij}}{x}$の一次式であり、その共変微分が0となるような二階共変テンソルは$G_{ij}$と$g_{ij}$の線形結合に限ることが分かっている。つまり、
        \[G_{ij} + \Lambda g_{ij} = \kappa T_{ij}\]
    である。両辺に$g^{ij}$を掛けて縮約する。
        \[g^{ij}R_{ij} - \frac{1}{2}Rg^{ij}g_{ij} + \Lambda g^{ij}g_{ij} = \kappa g^{ij}T_{ij}\]
    $g^{ij}g_{ij}$はクロネッカーのデルタのトレースなので4である。
    \begin{gather*}
        R - 2R + 4\Lambda = \kappa T\\
        R = 4\Lambda - \kappa T
    \end{gather*}
    これを元の式に代入して、
    \begin{gather*}
        R_{ij} - \frac{1}{2}(4\Lambda - \kappa T)g_{ij} + \Lambda g_{ij} = \kappa T_{ij}\\
        R_{ij} - \Lambda g_{ij} = \kappa\(T_{ij} + \frac{1}{2}Tg_{ij}\)
    \end{gather*}
    である。

    まず加速度系における計量を求める。慣性系$K$に対して加速度系$K'$が$x$軸方向に加速度$g$で進んでいるとする。時刻$t = 0$で両者が一致しているとすれば、十分小さい$t$について、
    \begin{align*}
        t' &= t\\
        x' &= x-\frac{1}{2}gt^2\\
        y' &= y\\
        z' &= z
    \end{align*}
    である。従って慣性系ではミンコフスキー計量なので、
    \begin{align*}
        ds^2 &= -c^2dt^2 + dx^2 + dy^2 + dz^2\\
             &= -c^2dt'^2 + (dx' + gtdt')^2 + dy'^2 + dz'^2\\
             &= \(-1 + \frac{(gt)^2}{c^2}\)(cdt')^2 + 2gtdt'dx' + dx'^2 + dy'^2 + dz'^2
    \end{align*}
    $K'$系の原点では$x = \frac{1}{2}gt'^2$、重力ポテンシャルは$\phi = -gx$なので
    \begin{align*}
        ds^2 &= \(-1 + \frac{2gx}{c^2}\)(cdt')^2 + 2\sqrt{2gx}dt'dx' + dx'^2 + dy'^2 + dz'^2\\
             &= \(-1 - \frac{2\phi}{c^2}\)(cdt')^2 - 2\sqrt{-2\phi}dt'dx' + dx'^2 + dy'^2 + dz'^2
    \end{align*}
    重力ポテンシャルはスカラーなので、加速の方向に依らず$g_{00}= -1 - 2\phi / c^2$である。

    次にリッチテンソルを計算する。$g_{ij} = \eta_{ij} + h_{ij}$とおく。$h_{ij}$は十分小さいとして$h_{ij}$の二次の項を無視する。クリストッフェル記号は、
    \begin{align*}
        \Gamma^k_{ij}
            &= \frac{1}{2}g^{kl}\(\pdv{g_{lj}}{x^i} + \pdv{g_{li}}{x^j} - \pdv{g_{ij}}{x^l}\)\\
            &= \frac{1}{2}\eta_{kl}\(\pdv{h_{lj}}{x^i} + \pdv{h_{li}}{x^j} - \pdv{h_{ij}}{x^l}\)
    \end{align*}
    $h_{00} = -2\phi / c^2$なので
    \begin{align*}
        \Gamma^k_{00}
            &= \frac{1}{2}\eta_{kl}\(\pdv{h_{l0}}{x^0} + \pdv{h_{l0}}{x^0} - \pdv{h_{00}}{x^l}\)\\
            &= -\frac{1}{2}\eta_{kl}\pdv{h_{00}}{x^l}
    \end{align*}
    である。リッチテンソルは後ろの二項が$h_{ij}$の二次になるので無視して
    \begin{align*}
        R_{00}  &= R^k_{0k0}\\
                &= \pdv{\Gamma^k_{00}}{x^k} - \pdv{\Gamma^k_{0k}}{x^0}\\
                &= \pdv{\Gamma^k_{00}}{x^k}\\
                &= -\frac{1}{2}\eta_{kl}\pdv{h_{00}}{x^k}{x^l}\\
                &= -\frac{1}{2}\partial^i\partial_ih_{00}\\
                &= -\frac{1}{2}\Delta h_{00}
    \end{align*}

    次にエネルギー運動量テンソルを計算する。$T_{ij} = \rho u_iu_j$だが、速さが十分小さいので$T_{00} = \rho c^2$で残りの成分は全て無視する。よって$T = g_{ij}T^{ij} = g_{00}\rho c^2$である。したがって方程式は、
    \begin{align*}
        R_{ij} - \Lambda g_{ij} = \kappa\(T_{ij} + \frac{1}{2}Tg_{ij}\)\\
        -\frac{1}{2}\Delta h_{00} - \Lambda g_{00} = \kappa\rho c^2\(1 - \frac{1}{2}g_{00}^2\)
    \end{align*}
    $h_{00} = -2\phi / c^2, g_{00} = -1$を代入すれば、
    \begin{align*}
        \frac{1}{c^2}\Delta\phi + \Lambda = \frac{c^2}{2}\kappa\rho
    \end{align*}
    ポアソン方程式$\Delta\phi = 4\pi G\rho$と比較すると、$\Lambda = 0, \kappa = 8\pi G / c^4$である。つまり、
        \[R_{ij} - \frac{1}{2}Rg_{ij} = \frac{8\pi G}{c^4}T_{ij}\]
    これを重力場の方程式またはアインシュタイン方程式と呼ぶ。

\subsection{弱い重力場}
    弱い重力場$g_{\mu\nu} = \eta_{\mu\nu} + h_{\mu\nu}$ ($|h_{ij}| \ll 1$)において、アインシュタイン方程式を$h_{\mu\nu}$について線形化する。

    クリストッフェル記号は前節と同様
    \begin{align*}
        \Gamma^\lambda_{\mu\nu}
            &= \frac{1}{2}g^{\lambda\sigma}(g_{\nu\sigma, \mu} + g_{\sigma\mu, \nu} - g_{\mu\nu, \sigma})\\
            &= \frac{1}{2}\eta^{\lambda\sigma}(h_{\nu\sigma, \mu} + h_{\sigma\mu, \nu} - h_{\mu\nu, \sigma})
    \end{align*}
    リーマン曲率テンソルは
    \begin{align*}
        R^\lambda_{\sigma\mu\nu}
            &= \partial_\mu \Gamma^\lambda_{\sigma\nu} - \partial_\nu \Gamma^\lambda_{\sigma\mu}\\
            &= \partial_\mu\left(\frac{1}{2}\eta^{\lambda\alpha}(h_{\nu\alpha, \sigma} + h_{\alpha\sigma, \nu} - h_{\sigma\nu, \alpha})\right) - \partial_\nu\left(\frac{1}{2}\eta^{\lambda\alpha}(h_{\mu\alpha, \sigma} + h_{\alpha\sigma, \mu} - h_{\sigma\mu, \alpha})\right)\\
            &= \partial_\mu\left(\frac{1}{2}\eta^{\lambda\alpha}(h_{\nu\alpha, \sigma} - h_{\sigma\nu, \alpha})\right) - \partial_\nu\left(\frac{1}{2}\eta^{\lambda\alpha}(h_{\mu\alpha, \sigma} - h_{\sigma\mu, \alpha})\right)
    \end{align*}
    リッチテンソルは
    \begin{align*}
        R_{\mu\nu}
            &= R^\lambda_{\mu\lambda\nu}\\
            &= \partial_\lambda\left(\frac{1}{2}\eta^{\lambda\alpha}(h_{\nu\alpha, \mu} - h_{\mu\nu, \alpha})\right) - \partial_\nu\left(\frac{1}{2}\eta^{\lambda\alpha}(h_{\lambda\alpha, \mu} - h_{\mu\lambda, \alpha})\right)\\
            &= \frac{1}{2}\eta^{\lambda\alpha}(h_{\nu\alpha, \mu\lambda} - h_{\mu\nu, \alpha\lambda} - h_{\lambda\alpha, \mu\nu} + h_{\mu\lambda, \alpha\nu})
    \end{align*}
    スカラー曲率は
    \begin{align*}
        R   &= g^{\mu\nu}R_{\mu\nu}\\
            &= \frac{1}{2}\eta^{\mu\nu}\eta^{\lambda\alpha}(h_{\nu\alpha, \mu\lambda} - h_{\mu\nu, \alpha\lambda} - h_{\lambda\alpha, \mu\nu} + h_{\mu\lambda, \alpha\nu})\\
            &= \partial^\alpha \partial^\beta h_{\alpha\beta} - \square h
    \end{align*}
    アインシュタインテンソルは
    \begin{align*}
        G_{\mu\nu}
            &= R_{\mu\nu} - \frac{1}{2}Rg_{\mu\nu}\\
            &= \frac{1}{2}\eta^{\lambda\alpha}(h_{\nu\alpha, \mu\lambda} - h_{\mu\nu, \alpha\lambda} - h_{\lambda\alpha, \mu\nu} + h_{\mu\lambda, \alpha\nu})\\
                &\quad - \frac{1}{2}(\partial^\alpha \partial^\beta h_{\alpha\beta} - \square h)\eta_{\mu\nu}\\
    \end{align*}

\subsection{重力場の作用}
    アインシュタインテンソルはゲージ変換に対して不変である。

    重力場の作用は
        \[S_g[g_{\mu\nu}(x)] = \int \frac{c^4}{16\pi G}(R - 2\Lambda) \sqrt{-g}dx^4\]
    であり、アインシュタイン=ヒルベルト作用と呼ばれている。これの変分を計算する。まず
        
    \begin{align*}
        \delta g
            &= g g^{\mu\nu} \delta g_{\mu\nu}\\
        \delta \sqrt{-g}
            &= \frac{1}{2}\sqrt{-g}g^{\mu\nu}\delta g_{\mu\nu}\\
            &= -\frac{1}{2}\sqrt{-g}g_{\mu\nu}\delta g^{\mu\nu}\\
        \delta R_{\mu\nu}
            &= \delta(\partial_\lambda \Gamma^\lambda_{\mu\nu} - \partial_\nu \Gamma^\lambda_{\mu\lambda} + \Gamma^\lambda_{\lambda\alpha}\Gamma^\alpha_{\mu\nu} - \Gamma^\lambda_{\nu\alpha}\Gamma^\alpha_{\mu\lambda})\\
            &= \partial_\lambda \delta\Gamma^\lambda_{\mu\nu} - \partial_\nu \delta\Gamma^\lambda_{\mu\lambda} + \delta\Gamma^\lambda_{\lambda\alpha} \Gamma^\alpha_{\mu\nu} + \Gamma^\lambda_{\lambda\alpha} \delta\Gamma^\alpha_{\mu\nu} - \delta\Gamma^\lambda_{\nu\alpha} \Gamma^\alpha_{\mu\lambda} - \Gamma^\lambda_{\nu\alpha} \delta\Gamma^\alpha_{\mu\lambda}\\
            &= (\partial_\lambda \delta\Gamma^\lambda_{\mu\nu} + \Gamma^\lambda_{\lambda\alpha} \delta\Gamma^\alpha_{\mu\nu} - \Gamma^\alpha_{\mu\lambda} \delta\Gamma^\lambda_{\nu\alpha} - \Gamma^\lambda_{\nu\alpha} \delta\Gamma^\alpha_{\mu\lambda}) - (\partial_\nu \delta\Gamma^\lambda_{\mu\lambda} - \Gamma^\alpha_{\mu\nu} \delta\Gamma^\lambda_{\lambda\alpha})\\
            &= \nabla_\lambda \delta\Gamma^\lambda_{\mu\nu} - \nabla_\nu \delta\Gamma^\lambda_{\mu\lambda}\\
        \delta R
            &= \delta g^{\mu\nu} R_{\mu\nu} + g^{\mu\nu} \delta R_{\mu\nu}\\
            &= R_{\mu\nu} \delta g^{\mu\nu} + g^{\mu\nu} (\nabla_\lambda \delta\Gamma^\lambda_{\mu\nu} - \nabla_\nu \delta\Gamma^\lambda_{\mu\lambda})\\
            &= R_{\mu\nu} \delta g^{\mu\nu} +  \nabla_\lambda (g^{\mu\nu} \delta\Gamma^\lambda_{\mu\nu}) - \nabla_\nu(g^{\mu\nu} \delta\Gamma^\lambda_{\mu\lambda})\\
            &= R_{\mu\nu} \delta g^{\mu\nu} +  \nabla_\lambda(g^{\mu\nu} \delta\Gamma^\lambda_{\mu\nu} - g^{\mu\lambda} \delta\Gamma^\nu_{\mu\nu})\\
        \delta S_g
            &= \frac{c^3}{16\pi G} \int \delta\left[(R - 2\Lambda)\sqrt{-g}\right] dx^4\\
            &= \frac{c^3}{16\pi G} \int (\delta R \sqrt{-g} - (R - 2\Lambda)\delta \sqrt{-g}) dx^4\\
            &= \frac{c^3}{16\pi G} \int \left(R_{\mu\nu} \delta g^{\mu\nu} \sqrt{-g} - (R - 2\Lambda)\frac{1}{2}\sqrt{-g}g_{\mu\nu}\delta g^{\mu\nu}\right) dx^4\\
            &= \frac{c^3}{16\pi G} \int \left(R_{\mu\nu} - \frac{1}{2}Rg^{\mu\nu} + \Lambda g^{\mu\nu}\right) \delta g^{\mu\nu} \sqrt{-g} dx^4
    \end{align*}
    となる。全体の作用
        \[S = S_m + S_g\]
    の変分が0となるので、
    \begin{gather*}
        \begin{aligned}
            \delta S
                &= -\frac{1}{2c} \int T_{\mu\nu} \delta g^{\mu\nu} \sqrt{-g} dx^4\\
                &\quad + \frac{c^3}{16\pi G} \int \left(R_{\mu\nu} - \frac{1}{2}Rg^{\mu\nu} + \Lambda g^{\mu\nu}\right) \delta g^{\mu\nu} \sqrt{-g} dx^4 = 0\\
        \end{aligned}\\
        R_{\mu\nu} - \frac{1}{2}Rg_{\mu\nu} + \Lambda g_{\mu\nu} = \frac{8\pi G}{c^4}T_{\mu\nu}
    \end{gather*}
    となりアインシュタイン方程式が導かれる。