\section{四次元の時空}

\subsection{ミンコフスキー時空}
    ローレンツ変換ではガリレイ変換と異なり、時間と空間が一体となって変換される。そこで相対論では$x, y, z$に時間成分$ct$を加えた4次元時空というものを考える。これをミンコフスキー時空と言い、$(x^0, x^1, x^2, x^3) = (ct, x, y, z)$と表す。相対論では出来事を事象(event)と呼ぶ。事象はミンコフスキー時空の点として表され、世界点という。また質点の運動はこの時空の曲線で表され、世界線という。二つの世界点$(ct_1, x_1, y_1, z_1), (ct_2, x_2, y_2, z_2)$に対して、世界間隔$s_{12}$を
        \[s_{12}^2 = -(ct_2 - ct_1)^2 + (x_2 - x_1)^2 + (y_2 - y_1)^2 + (z_2 - z_1)^2\]
    と定義する。特に原点との世界間隔を世界長さ$s$と言い、
        \[s^2 = -(ct)^2 + x^2 + y^2 + z^2\]
    である。微小な世界間隔は
        \[dx^2 = -(cdt)^2 + dx^2 + dy^2 + dz^2\]
    であり、ミンコフスキー計量
    \[
        \eta_{ij} =
        \begin{pmatrix}
            -1 & 0 & 0 & 0\\
            0 & 1 & 0 & 0\\
            0 & 0 & 1 & 0\\
            0 & 0 & 0 & 1
        \end{pmatrix}
    \]
    を用いると
        \[ds^2 = \eta_{ij}dx^idx^j\]
    となる。ミンコフスキー時空の原点を通過する光は、時間軸から45度傾いた円錐面(実際は面ではない)
        \[s^2 = -(ct)^2 + x^2 + y^2 + z^2 = 0\]
    の上を伝播する。これを光円錐と呼ぶ。光円錐の内側($s^2 < 0$)では、原点における事象と因果関係を持つことができ、時間的(time like)領域と呼ばれる。それに対して外側($s^2 > 0$)は空間的(space like)領域と呼ばれる。

    世界長さ$s$はローレンツ変換によって変化しない不変量である。ローレンツ変換はミンコフスキー時空内でのある種の回転を表していると考えられる。実際$v / c = \tanh\theta$とおくと
    \begin{align*}
        ct' &= +\cosh\theta \cdot ct - \sinh\theta \cdot x\\
        x'  &= -\sinh\theta \cdot ct + \cosh\theta \cdot x
    \end{align*}
    である。

\subsection{固有時}
    注目している物体の静止系で測った時間をその物体の固有時(固有時間)$\tau$と言い、
    \begin{gather*}
        -(cdt)^2 + dx^2 + dy^2 + dz^2 = -(cd\tau)^2\\
        \begin{aligned}
            d\tau^2 &= dt^2 - \frac{dx^2 + dy^2 + dz^2}{c^2}\\
            d\tau &= -\frac{ds}{c}
        \end{aligned}
    \end{gather*}
    である。固有時は世界長さ同様ローレンツ変換によって変わらない不変量である。両辺を$dt^2$で割ると、
    \begin{align*}
        \(\dv{\tau}{t}\)^2
            &= 1 - \frac{1}{c^2}\left[\left(\dv{x}{t}\right)^2 + \left(\dv{y}{t}\right)^2 + \left(\dv{z}{t}\right)^2\right]\\
            &= 1 - \frac{v^2}{c^2}
    \end{align*}
    となる。

\subsection{微分演算子}
    三次元の微分演算子$\nabla$を四次元に拡張する。微分演算子はそれ自身では共変ベクトルだが、テンソルに作用するときは一般にテンソルとはならない。しかし座標変換としてローレンツ変換のような線形変換のみ考えるならば、テンソルと同じように振る舞う。
        \[(\partial_0, \partial_1, \partial_2, \partial_3) = \left(\pdv{x^0}, \pdv{x^1}, \pdv{x^2}, \pdv{x^3}\right) = \left(\frac{1}{c}\pdv{t}, \pdv{x}, \pdv{y}, \pdv{z}\right)\]
    とする。これとミンコフスキー計量との縮約をとると、
        \[(\partial^0, \partial^1, \partial^2, \partial^3) = \left(-\pdv{x^0}, \pdv{x^1}, \pdv{x^2}, \pdv{x^3}\right) = \(-\frac{1}{c}\pdv{t}, \pdv{x}, \pdv{y}, \pdv{z}\)\]
    となる。更にこれらの縮約をとり、
        \[\partial^i\partial_i = -\frac{1}{c^2}\pdv[2]{t} + \pdv[2]{x} + \pdv[2]{y} + \pdv[2]{z}\]
    これはダランベール演算子と呼ばれ$\square$と書く。

\subsection{共変形式}
    特殊相対性原理によれば物理法則はローレンツ変換によって形を変えない。したがって方程式のローレンツ共変性が一見して分かるような形式に書き直せるはずである。このようなものを共変形式(covariant form)と呼ぶ。ニュートン力学における速度や運動量、エネルギーといった物理量はローレンツ変換によって大きく形を変えてしまうため、相対論的な概念に書き換えていく。

    四元速度を
        \[u^i = \dv{x^i}{\tau} = (\gamma c, \gamma v^1, \gamma v^2, \gamma v^3)\]
    と定義する。座標を不変量である固有時で微分しているので反変テンソルである。四元速度の空間成分はニュートン力学の速度に相当する。

    四元運動量を
        \[p^i = mu^i = \gamma m(c, v^1, v^2, v^3)\]
    と定義する。四元速度と同様反変テンソルである。亜光速では$mv^i$の運動量は保存せず、四元運動量が保存することが実験によって確認されている。ここで
        \[(mc)^2 = (p^0)^2 - p_x^2 - p_y^2 - p_z^2 = (p^0)^2 - p^2\]
    なので、テイラー展開して、
    \begin{align*}
        p^0c &= \sqrt{(mc^2)^2 + (pc)^2}\\
             &= mc^2\sqrt{1 + \frac{p^2}{(mc)^2}}\\
             &= mc^2 + \frac{p^2}{2m} + \cdots
    \end{align*}
    第二項は古典力学における運動エネルギーと一致する。そこで$p^0$は物体の全エネルギー$E$を$c$で割ったものだと解釈する。つまり相対論では、物体のエネルギーは四元運動量の時間成分に組み込まれる。
        \[p^i = \left(\frac{E}{c}, p_x, p_y, p_z\right)\]
    である。またエネルギーは
    \begin{align*}
        E^2 &= (mc^2)^2 + (pc)^2\\
            &= (mc^2)^2 + (\gamma mv \cdot c)^2\\
            &= (mc^2)^2 \(1 + \gamma^2\frac{v^2}{c^2}\)\\
            &= (\gamma mc^2)^2\\
        E   &= \gamma mc^2
    \end{align*}
    となる。特に物体が静止しているときは
        \[E = mc^2\]
    となる。