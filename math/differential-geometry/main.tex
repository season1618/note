\documentclass[uplatex]{jsarticle}
\usepackage{amssymb,amsmath}
\usepackage{bm}
\usepackage{amsthm}
\usepackage{physics}
\usepackage{ascmac}

\renewcommand{\epsilon}{\varepsilon}
\newcommand{\R}{\mathbb{R}}
\renewcommand{\tr}{\operatorname{tr}}
\newcommand{\Ric}{\text{Ric}}
\newcommand{\inner}[2]{\langle #1, #2 \rangle}
\newcommand{\chr}[2]{\left\{#1 \atop #2\right\}}

\renewcommand{\(}{\left(}
\renewcommand{\)}{\right)}

\theoremstyle{definition}
\renewcommand{\proofname}{\textbf{証明}}

\newtheorem{ax}{公理}
\newtheorem{dfn}{定義}
\newtheorem{prop}{命題}
\newtheorem{lem}{補題}
\newtheorem{thm}{定理}
\newtheorem{cor}{系}
\newtheorem{ex}{例}

\title{微分幾何学}
\author{season07001674}
\date{\today}

\begin{document}
\maketitle
\tableofcontents

\section{曲線}

空間曲線を分類する。

\subsection{曲線の表示}
    滑らかな空間曲線$p(t) = (x(t), y(t), z(t))$を考える。$p(t)$が動かなければ尖った部分ができてしまうので、$p'(t) \neq 0$と仮定する。ここで、時刻が0から$t$まで動いたときの曲線の長さは、
        \[s(t) = \int_0^t |p'(t')| \dd{t'}\]
    である。これを弧長パラメータという。$s$は$t$の単調関数なので逆関数$t(s)$が存在し、$p(s)$を考えることができる。

\subsection{Frenet-Serretの公式}
    曲線上を一定の速さで移動することを考える。進行方向の単位ベクトルを$e_1$とすると、$|p'(s)| = 1$より$e_1 = p'(s)$である。また
        \[p'(s) \cdot p'(s) = 1\]
    の両辺を微分すると
        \[p'(s) \cdot p''(s) = 0\]
    だから加速度ベクトル$p''(s)$は速度ベクトルと直交する。直感的には速度ベクトルが単位球面上を移動する様子を表している。点$p$が加速する方向を主法線方向と呼び、単位ベクトルを$e_2$とする。そして$e_3 = e_1 \times e_2$の方向を従法線方向と呼ぶ。$(e_1, e_2, e_3)$は正規直交基底になっており、Frenet-Serret標構と呼ばれる。

    $(e_1', e_2', e_3')$と$(e_1, e_2, e_3)$の関係式を導く。まず、$|p''(s)| = \kappa(s)\ (> 0)$とすれば
        \[e_1' = p''(s) = \kappa(s)e_2\]
    である。$|e_2| = 1$なので、$e_1$と同様に$e_2$と$e_2'$は直交する。つまり$e_2'$は$e_1, e_3$のなす平面上にある。$e_2' = k(s)e_1 + \tau(s)e_3$とおく。$e_1 \cdot e_2 = 0$の両辺を微分して
    \begin{align*}
        e_1' \cdot e_2 + e_1 \cdot e_2' &= 0\\
        \kappa e_2\cdot e_2 + e_1\cdot (ke_1 + \tau e_3) &= 0\\
        \kappa + k &= 0
    \end{align*}
    よって$k = -\kappa$である。また$e_3 = e_1 \times e_2$の両辺を微分して
    \begin{align*}
        e_3' &= e_1' \times e_2 + e_1 \times e_2'\\
             &= \kappa e_2 \times e_2 + e_1 \times (-\kappa e_1 + \tau e_3)\\
             &= \tau e_1 \times e_3\\
             &= -\tau e_2
    \end{align*}
    まとめると
    \begin{align*}
        e_1' &= \qquad\qquad \kappa e_2\\
        e_2' &= -\kappa e_1 \quad\qquad +\tau e_3\\
        e_3' &= \quad\qquad -\tau e_2
    \end{align*}
    が成立する。これをFrenet-Serretの公式という。$\kappa$を曲率、$\tau$を捩率(第二曲率)と呼ぶ。曲率と捩率はパラメータの始点と曲線の向きに依存する。つまり$\kappa(s), \tau(s)$に対して$\kappa(\pm s - c), \tau(\pm s - c)$も同じ曲線を表す。
    \begin{thm}[空間曲線の基本定理]
        曲率$\kappa(s)(> 0)$と捩率$\tau(s)$が与えられたとき$p(s)$が合同変換を除いて一意に存在する。
    \end{thm}
    \begin{proof}
        常微分方程式の解の一意性より従う。
    \end{proof}
\section{曲面}

\subsection{曲面の表示}
    滑らかな曲面
        \[p: (u, v) \in D \mapsto (x(u, v), y(u, v), z(u, v)) \in S\]
    を考える。また便宜上$(u, v)$を$(u_1, u_2)$、$(x, y, z)$を$(x_1, x_2, x_3)$と表すことがある。この曲面が点や曲線に退化せず、特異点を持たないよう以下の条件を設ける。以下の5つの条件は全て同値である。
    \begin{enumerate}
        \item $D$上に任意の曲線$s(t) = (u(t), v(t))$を与えると、$s'(t) \neq 0$ならば、$\dv{t}p(s(t)) \neq 0$である。つまり$D$上の曲線の像は$S$上の曲線となる。
        \item $S$の点$p$における接ベクトル全体は平面をなす。これを接平面という。
        \item $x, y, z$の内、任意の二つを$x_i, x_j$とすると、ヤコビアン
                \[\left|\pdv{(x_i, x_j)}{(u, v)}\right| \neq 0\]
            である。このとき十分小さい$D$で、写像
                  \[(u, v) \mapsto (x_i, x_j)\]
              は全単射で、逆写像も滑らかである(陰関数定理)。
        \item $x, y, z$の内、任意の二つを$x_i, x_j$とすると、
                  \[\dd{x_i} \wedge \dd{x_j} \neq 0\]
        \item ベクトル$\pdv*{p}{u}$と$\pdv*{p}{v}$は線形独立。
    \end{enumerate}
    曲面を表示するに当たって、正規直交枠とガウス枠を導入する。曲面上の点$p$を原点とし、その点ににおける法線方向を$z$軸とする直交座標系$(x, y, z)$を取り、それぞれの単位ベクトルを$e_1, e_2, e_3$とする。$e_3$は$p$が曲面上を移動するとき、一定の向きを持つように決める。$e_1, e_2$は$e_1 \times e_2 = e_3$となるように取る。$x$軸と$y$軸は回転の分の自由度を残しておく。$z = f(x, y)$とすれば、$f(0, 0) = f_x(0, 0) = f_y(0, 0) = 0$が成り立つ。$(e_1, e_2, e_3)$を正規直交枠と呼ぶ。一方$B_1 = \pdv*{p}{u}, B_2 = \pdv*{p}{v}$で、$n$を$B_1 \times B_2$の方向の単位法線ベクトルとしたとき、$(B_1, B_2, n)$をガウス枠と呼ぶ。

\subsection{第一基本形式}
    曲面上における二点間の微小長さは、空間内のユークリッド距離によって
    \begin{align*}
        \dd{s}^2 &= \dd{p}^2\\
                 &= \(\pdv{p}{u}\dd{u} + \pdv{p}{v}\dd{v}\)^2\\
                 &= p_u^2\dd{u}^2 + 2p_up_v\dd{u}\dd{v} + p_v^2\dd{v}^2
    \end{align*}
    で与えられる。$g_{ij} = \pdv*{p}{u_i} \cdot \pdv*{p}{u_j}$とおいて
    \begin{align*}
        \dd{s}^2 &= g_{11}\dd{u}^2 + 2g_{12}\dd{u}\dd{v} + g_{22}\dd{v}^2 \\
        I &= 
        \begin{pmatrix}
            g_{11} & g_{12}\\
            g_{21} & g_{22}
        \end{pmatrix}
    \end{align*}
    これを第一基本形式という。また$I$をリーマン計量(計量テンソル)と呼ぶ。今後はこのような曲面上の計量を考える。

\subsection{第二基本形式}
    曲面上の点$p(u, v)$とその接平面を考える。点$p(u + \dd{u}, v + \dd{v})$と接平面の距離は
        \[\dd{z}  = (p(u + \dd{u}, v + \dd{v}) - p(u, v)) \cdot n\]
    二次の項までテイラー展開すると、
        \[\dd{z} = \(\pdv{p}{u}\dd{u} + \pdv{p}{v}\dd{v} + \frac{1}{2}\pdv[2]{p}{u}\dd{u}^2 + \pdv{p}{u}{v}\dd{u}\dd{v} + \frac{1}{2}\pdv[2]{p}{v}\dd{v}^2\) \cdot n\]
    $p_u, p_v \perp n$であることに注意すると、
        \[\dd{z} = \(\frac{1}{2}p_{uu}\dd{u}^2 + p_{uv}\dd{u}\dd{v} + \frac{1}{2}p_{vv}\dd{v}^2\) \cdot n\]
    となる。$h_{ij} = \pdv*{p}{u_i}{v_i}$とおいて
    \begin{align*}
        2\dd{z} &= h_{11}\dd{u}^2 + 2h_{12}\dd{u}\dd{v} + h_{22}\dd{v}^2\\
        II &=
        \begin{pmatrix}
            h_{11} & h_{12}\\
            h_{21} & h_{22}
        \end{pmatrix}
    \end{align*}
    これを第二基本形式という。

    第一基本形式が表す計量は曲面に貼り付いた生物にも観測できるという意味で内在的であり、第二基本形式は曲面を外部の空間から見た様子を表しているという意味で外在的である。

\subsection{測地的曲率と法曲率}
    曲面上の曲線を弧長パラメータによって$p(u, v) = p(u(s), v(s))$と表し、正規直交枠$(e_1, e_2, e_3)$を導入する。$e_1$を曲線の接線方向、$e_3$を曲面の法線方向の単位ベクトルとすると、$p''(s) = e_1' = k_ge_2 + k_ne_3$と一意に表せる。このとき$k_g, k_n$をそれぞれ測地的曲率、法曲率という。$|p''|$は空間曲線としての曲率$k(s)$である。ここで、
    \begin{align*}
        p'  &= p_u\dv{u}{s} + p_v\dv{v}{s}\\
        p'' &= \dv{p_u}{s}\dv{u}{s} + p_u\dv[2]{u}{s} + \dv{p_v}{s}\dv{v}{s} + p_v\dv[2]{v}{s}\\
            &= p_{uu}\(\dv{u}{s}\)^2 + 2p_{uv}\dv{u}{s}\dv{v}{s} + p_{vv}\(\dv{v}{s}\)^2 + p_u\dv[2]{u}{s} + p_v\dv[2]{v}{s}\\
    \end{align*}
    なので$p_u \cdot e_3 = p_v \cdot e_3 = 0$より、
    \begin{align*}
        k_n &= p'' \cdot e_3\\
            &= (p_{uu} \cdot e_3)\(\dv{u}{s}\)^2 + 2(p_{uv} \cdot e_3)\dv{u}{s}\dv{v}{s} + (p_{vv} \cdot e_3)\(\dv{v}{s}\)^2\\
            &= h_{11}\(\dv{u}{s}\)^2 + 2h_{12}\dv{u}{s}\dv{v}{s} + h_{22}\(\dv{v}{s}\)^2\\
            &= \frac{h_{11}\dd{u}^2 + 2h_{12}\dd{u}\dd{v} + h_{22}\dd{v}^2}{g_{11}\dd{u}^2 + 2g_{12}\dd{u}\dd{v} + g_{22}\dd{v}^2}\\
    \end{align*}
    となる。つまり曲線上の点$p$における法曲率は$\lambda = \dv*{v}{u}$のみに依存する。

\subsection{主曲率}
    曲面上の点において、法線ベクトルを含む平面と曲面が交わってできる曲線を法切断または直截線という。点$p$における法切断の測地的曲率は0なので、曲率は法曲率に等しい。法切断の曲率が最大値最小値をとるとき、その曲率$k_1,k_2$を主曲率、接ベクトルを主方向$X_1,X_2$と呼ぶ。法曲率が$k$となる条件は、
    \begin{gather*}
        k = \frac{h_{11} + 2h_{12}\lambda + h_{22}\lambda^2}{g_{11} + 2g_{12}\lambda + g_{22}\lambda^2}\\
        (h_{22} - kg_{22})\lambda^2 + 2(h_{12} - kg_{12})\lambda + (h_{11} - kg_{11}) = 0
    \end{gather*}
    の解が存在することであり、この二次方程式の判別式$D$が0以上になることである。
    \begin{align*}
        \frac{D}{4}
            &= (h_{12} - kg_{12})^2 - (h_{11} - kg_{11})(h_{22} - kg_{22})\\
            &= (g_{12}^2 - g_{11}g_{22})k^2 + (g_{11}h_{22} - 2g_{12}h_{12} + g_{22}h_{11})k + (h_{12}^2 - h_{11}h_{22})\\
            &\geq 0
    \end{align*}
    $g$は正定値行列なので$g_{12}^2 - g_{11}g_{22} < 0$である。よって$k$が極値を取るのは$D = 0$のときである。
    \begin{align*}
        (h_{11} - kg_{11})(h_{22} - kg_{22}) - (h_{12} - kg_{12})(h_{21} - kg_{21}) &= 0\\
        \det(II - kI) &= 0\\
        \det(I^{-1}II - kE) &= 0
    \end{align*}
    となる。これは$I^{-1}II$の固有方程式である。$S = I^{-1}II$をシェイプ作用素という。つまり二つの主曲率はシェイプ作用素の固有値である。

    一般の方向における法切断の曲率を求める。
    \begin{lem}
        平面曲線$y = y(x)$で$y'(0) = 0$のとき、原点における曲率は、$y''(0)$である(平面曲線の場合、曲率に符号を付ける)。
    \end{lem}
    \begin{proof}
        $p = (x, y)$、弧長パラメータを$s$とすれば曲率は$\dv*{p}{s}$である。
            \[\dv{p}{x} = (1, y'), \quad \dv{s}{x} = \sqrt{1 + y'^2}\]
        より、
        \begin{align*}
            \dv{p}{s} &= \frac{(1, y')}{\sqrt{1 + y'^2}}\\
            \dv{x}\dv{p}{s} &= \frac{(0, y'')\sqrt{1 + y'^2} - (1, y')y'y'' / \sqrt{1 + y'^2}}{1 + y'^2}
        \end{align*}
        $y'(0) = 0$なので
            \[\dv{p}{s} = \frac{(0, y'')(1 + y'^2) - (1, y')y'y''}{(1 + y'^2)^2} = (0, y'')\]
        よって原点における曲率は$y''$となる。
    \end{proof}

    \begin{thm}[オイラーの定理(微分幾何)]
        二つの主方向は直交し、更に$X_1$に対して角$\theta$をなす法切断の曲率を$k_\theta$とすれば、
            \[k_\theta = k_1\cos^2\theta + k_2\sin^2\theta\]
        である。
    \end{thm}
    \begin{proof}
        曲面上に正規直交枠をとる。補題より、$x, y$軸の法切断の曲率は偏微分となり$\pdv[2]{z}{x}, \pdv[2]{z}{y}$である。$x$軸と角$\theta$をなす方向を$v = (\cos\theta, \sin\theta)$とすれば、曲率$k_\theta$は方向微分となり、
        \begin{align*}
            k_\theta &= \pdv[2]{z}{v}\\
                     &= \pdv{v}(z_x\cos\theta + z_y\sin\theta)\\
                     &= (z_{xx}\cos\theta + z_{xy}\sin\theta, z_{xy}\cos\theta + z_{yy}\sin\theta) \cdot (\cos\theta, \sin\theta)\\
                     &= z_{xx}\cos^2\theta + 2z_{xy}\sin\theta\cos\theta + z_{yy}\sin^2\theta
        \end{align*}
        この二次形式の最大値と最小値は行列$\begin{pmatrix}z_{xx} & z_{xy}\\ z_{yx} & z_{yy}\end{pmatrix}$の固有値であり、主方向はそれぞれの固有値の固有ベクトルである。実対称行列には直交する固有ベクトルが存在するので主方向も直交する。この固有ベクトルで主軸変換を行えば、主方向との成す角を新たに$\theta$とおいて
            \[k_\theta = k_1\cos^2\theta + k_2\sin^2\theta\]
        となる。
    \end{proof}

\subsection{ガウス曲率と平均曲率}
    二つの主曲率$k_1, k_2$の積をガウス曲率$K$、平均を平均曲率$H$という。主曲率の満たす二次方程式の解と係数の関係より、
    \begin{align*}
        K &= k_1k_2 = \det S = \frac{h_{11}h_{22} - h_{12}^2}{g_{11}g_{22} - g_{12}^2}\\
        2H &= k_1 + k_2 = \tr S = \frac{g_{11}h_{22} - 2g_{12}h_{12} + g_{22}h_{11}}{g_{11}g_{22} - g_{12}^2}\\
    \end{align*}
    となる。ガウス曲率が正の点を楕円点、負の点を双曲点という。また0の点を放物点と呼ぶこともある。
\section{曲面の方程式(ガウス枠)}
第一種及び第二種のクリストッフェル記号を次のように定義する。
\begin{align*}
    [jk, l] &= \frac{1}{2}\(\pdv{g_{kl}}{u^j} + \pdv{g_{jl}}{u^k} - \pdv{g_{jk}}{u^l}\)\\
    \chr{i}{jk} &= \frac{1}{2}g^{ih}\(\pdv{g_{jh}}{u^k} + \pdv{g_{kh}}{u^j} - \pdv{g_{jk}}{u^h}\)
\end{align*}
ここで$[jk, l] = g_{il}\chr{i}{jk}, \chr{i}{jk} = g^{ih}[jk, h]$が成り立っている。

\subsection{ガウス=ワインガルテンの公式}
    曲線におけるフレネ=セレの公式に当たるものを導出する。
    
    ガウス枠$\{B_1, B_2, n\}$をとる。ここで
        \[n = \frac{B_1 \times B_2}{|B_1 \times B_2|} = \frac{B_1 \times B_2}{\sqrt{g_{11}g_{22} - g_{12}^2}}\]
    である。三つのベクトルは線形独立なので、
        \[\pdv{B_j}{u^k} = \Gamma_{jk}^iB_i + h_{jk}n\]
    と書ける。$h_{jk} = \pdv*{B_j}{u^k} \cdot n$は定義より第二基本量である。また$g_{ij} = B_i \cdot B_j$を微分した
        \[\pdv{B_i}{u^k} \cdot B_j + B_i\cdot \pdv{B_j}{u^k} = \pdv{g_{ij}}{u^k}\]
    に代入して、
        \[\Gamma_{ik}^hg_{hj} + \Gamma_{jk}^hg_{ih} = \pdv{g_{ij}}{u^k}\]
    を得る。$\Gamma_{jk|i} = \Gamma_{jk}^hg_{ih}$と置けば、
        \[\Gamma_{ik|j} + \Gamma_{jk|i} = \pdv{g_{ij}}{u^k}\]
    となる。$\pdv*{B_j}{u^k} = \pdv*{p}{u^j}{u^k}$は$j,k$に関して対称なので、上式の添え字を循環的に入れ替えて、$\Gamma_{jk|i}$に関する連立方程式とみれば、
    \begin{align*}
        \Gamma_{jk|i} &= \frac{1}{2}\(\pdv{g_{ki}}{u^j} + \pdv{g_{ij}}{u^k} - \pdv{g_{jk}}{u^i}\) = [jk,i]\\
        \Gamma_{jk}^i &= g^{ih}\Gamma_{jk|h} = \chr{i}{jk}
    \end{align*}
    となることが分かる。次に$n \cdot n = 1$を微分すると$\pdv*{n}{u^j} \cdot n = 0$だから
        \[\pdv{n}{u^k} = r_k^hB_h\]
    となる$r_j^h$が一意に決まる。$B_i \cdot n = 0$を微分した$\pdv*{B_i}{u^k} \cdot n + B_i \cdot \pdv*{n}{u^k} = 0$に先程の式を代入して、
        \[h_{ij} + r_j^hg_{ih} = 0\]
    これは行列の積を成分ごとに表したものなので、両辺に右から$g$の逆行列をかけて、
        \[r_k^i = -h_{kl}g^{li}\]
    二つを合わせて
    \begin{align*}
        \pdv{B_j}{u^k} &= \chr{i}{jk}B_i + h_{jk}n \tag{ガウスの公式}\\
        \pdv{n}{u^k} &= -h_{kl}g^{li}B_i \tag{ワインガルテンの公式}
    \end{align*}

\subsection{ガウス=コダッチの方程式}
    ガウス=ワインガルテンの公式の係数は第一及び第二基本量から求められる。この方程式を解くことで求められたガウス枠$\{B_1, B_2, n\}$が実際の曲面と矛盾しないためには以下の条件が必要となる。
    \begin{align*}
        \pdv{p}{u^j}{u^i} &= \pdv{p}{u^i}{u^j}\\
        \pdv{B_i}{u^k}{u^j} &= \pdv{B_i}{u^j}{u^k}\\
        \pdv{n}{u^j}{u^i} &= \pdv{n}{u^i}{u^j}
    \end{align*}
    である。第一式に関しては既に成り立っている。なぜなら
        \[\pdv{p}{u^j}{u^i} = \pdv{B_i}{u^j} = \chr{h}{ij}B_h + h_{ij}n\]
    は$i,j$に関して対称だからである。次に第二式の条件を求める。
    \begin{align*}
        \pdv{B_i}{u^k}{u^j}
            &= \pdv{u^k}\(\chr{h}{ij}B_h + h_{ij}n\)\\
            &= \pdv{u^k}\chr{h}{ij}B_h + \chr{h}{ij}\pdv{B_h}{u^k} + \pdv{h_{ij}}{u^k}n + h_{ij}\pdv{n}{u^k}\\
            &= \sum_h \pdv{u^k}\chr{h}{ij}B_h + \sum_{h,l} \chr{h}{ij}\chr{l}{hk}B_l + \sum_h \chr{h}{ij}h_{hk}n\\
            &\quad + \pdv{h_{ij}}{u^k}n + h_{ij} \sum_{l,h}(-h_{kl}g^{lh}B_h)
    \end{align*}
    第二項で添え字$h, l$を入れ替えると
    \begin{align*}
        \pdv{B_i}{u^k}{u^j}
            &= \left[\pdv{u^k}\chr{h}{ij} + \chr{h}{kl}\chr{l}{ij} - h_{ij}h_{kl}g^{lh}\right]B_h\\
            &\quad + \left[\pdv{h_{ij}}{u^k} + \chr{h}{ij}h_{hk}\right]n
    \end{align*}
    となる。これが$j, k$に関して対称となる。$B_h, n$は線形独立なので、
    \begin{gather*}
        R^h_{ijk} = h_{ik}h_{jl}g^{lh} - h_{ij}h_{kl}g^{lh} \tag{ガウスの方程式}\\
        \pdv{h_{ij}}{u^k} - \pdv{h_{ik}}{u^j} + \chr{h}{ij}h_{hk} - \chr{h}{ik}h_{hj} = 0 \tag{コダッチの方程式}
    \end{gather*}
    となる。ただし
        \[R^h_{ijk} = \pdv{u^j}\chr{h}{ik} - \pdv{u^k}\chr{h}{ij} + \chr{l}{ik}\chr{h}{jl} - \chr{l}{ij}\chr{h}{kl}\]
    である。また両辺に$g^{hl}$をかけて縮約を取ったものを、
        \[R_{lijk} = g^{hl}R^h_{ijk} = \pdv{u^j}\chr{h}{ik}g^{hl} - \pdv{u^k}\chr{h}{ij}g^{hl} + \chr{a}{ik}\chr{h}{ja}g^{hl} - \chr{a}{ij}\chr{h}{ka}g^{hl}\]
    これらをリーマン曲率テンソルと呼ぶ。リーマン曲率テンソルを用いてガウスの方程式を表すと、
    \begin{align*}
        g^{ha}R^h_{ijk}
            &= h_{ik}h_{ja}g^{ah}g^{hl} - h_{ij}h_{ka}g^{ah}g^{hl}\\
        R_{lijk}
            &= h_{ik}h_{ja}\delta_l^a - h_{ij}h_{ka}\delta_l^a\\
            &= h_{ik}h_{jl} - h_{ij}h_{kl}
    \end{align*}
    また第三式については、コダッチの方程式と同じものが導かれるため、条件はこれで十分である。よって次の定理が成り立つ。
    \begin{thm}[曲面論の基本定理]
        対称テンソル$g_{ij}, h_{ij}$が与えられ、$g_{ij}$は正定値であるとする。これらがガウス=コダッチの方程式を満たすとき、これらを第一及び第二基本量とする曲面$p(u, v)$が合同変換を除いて一意に存在する。
    \end{thm}

\subsection{驚異の定理}
    上の式は右辺が第二基本量のみに依存している。元々リーマン曲率テンソル及びその定義に含まれるクリストッフェル記号は第一基本量のみから求めることができた。つまり第一基本テンソルと第二基本テンソルは独立ではない。特に
        \[R_{1212} = h_{11}h_{22} - h_{12}^2\]
    である。これを用いると、
        \[K = \frac{h_{11}h_{22} - h_{12}^2}{g_{11}g_{22} - g_{12}^2} = \frac{R_{1212}}{g_{11}g_{22}-g_{12}^2}\]
    となる。曲面のガウス曲率を第一基本テンソルだけから求めることができる。これを驚異の定理(Theorem Egregium)という。元々外在的な量から定義されたガウス曲率が、内在的な量のみから決定できることが示された。外界の情報を用いることなく空間の曲がり方を考察することができることを示唆している。


    % \paragraph{等長写像}
    %     二つの曲面が距離を保ったまま変形できるとき、等長的であると言い、そのような写像を等長写像という。等長的な二つの曲面は、伸縮せずに折り曲げることで変形できる。折り紙は平面と等長的な曲面(可展面)を作る遊びである(ウェットフォールディングをする場合はこの限りではない)。等長変換によって第一基本形式およびガウス曲率は不変である。対偶を取ると、ガウス曲率の一致しない曲面は等長的でない。平面のガウス曲率は任意の点で0であり、半径$r$の球面では$1/r^2$なので、球面を歪ませることなく平面に展開することはできない。つまり地球の正確な平面図を作成することは不可能である。
    % \paragraph{共形写像}
    %     始点が同じ任意の二つのベクトルの成す角を保存する写像を共形写像(等角写像)という。
    % \paragraph{等温座標}
    %     パラメータ$(u,v)$から平面への共形写像が存在するとき、$(u,v)$を等温座標という。これは第一基本形式が$ds^2 = E(du^2 + dv^2)$となることと同値である。複素数$z = u + iv$を使えば、$ds^2 = E|dz|^2$とも書ける。曲面上の任意の点で局所的には等温座標が存在することが証明できる。ここで、
    %     \begin{align*}
    %         \partial = \pd{z} = \frac{1}{2}\(\pd{u} - i\pd{v}\)\\
    %         \overline{\partial} = \pd{\overline{z}} = \frac{1}{2}\lr{\pd[]{u}+i\pd[]{v}}
    %     \end{align*}
    %     とするとガウス曲率は、
    %         \[K = -\frac{2\partial\overline{\partial}\log E}{E} = -\frac{\Delta \log E}{2E}\]
    %     となる。 

\section{曲面の方程式(正規直交枠)}
\subsection{構造方程式}
    正規直交枠$\{e_1, e_2, e_3\}$を取る。ただし$e_3$を法線ベクトルとして、$e_1 \times e_2 = e_3$となるように取る。$\{e_1, e_2\}$は接平面の基底なので
    \begin{align*}
        \pdv{p}{u} &= a^1_1e_1 + a^2_1e_2\\
        \pdv{p}{v} &= a^1_2e_1 + a^2_2e_2
    \end{align*}
    と書ける。係数行列の行列式は非零である。
        \[\theta^1 = a^1_1\dd{u} + a^1_2\dd{v}, \quad \theta^2 = a^2_1\dd{u} + a^2_2\dd{v}\]
    とおくと
    \begin{align*}
        \dd{p} &= \pdv{p}{u}\dd{u} + \pdv{p}{v}\dd{v}\\
                &= (a^1_1e_1 + a^2_1e_2)\dd{u} + (a^1_2e_1 + a^2_2e_2)\dd{v}\\
                &= (a^1_1\dd{u} + a^1_2\dd{v})e_1 + (a^2_1\dd{u} + a^2_2\dd{v})e_2\\
                &= \theta^1e_1 + \theta^2e_2
    \end{align*}
    となる。第一基本形式は
        \[I = \dd{p} \cdot \dd{p} = \theta^1\theta^1 + \theta^2\theta^2\]
    と表すことができる。

    正規直交枠$\{e_1, e_2, e_3\}$と第一基本形式$I = \theta^1\theta^1 + \theta^2\theta^2$が与えらえているとする。ここで
    \begin{align*}
        \dd{e_1} &= \omega^1_1e_1 + \omega^2_1e_2 + \omega^3_1e_3\\
        \dd{e_2} &= \omega^1_2e_1 + \omega^2_2e_2 + \omega^3_2e_3\\
        \dd{e_3} &= \omega^1_3e_1 + \omega^2_3e_2 + \omega^3_3e_3
    \end{align*}
    と表すことができる。$\omega_i^j$は1次微分形式である。$e_i \cdot e_j = \delta_{ij}$を微分して
        \[\dd{e_i} \cdot e_j + e_i \cdot \dd{e_j} = 0\]
    なので$\omega_i^j + \omega_j^i = 0$であり、特に$\omega_i^i = 0$である。これを用いると第二基本形式は
    \begin{align*}
        II  &= -\dd{p} \cdot \dd{e_3}\\
            &= - (\theta^1e_1 + \theta^2e_2) \cdot (\omega^1_3e_1 + \omega^2_3e_2)\\
            &= - \theta^1\omega^1_3 - \theta^2\omega^2_3\\
            &= \theta^1\omega^3_1 + \theta^2\omega^3_2
    \end{align*}
    $\omega^i_j$は$\theta^1, \theta^2$の線形結合だから、
    \begin{align*}
        \omega^3_1 &= b_{11}\theta^1 + b_{12}\theta^2\\
        \omega^3_2 &= b_{21}\theta^1 + b_{22}\theta^2
    \end{align*}
    となるので結局
        \[II = b_{11}\theta^1\theta^1 + b_{12}\theta^1\theta^2 + b_{21}\theta^2\theta^1 + b_{22}\theta^2\theta^2\]
    対称行列$B$の固有値は主曲率となる。つまりガウス曲率と平均曲率は
    \begin{align*}
        K &= \det B = b_{11}b_{22} - b_{12}^2\\
        2H &= \tr B = b_{11} + b_{22}
    \end{align*}
    となる。

    次に
        \[\dd{p} = \theta^1e_1 + \theta^2e_2\]
    の両辺の外微分を取ると
    \begin{align*}
        0   &= \dd{\theta^1}e_1 - \theta^1 \wedge \dd{e_1} + \dd{\theta^2}e_2 - \theta^2 \wedge \dd{e_2}\\
            &= \dd{\theta^1}e_1 - \theta^1 \wedge \sum_i \omega^i_1e_i + \dd{\theta^2}e_2 - \theta^2 \wedge \sum_i \omega^i_2e_i\\
            &= (\dd{\theta^1} - \theta^2 \wedge \omega^1_2)e_1 + (\dd{\theta^2} - \theta^1 \wedge \omega^2_1)e_2 - (\theta^1 \wedge \omega^3_1 + \theta^2 \wedge \omega^3_2)e_3
    \end{align*}
    成分ごとに比較すれば
    \begin{align*}
        \dd{\theta^1} &= \theta^2 \wedge \omega^1_2\\
        \dd{\theta^2} &= \theta^1 \wedge \omega^2_1\\
        0 &= \theta^1 \wedge \omega^3_1 + \theta^2 \wedge \omega^3_2
    \end{align*}
    これを第一構造式という。
        \[de_i = \sum_j \omega^j_ie_j\]
    の両辺の外微分を取ると
    \begin{align*}
        0    &= \sum_j d\omega^j_ie_j - \omega^j_i \wedge de_j\\
            &= \sum_j d\omega^j_ie_j - \omega^j_i \wedge \sum_k \omega^k_je_k\\
            &= \(d\omega^1_i - \sum_j \omega^j_i \wedge \omega^1_j\)e_1 + \(d\omega^2_i - \sum_j \omega^j_i \wedge \omega^2_j\)e_2 + \(d\omega^3_i - \sum_j \omega^j_i \wedge \omega^3_j\)e_3
    \end{align*}
    成分ごとに比較すると
        \[d\omega^j_i = \sum_k \omega^k_i \wedge \omega^j_k\]
    である。これを第二構造式という。$d\omega^1_2$は
    \begin{align*}
        d\omega^1_2
            &= \sum_k \omega^k_2 \wedge \omega^1_k\\
            &= \omega^3_2 \wedge \omega^1_3\\
            &= (b_{21}\theta^1 + b_{22}\theta^2) \wedge -(b_{11}\theta^1 + b_{12}\theta^2)\\
            &= (b_{11}b_{22} - b_{12}b_{21})\theta^1 \wedge \theta^2\\
            &= K\theta^1 \wedge \theta^2
    \end{align*}
    $K$は基底に依らないことが分かるので、第一基本形式のみに依存する。

    % $d\omega^3_i$は
    % \begin{align*}
    %     d\omega^3_i &= \sum_k \omega^k_i \wedge \omega^3_k\\
    %     &= \omega^1_i \wedge \omega^3_1 + \omega^2_i \wedge \omega^3_2\\
    % \end{align*}
    % $i = 1$のとき
    % \begin{align*}
    %     d\omega^3_1 &= \omega^2_1 \wedge \omega^3_2\\
    %     d(b_{11}\theta^1 + b_{12}\theta^2) &= \omega^2_1 \wedge (b_{21}\theta^1 + b_{22}\theta^2)\\   
    % \end{align*}

\subsection{ガウス=ボネの定理}
    平面の領域$D$上の正規直交基底を$\{e_1, e_2\}$、双対基底を$\{\theta^1, \theta^2\}$とする。第一基本形式$I = \theta^1\theta^1 + \theta^2\theta^2$が与えられている。

    $A \subset D$を考える。$\partial A$は滑らかな曲線$\alpha_1, \dots, \alpha_n$を繋いだものであり、$\alpha_i$から$\alpha_{i+1}$に移るときの内角と外角を$\tau_i, \epsilon_i\ (\tau_i + \epsilon_i = \pi)$とする。$\partial A$を弧長パラメータを用いて$\alpha(s)$と表すことにする。
        \[\frac{\alpha'(s)}{|\alpha'(s)|} = \cos\phi(s)e_1 + \sin\phi(s)e_2\]
    とおくと
    \begin{align*}
        \alpha''(s)
            &= \dv{s}(\cos\phi(s)e_1 + \sin\phi(s)e_2)\\
            &= -\sin\phi(s)\phi'(s)e_1 + \cos\phi(s)\dv{e_1}{s} + \cos\phi(s)\phi'(s)e_2 + \sin\phi(s)\dv{e_2}{s}\\
            &= -\sin\phi(s)\phi'(s)e_1 + \cos\phi(s)\frac{\omega^2_1}{\dd{s}}e_2 + \cos\phi(s)\phi'(s)e_2 + \sin\phi(s)\frac{\omega^1_2}{\dd{s}}e_1\\
            &= \sin\phi(s)\(-\dv{\phi}{s} + \frac{\omega^1_2}{\dd{s}}\)e_1 + \cos\phi(s)\(\dv{\phi}{s} + \frac{\omega^2_1}{\dd{s}}\)e_2
    \end{align*}
    測地的曲率$k_g$は
    \begin{align*}
        k_g &= |\alpha''(s)|\\
            &= \sqrt{\sin^2\phi(s)\(\(\dv{\phi}{s}\)^2 - 2\dv{\phi}{s}\frac{\omega^1_2}{\dd{s}} + \(\frac{\omega^1_2}{\dd{s}}\)^2\) + \cos^2\phi(s)\(\(\dv{\phi}{s}\)^2 + 2\dv{\phi}{s}\frac{\omega^2_1}{\dd{s}} + \(\frac{\omega^2_1}{\dd{s}}\)^2\)}\\
            &= \sqrt{\(\dv{\phi}{s}\)^2 - 2\dv{\phi}{s}\frac{\omega^1_2}{\dd{s}} + \(\frac{\omega^1_2}{\dd{s}}\)^2}\\
            &= \dv{\phi}{s} - \frac{\omega^1_2}{\dd{s}}\\
        k_g\dd{s} &= \dd{\phi} - \omega^1_2
    \end{align*}
    曲線に沿って一周積分すると
        \[\int_{\partial A} \omega^1_2 + \int_{\partial A} k_g \dd{s} = \int_{\partial A} \dd{\phi} = 2\pi - \sum_i \epsilon\]
    ストークスの定理より
        \[\int_{\partial A} \omega^1_2 = \int_A \dd{\omega^1_2} = \int_A K\theta^1 \wedge \theta^2\]
    だから
        \[\int_A K\theta^1 \wedge \theta^2 + \int_{\partial A} k_g \dd{s} = 2\pi - \sum_i \epsilon\]
    となる。$A$が3つの滑らかな曲線からなる場合は
        \[\int_A K\theta^1 \wedge \theta^2 + \int_{\partial A} k_g \dd{s} = \tau_1 + \tau_2 + \tau_3 - \pi\]
    特に測地三角形の場合は
        \[\int_A K\theta^1 \wedge \theta^2 = \tau_1 + \tau_2 + \tau_3 - \pi\]
    つまりガウス曲率$K$の符号によって三角形の内角の和と$\pi$の大小が決まる。

    向き付け可能な閉曲面$S$について考える。ここで三角形分割を、三角形を敷き詰めたものであって、
    \begin{enumerate}
        \item 点$p$がある三角形の面上にあり辺上にないとき、$p$を含む三角形は1つ。
        \item 点$p$がある三角形の辺上にあり頂点でないとき、$p$を含む三角形は2つ。
        \item 点$p$がある三角形の頂点であるとき、$p$を含む三角形は有限個。
    \end{enumerate}
    を満たすものとする。曲面を必ずこのように三角形分割することができる。閉曲面を三角形分割したとき、頂点の数を$v$、辺の数を$e$、面の数を$f$とすると、
    \begin{align*}
        \sum_i \int_{T_i} K\theta^1 \wedge \theta^2 + \sum_i \int_{\partial T_i} k_g \dd{s} &= \sum_i \tau_{i1} + \tau_{i2} + \tau_{i3} - \pi\\
        \int_S K\theta^1 \wedge \theta^2 &= (2v - f)\pi
    \end{align*}
    ここで$2e = 3f$だから、$2v - f = 2v - 3f + 2f = 2(v - e + f)$より
        \[\int_S K\theta^1 \wedge \theta^2 = 2\pi(v - e + f)\]
    となる。これをガウス=ボネの定理という。閉曲面$S$が与えられたとき、左辺はリーマン計量のみに依存し、右辺は三角形分割のみに依存する。よって両辺ともにリーマン計量にも三角形分割にも依存しない。特に右辺に関して、$\chi(S) = v - e + f$と書くことができる。これを曲面のオイラー標数という。
\section{変分問題}

\subsection{測地線}
    曲面上の二点を結ぶ曲線のうち、最も短いものを最短線という。曲線が最短線となる必要条件を考える。助変数を$(u_1(t), u_2(t))\ (a \leq t \leq b)$とし、対応する端点を$A, B$とする。すると$AB$の長さは汎関数
        \[s[r(t)] = \int_a^b F(u(t),\dot{u}(t)) \dd{t}\]
    で与えられる。ただし$F(u(t), \dot{u}(t)) = \sqrt{g_{ij}(u)\dot{u}^i\dot{u}^j}$である。また$\dd{s} = F\dd{t}$である。汎関数$s[r(t)]$が極値をとる条件は$F$がEuler-Lagrange方程式を満たすことである。そのような曲線を測地線と呼ぶ。最短線は測地線だが測地線は最短線ではない。まずEuler-Lagrange方程式は、
        \[\dv{t}\pdv{F}{\dot{u}^i} - \pdv{F}{u^i} = 0\]
    で表される。第一項は、
        \[\pdv{F}{\dot{u}^l} = \frac{1}{2F}g_{jk}\left\{\pdv{\dot{u}^j}{\dot{u}^l}\dot{u}^k + \dot{u}^j\pdv{\dot{u}^k}{\dot{u}^l}\right\}\]
    $\pdv*{\dot{u}^j}{\dot{u}^l}$は$j = l$のときに限り1なので、
    \begin{align*}
        \pdv{F}{\dot{u}^l}
            &= \frac{1}{2F}\(\sum_k g_{lk}\dot{u}^k + \sum_j g_{jl}\dot{u}^j\)\\
            &= \frac{1}{F}g_{il}\dot{u}^i = g_{il}\dv{u^i}{s}
    \end{align*}
    従って
        \[\dv{t}\(\pdv{F}{\dot{u}^l}\) = \(\sum_i g_{il}\dv[2]{u^i}{s} + \sum_{j,k} \pdv{g_{jl}}{u^k}\dv{u^k}{s}\dv{u^j}{s}\)\dv{s}{t}\]
    $\pdv*{g_{jl}}{u^k} = [jk, l] + [lk, j]$より、
    \begin{align*}
        &= \left[g_{il}\dv{u^i}{s} + ([jk,l] + [lk,j])\dv{u^j}{s}\dv{u^k}{s}\right]\dv{s}{t}
    \end{align*}
    そして第二項は、$[jk,l] = [kj,l], \pdv*{g_{jk}}{u^l} = [jl,k] + [kl,j]$であることに注意すれば、
    \begin{align*}
        \pdv{F}{u^l}
            &= \frac{1}{2F}\pdv{g_{jk}}{u^l}\dot{u}^j\dot{u}^k\\
            &= \frac{1}{2F}([jl, k] + [kl, j])\dot{u}^j\dot{u}^k\\
            &= \frac{1}{F}[kl, j]\dot{u}^j\dot{u}^k\\
            &= [kl, j]\dv{u^j}{s}\dv{u^k}{s}\dv{s}{t}
    \end{align*}
    なので、
    \begin{align*}
        \dv{t}\pdv{F}{\dot{u}^l} - \pdv{F}{u^l}
            &= \(g_{il}\dv[2]{u^i}{s} + [jk, l]\dv{u^j}{s}\dv{u^k}{s}\)\dv{s}{t}\\
            &= g_{il}\(\dv[2]{u^i}{s} + \chr{i}{jk}\dv{u^j}{s}\dv{u^k}{s}\)\dv{s}{t}\\
            &= 0
    \end{align*}
    $\det g_{il} \neq 0, \dv*{s}{t} = F \neq 0$であるので、
        \[
            \begin{aligned}
                \dv[2]{u^i}{s} + \chr{i}{jk}\dv{u^j}{s}\dv{u^k}{s} &= 0 & (i = 1, 2, \dots, n)
            \end{aligned}
        \]
    これを測地線の方程式という。またこれを
    \begin{align*}
        \dv{u^i}{s} &= v^i\\
        \dv{v^i}{s} &= -\chr{i}{jk}v^jv^k
    \end{align*}
    と書き換えると、1階の連立常微分方程式となるので、次の定理が導かれる。
    \begin{thm}[測地線]
        リーマン多様体$M$の任意の点で任意の方向にただ1本の測地線が引ける。
    \end{thm}

\subsection{Weierstrassの表現}
    二次元リーマン多様体、つまり曲面の場合を考える。このとき$k > 0$として$F(u, v, k\dot{u}, k\dot{v}) = kF(u, v, \dot{u}, \dot{v})$となって$F$は$\dot{u}, \dot{v}$について正斉次である。この式を$k$で微分し、$k = 1$とすると、
        \[\pdv{F}{\dot{u}}\dot{u} + \pdv{F}{\dot{v}}\dot{v} = F\]
    となる。両辺を$\dot{u^i}$で微分して
    \begin{align*}
        \pdv[2]{F}{\dot{u}}\dot{u} + \pdv{F}{\dot{v}}{\dot{u}}\dot{v} &= 0\\
        \pdv{F}{\dot{u}}{\dot{v}}\dot{u} + \pdv[2]{F}{\dot{v}}\dot{v} &= 0
    \end{align*}
    が導かれ、
        \[\pdv[2]{F}{\dot{u}} : \pdv{F}{\dot{u}}{\dot{v}} : \pdv[2]{F}{\dot{v}} = \dot{v}^2 : -\dot{u}\dot{v} : \dot{u}^2\]
    となる。比例因子を$c$として
    \begin{align*}
        \pdv[2]{F}{\dot{u}} &= c\dot{v}^2\\
        \pdv{F}{\dot{u}}{\dot{v}} &= -c\dot{u}\dot{v}\\
        \pdv[2]{F}{\dot{v}} &= c\dot{u}^2
    \end{align*}
    また
    \begin{align*}
        \pdv{F}{\dot{u}} &= \frac{g_{11}\dot{u} + g_{12}\dot{v}}{F}\\
        \pdv{F}{\dot{v}} &= \frac{g_{21}\dot{u} + g_{22}\dot{v}}{F}
    \end{align*}
    なので
    \begin{align*}
        c   &= \frac{F_{\dot{u}\dot{u}}}{\dot{v}^2}\\
            &= \frac{1}{\dot{v}^2}\frac{g_{11}F - (g_{11}\dot{u} + g_{12}\dot{v})F_{\dot{u}}}{F^2}\\
            &= \frac{1}{\dot{v}^2}\frac{g_{11}F^2 - (g_{11}\dot{u} + g_{12}\dot{v})^2}{F^3}\\
            &= \frac{g_{11}g_{22} - g_{12}^2}{F^3}
    \end{align*}
    これをEuler-Lagrange方程式に代入すると
    \begin{align*}
        F_{u} - \dv{t}F{\dot{u}} &= \dot{v}T\\
        F_{v} - \dv{t}F{\dot{v}} &= -\dot{u}T
    \end{align*}
    ただし
        \[T = F_{u\dot{v}} - F_{\dot{u}v} + c(\dot{u}\ddot{v} - \dot{v}\ddot{u})\]
    従って曲面の場合、測地線の方程式は$T = 0$と同値である。これをWeierstrassの表現という。
            
% \subsection{極小曲面}
\section{接続}
多様体上のスカラー場の微分は接ベクトルを作用させるだけで良かった。同様に、多様体上の点$p$における$X(p)$に沿ったベクトル場$Y$の方向微分として、ユークリッド空間の場合に倣い、
    \[D_XY(p) = \lim_{t \to 0} \frac{Y(p + X(p)t) - Y(p)}{t}\]
が考えられる。しかし、$Y(p)$と$Y(p + X(p)t)$はもはや異なる接空間に属する接ベクトルであり、単純に引き算を考えることはできない。ユークリッド空間の場合は異なる位置にあるベクトルに対して自然な対応を考えることができたのだった。異なる接空間に属する接ベクトルを比較するには、接空間を何らかの方法で結び付ける必要がある。これを接続と呼ぶ。

\subsection{ベクトル束の接続}
    空間$M$の各点に対してベクトル空間を対応させたものを$M$上のベクトル束という。

    \begin{dfn}[ベクトル束]
        位相空間$M, E$、写像$\pi: E \rightarrow M$を考える。$E_p = \pi^{-1}(p)$が$r$次元のベクトル空間を成し、$p$上のファイバーと呼ばれる。$E$が局所的に$M$とファイバーの直積となる、つまり$p \in M$に対して、近傍$U$と微分同相写像$\phi: \pi^{-1}(U) \rightarrow U \times \R^r$が存在し、$v \in \pi^{-1}(p) \iff \phi(v) \in \{p\} \times \R^r$であり、線形同型となっている。このとき、$(E, \pi, M, \phi)$または単に$E$を$M$上のベクトル束と呼ぶ。ベクトル束$(E, \pi, M, \phi)$に対して、$M$を底空間、$E$を全空間、$\pi$を射影という。

        $p \in M \mapsto v_p \in E_p$を$E$の切断という。
    \end{dfn}
    
    可微分多様体の各点に接空間を対応させた接ベクトル束(接束)や余接空間を対応させた余接ベクトル束(余接束)はベクトル束の例である。接ベクトル束の切断は(接)ベクトル場であり、余接ベクトル束の切断は1次微分形式である。

    ベクトル束の直和、テンソル積、双対空間、外積、対称積をそれぞれ各点上のファイバーに適用することで定義することができる。
    
    $E$の微分可能な切断全体を$\Gamma(E)$と書く。$\Gamma(E)$は$\Omega^0(M)$上の無限次元ベクトル空間である。また$T^*M \otimes E$の切断は$p \in M \mapsto \omega_p \otimes v_p \in T_p^*(M) \otimes E_p$となる。
    \begin{dfn}[共変微分]
        $n$次元多様体$M$上のベクトル束$(E, \pi, M)$に対して、線形写像$\nabla: \Gamma(E) \rightarrow \Gamma(T^*M \otimes E)$であって、$f \in \Omega^0(M), v \in \Gamma(E)$に対して
            \[\nabla(fv) = df \otimes v + f\nabla v\]
        を満たすものを$E$の共変微分という。また$X: p \in M \mapsto X_p \in T_p(M)$に対して
            \[\nabla_Xv(p) = \nabla v(X_p) \in E_p\]
        なる$\nabla_Xv \in \Gamma(E)$を$X$に沿った$v$の共変微分という。
    \end{dfn}
    $M$の各点$p$にベクトル空間$\R^r$の基底$\{e_1, e_2, \dots, e_r\}$を対応させたものを局所標構場という。
    
    \begin{dfn}[接続形式]
        ファイバー$\R^r$の局所標構場$e = \{e_1, e_2, \dots, e_r\}$を定めたとき、その共変微分は1次微分形式$\omega_i^j: \mathfrak{X}(M) \rightarrow \mathfrak{F}(M)$を用いて
            \[\nabla e_i = \sum_j \omega_i^j \otimes e_j\]
        と書ける。行列$\omega = (\omega_i^j)$を$\nabla$の接続形式(connection form)という。
    \end{dfn}
    一般のベクトル$v = \sum_i v^i e_i$に対して、
    \begin{align*}
        \nabla v &= \nabla \sum_i v^i e_i\\
                 &= \sum_i (dv^i \otimes e_i + v^i \nabla e_i)\\
                 &= \sum_i \(dv^i \otimes e_i + v^i \sum_j \omega_i^j \otimes e_j\)\\
                 &= \sum_i \(dv^i + \sum_j \omega^i_j v^j\) \otimes e_i\\
        (\nabla v)^i &= dv^i + \sum_j \omega^i_j v^j
    \end{align*}
    なので
        \[\nabla = d + \omega\]
    となる。つまり接続形式は外微分からのずれを表す。

\subsection{曲率}
    微分形式の外微分の拡張として、共変外微分$D: \Omega^k(M, E) \rightarrow \Omega^{k+1}(M, E)$を
    \begin{align*}
        D\omega(X_0, X_1, X_2, \dots, X_k)
            &= \frac{1}{n!}\sum_i (-1)^i \nabla_{X_i}(\omega(X_0, \dots, X_{i-1}, X_{i+1}, \dots, X_k))\\
            &\quad + \sum_{i<j} (-1)^{i+j} \omega([X_i, X_j], X_0, \dots, X_{i-1}, X_{i+1}, \dots, X_{j-1}, X_{j+1}, \dots, X_k)
    \end{align*}
    と定義する。共変外微分は
    \begin{enumerate}
        \item $f \in \Omega^0(M, E) = \Gamma(E)$に対して$Df = \nabla f \in \Gamma(T^*M \otimes E) \simeq \Omega^1(M, E)$
        \item $D(f\omega) = \nabla f \wedge \omega + f d\omega$
        \item $\omega_1 \in \Omega^r(M, E), \omega_2 \in \Omega^s(M, E)$に対して、$D(\omega_1 \wedge \omega_2) = D\omega_1 \wedge \omega_2 + (-1)^r \omega_1 \wedge D\omega_2$
    \end{enumerate}
    を満たす。特に$\omega^0 \in \Omega^0(M, E), \omega^1 \in \Omega^1(M, E)$に対して
    \begin{align*}
        D\omega^0(X) &= \nabla_X\omega\\
        D\omega^1(X, Y) &= \frac{1}{2}(\nabla_X\omega^1(Y) - \nabla_Y\omega^1(X) - \omega^1([X, Y]))
    \end{align*}
    である。$D^2\omega^0(X, Y)$を計算すると
    \begin{align*}
        D^2\omega^0(X, Y)
            &= D(\nabla_X\omega)(X, Y)\\
            &= \frac{1}{2}(\nabla_X\nabla_Y\omega^0 - \nabla_Y\nabla_X\omega^0 - \nabla_{[X, Y]}\omega^0)
    \end{align*}
    外微分と異なり$D^2$は一般に0にはならない。$R(X, Y)v = 2(D^2v)(X, Y)$は曲面の曲がり具合を表す量であり、曲率と呼ばれる。
    \begin{align*}
        &R(X, Y): \Gamma(E) \rightarrow \Gamma(E)\\
        &R(X, Y)v = (\nabla_X\nabla_Y - \nabla_Y\nabla_X - \nabla_{[X, Y]})v
    \end{align*}
    である。

    \begin{align*}
        D^2 e_i &= D\(\sum_j \omega_i^j e_j\)\\
                &= \sum_j (d\omega_i^j e_j - \omega_i^j \wedge De_j)\\
                &= \sum_j \(d\omega_i^j \otimes e_j - \omega_i^j \wedge \(\sum_k \omega_j^k e_k\)\)\\
                &= \sum_j \(d\omega_i^j - \sum_k \omega_i^k \wedge \omega_k^j\)e_j\\
                &= \sum_j \(d\omega_i^j + \sum_k \omega_k^j \wedge \omega_i^k\)e_j
    \end{align*}
    となる。
    \begin{dfn}[曲率形式]
        局所標構場$e$が与えらえたとき、2次微分形式$\Omega_i^j: \mathfrak{X}(M) \times \mathfrak{X}(M) \rightarrow \mathfrak{F}(M)$を接続形式によって
            \[\Omega_i^j = d\omega_i^j + \sum_k \omega_k^j \wedge \omega_i^k\]
        と定義したとき、$\Omega = (\Omega_i^j)$を曲率形式(curvature form)という。
    \end{dfn}
        \[Re_i = \sum_j \Omega_i^j e_j\]
    となる。これを構造方程式という。

\subsection*{平行移動}
    \begin{dfn}[平行]
        任意のスカラー場$f$に対して$Xf(c(t)) = \dv*{t}f(c(t))$なる$M$上の曲線$c(t)\ (a \leq t \leq b)$を考える。この曲線上において$E$の切断$v$の共変微分$\nabla_Xv$が常に0になるとき、$v$は$c(t)$に沿って平行であるという。
    \end{dfn}
    一般に二点におけるファイバーの元が平行かどうかは経路に依存する。

    曲線と始点におけるファイバーの元を定めれば、終点においてそれと平行であるようなファイバーの元が一意に定まる。これを$c(t)$に沿った平行移動という。

    $p \in M$に対して、$p$から$p$へと至る閉曲線$c(t)\ (a \leq t \leq b, c(a) = c(b) = p)$を考える。この閉曲線に沿ってファイバーの元を平行移動させることによってファイバーの自己同型写像$\tau_c$が得られる。$c$を逆に辿った曲線を$c^{-1}$、$c_1, c_2$の連結を$c_2 \circ c_1$として
        \[\tau_{c^{-1}} = \tau_c^{-1}, \quad \tau_{c_2 \circ c_1} = \tau_{c_2} \circ \tau_{c_1}\]
    と定義すれば、$\tau_c$の全体は群となる。これを$p$を基点とするホロノミ―群$\Psi_p$と呼ぶ。

\subsection{主ファイバー束の接続}
    \begin{dfn}[主ファイバー束]
        多様体$P, M$、全射な微分可能写像$\pi: P \rightarrow M$、リー群$G$を考え、以下の条件を満たすとき、$(P, \pi, M, G)$または単に$P$を、$G$を構造群とする$M$上の主ファイバー束(主$G$束)という。
        \begin{enumerate}
            \item $G$は$P$に右から作用する。つまり
                \[(u, a) \in P \times G \mapsto ua \in P\]
            は微分可能であり、$a \in G$を固定したとき
                \[R_a: u \in P \mapsto ua \in P\]
            は微分同相写像であり、
            \begin{align*}
                (ua)b &= u(ab)\\
                ue &= u
            \end{align*}
            となる。
            \item $u \in \pi^{-1}(x)$ならば$ua \in \pi^{-1}(x)$である。
            \item $R_a$は単純推移的である、つまり$\pi(u_1) = \pi(u_2)$ならば$a \in G$が唯一存在して$u_1a = u_2$となる。
            \item $P$は局所的に$M \times G$と微分同相になる。$p \in M$に対してその近傍$U$があり、微分同相写像
                \[\phi: u \in \pi^{-1}(U) \mapsto (\pi(u), \phi(u)) \in U \times G\]
            であって、任意の$u, a$に対して$\phi(ua) = \phi(u)a$が成り立つものがある。
        \end{enumerate}
    \end{dfn}

    \begin{dfn}[基本ベクトル場]
        主$G$束$P$とリー群$G$に付随するリー環$\mathfrak{g}$を考える。$u \in P$に対して、軌道$u\exp(tA)$の$t = 0$における接ベクトル$A^*_u \in T_uP$を対応させる。$P$上のベクトル場
            \[A^*: u \in P \mapsto A^*_u \in T_uP\]
        を$A$に対応する基本ベクトル場という。
    \end{dfn}

    \begin{dfn}[接続(主ファイバー束)]
        $M$の近傍$U_1, U_2$について同型写像
            \[\phi_1(x): E_x \rightarrow \R^r, \quad \phi_2(x): E_x \rightarrow \R^r\]
        があり、変換関数を
            \[\phi: U_1 \cap U_2 \rightarrow GL(r, \R)\]
        と定義する。$\mathfrak{gl}(r, \R)$に値を取る$P$上の1次微分形式であって、
            \[\omega_2 = \phi^{-1}\omega_1\phi + \phi^{-1}d\phi\]
        なる変換を受けるものを$P$の接続と呼ぶ。
    \end{dfn}

    \begin{thm}
        リー環$\mathfrak{g}$に値を持つ$P$上の1次微分形式
        $\omega$について、$\omega$が$P$の接続形式であることは以下と同値である。
        \begin{enumerate}
            \item $R_a^*\omega = a^{-1}\omega a$
            \item $\omega(A^*) = A$
        \end{enumerate}
    \end{thm}

    \begin{dfn}[接続]
        $G$を構造群とする$M$上の主ファイバー束$P$の点$u$において、$T_u(P)$の垂直部分空間及び水平部分空間を
        \begin{align*}
            V_u &= \{X \in T_uP \mid \pi^*(X) = 0\}\\
            H_u &= \{X \in T_uP \mid \omega_u(X) = 0\}
        \end{align*}
        で定義する。
        \begin{enumerate}
            \item $T_u(P) = V_u \oplus H_u$
            \item $H_{ua} = R_a(H_u)$
        \end{enumerate}
        を満たすとき対応$\Gamma: u \mapsto H_u$を$P$の接続という。
    \end{dfn}

    $P$上の1次微分形式$\omega$と$H_u \subset T_u(P)$に対して、$\omega$が接続形式となることと$H_u$が以下を満たすことは同値である。
    \begin{enumerate}
        \item $T_u(P) = V_u \oplus H_u$
        \item $H_{ua} = R_a(H_u)$
    \end{enumerate}

    $X \in T_u(P)$に対して
        \[X = vX + hX \quad vX \in V_u, hX \in H_u\]
    と書けるとき、$k$次微分形式$\omega: P \rightarrow V$に対して共変外微分を
        \[D\omega(X_1, \dots, X_{k+1}) = (d\omega)(hX_1, \dots, hX_{k+1})\]
    と定義することができる。
    \begin{gather*}
        D(fX) = X \otimes df + f DX\\
        D(\omega_1 \wedge \omega_2) = D\omega_1 \wedge \omega_2 + (-1)^{\deg\omega_1}\omega_1 \wedge d\omega_2
    \end{gather*}
    が成り立つ。

% \subsection{接続形式}
%         \[L_a: x \mapsto ax\]
%     を左移動という。$G$上のベクトル場$X$が任意の$a, x \in G$に対して$L_aX_x = X_{ax}$を満たすとき、左不変と呼ぶ。$G$上の左不変なベクトル場全体$\mathfrak{g}$をリー環と呼ぶ。任意の$A \in T_e(G)$に対して
%         \[X_x = dL_x(A)\]
%     と定義すると、$X \in \mathfrak{g}$となる。

%     $A$の局所1パラメータ変換群$\{\phi_t\}$に対して$a_t = \phi_t(e) = \exp tX$とおく。$R_{a_t}: u \mapsto ua_t$が引き起こすベクトル場を基本ベクトル場$A*$という。
%     $a \in G$に対して
%         \[ad(a): x \in \mathfrak{g} \mapsto axa^{-1} \in \mathfrak{g}\]
%     を随伴表現という。
%     \begin{dfn}[接続形式]
%         1次微分形式$\omega: T_u(P) \rightarrow \mathfrak{g}$であって、
%         \begin{enumerate}
%             \item $\omega(A*) = A$
%             \item $\omega(dR_a(X)) = ad(a^{-1})(\omega(X))$
%         \end{enumerate}
%         を満たすものを接続形式(connection form)と呼ぶ。
%     \end{dfn}

%     \begin{dfn}[曲率形式]
%         接続形式$\omega$に対して、その共変外微分
%             \[\Omega = D\omega = d\omega + \omega \wedge \omega\]
%         で定義される2次微分形式$\Omega: T_u(P) \times T_u(P) \rightarrow \mathfrak{g}$を曲率形式(curvature form)と呼ぶ。
%     \end{dfn}
    % \begin{proof}
    %     (1)$X, Y \in H_u$のとき
    %     $\omega(X) = \omega(Y) = 0$より$\Omega(X, Y) = d\omega(X, Y)$は定義そのもの。
    %     (2)$X = A*, Y = B*(A, B \in \mathfrak{g})$のとき
    %     \begin{align*}
    %         d\omega(A*, B*)
    %         &= \frac{1}{2}(A*\omega(B*) - B*\omega(A*) - \omega([A*, B*]))\\
    %         &= \frac{1}{2}(A*B - B*A - [A, B]) = 0\\
    %         d\omega(A*, B*) + \frac{1}{2}(\omega(A*)\omega(B*) - \omega(B*)\omega(A*)) = 0\\
    %         \Omega(A*, B*) = 0\\
    %     \end{align*}
    %     (3)$X \in H_u, Y = A*(A \in \mathfrak{g})$
    %     \begin{align*}
    %         d\omega(X, A*)
    %         &= \frac{1}{2}(X\omega(A*) - A*\omega(X) - \omega([X, A*]))\\
    %         &= XA - A*\omega(X) - \omega([X, A*])\\
    %         &= A*\omega(X)
    %     \end{align*}
    % \end{proof}

    % $\omega$は1次微分形式なので、水平ベクトル場$X, Y$に関して
    %     \[2d\omega(X, Y) = X\omega(Y) - Y\omega(X) - \omega([X, Y])\]
    % $\omega = \alpha dx$なら
    % \begin{align*}
    %     2d\omega(X, Y)
    %     &= 2d\alpha \wedge dx(X, Y) = d\alpha(X)dx(Y) - d\alpha(Y)dx(X)\\
    %     &= X\alpha dx(Y) - Y\alpha dx(X)\\
    %     &= X(\alpha dx(Y)) - Y(\alpha dx(X)) - \alpha X(dx(Y)) + \alpha Y(dx(X))\\
    %     &= X(\alpha dx(Y)) - Y(\alpha dx(X)) - (\alpha dx(XY) - \alpha dx(YX))\\
    %     &= X\omega(Y) - Y\omega(X) - \omega([X, Y])\\
    % \end{align*}

    %     \[2\Omega(X, Y) = 2d\omega(X, Y) = -\omega([X, Y])\]

% \begin{dfn}{持ち上げ}
%     $M$上の曲線$\tau = \{x(t) \mid 0 \leq t \leq 1\}$に対して$P$上の曲線$\tau^{*} = \{u(t) \mid 0 \leq t \leq 1\}$が
%     \begin{enumerate}
%         \item $\tau^{*}$の各点での接ベクトルは水平
%         \item $\pi(u(t)) = x(t)$
%     \end{enumerate}
%     を満たすとき持ち上げと呼ぶ。
% \end{dfn}
% \begin{thm}
%     $M$上の曲線$\tau = \{x(t) \mid 0 \leq t \leq 1\}$が与えられたとき、$\pi(u_0) = x(0)$となるような$u_0$に対して$u(0) = u_0$であるような持ち上げ$\tau^{*} = \{u(t) \mid 0 \leq t \leq 1\}$がただ一つ存在する。
% \end{thm}
% \begin{thm}
%     リー群$G$、リー環$\mathfrak{g}$として、$\mathfrak{g}$を$e$における接空間と同一視する。$\mathfrak{g}$上の曲線$Y(t)(0 \leq t \leq 1)$が与えられたとき、$G$上の曲線$a(t)(0 \leq t \leq 1)$であって、
%     \begin{align*}
%         a'(t)a(t)^{-1} = Y(t)\\
%         a(0) = e
%     \end{align*}
%     を満たすものがただ一つ存在する。
% \end{thm}
\section{アフィン接続}

\subsection{アフィン接続}
    \begin{dfn}[接$n$枠束(frame bundle)]
        線形同型写像$u: (a_1, a_2, \dots, a_n) \in \R^n \mapsto a_1X_1 + a_2X_2 + \dots + a_nX_n \in T_p(M)$を$p \in M$における接$n$枠と呼ぶ。$p$における接$n$枠全体を$L_p(M)$として、$L(M) = \cup \{p\} \times L_p(M)$を接$n$枠束という。
    \end{dfn}

    \begin{dfn}[アフィン接続]
        主ファイバー束$L(M)$の切断をアフィン接続(線形接続)という。
    \end{dfn}

    \begin{dfn}[アフィン接続(線形接続)]
        接束$TM$の接続を$M$のアフィン接続または線形接続という。
    \end{dfn}

    \begin{dfn}[共変微分]
        アフィン接続の共変微分は写像$\nabla: (X, Y) \in \mathfrak{X}(M) \times \mathfrak{X}(M) \mapsto \nabla_XY \in \mathfrak{X}(M)$であって、以下を満たすものを言う。
        \begin{enumerate}
            \item $\nabla_X(Y_1 + Y_2) = \nabla_XY_1 + \nabla_XY_2$
            \item $\nabla_{X_1 + X_2}(Y) = \nabla_{X_1}Y + \nabla_{X_2}Y$
            \item $\nabla_{fX}Y = f\nabla_XY$
            \item $\nabla_X(fY) = (Xf)Y + f\nabla_XY$
        \end{enumerate}
        ただし$f \in \mathfrak{F}(M)$である。
    \end{dfn}

    % \begin{thm}
    %     多様体$M$と上の条件を満たす写像$\nabla$が与えられているとき、これを共変微分とするアフィン接続が存在する。
    % \end{thm}
    % つまり共変微分を与えることでアフィン接続を考えることができる。

    座標系$(x^i)$を定めたとき$\nabla_{\pdv{x^i}}Y$を単に$\nabla_iY$と書く。
        \[\nabla_{\pdv{x^i}}\pdv{x^j} = \sum_k \Gamma^k_{ij}\pdv{x^k}\]
    となる$\Gamma^k_{ij}$を座標系$(x^i)$に関する接続係数またはクリストッフェル記号という。$X = \sum X^i\pdv*{x^i}, Y = \sum Y^i\pdv*{x^i}$のとき
    \begin{align*}
        \nabla_XY
            &= \sum_i X^i\nabla_{\pdv{x^i}}\(\sum_j Y^j\pdv{x^j}\)\\
            &= \sum_{i,j} X^i\nabla_{\pdv{x^i}}\(Y^j\pdv{x^j}\)\\
            &= \sum_{i,j} X^i\pdv{Y^j}{x^i}\pdv{x^j} + X^iY^j\nabla_{\pdv{x^i}}\pdv{x^j}\\
            &= \sum_{i,j} X^i\pdv{Y^j}{x^i}\pdv{x^j} + X^iY^j\sum_k \Gamma^k_{ij}\pdv{x^k}
    \end{align*}
    となる。

    アフィン接続の不変量として曲率と捩率がある。

    % 可微分写像$f: M_1 \rightarrow M_2$に対して、可微分写像
    %     \[X: p \in M_1 \mapsto X_p \in T(M_2)\]
    % であって、
    %     \[\pi(X_p) = f(p)\]
    % を満たすようなものを$f$に沿うベクトル場という。$f$に沿うベクトル場全体を$\mathfrak{X}_f$で表す。

    % $M$がアフィン接続を持ち、共変微分$\nabla$が与えられているとする。写像
    %     \[(X, Y) \in \mathfrak{X}(M) \times \mathfrak{X}_f \mapsto \nabla_XY \in \mathfrak{X}\]
    % で
    % \begin{enumerate}
    %     \item $\nabla_{X_1 + X_2}Y = \nabla_{X_1}Y + \nabla_{X_2}Y$
    %     \item $\nabla_{\phi X}Y = \phi\nabla_XY$
    %     \item $\nabla_X(Y_1 + Y_2) = \nabla_XY_1 + \nabla_XY_2$
    %     \item $\nabla_X \phi Y = (X\phi)Y + \phi\nabla_XY$
    % \end{enumerate}
    % を満たすものがただ一つ存在する。これを$f$に沿う共変微分という。

\subsection{曲率テンソル}
    \begin{dfn}[曲率テンソル]
        接ベクトル場$X, Y, Z \in \mathfrak{X}(M)$に対して$(1, 3)$型テンソル場
            \[R(X, Y)Z = (\nabla_X\nabla_Y - \nabla_Y\nabla_X - \nabla_{[X, Y]})Z\]
        を曲率テンソル場という。曲率テンソルまたは単に曲率ともいう。
    \end{dfn}
    \begin{proof}
        $R$がテンソルになることを示す。和に関しては自明。関数倍については
        \begin{align*}
            R(fX, Y)Z &= \nabla_{fX}\nabla_YZ - \nabla_Y\nabla_{fX}Z - \nabla_{[fX, Y]}Z\\
                      &= f\nabla_X\nabla_YZ - \nabla_Y(f\nabla_XZ) - \nabla_{[fXY - Y(fX)]}Z\\
                      &= f\nabla_X\nabla_YZ - Yf\nabla_XZ - f\nabla_Y\nabla_XZ - \nabla_{[fXY - YfX - fYX]}Z\\
                      &= f\nabla_X\nabla_YZ - Yf\nabla_XZ - f\nabla_Y\nabla_XZ - f\nabla_{[XY - YX]}Z + Yf\nabla_XZ\\
                      &= fR(X, Y)Z
        \end{align*}
        同様に
            \[R(X, fY)Z = fR(X, Y)Z\]
        また
        \begin{align*}
            R(X, Y)(fZ)
                &= \nabla_X\nabla_Y(fZ) - \nabla_Y\nabla_X(fZ) - \nabla_{[X, Y]}(fZ)\\
                &= \nabla_X(YfZ + f\nabla_YZ) - \nabla_Y(XfZ + f\nabla_XZ) - ([X, Y]fZ + f\nabla_{[X, Y]}Z)\\
                &= (XYfZ + Yf\nabla_XZ + Xf\nabla_YZ + f\nabla_X\nabla_YZ)\\
                &\quad - (YXfZ + Xf\nabla_YZ + Yf\nabla_XZ + f\nabla_Y\nabla_XZ)\\
                &\quad - ([X, Y]fZ + f\nabla_{[X, Y]}Z)\\
                &= fR(X, Y)Z
        \end{align*}
        だから、$R(X, Y)Z$は多重線形写像つまりテンソルである。
    \end{proof}

    $R$の成分を
        \[R\(\pdv{x^i}, \pdv{x^j}\)\pdv{x^k} = \sum_l R^l_{kij}\pdv{x^l}\]
    とする。

    $Z = Z^k\pdv*{x^k}$について
    \begin{align*}
        R\(\pdv{x^i}, \pdv{x^j}\)Z
            &= \sum_{k, l} R^l_{kij}Z^k \pdv{x^l}\\
        R\(\pdv{x^i}, \pdv{x^j}\)Z
            &= \sum_k Z^k R\(\pdv{x^i}, \pdv{x^j}\)\pdv{x^k}\\
            &= \sum_k Z^k[\nabla_i, \nabla_j]\pdv{x^k}
    \end{align*}
    つまり
        \[[\nabla_X, \nabla_Y]A^\mu = R^\mu_{ijk}A^i\]
    である。通常の偏微分が可換、つまり$[\partial_X, \partial_Y] = 0$なのに対して、共変微分は非可換であることが分かる。また非可換さの度合いを表す$R^\mu_{ijk}$が空間の曲がり具合を表すことが確認できる。
    
    曲率テンソルの成分を接続係数を用いて表す。
    \begin{align*}
        R\(\pdv{x^i}, \pdv{x^j}\)\pdv{x^k}
            &= \nabla_{\pdv{x^i}}\sum_l \Gamma^l_{jk}\pdv{x^l} - \nabla_{\pdv{x^j}}\sum_l \Gamma^l_{ik}\pdv{x^l} - \nabla_0Z\\
            &= \sum_l \nabla_{\pdv{x^i}}\(\Gamma^l_{jk}\pdv{x^l}\) - \sum_l \nabla_{\pdv{x^j}}\(\Gamma^l_{ik}\pdv{x^l}\)\\
            &= \sum_l \pdv{\Gamma^l_{jk}}{x^i}\pdv{x^l} + \Gamma^l_{jk}\nabla_{\pdv{x^i}}\pdv{x^l} - \sum_l \pdv{\Gamma^l{ik}}{x^j}\pdv{x^l} + \Gamma^l_{ik}\nabla_{\pdv{x^j}}\pdv{x^l}\\
            &= \sum_l \pdv{\Gamma^l_{jk}}{x^i}\pdv{x^l} + \Gamma^l_{jk}\sum_m \Gamma^m_{il}\pdv{x^m} - \sum_l \pdv{\Gamma^l_{ik}}{x^j}\pdv{x^l} + \Gamma^l_{ik}\sum_m \Gamma^m_{jl}\pdv{x^m}\\
            &= \sum_l \pdv{\Gamma^l_{jk}}{x^i}\pdv{x^l} + \sum_l\sum_m \Gamma^l_{jk}\Gamma^m_{il}\pdv{x^m} - \sum_l \pdv{\Gamma^l_{ik}}{x^j}\pdv{x^l} - \sum_l\sum_m \Gamma^l_{ik}\Gamma^m_{jl}\pdv{x^m}\\
    \intertext{第二項と第四項で添え字$l, m$を入れ替える。}
            &= \sum_l \pdv{\Gamma^l_{jk}}{x^i}\pdv{x^l} + \sum_l\sum_m \Gamma^m_{jk}\Gamma^l_{im}\pdv{x^l} - \sum_l \pdv{\Gamma^l_{ik}}{x^j}\pdv{x^l} - \sum_l\sum_m \Gamma^m_{ik}\Gamma^l_{jm}\pdv{x^l}\\
            &= \sum_l \(\pdv{\Gamma^l_{jk}}{x^i} - \pdv{\Gamma^l_{ik}}{x^j} + \sum_m \Gamma^m_{jk}\Gamma^l_{im} - \Gamma^m_{ik}\Gamma^l_{jm}\)\pdv{x^l}\\
        R^l_{kij}
            &= \pdv{\Gamma^l_{jk}}{x^i} - \pdv{\Gamma^l_{ik}}{x^j} + \sum_m \Gamma^m_{jk}\Gamma^l_{im} - \Gamma^m_{ik}\Gamma^l_{jm}
    \end{align*}
    となる。$R^l_{kij} = - R^l_{kji}$が成り立つ。

\subsection{捩率テンソル}
    \begin{dfn}[標準形式]
        局所標構場$e$が与えらえたとき、双対基底(1次微分形式)$\theta^i: \mathfrak{X}(M) \rightarrow \mathfrak{F}(M)$は
            \[\theta^i(e_j) = \delta^i_j\]
        となる。$\theta = (\theta^i)$を標準形式(solder form)という。
    \end{dfn}

    \begin{dfn}[捩率形式]
        局所標構場$e$が与えられたとき、2次微分形式$\Theta^i: \mathfrak{X}(M) \times \mathfrak{X}(M) \rightarrow \mathfrak{F}(M)$を標準形式によって
            \[\Theta^i = d\theta^i + \sum_j \omega^i_j \wedge \theta^j\]
        と定義する。$\Theta = (\Theta^i)$を捩率形式(torsion form)という。
    \end{dfn}

    曲率形式と捩率形式の定義
    \begin{align*}
        \Omega^i_j &= d\omega^i_j + \sum_k \omega^i_k \wedge \omega^k_j\\
        \Theta^i   &= d\theta^i + \sum_j \omega^i_j \wedge \theta^j
    \end{align*}
    を第一及び第二構造式と呼ぶ。

    \begin{dfn}[標準形式]
        1次微分形式$\theta: \mathfrak{X}(M) \rightarrow \mathfrak{F}(M)$であって、$u \in L_p(M), X \in T_u(L(M))$に対して
            \[\theta(X) = u^{-1}(d\pi^*(X))\]
        となるものを標準形式(solder form)と呼ぶ。
    \end{dfn}

        \[\theta(X) = 0 \iff \pi^*(X) = 0(\text{ファイバーに沿ったベクトル、つまり垂直})\]
    $u = (e_1, e_2, \dots, e_n), X = d\pi^{-1}(a_1e_1 + a_2e_2 + \dots + a_ne_n)$なら
        \[\theta(X) = (a_1, a_2, \dots, a_n)\]
    となる。$\theta^i(e_j) = \delta^i_j$だから$\{\theta^i\}$は双対基底となる。

    % \begin{dfn}[捩率形式]
    %     標準形式の共変外微分
    %         \[\Theta = D\theta = \dd{\theta} + \omega \wedge \theta\]
    %     で定義される2次微分形式$\Theta: T_u(P) \times T_u(P) \rightarrow \R^n$を捩率形式(torsion form)と呼ぶ。
    % \end{dfn}

    \begin{dfn}[捩率テンソル]
        $u \in P$、ベクトル場$X, Y$に対して$(1, 2)$型テンソル場
            \[T(X, Y)_u = u(2\Theta_u(\pi^{-1}(X), \pi^{-1}(Y)))\]
        を捩率テンソル場という。捩率テンソルまたは単に捩率ともいう。共変微分を用いて表すと
        \begin{align*}
            2\Theta(X*, Y*)
                &= 2d\theta(X*, Y*) = X*\theta(Y*) - Y*\theta(X*) - \theta([X*, Y*])\\
                &= \nabla_XY - \nabla_YX - \theta([X, Y]*)\\
            T(X, Y)_u
                &= u(2\Theta_u(\pi^{-1}(X), \pi^{-1}(Y)))\\
                &= \nabla_XY - \nabla_YX - [X, Y]
        \end{align*}
        となる。
    \end{dfn}
    \begin{proof}
        $T$がテンソルであることを示す。曲率テンソルと同様に和に関しては自明。関数倍に関しては
        \begin{align*}
            T(fX, Y)
                &= (\nabla_{fX}Y - \nabla_Y(fX)) - (fXY - YfX)\\
                &= (f\nabla_XY - ((Yf)X + f\nabla_YX)) - (fXY - ((Yf)X + fYX))\\
                &= f\nabla_XY - f\nabla_YX - f(XY - YX)\\
                &= fT(X, Y)
        \end{align*}
        同様に
            \[T(X, fY) = fT(X, Y)\]
        だから、$T(X, Y)$は双線形写像つまりテンソルである。
    \end{proof}

    捩率テンソルは
        \[T(X, Y) = -T(Y, X)\]
    なので反対称テンソルである。

    捩率テンソルの成分を
        \[T\(\pdv{x^i}, \pdv{x^j}\) = \sum_k T^k_{ij}\pdv{x^k}\]
    とおくと
    \begin{align*}
        T\(\pdv{x^i}, \pdv{x^j}\)
            &= \nabla_{\pdv{x^i}}\pdv{x^j} - \nabla_{\pdv{x^j}}\pdv{x^i} - 0\\
            &= \sum_k \(\Gamma^k_{ij} - \Gamma^k_{ji}\)\pdv{x^k}\\
        T^k_{ij} &= \Gamma^k_{ij} - \Gamma^k_{ji}
    \end{align*}
    となる。$T^k_{ij} = -T^k_{ji}$が成り立つことからも反対称であることが分かる。

    % 曲率形式を曲率テンソルを用いて書くと
    %    \[\Omega_i^j = \sum_{k,l}\frac{1}{2}R^j_{ikl}\theta^k \wedge \theta^l\]
    % 捩率形式を捩率テンソルを用いて書くと
    %     \[\Theta^k = \sum_{i,j} T^k_{ij}\theta^i \wedge \theta^j\]

\subsection{ビアンキの恒等式}
    $K(X, Y, Z)$の巡回和を
        \[\mathfrak{G}{K(X, Y, Z)} = K(X, Y, Z) + K(Y, Z, X) + K(Z, X, Y)\]
    とすると以下の定理が成り立つ。
    \begin{thm}[ビアンキの恒等式]
        任意のベクトル場$X, Y, Z$に対して
        \begin{gather*}
            \mathfrak{G}{R(X, Y)Z} = \mathfrak{G}{T(T(X, Y), Z)} + \mathfrak{G}{(\nabla_XT)(Y, Z)}\\
            \mathfrak{G}{(\nabla_ZR)(X, Y) + R(T(X, Y), Z)} = 0
        \end{gather*}
        が成り立つ。
    \end{thm}

    % \begin{align*}
    %     D\Omega &= 0\\
    %     D\Theta &= \Omega \wedge \theta\\
    % \end{align*}

% \subsection{測地線}
\section{リーマン幾何学}

リーマン多様体または擬リーマン多様体に関する微分幾何学をリーマン幾何学という。

\subsection{リーマン計量}
    \begin{dfn}[リーマン計量]
        $n$次元可微分多様体$M$に関して、$(0, 2)$型テンソル$g: \mathfrak{X}(M) \times \mathfrak{X}(M) \to \mathfrak{F}$が
        \begin{description}
            \item[対称性] $g(X, Y) = g(Y, X)$
            \item[正値性] $g(X, X) \geq 0, g(X, X) = 0 \iff X = 0$
        \end{description}
        を満たすとき$M$のリーマン計量という。
    \end{dfn}
    余接束の双対基底$\{dx^1, dx^2, \dots, dx^n\}$を用いて
        \[g = \sum_{i,j} g_{ij} dx^i \otimes dx^j\]
    と書くこともできる。つまり
        \[g_{ij} = g\(\pdv{x^i}, \pdv{x^j}\)\]
    である。また内積を持つベクトル空間$V$に対して
        \[X \wedge Y: Z \in V \mapsto g(Y, Z)X - g(X, Z)Y \in V\]
    と定義する。
    \begin{dfn}[リーマン多様体]
        $(M, g)$をリーマン多様体と呼ぶ。
    \end{dfn}

    \begin{dfn}[等長写像]
        リーマン多様体$(M_1, g_1), (M_2, g_2)$の間の写像$f: M_1 \to M_2$であって
            \[g_1(X, Y) = g_2(df(X), df(Y))\]
        を満たすものを等長写像という。全単射な等長写像を等長変換という。
    \end{dfn}

\subsection{レヴィ=チヴィタ接続}
    \begin{dfn}[レヴィ=チヴィタ接続(リーマン接続),リーマン幾何学の基本定理]
        リーマン多様体$(M, g)$について、捩れのない計量接続(計量テンソルを保存する接続)が一意に存在する。これをレヴィ=チヴィタ接続またはリーマン接続という。
        \begin{enumerate}
            \item $T(X, Y) = \nabla_XY - \nabla_YX - [X, Y] = 0$
            \item $\nabla_Xg(Y, Z) = Xg(Y, Z) - g(\nabla_XY, Z) - g(Y, \nabla_XZ) = 0$
        \end{enumerate}
    \end{dfn}
    レヴィ=チヴィタ接続の条件より、接続係数をリーマン計量で表すことができる。第一式より
        \[\Gamma^k_{ij} = \Gamma^k_{ji}\]
    第二式で$X = \pdv*{x^i}, Y = \pdv*{x^j}, Z = \pdv*{x^k}$として
    \begin{align*}
        \nabla_{\pdv{x^i}}g\(\pdv{x^j}, \pdv{x^k}\)
            &= \pdv{x^i}g\(\pdv{x^j}, \pdv{x^k}\) - g\(\sum_l \Gamma^l_{ij}\pdv{x^l}, \pdv{x^k}\) - g\(\pdv{x^j}, \sum_l \Gamma^l_{ik}\pdv{x^l}\)\\
            &= \pdv{g_{jk}}{x^i} - \sum_l \Gamma^l_{ij}g_{lk} - \sum_l \Gamma^l_{ik}g_{jl}\\
            &= 0
    \end{align*}
    添え字を巡回的に入れ替えることで同様に
    \begin{align*}
        \sum_l \Gamma^l_{ij}g_{lk} + \sum_l \Gamma^l_{ik}g_{jl} &= \pdv{g_{jk}}{x^i}\\
        \sum_l \Gamma^l_{jk}g_{li} + \sum_l \Gamma^l_{ji}g_{kl} &= \pdv{g_{ki}}{x^j}\\
        \sum_l \Gamma^l_{ki}g_{lj} + \sum_l \Gamma^l_{kj}g_{il} &= \pdv{g_{ij}}{x^k}
    \end{align*}
    第一式$+$第二式$-$第三式を計算する。$\Gamma^k_{ij} = \Gamma^k_{ji}, g_{ij} = g_{ji}$であることに注意すると
        \[2\sum_l \Gamma^l_{ij}g_{lk} = \pdv{g_{jk}}{x^i} + \pdv{g_{ki}}{x^j} - \pdv{g_{ij}}{x^k}\]
    $g_{ij}$の逆行列の成分を$g^{ij}$とおく。両辺に$g^{mk}$を掛けて$k$について和を取ると
        \[\Gamma^m_{ij} = \frac{1}{2}\sum_k g^{mk}\(\pdv{g_{jk}}{x^i} + \pdv{g_{ki}}{x^j} - \pdv{g_{ij}}{x^k}\)\]
    となる。

    % \subsection{共変微分}
    % 反変ベクトル$A^i(X^j)$を座標変換したものが$a^i(x^j)$であるとする。
    % \begin{align*}
    %     \pd[a^i]{x^j} &= \pdv{x^j}\lr{\pd[x^i]{X^l}A^l}\\
    %     &= \ppd{x^i}{x^j}{X^l}A^l + \pd[x^i]{X^l}\pd[X^k]{x^j}\pd[A^l]{X^k}\\
    %     \intertext{共変ベクトルも同様に、}
    %     \pd[a_i]{x^j} &= \pdv{x^j}\lr{\pd[X^l]{x^i}A^l}\\
    %     &= \ppd{X^l}{x^i}{x^j}A_l + \pd[X^l]{x^i}\pd[X^k]{x^j}\pd[A_l]{X^k}
    % \end{align*}
    % 第二項だけを見ればそれぞれ混合テンソル、共変テンソルのようになっている。
    % \begin{align*}
    %     \na_ja^i &= \pd[a^i]{x^j}-\ppd{x^i}{x^j}{X^l}A^l\\
    %     &= \pd[a^i]{x^j}-\ppd{x^i}{x^j}{X^l}\pd[X^l]{x^k}a^k\\
    %     \na_ja_i &= \pd[a_i]{x^j}-\ppd{X^l}{x^i}{x^j}A_l\\
    %     &= \pd[a_i]{x^j}-\ppd{X^l}{x^i}{x^j}\pd[x^k]{X^l}a_k
    % \end{align*}
    % とすれば$X$と$x$の二つの座標系間の変換で不変となる。これを共変微分という。第二項の$a^k,a_k$の係数は接続係数と呼ばれる。座標に依存しないように計量を使って書き直す。ユークリッド計量を$\delta_{ij}$とすると、
    %     \[g_{ij} = \pd[X^m]{x^i}\pd[X^n]{x^j}\delta_{mn}\]
    % これを微分して、
    %     \[\pd[g_{ij}]{x^k} = \delta_{mn}\(\ppd{X^m}{x^i}{x^k}\pd[X^n]{x^j} + \pd[X^m]{x^i}\ppd{X^n}{x^j}{x^k}\)\]
    % 計量は対称テンソルなので、添え字を巡回的に入れ替えて足し引きする。
    %     \[\frac{1}{2}\(\pd[g_{jk}]{x^i}+\pd[g_{ki}]{x^j}-\pd[g_{ij}]{x^k}\) = \delta_{mn}\ppd{X^m}{x^i}{x^j}\pd[X^n]{x^k}\]
    % とすると左辺は第一種クリストッフェル記号そのものであり$\Ga_{kij}$と書く。第二種クリストッフェル記号を$\Ga^l_{ij}$と書き、
    % \begin{align*}
    %     \Ga^l_{ij} &= g^{lk}\Ga_{kij}\\
    %     &= g^{lk}\delta_{mn}\ppd{X^m}{x^i}{x^j}\pd[X^n]{x^k}\\
    %     &= \pd[x^l]{X^u}\pd[x^k]{X^v}\delta^{uv}\delta_{mn}\ppd{X^m}{x^i}{x^j}\pd[X^n]{x^k}\\
    %     \intertext{$\delta^{ij},\delta_{ij}=0(i\neq j)$なので}
    %     &= \pd[x^l]{X^u}\pd[x^k]{X^u}\ppd{X^m}{x^i}{x^j}\pd[X^m]{x^k}\\
    %     &= \ppd{X^m}{x^i}{x^j}\pd[x^l]{X^u}\pd[X^m]{X^u}\\
    %     \intertext{$\pd[X^m]{X^u}=\delta^m_u$より}
    %     &= \ppd{X^m}{x^i}{x^j}\pd[x^l]{X^m}
    % \end{align*}
    % となって共変ベクトルの共変微分の接続係数となる。一方、反変ベクトルの共変微分の接続係数は、
    % \begin{align*}
    %     \ppd{x^i}{x^j}{X^l}\pd[X^l]{x^k} &= \pdv{x^j}\lr{\pd[x^i]{X^l}}\pd[X^l]{x^k}\\
    %     &= \pdv{x^j}\lr{\pd[x^i]{X^l}\pd[X^l]{x^k}}-\pd[x^i]{X^l}\pdv{x^j}\pd[X^l]{x^k}\\
    %     &= \ppd{x^i}{x^j}{x^k}-\ppd{X^l}{x^j}{x^k}\pd[x^i]{X^l}\\
    %     &= -\Ga^i_{jk}
    % \end{align*}
    % となる。つまり接続係数はどちらもクリストッフェル記号で表すことができる。共変微分の定義を書き直すと、
    % \begin{gather*}
    %     \na_ja^i = \pd[a^i]{x^j}+\Ga^i_{jk}a^k\\
    %     \na_ja_i = \pd[a_i]{x^j}-\Ga^k_{ij}a_k
    % \end{gather*}
    % となる。一般のテンソルに対しても同様に定義することができる。二階のテンソルの場合は、
    % \begin{align*}
    %     \na_kT^{ij} &= \pd[T^{ij}]{x^k}+\Ga^i_{km}T^{mj}+\Ga^j_{km}T^{mj}\\
    %     \na_kT^i_j &= \pd[T^i_j]{x^k}+\Ga^i_{km}T^m_j-\Ga^m_{jk}T^i_m\\
    %     \na_kT_{ij} &= \pd[T_{ij}]{x^k}-\Ga^m_{ik}T_{mj}-\Ga^m_{jk}T_{im}
    % \end{align*}
    % となる。
    % 計量テンソルを共変微分すると、デカルト座標で考えることにより、
    %     \[\na_kg_{ij} = \pa_k\delta_{ij} = 0\]
    % となる。これを計量条件という。反変の計量テンソルについても同様に0となる。

\subsection{リーマン曲率}
    アフィン接続における曲率テンソル及び、それによって定義される$(0, 4)$型テンソル
        \[R(X, Y, Z, W) = g(R(Z, W)Y, X)\]
    をリーマン曲率テンソルという。
    \begin{align*}
        R_{ijkl}
            &= R\(\pdv{x^i}, \pdv{x^j}, \pdv{x^k}, \pdv{x^l}\)\\
            &= g\(R\(\pdv{x^k}, \pdv{x^l}\)\pdv{x^j}, \pdv{x^i}\)\\
            &= g\(\sum_h R^h_{jkl}\pdv{x^h}, \pdv{x^i}\)\\
            &= \sum_h R^h_{jkl}g\(\pdv{x^h}, \pdv{x^i}\)\\
            &= g_{ih}R^h_{jkl}
    \end{align*}
    となる。

    リッチテンソル$\Ric(X, Y)$を線形変換$V \mapsto R(V, X)Y$のトレースとして定義する。
        \[\Ric = \sum_{i,j} R_{ij} dx^i \otimes dx^j\]
    とおくと
        \[R\(\pdv{x^k}, \pdv{x^i}\)\pdv{x^j} = R^l_{jki}\pdv{x^l}\]
    より
        \[R_{ij} = \sum_k R^k_{ikj}\]
    また$\Ric$のトレースをリッチスカラー(スカラー曲率)$R$と呼ぶ。(0, 2)型テンソル
        \[G = \Ric - \frac{1}{2}Rg\]
    をアインシュタインテンソルと呼ぶ。

\subsection{ビアンキの恒等式}
    アフィン接続におけるビアンキの恒等式で$T = 0$とすると
    \begin{gather*}
        \mathfrak{G}{R(X, Y)Z} = 0\\
        \mathfrak{G}{(\nabla_ZR)(X, Y)} = 0
    \end{gather*}
    これをさらに局所座標の成分で書くと
    \begin{gather*}
        R^l_{ijk} + R^l_{jki} + R^l_{kij} = 0\\
        \nabla_mR^i_{jkl} + \nabla_kR^i_{jlm} + \nabla_lR^i_{jmk} = 0
    \end{gather*}
    となる。

    % 共変微分のヤコビ恒等式や曲率テンソルの接続係数による表示から導出することもできる。
    % \begin{align*}
    %     [\nabla_i, \nabla_j]\(X^k\pdv{x^k}\)
    %     &= R\(\pdv{x^i}, \pdv{x^j}\)\(X^k\pdv{x^k}\)\\
    %     &= R\(\pdv{x^i}, \pdv{x^j}\)\pdv{x^k} = \sum_l R^l_{kij}\pdv{x^l}\\
    % \end{align*}
    % だから
    % \begin{align*} 
    %     \nabla_i[\nabla_j, \nabla_k]\(X^l\pdv{x^l}\)
    %     &= \nabla_i\(X^l\sum_h R^h_{ljk}\pdv{x^h}\)\\
    %     &= \sum_h \(\pd[X^l]{x^i}R^h_{ljk} + X^l\pd[R^h_{ljk}]{x^i}\)\pdv{x^h} + \sum_h X^lR^h_{ljk}\nabla_i\pdv{x^h}\\
    % \end{align*}
    % となる。また
    % \begin{align*}
    %     [\nabla_j, \nabla_k]\nabla_i\(X^l\pdv{x^l}\)
    %     &= \pd[X^l]{x^i}[\nabla_j, \nabla_k]\pdv{x^l}\\
    %     &= \pd[X^l]{x^i}\sum_h R^h_{ljk}\pdv{x^h}
    %     \nabla_j\nabla_k\nabla_i\pdv{x^l} - \nabla_k\nabla_j\nabla_i\pdv{x^l}
    %     \nabla_j[\nabla_k, \nabla_i]\pdv{x^l} + \nabla_j\nabla_i\nabla_k\pdv{x^l} - \nabla_k[\nabla_j, \nabla_i]\pdv{x^l} - \nabla_k\nabla_i\nabla_j\pdv{x^l}

    %     [\nabla_i, [\nabla_j, \nabla_k]]\(X^l\pdv{x^l}\)
    % \end{align*}
    % \begin{gather*}
    %     R^l_{ijk} + R^l_{jki} + R^l_{kij} = 0\\
    %     \nabla_mR^i{jkl} + \nabla_kR^i_{jlm} + \nabla_lR^i_{jmk} = 0\\
    % \end{gather*}
    % となる。

        \[\mathfrak{G}{(\nabla_ZR)(X, Y)} = 0\]


    ビアンキの第二恒等式を変形する。
        \[\nabla_k R^h_{ilj} - \nabla_jR^h_{ilk} + \nabla_lR^h_{ijk} = 0\]
    $h=l$として縮約し、第三項を書き換える。
        \[\nabla_kR_{ij} - \nabla_jR^h_{ik} + \nabla_lg^{ih}R_{hijk} = 0\]
    両辺に$g^{ij}$を掛ける。計量条件より共変微分の中に入れることができて、
    \begin{align*}
        \nabla_kR - \nabla_jR^j_k - \nabla_lg^{ih}R^j_{hjk}
        &= \nabla_kR - \nabla_jR^j_k - \nabla_lR^i_k\\
        &= \nabla_kR - 2\nabla_jR^j_k\\
        \intertext{第一項を書き換えて、}
        &= \nabla_j(\delta^j_kR) - 2\nabla_jR^j_k = 0
    \end{align*}
    となる。
        \[G^j_k = R^j_k - \frac{1}{2}\delta^j_kR\]
    とおく。添え字を下げて二階の共変テンソルにすれば、
        \[G^{ij} = g^{ik}G^j_k = R^{ij} - \frac{1}{2}Rg^{ij}\]
    計量条件より
        \[\nabla_jG^{ij} = 0\]
    である。

\subsection{断面曲率}
    $T_p(M)$の2次元部分空間$\sigma$を平面または平面切り口という。$\sigma$の基底$\{X, Y\}$を取ると、シュワルツの不等式より$g(X, X)g(Y, Y) - g(X, Y)^2 > 0$となる。そこで$\sigma$の断面曲率を
        \[K(\sigma) = \frac{g(R(X, Y)Y, X)}{g(X, X)g(Y, Y) - g(X, Y)^2}\]
    と定義する。
    % $K(\sigma)$が基底に依らないことを示す。$ad - bc \neq 0$なる$a,b,c,d$を取って
    %     \[X' = aX + bY, Y' = cX + dY\]
    % とする。
    % \begin{align*}
    %     g(R(X', Y')Y', X')
    %     &= g(R(aX + bY, cX + dY)(cX + dY), aX + bY)\\
    %     &= g((adR(X, Y) + bcR(Y, X))(cX + dY), aX + bY)\\
    %     &= (ad - bc)g(R(X, Y)(cX + dY), aX + bY)\\
    %     &= (ad - bc)g(cR(X, Y)X + dR(X, Y)Y, aX + bY)\\
    %     \bmat{g(X', X') & g(X', Y')\\ g(Y', X') & g(Y', Y')}
    %     &= \bmat{a^2g(X, X) + 2abg(X, Y) + b^2g(Y, Y) & acg(X, X) + (ad + bc)g(X, Y) + bdg(Y, Y)\\ acg(X, X) + (ad + bc)g(X, Y) + bdg(Y, Y) & c^2g(X, X) + 2cdg(X, Y) + d^2g(Y, Y)}\\
    %     &= \bmat{a & c\\ b & d}\bmat{g(X, X) & g(X, Y)\\ g(Y, X) & g(Y, Y)}\bmat{a & b\\ c & d}
    %     g(X', X')g(Y', Y') - g(X', Y')^2 = (ad - bc)^2(g(X, X)g(Y, Y) - g(X, Y)^2)
    % \end{align*}
    % より示された。
    正規直交基底$\{X, Y\}$を取れば
        \[K(\sigma) = g(R(X, Y)Y, X)\]
    である。
    \begin{thm}
        $p \in M$について以下は同値である。
        \begin{itemize}
            \item $T_p(M)$の任意の平面$\sigma$に対して$K(\sigma) = c$
            \item $R(X, Y) = cX \wedge Y$
        \end{itemize}
    \end{thm}
    \begin{thm}[シューアの定理]
        $n(\geq 3)$次元の連結リーマン多様体$M$の各点で$K(\sigma)$が定数なら、$M$全体で$K(\sigma)$は定数
    \end{thm}

    $T_p(M)$の正規直交基底を$\{X_1, X_2, \dots, X_n\}$とする。$X_1, X_i(i \geq 2)$で張られる平面を$\sigma_i$とすると、リッチテンソルは
    \begin{align*}
        Ric(X, X)
            &= \sum_{i=1}^n g(R(X_i, X)X, X_i)\\
            &= \sum_{i=2}^n g(R(X_i, X)X, X_i)\\
            &= \sum_{i=2} K(\sigma_i)
    \end{align*}
\section{ユークリッド空間内の超曲面}

\subsection{基本テンソルとガウス=ワインガルテンの公式}
    $n$次元多様体$M$とユークリッド空間$E^{n+1}$へのはめ込み$f$が与えられている。$p \in M$に対して$M$上のリーマン計量を$E^{n+1}$内の通常の内積を用いて
        \[g_p(X, Y) = \<df(X), df(Y)\>\]
    と定義する。これを$M$の第一基本形式または誘導されたリーマン計量という。$M$の局所座標を$\{x^1, x^2, \dots, x^n\}$、$E^{n+1}$のユークリッド座標を$\{y^1, y^2, \dots, y^{n+1}\}$とすると
        \[f*\(\pdv{x^i}\) = \sum_{k=1}^{n+1} \pdv{f^k}{x^i}\pdv{y^k}\]
    なので
    \begin{align*}
        g_{ij} &= g\(\pdv{x^i}, \pdv{x^j}\)\\
        &= \<\sum_{k=1}^{n+1} \pdv{f^k}{x^i}\pdv{y^k}, \sum_{k=1}^{n+1} \pdv{f^k}{x^j}\pdv{y^k}\>\\
        &= \sum_{k=1}^{n+1} \pdv{f^k}{x^i}\pdv{f^k}{x^j}
    \end{align*}
    となる。

    $E^{n+1}$の通常の共変微分を$D$とおく。$(D_Xdf(Y))_p$を$M$に接する成分と垂直な成分に分解すると
        \[(D_Xdf(Y))_p = df(\nabla_XY)_p + \alpha_p(X, Y)\]
    となる。$\alpha_p$は単位法ベクトル場$\xi \in \mathfrak{X}(E^{n+1})\ (\inner{\xi}{df(X)} = 0, \inner{\xi}{\xi} = 1)$を用いて
        \[\alpha_p(X, Y) = h(X, Y)\xi_p\]
    と書くことができる。$h(X, Y)$は双線形写像であり、第二基本形式と呼ぶ。また$\xi \in \mathfrak{X}_f$とすると
    \begin{align*}
        \inner{\xi}{\xi} = 1\\
        \inner{D_X\xi}{\xi} = 0
    \end{align*}
    より$D_X\xi$は$M$に接しており、$X \mapsto D_X\xi$は線形写像であるから
        \[D_X\xi = -df(SX)\]
    となる$(1, 1)$型テンソル場$S$が存在する。これをシェイプ(Shape)作用素と呼ぶ。シェイプ作用素は曲面の外界への入り方を表す外在的な量である。まとめると
    \begin{align*}
        D_Xdf(Y) &= df(\nabla_XY) + h(X, Y)\xi \tag{ガウスの公式}\\
        D_X\xi &= -df(SX) \tag{ワインガルテンの公式}
    \end{align*}
    それぞれの公式は$M$上を移動したときの接ベクトル、法ベクトルの変化率を与えるもので、$M$上のベクトルと$E^{n+1}$上のベクトルの関係を表す。

\subsection{ガウス=コダッチ方程式}
    \begin{align*}
        \<df(Y), \xi\> = 0\\
        \<D_Xdf(Y), \xi\> + \<df(Y), D_X\xi\> = 0\\
        h(X, Y) + \<df(Y), -df(SX)\> = 0\\
        h(X, Y) - g(SX, Y) = 0
    \end{align*}

    ガウス=ワインガルテンの公式を満たす超曲面が存在するためにシェイプ作用素が満たすべき条件を求める。$E^{n+1}$の共変微分$D$が満たす性質を$M$の共変微分$\nabla$を用いて書き換える。$D$は曲率テンソル、捩率テンソルが共に0で、ユークリッド計量に対して計量的である。つまり
    \begin{gather*}
        D_XD_Ydf(Z) - D_YD_Xdf(Z) - D_[X, Y]df(Z) = 0\\
        D_Xdf(Y) - D_Ydf(X) - df([X, Y]) = 0\\
        X\<df(Y), df(Z)\> - \<D_Xdf(Y), df(Z)\> - \<df(Y), D_Xdf(Z)\> = 0\\
    \end{gather*}

    捩率テンソルの式にガウスの公式を適用すると
    \begin{align*}
        (df(\nabla_XY) + h(X, Y)\xi) - (df(\nabla_YX) + h(Y, X)\xi) - df([X, Y]) = 0
    \end{align*}
    $M$に接する成分と垂直な成分に分解すると
    \begin{gather*}
        \nabla_XY - \nabla_YX - [X, Y] = 0\\
        h(X, Y) = h(Y, X)
    \end{gather*}
    つまり$\nabla$の捩率テンソルも0になる。また$h(X, Y), S$が対称線形であることが分かる。

    次に計量条件の式を書き換える
    \begin{align*}
        Xg(Y, Z) - \<df(\nabla_XY), df(Z)\> - \<df(Y), df(\nabla_XZ)\> &= 0\\
        Xg(Y, Z) - g(\nabla_XY, Z) - g(Y, \nabla_XZ) &= 0
    \end{align*}
    つまり$\nabla$も計量的である。捩率テンソルが0でかつ計量的なので$\nabla$はレヴィ=チヴィタ接続である必要がある。

    曲率テンソルの式にガウス=ワインガルテンの公式を適用すると
    \begin{align*}
        D_XD_Ydf(Z)
        &= D_X(df(\nabla_YZ) + h(Y, Z)\xi)\\
        &= df(\nabla_X\nabla_YZ) + h(X, \nabla_YZ)\xi + X(h(Y, Z))\xi + h(Y, Z)D_X\xi\\
        &= df(\nabla_X\nabla_YZ - h(Y, Z)SX) + (h(X, \nabla_YZ) + X(h(Y, Z)))\xi
    \end{align*}
    同様に
        \[D_YD_Xdf(Z) = df(\nabla_Y\nabla_XZ - h(X, Z)SY) + (h(Y, \nabla_XZ) + Y(h(X, Z)))\xi\]
    であり、また
        \[D_[X, Y]df(Z) = df(\nabla_[X, Y]Z) + h([X, Y], Z)\xi\]
    より
    \begin{align*}
        \{df(\nabla_X\nabla_YZ - h(Y, Z)SX) - df(\nabla_Y\nabla_XZ - h(X, Z)SY) - df(\nabla_[X, Y]Z)\} + \{(h(X, \nabla_YZ) + X(h(Y, Z)))\xi - (h(Y, \nabla_XZ) + Y(h(X, Z)))\xi - h([X, Y], Z)\xi\} = 0
    \end{align*}
    $M$に接する成分と垂直な成分に分解する。接する成分は
    \begin{align*}
        \nabla_X\nabla_YZ - \nabla_Y\nabla_XZ - \nabla_[X, Y]Z &= h(Y, Z)SX - h(X, Z)SY\\
        R(X, Y)Z &= g(SY, Z)SX - g(SX, Z)SY
        R(X, Y) &= SX \wedge SX
    \end{align*}
    垂直な成分は
    \begin{align*}
        X(h(Y, Z)) - Y(h(X, Z)) + h(X, \nabla_YZ) - h(Y, \nabla_XZ) - h([X, Y], Z) = 0\\
        X(h(Y, Z)) - Y(h(X, Z)) + h(X, \nabla_YZ) - h(Y, \nabla_XZ) + h(\nabla_YX, Z) - h(\nabla_XY, Z) = 0\\
        \nabla_Xh(Y, Z) = \nabla_Yh(X, Z)\\
        \nabla_XS(Y) = \nabla_YS(X)
    \end{align*}
    まとめると
    \begin{gather*}
        R(X, Y) = SX \wedge SX \tag{ガウスの方程式}\\
        \nabla_XS(Y) = \nabla_YS(X) \tag{コダッチの方程式}\\
    \end{gather*}
    ガウス=コダッチの方程式はリーマン計量によって定まる曲率テンソルとシェイプ作用素が満足すべき条件を示す。実際次の定理が成り立つ。
    \begin{thm}[超曲面の基本定理]
        単連結なリーマン多様体$(M, g)$上で$(1, 1)$型テンソル場$S$が$g$のレヴィ=チヴィタ接続に関してコダッチの方程式を満たすとき、$M$から$E^{n+1}$への等長的なはめ込みが合同変換を除いて一意的に存在する。
    \end{thm}

\subsection{驚異の定理}
    $M$が$E^3$の2次元リーマン多様体つまり曲面のときを考える。$S$の固有値を$\lambda_1, \lambda_2$、それに対応する単位固有ベクトルを$X_1, X_2$とおく。ここで$S$は対称変換である。つまり
    \begin{gather*}
        h(X_1, X_2) = g(AX_1, X_2) = \lambda_1g(X_1, X_2)\\
        h(X_1, X_2) = g(X_1, AX_2) = \lambda_2g(X_1, X_2)\\
        (\lambda_2 - \lambda_1)g(X_1, X_2) = 0
    \end{gather*}
    なので、異なる固有値に属する固有ベクトルは直交する。固有値が縮退する場合も直交する固有ベクトルを取ることができる。つまり$g(X_1, X_1) = g(X_2, X_2) = 1, g(X_1, X_2) = 0$である。$K = \lambda_1\lambda_2 = \det S$をガウス曲率という。ガウスの方程式より
    \begin{align*}
        R(X_1, X_2)
            &= SX_1 \wedge SX_2 = \lambda_1X_1 \wedge \lambda_2X_2\\
            &= KX_1 \wedge X_2
    \end{align*}
    断面曲率は
    \begin{align*}
        K(\sigma)
            &= g(R(X_1, X_2)X_2, X_1)\\
            &= g(K(X_1 \wedge X_2)X_2, X_1)\\
            &= g(Kg(X_2, X_2)X_1, X_1)\\
            &= Kg(X_1, X_1)g(X_2, X_2) = K
    \end{align*}
    となりガウス曲率に一致する。断面曲率は基底に依らないので、ガウス曲率はリーマン計量によって完全に決定する。

\section{定曲率空間}
断面曲率が一定のリーマン多様体を定曲率空間という。

\subsection{$n$次元球面}
    $E^{n+1}$内の半径$r$の$n$次元球面を
        \[S^n(r) = {x \in E^{n+1} \mid \<x, x\> = r^2}\]
    と定義する。また$n$次元単位球面を単に$S^n = S^n(1)$と書く。点$p$における接ベクトル$X \in T_p(S^n(r))$、単位法ベクトル$\xi = -p/r$に対して
        \[D_X\xi = -\frac{X}{r}\]
    よりシェイプ作用素は
        \[S = \frac{1}{r}I\]
    となる。ガウスの方程式より
    \begin{align*}
        R(X, Y) = \frac{1}{r^2}X \wedge Y\\
        K(\sigma) = \frac{1}{r^2}
    \end{align*}
    となる。

\subsection{双曲空間}
        \[\<x, y\> = \sum_{i=1}^n x^iy^i - x^{n+1}y^{n+1}\]
    をローレンツ内積と呼ぶ。$R^{n+1}$にローレンツ内積が定義された空間を$n+1$次元ローレンツ空間$L^{n+1}$と呼ぶ。
        \[\{x \in L^{n+1} \mid \<x, x\> = -r^2\}\]
    は$x^{n+1} > 0$と$x^{n+1} < 0$の二つの連結成分を持つ。そこで$n$次元双曲空間を
        \[H^n(r) = \{x \in L^{n+1} \mid \<x, x\> = -r^2, x^{n+1} > 0\}\]
    と定義する。点$p$における接ベクトル$X \in T_p(H^n(r))$、単位法ベクトル$\xi = -p/r$に対して
        \[D_X\xi = -\frac{X}{r}\]
    よりシェイプ作用素は
        \[S = \frac{1}{r}I\]
    ガウスの方程式より
    \begin{align*}
        R(X, Y) &= -\frac{1}{r^2}X \wedge Y\\
        K(\sigma) &= -\frac{1}{r^2}
    \end{align*}
    となる。

\begin{thm}
    $M_1, M_2$が同じ曲率を持つ$n$次元定曲率空間であるとする。$x_0 \in M_1$と$T_{x_0}(M_1)$の正規直交基底$\{X_1, X_2, \dots, X_n\}$、$y_0 \in M_2$と$T_{y_0}(M_2)$の正規直交基底$\{Y_1, Y_2, \dots, Y_n\}$に対して、$x_0$の近傍から$y_0$の近傍への等長変換$f$で
        \[f(x_0) = y_0, df(X_i) = Y_i\]
    となるものが存在する。
\end{thm}
\begin{thm}
    $n$次元局所対称的リーマン多様体$M_1, M_2$に対して、$x_0 \in M_1$の近傍から$y_0 \in M_2$の近傍への等長変換$f$で
        \[f(x_0) = y_0, df(R_1(X, Y)Z) = R_2(df(X), df(Y))df(Z)\]
    となるものが存在する。
\end{thm}
\begin{cor}
    一定曲率$c$を持つ$n$次元リーマン多様体は
    \[\begin{cases}
        E^n & (c = 0)\\
        S^n(r) & (c > 0, r = \frac{1}{\sqrt{c}})\\
        H^n(r) & (c < 0, r = \frac{1}{-\sqrt{c}})
    \end{cases}\]
    のいずれかと局所等長的である。
\end{cor}
\begin{cor}
    $n$次元定曲率空間の2点$x, y$とその正規直交基底$\{X_1, X_2, \dots, X_n\}, \{Y_1, Y_2, \dots, Y_n\}$に対して、局所等長写像
        \[f(x) = y, df(X_i) = Y_i\]
    が存在する。
\end{cor}
\begin{cor}
    局所対称的リーマン多様体の2点$x, y$に対して、$x$の近傍から$y$の近傍への等長変換が存在する。
\end{cor}

\nocite{*}
\bibliographystyle{plain}
\bibliography{refs.bib}

\end{document}